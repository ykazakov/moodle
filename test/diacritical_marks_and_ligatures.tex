% !TeX encoding = UTF-8
% !TeX spellcheck = en_US
\documentclass{article}
%% FOR PDFLATEX
\usepackage[T1]{fontenc} % necessary
\usepackage[utf8]{inputenc} % necessary
%% FOR XELATEX
%\usepackage{fontspec}
%%
\usepackage[french]{babel} % for french quotes
%\usepackage{moodle}
%\usepackage{../moodlev5}
\usepackage{../moodle}


\def\UTFdiacriticLC
{à   â   é   è   ë   ê   î   ï   ô   ö   ù   ü   û   ç  å }
\def\UTFdiacriticUC
{À   Â   É   È   Ë   Ê   Î   Ï   Ô   Ö   Ù   Ü   Û   Ç  Å }
\def\LaTeXdiacriticLC
{\`a \^a \'e \`e \"e \^e \^i \"i \"o \^o \`u \"u \^u \c{c} \c{s} \c{t} \H{o} 
\H{u} \aa }
\def\LaTeXdiacriticUC
{\`A \^A \'E \`E \"E \^E \^I \"I \"O \^O \`U \"U \^U \c{C} \c{S} \c{T} \H{O} 
\H{U} \AA }

\def\UTFligaturesLC{œ   æ } % XML is only correct when compiled with xelatex
\def\UTFligaturesUC{Œ   Æ } % XML is only correct when compiled with xelatex
\def\LaTeXligaturesLC{\oe\ \ae\ } % XML is only correct when compiled with 
%xelatex
\def\LaTeXligaturesUC{\OE\ \AE\ } % XML is only correct when compiled with 
%xelatex

\htmlregister{\UTFdiacriticLC}
\htmlregister{\UTFdiacriticUC}
\htmlregister{\LaTeXdiacriticLC}
\htmlregister{\LaTeXdiacriticUC}
\htmlregister{\UTFligaturesLC}
\htmlregister{\UTFligaturesUC}
\htmlregister{\LaTeXligaturesLC}
\htmlregister{\LaTeXligaturesUC}

\begin{document}

\section*{Introduction}

This document is intended to check the support of diacritical marks (accents) 
and ligatures found in french typesetting.

For most of them, two ways of encoding are tested: UTF8 et \LaTeX{} syntax.

\begin{quiz}{Diacritical marks and ligatures}

\begin{multi}{\UTFdiacriticUC}
%\UTFligaturesLC \UTFligaturesUC\\ % UTF8-coded ligatures only work with xelatex
\og\UTFdiacriticLC\\
\UTFdiacriticUC\fg
\item[feedback={\UTFdiacriticUC}] \UTFdiacriticLC
\item[feedback={\UTFdiacriticLC}]* \UTFdiacriticUC
\end{multi}

\begin{multi}{\LaTeXdiacriticUC}
\LaTeXligaturesLC \LaTeXligaturesUC\\
\og\LaTeXdiacriticLC\\
\LaTeXdiacriticUC\fg
\item[feedback={\LaTeXdiacriticUC}] \LaTeXdiacriticLC 
\item[feedback={\LaTeXdiacriticLC}]* \LaTeXdiacriticUC
\end{multi}

\end{quiz}

\end{document}
