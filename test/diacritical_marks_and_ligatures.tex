% !TeX encoding = UTF-8
% !TeX spellcheck = en_US
\documentclass{article}

\usepackage{../moodle}

\ifxetex % FOR XELATEX
	\usepackage{fontspec}
\else %% FOR PDFLATEX
	\usepackage[utf8]{inputenc} % necessary
	\usepackage[T1]{fontenc} % necessary
\fi
\usepackage{libertine} % includes the "großes Eszett" ß

\usepackage[french]{babel} % for french quotes

\def\UTFdiacriticLC
{à â é è ë ê î ï ô ö ù ü û ã ñ õ ç å 
ş ţ ő ű ă ĕ ğ ĭ ŏ ŭ č ď ě ľ ň ř š ť ž } % in this line XML is only correct when 
%compiled with xelatex
\def\UTFdiacriticUC
{À Â É È Ë Ê Î Ï Ô Ö Ù Ü Û Ã Ñ Õ Ç Å 
Ş Ţ Ő Ű Ă Ĕ Ğ Ĭ Ŏ Ŭ Č Ď Ě Ľ Ň Ř Š Ť Ž } % in this line XML is only correct when 
%compiled with xelatex
\def\LaTeXdiacriticLC
{\`a \^a \'e \`e \"e \^e \^i \"i \"o \^o \`u \"u \^u \~a \~n \~o \c{c} \aa\ 
\c{s} \c{t} \H{o} \H{u} \u{a} \u{e} \u{g} \u{\i} \u{o} \u{u} \v{c} \v{d} \v{e} 
\v{l} \v{n} \v{r} \v{s} \v{t} \v{z} }
\def\LaTeXdiacriticUC
{\`A \^A \'E \`E \"E \^E \^I \"I \"O \^O \`U \"U \^U \~A \~N \~O \c{C} \AA\  
\c{S} \c{T} \H{O} \H{U} \u{A} \u{E} \u{G} \u{I} \u{O} \u{U} \v{C} \v{D} \v{E} 
\v{L} \v{N} \v{R} \v{S} \v{T} \v{Z} }

\def\UTFligaturesLC{œ   æ   ß } % XML is only correct when compiled with xelatex
\def\UTFligaturesUC{Œ   Æ   ẞ } % XML is only correct when compiled with xelatex
\def\LaTeXligaturesLC{\oe\ \ae\ \ss\ }
\def\LaTeXligaturesUC{\OE\ \AE\ \SS\ }

\def\UTFotherLC{ø ł }
\def\UTFotherUC{Ø Ł }
\def\LaTeXotherLC{\o\ \l\ }
\def\LaTeXotherUC{\O\ \L\ }

\htmlregister{\UTFdiacriticLC}
\htmlregister{\UTFdiacriticUC}
\htmlregister{\LaTeXdiacriticLC}
\htmlregister{\LaTeXdiacriticUC}
\htmlregister{\UTFligaturesLC}
\htmlregister{\UTFligaturesUC}
\htmlregister{\LaTeXligaturesLC}
\htmlregister{\LaTeXligaturesUC}
\htmlregister{\UTFotherLC}
\htmlregister{\UTFotherUC}
\htmlregister{\LaTeXotherLC}
\htmlregister{\LaTeXotherUC}

\begin{document}

\section*{Introduction}

This document is intended to check the support of diacritical marks (accents) 
and ligatures found in french typesetting.

For most of them, two ways of encoding are tested: UTF8 et \LaTeX{} syntax.

\begin{quiz}[points=1]{Diacritical marks and ligatures}
\NoAutoSpacing
\begin{multi}{\UTFdiacriticUC}
%«~\UTFligaturesLC \UTFligaturesUC»\\ % only xelatex produces an adequate XML
%when using UTF8-coded ligatures and french quotes
\UTFdiacriticLC\\
\UTFdiacriticUC\\
%\UTFotherUC \UTFotherLC\\ % only xelatex produces an adequate XML
%when using UTF8-coded ligatures and french quotes
\item[feedback={\UTFdiacriticUC}] \UTFdiacriticLC
\item[feedback={\UTFdiacriticLC}]* \UTFdiacriticUC
\end{multi}

\begin{multi}{\LaTeXdiacriticUC} % Here both pdflatex and xelatex should work
\og\LaTeXligaturesLC \LaTeXligaturesUC\fg\\
\LaTeXdiacriticLC\\
\LaTeXdiacriticUC
\LaTeXotherUC \LaTeXotherLC\\
\item[feedback={\LaTeXdiacriticUC}] \LaTeXdiacriticLC 
\item[feedback={\LaTeXdiacriticLC}]* \LaTeXdiacriticUC
\end{multi}

\end{quiz}

\end{document}
