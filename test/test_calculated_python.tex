% !TeX encoding = UTF-8
% !TeX spellcheck = en_US
% !TEX TS-program = lualatex
\documentclass{article}
\usepackage[nostamp]{moodle}
\ifPDFTeX % FOR LATEX and PDFLATEX
	\usepackage[utf8]{inputenc} % necessary
	\usepackage[T1]{fontenc} % necessary
\else % assuming XELATEX or LUALATEX
	\usepackage{fontspec}
\fi
\usepackage{hyperref}
\usepackage{python}
\begin{document}

\section*{Introduction}

This document is intended to check the possibility of generating 
pseudo-calculated questions with the help of Python scripts.

Inspired by
\url{https://github.com/avohns/python-latex-moodle-quiz/blob/master/simple-examples-eng/example1_arithmetic.tex}

\begin{quiz}[tags={calculated}]{Example Quiz}
% the following syntax is Python 3
% on Ubuntu 20.10, I had to force python to default to python 3
% https://stackoverflow.com/a/50331137
\begin{python}
def clozenum_print(pair,op,result,initialText='',finalText='\n'):
  clozeNumStr = rf"""\begin{{numerical}}
{initialText}${pair[0]} {op} {pair[1]} =$
\item {result} 
\end{{numerical}}{finalText}"""
  print(clozeNumStr)
def cloze_print(pair,points):
  print(rf"""\begin{{cloze}}[points={points}]{{Arithmetic Quiz 
  {(pair[0],pair[1])}}}""")
  initialText = 'Solve the following tasks!\\newline\n'
  clozenum_print([x,y],"+",x+y,initialText=initialText)
  clozenum_print([x,y],"-",x-y)
  if pair[0]/pair[1] != pair[0]//pair[1]:
    clozenum_print([x,y],"*",x*y,finalText='')
  else:
    clozenum_print([x,y],"*",x*y)
  if pair[0]/pair[1] == pair[0]//pair[1]:
    clozenum_print([x,y],":",x//y,finalText='')
  print('\end{cloze}\n')
for x in range(2,5):
  for y in range(2,5):
    if x > y:
      if x/y == x//y:
        points=1
      else:
        points=2  
      cloze_print([x,y],points)
\end{python}
\end{quiz}
\end{document}