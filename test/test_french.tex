% !TeX encoding = UTF-8
% !TeX spellcheck = en_US
% !TEX TS-program = lualatex
\documentclass{article}
\usepackage[nostamp]{moodle}
\ifPDFTeX % FOR LATEX and PDFLATEX
	\usepackage[utf8]{inputenc} % necessary
	\usepackage[T1]{fontenc} % necessary
\else % assuming XELATEX or LUALATEX
	% \usepackage{fontspec}
\fi
\usepackage[french]{babel} % for french quotes
\usepackage{hyperref}
% pangramme conçu par Gilles Esposito-Farèse et cité dans "Petites leçons de 
% typographie" de Jacques André: https://jacques-andre.fr/faqtypo/lessons.pdf
\ifPDFTeX
  \def\myquestiontext{ \underline{Pangramme} con\c{c}u par 
  Gilles \textsc{Esposito-Far\`ese}\,:}
  \def\pangramme{D\`es No\"el o\`u un z\'ephyr ha\"i me 
  v\^et de gla\c{c}ons w\"urmiens je d\^ine d'exquis 
  r\^otis de b\oe uf au kir \`a l'a\"y d'\^age m\^ur \& 
  c\ae tera\,!}
  \def\myfeedback{cit\'e dans 
\og\,\href{https://jacques-andre.fr/faqtypo/lessons.pdf}{Petites
 le\c{c}ons de typographie}\,\fg\ de Jacques 
 \textsc{Andr\'e}.}
\else % assuming XELATEX or LUALATEX
  \def\myquestiontext{ \underline{Pangramme} conçu par 
  Gilles \textsc{Esposito-Farèse}:}
  \def\pangramme{Dès Noël où un zéphyr haï me vêt de 
  glaçons würmiens je dîne d’exquis rôtis de bœuf au kir à 
  l’aÿ d’âge mûr \& cætera\,!}
  \def\myfeedback{cité dans 
  \og\href{https://jacques-andre.fr/faqtypo/lessons.pdf}
  {Petites leçons de typographie}\fg\ de Jacques 
  \textsc{André}.}
\fi
\htmlregister{\myquestiontext}
\htmlregister{\pangramme}
\htmlregister{\myfeedback}
\begin{document}

\section*{Introduction}

This document is intended to check the support of the \texttt{babel} package 
with option \texttt{french} that causes problems during \texttt{pdfLaTeX} 
compilation, due to autospacing.

Secondarily, the use of \verb|\textsuperscript| (and french adaptation 
\verb|\fup|), \verb|\textsubscript|, \verb|\textsc|, \verb|\underline|, 
\verb|\url|, and \verb|\href| is also demonstrated.

\begin{quiz}[points=1]{French Issues}
\ifPDFTeX
  \NoAutoSpacing% this option preserves pdflatex compilation
\fi
\begin{multi}[feedback={1\fup{er}, M\fup{me}, N\fup{o}\\
1\textsuperscript{er}, M\textsuperscript{me}, N\textsuperscript{o}\\
H\textsubscript{2}O}]{Test french}
\myquestiontext
\item[feedback={\myfeedback}]* \pangramme
\item Portez ce vieux whisky au juge blond qui fume.
\end{multi}

\end{quiz}
\end{document}
