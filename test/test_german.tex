% !TeX encoding = UTF-8
% !TeX spellcheck = en_US
% !TEX TS-program = lualatex
\documentclass{article}
\usepackage[nostamp]{moodle}
\ifPDFTeX % FOR LATEX and PDFLATEX
	\usepackage[utf8]{inputenc} % necessary
	\usepackage[T1]{fontenc} % necessary
\else % assuming XELATEX or LUALATEX
	\usepackage{fontspec}
\fi
\usepackage[ngerman]{babel}

\ifPDFTeX
  \def\myquestiontext{ \underline{Pangramm} :}
  \def\pangramme{\glqq Fix, Schwyz!\grqq, qu\"akt J\"urgen bl\"od vom Pa\ss.}
  \def\pangrammeother{Wie schon Carl-Friedrich Gau\ss\ %
  sagte: \glqq Nicht alles \"Uble in \"Osterreich ist 
  \"osterreichischer Herkunft -- aber manches, was als \glq 
  \"ubel\grq\ empfunden wird, k\"onnte durch entsprechende 
  \"Anderungen zum Guten ver\"andert werden\grqq.}
  \def\myfeedback{}
\else % assuming XELATEX or LUALATEX
  \def\myquestiontext{ \underline{Pangramm} :}
  \def\pangramme{„Fix, Schwyz!“, quäkt Jürgen blöd vom Paß.}
  \def\pangrammeother{Wie schon Carl-Friedrich Gauß sagte: 
  „Nicht alles Üble in Österreich ist österreichischer 
  Herkunft – aber manches, was als ‚übel‘ empfunden wird, 
  könnte durch entsprechende Änderungen zum Guten verändert 
  werden“.}
  \def\myfeedback{Wie schon Carl-Friedrich Gau\ss\ %
  sagte: \glqq Nicht alles \"Uble in \"Osterreich ist 
  \"osterreichischer Herkunft -- aber manches, was als \glq 
  \"ubel\grq\ empfunden wird, k\"onnte durch entsprechende 
  \"Anderungen zum Guten ver\"andert werden\grqq.}
\fi
\htmlregister{\myquestiontext}
\htmlregister{\pangramme}
\htmlregister{\pangrammeother}
\htmlregister{\myfeedback}
\begin{document}

\section*{Introduction}

This document is intended to check the support of the \texttt{babel} package 
with option \texttt{ngerman} that might cause problems during compilation, due 
to the special meaning given to \verb|"|.

\begin{quiz}[points=1.0]{German Issues}
\shorthandoff{"}%preserves pdftex compilation and fixes the use of \" for 
%other engines
\begin{multi}[feedback={}]{Test German}
\myquestiontext
\item[feedback={\myfeedback}]* \pangramme
\item \pangrammeother
\end{multi}

\ifTUTeX
  \shorthandon{"} % This time we use " as an active character for Umlauts
  \begin{description}{Additionnal test}
    Wie schon Carl-Friedrich Gau\ss\ %
    sagte: \glqq Nicht alles "Uble in "Osterreich ist 
    "osterreichischer Herkunft -- aber manches, was als \glq 
    "ubel\grq\ empfunden wird, k"onnte durch entsprechende 
    "Anderungen zum Guten ver"andert werden\grqq.
  \end{description}
\fi

\end{quiz}
\end{document}
