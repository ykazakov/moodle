% !TeX encoding = UTF-8
% !TeX spellcheck = en_US
% !TEX TS-program = lualatex
\documentclass{article}
\usepackage[nostamp]{moodle}
\ifPDFTeX % FOR LATEX and PDFLATEX
	\usepackage[utf8]{inputenc} % necessary
	\usepackage[T1]{fontenc} % necessary
\else % assuming XELATEX or LUALATEX
	\usepackage{fontspec}
\fi
\begin{document}

\section*{Introduction}

This document is intended to check the support of the \texttt{allornothing} 
option for multiple choice questions.

\begin{quiz}[]{All Or Nothing}
%\begin{quiz}[allornothing]{All Or Nothing}

\begin{multi}[allornothing]{All Or Nothing}%
text
\item* a
\item[fraction=20] b
\item[fraction=-30] c
\item d
\end{multi}

\begin{multi}[allornothing,multiple]{All Or Nothing Wins Conflict 1}%
text
\item* a
\item* b
\item c
\item d
\end{multi}

\begin{multi}[multiple,allornothing]{All Or Nothing Wins Conflict 2}%
text
\item* a
\item* b
\item c
\item d
\end{multi}

\begin{multi}[allornothing,single=false]{All Or Nothing Wins Conflict 3}%
text
\item* a
\item* b
\item c
\item d
\end{multi}

\begin{multi}[single=false,allornothing]{All Or Nothing Wins Conflict 4}%
text
\item* a
\item* b
\item c
\item d
\end{multi}

\begin{cloze}{Title}
\begin{multi}[allornothing=false] % <- true results in an error issued
text
\item* a
\item[fraction=20] b
\item[fraction=-30] c
\item d
\end{multi}
\end{cloze}

\end{quiz}

\end{document}