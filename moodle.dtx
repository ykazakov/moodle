% \iffalse meta-comment
%
% moodle.dtx
% Copyright 2016 by Anders O.F. Hendrickson (anders.o.f.hendrickson AT gmail DOT com)
% and 2019-2021 by Matthieu Guerquin-Kern (guerquin-kern AT crans DOT org).
%
% This work may be distributed and/or modified under the
% conditions of the LaTeX Project Public License, either version 1.3
% of this license or (at your option) any later version.
% The latest version of this license is in
%   http://www.latex-project.org/lppl.txt
% and version 1.3 or later is part of all distributions of LaTeX
% version 2005/12/01 or later.
%
% This work has the LPPL maintenance status `maintained'.
%
% The Current Maintainer of this work is Matthieu Guerquin-Kern.
%
% This work consists of the files moodle.dtx and moodle.ins
% and the derived file moodle.sty.
%
% \fi
%
% \iffalse
%<*driver>
\ProvidesFile{moodle.dtx}
%</driver>
%<package>\NeedsTeXFormat{LaTeX2e}[1999/12/01]
%<package>\ProvidesPackage{moodle}
%<*package>
    [2021/03/10 v1.0dev Moodle quiz XML generation]
%</package>
%
%<*driver>
\documentclass[10pt,a4paper]{ltxdoc}
%\usepackage[draft]{moodle}
\usepackage{iftex}
\ifPDFTeX
  \usepackage[utf8]{inputenc} % necessary
  \usepackage[T1]{fontenc} % necessary
  \usepackage{libertine}
  \usepackage[scaled=0.83]{beramono}
\else % assuming LuaLaTeX or XeLaTeX
  \usepackage{fontspec}
  \setmainfont{Linux Libertine O}
\fi
\usepackage{microtype}
\usepackage{dtxdescribe,varioref}
\addtolength\marginparwidth{30pt}
\addtolength\oddsidemargin{20pt}
\addtolength\evensidemargin{20pt}
\usepackage[main=english,french,german]{babel}
\usepackage{amssymb,threeparttable,booktabs}
\usepackage{eurosym,longtable,tikz,minted,changelog}
\usepackage[pdfpagelabels]{hyperref}
\usetikzlibrary{arrows,positioning,decorations.text,calc}
%\usemintedstyle{Wombat}
\EnableCrossrefs
\CodelineIndex
\OnlyDescription
\RecordChanges
\setcounter{IndexColumns}{2}
\begin{document}
  \DocInput{moodle.dtx}
\end{document}
%</driver>
% \fi
%
% \CheckSum{7729} ^^A Comment \OnlyDescription above to adjust
%
% \CharacterTable
%  {Upper-case    \A\B\C\D\E\F\G\H\I\J\K\L\M\N\O\P\Q\R\S\T\U\V\W\X\Y\Z
%   Lower-case    \a\b\c\d\e\f\g\h\i\j\k\l\m\n\o\p\q\r\s\t\u\v\w\x\y\z
%   Digits        \0\1\2\3\4\5\6\7\8\9
%   Exclamation   \!     Double quote  \"     Hash (number) \#
%   Dollar        \$     Percent       \%     Ampersand     \&
%   Acute accent  \'     Left paren    \(     Right paren   \)
%   Asterisk      \*     Plus          \+     Comma         \,
%   Minus         \-     Point         \.     Solidus       \/
%   Colon         \:     Semicolon     \;     Less than     \<
%   Equals        \=     Greater than  \>     Question mark \?
%   Commercial at \@     Left bracket  \[     Backslash     \\
%   Right bracket \]     Circumflex    \^     Underscore    \_
%   Grave accent  \`     Left brace    \{     Vertical bar  \|
%   Right brace   \}     Tilde         \~}
%
%
% \GetFileInfo{\jobname.dtx}
%
% \DoNotIndex{\newcommand,\newenvironment,\def}
%
% \providecommand\TikZ{\mbox{Ti\emph{k}Z}}%
% \providecommand\Moodle{\mbox{\brand{Moodle}}}%
% \providecommand\LMS{\mbox{\acro{LMS}}}%
% \providecommand\XML{\mbox{\acro{XML}}}%
%
% \title{%
%   The \pkg{moodle} package: \\
%   generating \Moodle\ quizzes via \LaTeX%
%   \thanks{This document corresponds to \pkg{moodle.sty}~\fileversion, dated \filedate.}
% }%
%
% \author{%
%   Anders Hendrickson\footnote{original author of the package (\texttt{v0.5}), inactive.}\\
%  \texttt{anders.o.f.hendrickson AT gmail DOT com} \and
%  Matthieu Guerquin-Kern\footnote{current maintainer, %
%  author of the updates (\texttt{v0.6} to \texttt{\fileversion}), %
%  partially funded in 2019 by \href{https://www.ensea.fr/en/}{ENSEA Graduate School}, France.}\\
%  \href{mailto:guerquin-kernATcransDOTorg}{\texttt{guerquin-kern AT crans DOT org}}%
% }%
%
% \date{\today}
%
% \maketitle
%
% \begin{abstract}
% This document describes the \pkg{moodle} package, made for writing \Moodle\
% quizzes in \LaTeX. In addition to typesetting the quizzes for proofreading or
% giving to students as handout, the package generates an \XML\ file to be uploaded
% to a \Moodle\ server.
% \end{abstract}
%
% \tableofcontents
%
% \section{Introduction}
% \subsection{Motivation}
%
% The acronym \acro{Moodle} stands for ``Modular Object-Oriented Dynamic Learning Environment.''
% It is an open source learning management system (\LMS) employed by many universities,
% colleges, and high schools to provide digital access to course materials, such as
% notes, video lectures, forums, and the like; see
% \url{https://moodle.com/moodle-lms/} for more information.
% One of the many useful
% features of \Moodle\ is that mathematical and scientific notation can be entered in
% \LaTeX\ or \TeX\ code, which will be typeset either through a built-in \TeX\ filter
% or by invoking MathJax.
%
% For instructors who want to give students frequent feedback,
% but lack the time to do so, a particularly valuable module in \Moodle\
% is the \emph{quiz}.  A \Moodle\ quiz can consist of several different types of
% questions---not only multiple choice or true/false questions, but also
% questions requiring a short phrase or numerical answer, and even essay
% questions.  All but the essay questions are automatically graded by the
% system, and the instructor has full control over how often the quiz may be
% attempted, its duration, and so forth.  Feedback can be tailored to specific
% mistakes the student makes.
%
% All these features make \Moodle\ quizzes very useful tools for instructors
% who have access to them.
% Unfortunately, the primary way to create or edit a \Moodle\ quiz
% is through a web-based interface that can be slow to operate.
% To users of \LaTeX, accustomed to the speed of typing source code on
% a keyboard alone, the agonizing slowness of switching between mouse and keyboard
% to navigate a web form with its myriad dropdown boxes, radio buttons,
% compounded with a perceptible time lag as one's \Moodle\ server responds to requests,
% can produce a very frustrating experience.  Moreover, editing is entirely
% impossible without network access.
%
% Once the quiz is written, there is no easy way to view and proofread all the
% information of which it is made.  Each question is edited on a separate webpage,
% which is so full of options that it cannot be viewed on a single screen.
% An instructor has to spend much time checking over the newly created quiz in
% order to be confident there are no errors.
%
% Added to all this is the frustration of managing graphics.
% If a question requires an image---say, asking a calculus student to interpret
% the graph of a function---the image must first be produced as a standalone file
% (e.g., in \JPG\ or \PNG\ format), uploaded to \Moodle, and then chosen in a web-based
% \HTML\ editor.  Great is the vexation of the instructor who decided to alter a
% question, as there are more and more possibilities of error whenever multiple
% files must be kept synchronized.
%
% Users of \LaTeX\ are also accustomed to the speed and flexibility that
% comes from defining their own macros, which may be as brief as writing
% \cmd{\R} instead of |\mathbb{R}|
% or as complex as macros that generate entire paragraphs of text.
% The \Moodle\ editor, by contrast, requires you to type |\mathbb{R}|
% every single time you want $\mathbb{R}$.
%
% Finally, there is the question of archiving and reusing one's work.
% Much, much work goes into creating \Moodle\ quizzes, which then reside
% on a \Moodle\ server somewhere in the cloud in a format neither easily browsable
% nor easily modifiable.
%
% \LaTeX\ itself has the power to solve all these difficulties:
% it is swift to edit and swifter to compile a \LaTeX\ document,
% and the \PDF\ may be previewed onscreen or printed out for ease of proofreading.
% Mathematical graphics can be integrated within the main file through \TikZ,
% and of course \LaTeX\ macros can be customized.
% Using the present \pkg{moodle} package,
% a quiz author can type a quiz using familiar \LaTeX\ syntax and document
% structure.  Upon compilation, \LaTeX\ will generate both a well-organized
% \PDF\ that is easy to proofread and an \XML\ file that can be uploaded directly
% to \Moodle.  The entire process is far faster than using \Moodle's own
% web-based editor, makes it easier to catch one's mistakes,
% and the ultimate source code of one's work is a human-readable \filenm{.tex} file
% that can be archived, versioned, browsed, and edited offline.
%
% Strictly speaking, the \pkg{moodle} package does not generate quizzes:
% it generates question banks that can be imported in the \LMS. The teacher
% still needs to compose manually a quiz from the question banks. Hopefully,
% two \Moodle\ features supported by the package make this task easier:
% categories and tags.
%
% In this documentation the \LMS\ is referred to as \Moodle\
% (uppercase letters and roman font) while the \LaTeX\ package
% is referred to as \pkg{moodle} (lower case and sans serif font).
%
% \subsection{Workflow}
%
%% \begin{figure}[bp]
% \centering
% \begin{tikzpicture}[node distance=1,auto,bend
% angle=45,box/.style={rectangle,draw=blue!50,rounded corners=3,top
% 	color=white,bottom color=black!20,thick,align=center,text
% 	width=2.5cm},elmt/.style={font=\itshape,align=left},
% cmnt/.style={font=\footnotesize,align=center},
% bigcmnt/.style={font=\normalsize,align=center},pre/.style={<-,>=stealth',thick},
% post/.style={->,>=stealth',thick}, prepost/.style={<->,>=stealth',thick}]
% \draw[white,fill=orange!20,rounded corners=10]
% (-1.5,-.7)--++(0,1.4)--++(10.3,0)--++(0,-2.3)--++(-6.7,0)--++(-.25,.725)--cycle;
% \node[orange!80!black,above] (dev) at (7.5,-1.6) {\textit{Development}};
% \fill[green,fill opacity=.2,rounded corners=10]
% (-1.5,.7)--++(3,0)--++(0.5,-2.4)--++(6.8,0)--++(0,-3)--++(-10.3,0)--cycle;
% \node[green!80!black,below] (dev) at (7.5,-1.7) {\textit{Publishing}};
% \node[box,text width=2cm] (tex) {\filenm{.tex} source file};
% \node[rectangle,fill=white,draw,align=center,text
% width=1.5cm,below=of tex] (compfinal) {\LaTeX{}\\ engine};
% \node[rectangle,fill=white,draw,align=center,text width=1.5cm,right=of tex,xshift=4.5cm]
% (compdraft) {\LaTeX{}\\ engine};
% \draw ($(tex)!.5!(compdraft)+(0,-.7)$) node[box] (pdf) {\filenm{.pdf} file for proofreading};
% \node[box,below=of pdf,yshift=.1cm] (pdfhandout) {\filenm{.pdf} file for students};
% \node[rectangle,fill=white,draw,align=center,text width=2cm,below=of compfinal] (extern)
% {Picture\\processing};
% \draw (pdf|-extern) node[box,anchor=center] (xml) {\filenm{.xml} file\\\footnotesize(pictures embedded)};
% \draw (compdraft|-pdfhandout)
% node[rectangle,fill=white,draw,align=center,text width=1.5cm,anchor=center] (students)
% {Students};
% \draw (compdraft|-xml)
% node[rectangle,fill=white,draw,align=center,text width=1.5cm,anchor=center] (moodle)
% {\Moodle\ \\Server};
% \draw (tex) edge [post,bend right=10] node[cmnt,pos=.5,left] {\optn{final}}
% node[cmnt,pos=.5,right] {\optn{handout}} (compfinal);
% \draw (tex) edge [post,bend left=10] node[cmnt,pos=.5,below] {\optn{draft}} (compdraft);
% \draw (compdraft) edge [post,bend left=10] (pdf);
% \draw (compfinal) edge [post] node[cmnt,pos=.5,above,sloped]
% {\footnotesize(\optn{handout})} (pdfhandout);
% \draw[dashed] (compfinal) edge [post] (pdf);
% \draw (compfinal) edge [post,bend left=0] node[cmnt,pos=.55,below,sloped]
% {\footnotesize(\optn{final})} (xml);
% \draw (compfinal) edge [post,bend right=15] node[cmnt,black!40,pos=.5,text width=1cm,left]
% {\optn{tikz}, \filenm{.png}, \filenm{.jpg}} (extern);
% \draw (extern) edge [post,green,bend right=15] node[cmnt,black!40,pos=.5,below,right]
% {\prog{base64}} (compfinal);
% \draw (pdf) edge [post,red,bend left=15] node[sloped,cmnt,pos=.5,above] {improve} (tex);
% \draw (pdfhandout) edge [post,red] node[cmnt,pos=.5,below] {distribute} (students);
% \draw (xml) edge [post,red] node[cmnt,pos=.5,below] {import} (moodle);
% \end{tikzpicture}
% \caption{Block diagram describing a typical workflow using the \pkg{moodle} package.}
% \label{fig:workflow}
% \end{figure}
%
% The process of creating a quiz in \Moodle\ using this package is depicted in
% Figure~\vref{fig:workflow}. It follows a few
% steps:
% \begin{enumerate}
%   \item Write a \LaTeX\ document using |\usepackage{moodle}| as described
% below.
%   \item Compile the document to \PDF\ using pdf\LaTeX\ (\acro{ASCII} characters only),
%         \XeLaTeX, or \LuaLaTeX.
% This will also produce the file \meta{jobname}\filenm{-moodle.xml}.
%   \item navigate to your course on \Moodle\ and, under ``Question bank'', select ``Import.''
%   \item Select ``Moodle \XML\ format,'' choose the \XML\ file to upload, and press ``Import.''
%   \item After \Moodle\ verifies that the questions have been imported correctly,
%         you may add them to your quizzes.
% \end{enumerate}
%
% \section{Usage}\label{sect:usage}
% The following pages presume the reader already has some familiarity with creating
% and editing \Moodle\ quizzes through the web interface.
% Users that are not familiar with \Moodle\ quizzes can learn more in the \Moodle\
% documentation. For instance, \url{https://docs.moodle.org/en/Question_types}.
%
% The \pkg{xkeyval} package is used to provide a key-value interface.
%
% \subsection{Example Document}
%
% Here is a simple example document:
% \begin{VerbatimOut}[gobble=1]{minted.doc.out}
%   \documentclass[12pt]{article}
%   \usepackage[section]{moodle}
%   \moodleregisternewcommands
%   \newcommand\monomial[1]{x^{#1}}
%   \newcommand\sillyanswer{What!?}
%   \begin{document}
%   Quiz generated \LaTeX's \textsf{moodle} (\moodleversion, \moodledate).
%   Import the derived file \texttt{\jobname-moodle.xml} on Moodle.
%   \begin{quiz}{My first quiz}
%     \begin{numerical}[points=2]{Basic addition}
%       What is $8+3$?
%       \item 11
%     \end{numerical}
%     \begin{shortanswer}[usecase]{Newton's name}
%       What was Newton's first name?
%       \item Isaac
%       \item[fraction=0, feedback={\sillyanswer}] Fig
%       \item[fraction=0] Sir
%     \end{shortanswer}
%     \begin{multi}[points=3]{A first derivative}
%       What is the first derivative of $\monomial{3}$?
%       \item $\frac{1}{4}\monomial{4}+C$
%       \item[feedback={yes!}]* $3\monomial{2}$
%       \item[feedback={\sillyanswer}]  $51$
%     \end{multi}
%   \end{quiz}
%   \end{document}
% \end{VerbatimOut}
% \inputminted[gobble=2,frame=lines]{latex}{minted.doc.out}
% Key features to note in this first example are that a \env{quiz} environment
% contains several question environments.
% Each question takes a name as a mandatory argument,
% and it may also take optional key-value arguments within brackets.
% The question environments resemble list environments
% such as \env{itemize} or \env{enumerate}, in that answers are set off by
% \cmd{\item}'s, but the question itself is the text that occurs before
% the first \cmd{\item}.
%
%^^A \DescribeMacro[moodle]{\moodleregisternewcommands}
%^^A \DescribeMacro[moodle]{\htmlregister}
%^^A Calling \cmb{\moodleregisternewcommands} tells the package to treat
%^^A specifically the macros defined subsequently.
%^^A This way, the macros that \Moodle's \LaTeX\ renderer does not know
%^^A about can be properly expanded in the \XML\ file.
%^^A This mechanism applies only to the macros defined using \cmd{\newcommand}
%^^A and \emph{without} optional argument.
%^^A Using \cmd{\htmlregister}\marg{macroname}, lets you declare a specific
%^^A macro for expansion. This mechanism, instead, also applies to the
%^^A macros defined using \TeX's primitive \cmd{\def}.
%
% \subsection{Package Options}
%
% \DescribeOption[moodle]{draft} If the package option \optn{draft} is invoked,
% by calling |\usepackage[draft]{moodle}| or |\documentclass[draft]{...}|,
% then no \XML\ file will be generated. This is especially useful while
% editing a quiz containing graphics, so as to avoid the time spent
% converting image files.
% \DescribeOption[moodle]{final} The package option \optn{final} might be
% useful if one wants to avoid the option \optn{draft} to be inherited from
% the \cmd{\documentclass}.
%
% \DescribeOption[moodle]{handout}
% If the package option \optn{handout} is invoked (|\usepackage[handout]{moodle}|),
% the \PDF\ file is generated clean from teacher-only information (answers,
% points, penalty, feedback, tags) and, hence, can be given  to students
% for classroom work. In particular, as would \Moodle\ do, answers in
% \env{matching} questions are shuffled and the option \optn{shuffle}
% triggers the shuffling of choices offered (\env{multi} and
% \env{matching}). This is achieved thanks to the package
% \pkg{randomlist}, loaded if the option is invoked. The seed of its
% random generator is controlled by the macro \cmd{\RLsetrandomseed}\marg{integer}
% \DescribeMacro[randomlist]{\RLsetrandomseed}.
% This option does not interfere with the generation of the \XML\ file.
%
% \DescribeOption[moodle]{samepage}^^A\watchout[experimental]
% If the package option \optn{samepage} is invoked, preferably used together
% with \optn{handout} (|\usepackage[handout,samepage]{moodle}|), the package
% will try to keep every question on the same page. Very bad spacing
% can result from this.
% This option is experimental. Subquestions inside a \env{cloze} question
% are protected but the \env{cloze} question itself is not protected.
%
% \DescribeOption[moodle]{nostamp}
% By default, the package will output a stamp as a comment in the XML
% file. This stamp contains information gathered about the \TeX\ engine,
% the platform used and the package version. For instance:%
% \begin{quote}\small
% \makeatletter\def\today{\the\year-\two@digits\month-\two@digits\day}\makeatother
% \newcount\hour\hour=\time
% \divide\hour by 60\relax
% \newcount\minute\minute=\hour
% \multiply\minute by -60\relax
% \advance\minute by \time\relax
% |<!-- This XML file is a question bank made for Moodle. -->|\\
% |<!-- It was generated on |\texttt{\today\space\the\hour:\the\minute}| by LuaLaTeX running -->|\\
% |<!-- on Linux with the package moodle |\texttt{\fileversion}| (|\texttt{\filedate}|) -->|
% \end{quote}
% The package option |nostamp| prevents this
% stamp from being written in the \XML\ file.
%
% \DescribeOption[moodle]{section}\DescribeOption[moodle]{subsection}
% \DescribeOption[moodle]{section*}\DescribeOption[moodle]{subsection*}
% The package options \optn{section} and \optn{subsection} place each quiz as a
% new section or subsection, respectively. Starred variants
% correspond to unnumbered sections or subsections. To preserve
% compatibility with Version 0.5 of this package, the default is
% \optn{subsection*}. Consequently, |\usepackage[subsection*]{moodle}| is equivalent
% to |\usepackage{moodle}| .
%
% \DescribeOption[moodle]{tikz}
% The package option \optn{tikz} is described in Section~\vref{subsec:tikz}.
%
% \DescribeOption[moodle]{svg}^^A\watchout[experimental]
% The package option \optn{svg} is described in Section~\vref{subsec:svg}.
%
% \DescribeOption[moodle]{LMS}\texttt{=}\meta{\texttt{X.Y}}
% \DescribeDefault{warn only} lets you specify version numbers
% for the target \Moodle\ LMS instance (\texttt{X} and \texttt{Y} are
% major and minor version integers). When version numbers are provided,
% the use of recent quiz features is secured by a compatibility
% check and \pkg{moodle} raises relevant errors. The XML stamp
% (see \optn{nostamp} above) also mentions \Moodle's target version.
% By default, \pkg{moodle} will just issue warnings when recent quiz
% features are used.
%
% \subsection{Quiz and Question Environments}
%
% A \filenm{.tex} document to generate \Moodle\ quizzes contains one or more
% \env{quiz} environments.
%
% \DescribeEnv[moodle]{quiz}\oarg{common options}\marg{category name}
% defines a quiz, within which various question environments are nested.
% The mandatory argument, \meta{category name}, names a category for \Moodle's
% ``question bank'': after import, the questions defined in this environment
% will be gathered in this category.
% Using the optional argument, options can be set at the quiz level.
% Although there are no \env{quiz}-specific options, any \meta{common options} set
% with the quiz will be inherited by all questions contained within that environment.
%
% \DescribeMacro[moodle]{\moodleset}\marg{options}
% is to be used to set options outside question environments; the option
% settings are local to \TeX-groups.
%
% \DescribeMacro[moodle]{\setcategory}\marg{category name}
% is to be used to change the current category inside a quiz environment and
% in between questions. Note that the \env{quiz} environment defines a category
% by its own.
%
% \DescribeMacro[moodle]{\setsubcategory}\marg{subcategory name}
% does the same with subcategories.

% The categories and subcategories are reflected in the \PDF\ file as sections, subsections, or
% subsubsections, in accordance to the package setting \optn{section}, \optn{section*},
% \optn{subsection}, or \optn{subsection*}.
%
% The syntax for each question environment is
% \begin{quote}
%   |\begin|\marg{question type}\oarg{question options}\marg{question name} \\
%   \rule{2em}{0pt}\meta{question text} \\
%   \rule{2em}{0pt}|\item|\oarg{item options} \meta{item} \\
%   \rule{2em}{0pt}\quad$\vdots$ \\
%   \rule{2em}{0pt}|\item|\oarg{item options} \meta{item} \\
%   |\end|\marg{question type}
% \end{quote}
% The meaning of the \meta{item}s varies depending on the question type,
% but they usually are answers to the question.
% Details will be given below.
%
% The following key-value options may be set for all questions:
%
% \DescribeKey[quiz,question]{points}\DescribeDefault{1}
% \DescribeKey[quiz,question]{default grade}
% By default, each question is worth 1 point on the quiz.
% This setting may be changed with the \optn{points} key or its synonym, \optn{default grade}.
% For example, \optn{points=2} makes that question worth two points.
%
% \DescribeKey[quiz,question]{penalty}\DescribeDefault{0.1}
% The \optn{penalty} is the fraction of points that is taken off for each wrong attempt;
% it may be set to any value between 0 and 1.
%
% \DescribeKey[answer]{fraction}^^A\DescribeDefault{0}
% In most question types, it is possible to designate some answers as being
% worth partial credit---that is, some fraction of a completely correct answer.
% The \optn{fraction} key may be set to any of the values given in Table~\vref{tab:fraction},
% from \texttt{0} (entirely wrong) to \texttt{100} (entirely correct).
%
% \begin{table}[tbp]
% \centering
% \caption{\href{https://github.com/moodle/moodle/blob/MOODLE\_310\_STABLE/question/engine/bank.php\#L339}
%     {Admissible positive values} for the \optn{fraction} key outside \env{cloze} environments: $100\cdot(p/q)$.}
% \label{tab:fraction}
% \footnotesize
% \begin{tabular}{l*{10}{l}}
% \toprule
% Denominator $q$ & \multicolumn{10}{c}{Numerator $p$}\\
% \cmidrule{2-11}
% & 0 & 1 & 2 & 3 & 4 & 5 & 6 & 7 & 8 & 9\\
% \cmidrule(lr){1-1}\cmidrule(lr){2-2}\cmidrule(lr){3-3}\cmidrule(lr){4-4}
% \cmidrule(lr){5-5}\cmidrule(lr){6-6}\cmidrule(lr){7-7}\cmidrule(lr){8-8}
% \cmidrule(lr){9-9}\cmidrule(lr){10-10}\cmidrule(lr){11-11}
% 20 & 0&5&&&&&&&&\\\cmidrule(lr){1-1}
% 10 & &10&20&30&40&50&60&70&80&90\\\cmidrule(lr){1-1}
% 9 & &11.11111&&&&&&&&100\\\cmidrule(lr){1-1}
% 8 & &12.5&&&&&&&100&\\\cmidrule(lr){1-1}
% 7 & &14.28571&&&&&&100&&\\\cmidrule(lr){1-1}
% 6 & &16.66667&&&&83.33333&100&&&\\\cmidrule(lr){1-1}
% 5 & &20&&&80&100&&&&\\\cmidrule(lr){1-1}
% 4 & &25&&75&100&&&&&\\\cmidrule(lr){1-1}
% 3 & &33.33333&66.66667&100&&&&&&\\\cmidrule(lr){1-1}
% 2 & &50&100&&&&&&&\\\cmidrule(lr){1-1}
% 1 & 0&100&&&&&&&&\\\cmidrule(lr){1-1}
% \bottomrule
% \end{tabular}
% \end{table}
%
% In questions where several choices can be selected (see \env{multi} with option
% \optn{multiple}), positive fractions must add up to exactly 100. It is also possible to set
% negative fractions (from -100 to 0) for wrong choices, in order to prevent the
% selection of all choices from leading to a good grade.
% In this case, the value ranging from -100 to 0 must be the opposite of one of the
% values listed in Table~\vref{tab:fraction}.
%
% \DescribeKey[answer]{fractiontol}\DescribeDefault{0.01}
% The package tries to match the \optn{fraction} key to one of the admissible values.
% To this end, the tolerance is controlled by the \optn{fractiontol} key. The default
% value, \texttt{0.01}, may be changed. When no admissible fraction value is matched, the
% package raises an error.
%
% \DescribeKey{feedback}
% The \optn{feedback} key sets text that will appear to the student after completing the quiz.
% For example, one might set
% \begin{center}
%   |feedback={This question might show up in the final exam.}|
% \end{center}
% The desired feedback should be included in braces.
%
% \DescribeKey[question]{feedback}If the \optn{feedback} key is set for a question,
% then that feedback will appear to each student regardless of the student's answer.
%
% \DescribeKey[answer]{feedback}
% Answer-specific feedback (perhaps explaining a common mistake)
% may also be given by setting the \optn{feedback} key \emph{at the individual answer}.
%
% \DescribeKey{tags}
% The \optn{tags} key sets a list of keywords for the question that will be taken into account
% by \Moodle\ for filtering purposes or classification of questions inside the question bank.
% It is possible for instance to build a quiz with questions cherry-picked among the set of
% questions holding a particular tag.
% For example, one might set
% \begin{center}
%   |tags={easy}|
% \end{center}
% The desired tag should be specified in between braces. Multiple tags can be set as a
% comma-separated list:
% \begin{center}
%   |tags={tag1,tag2,{ leading whitespace},{including, comma}}|
% \end{center}
%
% \DescribeKey[quiz]{tags}
% If the \optn{tags} key is set at the quiz level,
% then that tag list will serve as a default for each question of the quiz.
%
% \DescribeKey[question]{tags}
% Question-specific tags can be assigned by setting the \optn{tags} key \emph{at the question level}.
% The question-level \optn{tags} key overrides eventual quiz-level tags.
%
% Users willing to specify a same tag for all questions of the quiz could
% also consider relying on \Moodle's category mechanism.
%
% \subsection{\href{https://docs.moodle.org/en/Question_types}{Question Types}}
%
% We next discuss the various question types supported by \pkg{moodle}
% and the options that may be set.
%
%\subsubsection{\href{https://docs.moodle.org/en/True/False_question_type}{True/False}}
%
% \DescribeEnv[quiz]{truefalse}\oarg{question options}\marg{question name}
% is an environment that defines a \emph{True/False} question.
%
% The syntax for a True/False question is as follows:
% \begin{quote}
%   |\begin{truefalse}|\oarg{question options}\marg{question name} \\
%   \rule{2em}{0pt}\meta{question text} \\
%   \rule{2em}{0pt}|\item*| \meta{feedback when ``true" is chosen} \\
%   \rule{2em}{0pt}|\item| \meta{feedback when ``false" is chosen} \\
%   |\end{truefalse}|
% \end{quote}
% The correct answer is designated by the asterisk \texttt{*} after the \cmd{\item};
% it need not appear first in the list.
%
% Answer-specific feedback can also be defined as an item option, similarly to
% other types.
% \begin{quote}
%   |\begin{truefalse}|\oarg{question options}\marg{question name} \\
%   \rule{2em}{0pt}\meta{question text} \\
%   \rule{2em}{0pt}|\item[feedback={|\meta{When ``true" is chosen}|}]*| \\
%   |\end{truefalse}|
% \end{quote}
% Note that, in this example, no feedback is defined for the incorrect answer
% ``False": the corresponding item can be omitted.
%
% With the True/False question type, the |penalty| key has no effect.
%
% \subsubsection{\href{https://docs.moodle.org/en/Multiple_Choice_question_type}{Multiple Choice}}
% \label{subsec:multi}
%
% \DescribeEnv[quiz]{multi}\oarg{question options}\marg{question name}
% is an environment that defines a \emph{Multiple Choice} question.
%
% The syntax for a classic multiple choice question,
% with only one correct answer, is as follows:
% \begin{quote}
%   |\begin{multi}|\oarg{question options}\marg{question name} \\
%   \rule{2em}{0pt}\meta{question text} \\
%   \rule{2em}{0pt}|\item|\oarg{answer options}|*| \meta{correct answer} \\
%   \rule{2em}{0pt}|\item|\oarg{answer options} \meta{wrong answer} \\
%   \rule{2em}{0pt}\quad$\vdots$ \\
%   \rule{2em}{0pt}|\item|\oarg{answer options} \meta{wrong answer} \\
%   |\end{multi}|
% \end{quote}
% The correct answer is designated by the asterisk \texttt{*} after the \cmd{\item};
% it need not appear first in the list.
%
%
% \DescribeBoolean[multi]{shuffle}\DescribeDefault{true}
% The Boolean key \optn{shuffle} determines whether \Moodle\ will
% rearrange the possible answers in a random order.
% Setting \optn{shuffle=false} will guarantee that the answers will
% be presented to the student in the order they were typed.
%
% \DescribeKey[multi]{numbering}\DescribeDefault{abc}
% \Moodle\ offers different options for numbering the possible answers.
% You may set the \optn{numbering} key to any of the values listed in
% Table~\vref{tab:numbering_options}.
% \begin{table}[tbp]
% \centering
% \caption{Numbering modalities offered for \env{multi} questions.}
% \label{tab:numbering_options}
% \begin{tabular}{rllllll}
% \toprule
% Sample & a., b., \dots & A., B., \dots & 1., 2., \dots & i., ii., \dots &
% I., II., \dots & $\bullet$ \\
% \cmidrule(lr){2-2}\cmidrule(lr){3-3}\cmidrule(lr){4-4}\cmidrule(lr){5-5}\cmidrule(lr){6-6}
% \cmidrule(lr){7-7}
% \LaTeX\ syntax & \optn{alph} & \optn{Alph} & \optn{arabic} & \optn{roman} & \optn{Roman} & \optn{none}\\
% \Moodle\ syntax & \optn{abc} & \optn{ABCD} & \optn{123} & \optn{iii} & \optn{IIII} & \optn{none} \\
% \bottomrule
% \end{tabular}
% \end{table}
% \optn{numbering=none} produces an unnumbered list of answers.
% Note the \emph{four} capital letters required by \Moodle's syntax
% to obtain upper-case Roman or alphabetic numerals.
%
% \DescribeKey[item]{fraction}\DescribeDefault{0}
% \DescribeKey[item*]{fraction}\DescribeDefault{100}
% The \optn{fraction} key may be automatically set depending on the
% presence of an asterisk \texttt{*} right after the \cmd{\item}. By default,
% Starred items designate correct answers (\optn{fraction=100}) while
% bare items designate incorrect answers (\optn{fraction=0}).
% The key can be used to designate some wrong answers as being worth
% partial credit or sanction.  For example, a question might read
% \begin{VerbatimOut}[gobble=1]{minted.doc.out}
%   \begin{multi}{my question}
%     Compute $\int 4x^3\,dx$.
%     \item* $x^4+C$
%     \item[fraction=50] $x^4$
%     \item $12x^2$
%   \end{multi}
% \end{VerbatimOut}
% \inputminted[gobble=2,frame=lines]{latex}{minted.doc.out}
%
% \DescribeBoolean[multi]{single}\DescribeDefault{true}
% By default, the \env{multi} environment produces a multiple choice
% question where only one answer can be selected. This is called
% \optn{single} mode.
%^^A On \Moodle, the choices are displayed with radio
%^^A buttons and only one of them can be selected by the student.
%
% In \optn{single} mode, the teacher can set negative fractions for
% incorrect choices. This way, the expected grade of a student
% picking up choices randomly across a quiz is lowered.
% By default, incorrect answers that do not have a fraction
% specified correspond to \optn{fraction=0}.
% \DescribeKey[quiz,multi]{sanction}\DescribeDefault{0} However,
% if the \optn{sanction} key is set to a positive value, at quiz or
% question levels, such incorrect answers are sanctioned as if
% \optn{fraction=-}\meta{sanction value} was set for each of them.
%
% \DescribeKey[multi]{multiple}
% It is also possible to write questions with possibly more than
% one correct answer, asking the user to check all correct answers.
% To do this, use the key \optn{multiple} or \optn{single=false}.
% For this kind of question, the student gets a grade that
% corresponds to the total of the weights of the answers selected.
% Users may rely on two modalities to set the weights of the answers:
% \begin{enumerate}
% \item the \emph{advanced mode} applies whenever a \optn{fraction} key
% is set among the proposed answers. In this case, it is recommended
% that the user sets manually everything. Answers that are left with
% no fraction set are considered as neutral, selecting them will
% not change the grade. If \cmd{\item*} is used to designate some correct
% answers, \pkg{moodle} will distribute equally among those answers
% the points that are left for a total of 100\% (only positive
% fractions count here). Note that doing so will easily yield
% situations where the weights are inadmissible (see
% Table~\vref{tab:fraction}).
% \item the \emph{automatic mode} applies when correct answers are
% designated using \cmd{\item*} and no answer has a \optn{fraction} key set.
% In this case, each correct answer is weighted to bring the same
% fraction of the maximum grade and each incorrect answer is
% weighted to cancel the benefits of the selection of one correct
% answer.
% \end{enumerate}
% For example, the following two examples are equivalent:
% \begin{VerbatimOut}[gobble=1]{minted.doc.out}
%   \begin{multi}[multiple]{Automatic Mode}
%     Which numbers are prime?
%     \item 2
%     \item* 5
%     \item* 7
%     \item 1
%     \item 6
%   \end{multi}
%   \begin{multi}[multiple]{Advanced Mode}
%     Which numbers are prime?
%     \item[fraction=-50] 2
%     \item[fraction=50] 5
%     \item[fraction=50] 7
%     \item[fraction=-50] 1
%     \item[fraction=-50] 6
%   \end{multi}
% \end{VerbatimOut}
% \inputminted[gobble=2,frame=lines]{latex}{minted.doc.out}
% Note that, in this example, negative fractions are set for wrong
% choices. This prevents students from obtaining a good grade with
% no merit if they select all answers.
%
% Contrarily to the \optn{single} mode where questions with negative fractions may
% lead to a overall negative grade, when multiple choices can be selected, the
% lowest grade \Moodle\ will take into account for the question is 0.
%
% \DescribeKey[multi]{allornothing}
% There also exists a \href{https://moodle.org/plugins/qtype_multichoiceset}
% {``All-or-Nothing Multiple Choice''} plugin for \Moodle\ that introduces a
% question type similar to a multiple choice with multiple correct answers,
% with the specificity that the points are given if and only if the student
% selects all correct answers. This kind of question is set up using the
% \optn{allornothing} key.
% The recommended way for designating correct answers is with \cmd{\item*}. If instead
% the \optn{fraction} is used, \pkg{moodle} will consider that non-negative
% fractions ($>0$) designate correct answers and negative fractions ($\leq 0$)
% designate incorrect choices.
% The option \optn{allornothing} supersedes the options \optn{multiple} and \optn{single}.
% To the best of our knowledge, \Moodle\ does not offer the all-or-nothing behavior
% for multiple choice questions embedded inside a \env{cloze} question.
%
% \subsubsection{\href{https://docs.moodle.org/en/Numerical_question_type}{Numerical}}
%
% \DescribeEnv[quiz]{numerical}\oarg{question options}\marg{question name}
% is an environment that defines a \emph{Numerical} question which, in \Moodle, requires the student
% to input a real number in decimal form.
%
% The typical format for this question type is:
% \begin{quote}
%   |\begin{numerical}|\oarg{question options}\marg{question name} \\
%   \rule{2em}{0pt}\meta{question text} \\
%   \rule{2em}{0pt}|\item|\oarg{answer options} \meta{correct answer} \\
%   |\end{numerical}|
% \end{quote}
% If there is more than one correct answer, additional \cmd{\item}'s may be included.
% Because this is not a multiple choice question, there is no need to provide
% incorrect answers. There may nevertheless be reasons to include incorrect answers.
% For example, partially correct answers may be specified by setting the \optn{fraction} key.
% Feedback for a common mistake may be given by including the incorrect answer like this:
% \begin{quote}\footnotesize
%   |\item[fraction=0,feedback={Forgot to antidifferentiate?}]| \meta{incorrect answer}
% \end{quote}
%
% \DescribeKey[numerical]{tolerance}\DescribeDefault{0}
% The \optn{tolerance} key can be used to specify the validity of answers within some margin.
% This key can be set at different levels: quiz, question, item.
% For example, with the question
% \begin{VerbatimOut}[gobble=1]{minted.doc.out}
%   \begin{numerical}[tolerance=0.01]{my question}
%     Approximate value of $\sqrt{2}$?
%     \item[tolerance={1e-1}] 1.4142
%     \item[fraction=20,feedback={twice this!}] 7.0711e-1
%     \item[fraction=0,feedback={Wrong!}] *
%   \end{numerical}
% \end{VerbatimOut}
% \inputminted[gobble=2,frame=lines]{latex}{minted.doc.out}
% In this example,
% \begin{itemize}
%   \item any answer in the range $[1.4042,1.4242]$ will be validated,
%   \item any answer in the range $[0.69711,0.71711]$ will get the specific feedback
%         \emph{twice this!} and 20\% of points,
%   \item any other answer is incorrect and will get the specific feedback
%         \emph{Wrong!}.
% \end{itemize}
%
% When feedback is to be given for any non-specified answer, one can add a \emph{last} answer
% item containing the wildcard character \texttt{*} only. In this case, the \optn{tolerance} key is irrelevant.
%
% Both answers and tolerance can be specified with the comma (\texttt{,}) as a decimal separator.
% Exponent notation is accepted. After import, \Moodle\ will recognize indifferently \texttt{0.000165},
% \texttt{0,000165}, \texttt{1.65E-4}, \texttt{1.65e-4}, \texttt{1,65E-4}, and \texttt{1,65e-4}.
%
% If the \pkg{siunitx} package is loaded, \pkg{moodle} will detect it and
% numbers will be rendered nicely in the \PDF\ output.
%
% Units, unit-handling and multipliers are currently unsupported.\watchout
%
% \subsubsection{\href{https://docs.moodle.org/en/Short-Answer_question_type}{Short Answer}}
%
% \DescribeEnv[quiz]{shortanswer}\oarg{question options}\marg{question name}
% is an environment that defines a \emph{Short Answer} question. It resembles
% a \env{numerical} question: the student is to fill in a text box with a
% missing word or phrase.
% \begin{quote}
%   |\begin{shortanswer}|\oarg{question options}\marg{question name} \\
%   \rule{2em}{0pt}\meta{question text} \\
%   \rule{2em}{0pt}|\item|\oarg{answer options} \meta{correct answer} \\
%   \rule{2em}{0pt}\quad$\vdots$ \\
%   \rule{2em}{0pt}|\item|\oarg{answer options} \meta{correct answer} \\
%   |\end{shortanswer}|
% \end{quote}
% You can make the text box appear as part of the question with the
% control sequence \cmd{\blank}.  For example,
% your question might read
% \begin{VerbatimOut}[gobble=1]{minted.doc.out}
%   \begin{shortanswer}{Leibniz}
%     Newton's rival was Gottfried Wilhelm \blank.
%     \item Leibniz
%     \item Leibniz.
%   \end{shortanswer}
% \end{VerbatimOut}
% \inputminted[gobble=2,frame=lines]{latex}{minted.doc.out}
% Note that as the blank occurred at the end of a sentence,
% we included two answers,
% lest students get the question wrong merely by
% including or omitting a period.
%
% \DescribeBoolean[shortanswer]{case sensitive}\DescribeDefault{false}
% The default setting when creating a Short Answer question in \Moodle\
% is to ignore the distinction between upper- and lower-case letters
% when grading a short answer question. This default is preserved by
% \pkg{moodle}.
% You can make a question case-sensitive with the Boolean key
% \optn{case sensitive} or its shorter synonym \optn{usecase}\DescribeBoolean[shortanswer]{usecase}.
%
% The \href{https://docs.moodle.org/en/Short-Answer_question_type#Wildcard_usage}%
% {wildcard character} \texttt{*} can used to grab answers that match
% a specific pattern. Following the order of answers, the first match will lead
% to the corresponding score and eventual feedback. As an example, take the
% following question
% \begin{VerbatimOut}[gobble=1]{minted.doc.out}
%   \begin{shortanswer}[usecase]{Newton's name}
%     What was Newton's first name?
%     \item Isaac
%     \item[fraction=0,feedback={Simply Isaac!}] Isaa*
%     \item[fraction=0,feedback={This one is Leibniz!}] *Gottfried*
%     \item[fraction=0,feedback={First name, not title!}] Sir*
%     \item[fraction=0,feedback={No\dots}] *
%   \end{shortanswer}
% \end{VerbatimOut}
% \inputminted[gobble=2,frame=lines]{latex}{minted.doc.out}
% \begin{itemize}\def\answ#1{\framebox{\texttt{#1}}}%
%   \item the answer \answ{Isaac} is the only one that gets rewarded,
%   \item answers \answ{Isaac Leibniz}, \answ{Isaac Newton}, and \answ{Isaak} yield
%   the feedback \emph{Simply Isaac!},
%   \item answers \answ{Sir Gottfried Wilhelm}, \answ{Sir Gottfried},
%         \answ{Gottfried Wilhem}, and \answ{Gottfried} yield
%         the feedback \emph{This one is Leibniz!},
%   \item answers \answ{Sir Isaac}, and \answ{Sir} yield
%         the feedback \emph{First name, not title!},
%   \item any answer that does not match the previous patterns
%         yields the feedback \emph{No\dots}.
% \end{itemize}
%
% \subsubsection{\href{https://docs.moodle.org/en/Essay_question_type}{Essay}}\label{subsubsect:essay}
%
% \DescribeEnv[quiz]{essay}\oarg{question options}\marg{question name}
% is an environment that defines an \emph{Essay} question.
%
% Instructors may ask essay questions on a \Moodle\ quiz,
% although \Moodle's software is not up to the task of grading them!
% Instead, each essay question answer must be graded manually by the
% instructor or a teaching assistant.
% \begin{quote}
%   |\begin{essay}|\oarg{question options}\marg{question name} \\
%   \rule{2em}{0pt}\meta{question text} \\
%   \rule{2em}{0pt}|\item| \meta{information 1 for grader} \\
%   \rule{2em}{0pt}\quad$\vdots$ \\
%   \rule{2em}{0pt}|\item| \meta{information $n$ for grader} \\
%   |\end{essay}|
% \end{quote}
% Instead of containing answers, the \cmd{\item} entries included in the \env{essay}
% question contain notes that will appear to whoever is grading the question manually.
% Contrarily to other question types, \cmd{\item} in \env{essay} questions
% do not take options.
%
% \DescribeBoolean[essay]{response required}\DescribeDefault{false}
% Although \Moodle\ cannot grade the content of an essay question,
% it can at least determine whether the question has been left blank.
% If the \optn{response required} key is set, \Moodle\ will insist that the student
% enter something in the blank before accepting the quiz as completed.
%
% \DescribeKey[essay]{response format}\DescribeDefault{html}
% \Moodle\ offers five different ways for students to enter and/or upload their
% answers to an essay question. You may choose one of these five options:
% \begin{description}
%   \ItemDescribeOther[response format]{html} An editor with the ability to format \HTML\ responses
%         including markup for italics, boldface, etc.  This is the default.
%   \ItemDescribeOther[response format]{file} A file picker allowing the student to upload a
%         file, such as a \PDF\ or DOC file, containing the essay.
%   \ItemDescribeOther[response format]{html+file} The same \HTML\ editor as above, but with the
%         ability to upload files as well.  This permits some students to type
%         answers directly into the web form, and others to compose their
%         essays in another program first.
%   \ItemDescribeOther[response format]{text} This editor allows only for entering plain text
%         without any markup.
%   \ItemDescribeOther[response format]{monospaced} This yields a plain text editor, without any
%         markup, and with a fixed-width font.  This could be useful for
%         entering code snippets, for example.
% \end{description}
%
% \DescribeKey[essay]{response field lines}\DescribeDefault{15}
% The key \optn{response field lines} controls the height of the input box.
% For \Moodle, the admissible values are: 5, 10, 15, 20, 25, 30, 35, and 40.
% If the value set is not admissible, \pkg{moodle} will approximate the value:
% \begin{itemize}
%   \item with either 5 or 40 if the value set was out of range, or
%   \item with the next multiple of 5 otherwise.
% \end{itemize}
%
% \DescribeKey[essay]{attachments allowed}\DescribeDefault{0}
% The \optn{attachments allowed} key controls \emph{how many} attachments a student is
% allowed to upload. Admissible values are \texttt{0}, \texttt{1}, \texttt{2},
% \texttt{3}, or \texttt{unlimited}.
%
% \DescribeKey[essay]{attachments required}\DescribeDefault{0}
% You may also require the student to upload a certain number of attachments
% by setting \optn{attachments required} to \texttt{0}, \texttt{1}, \texttt{2}, or \texttt{3}.
%
% \DescribeKey[essay]{template}
% Finally, you may preload the essay question with a template that the student
% will edit and/or type over, with the key \optn{template=}\marg{template}.
% The \meta{template} should be enclosed in braces.
%
% \subsubsection{\href{https://docs.moodle.org/en/Matching_question_type}{Matching}}
%
% \DescribeEnv[quiz]{matching}\oarg{question options}\marg{question name}
% is an environment that defines a \emph{Matching} question. It typically
% looks like this:
% \begin{quote}
%   |\begin{matching}|\oarg{question options}\marg{question name} \\
%   \rule{2em}{0pt}\meta{question text} \\
%   \rule{2em}{0pt}|\item| \meta{item 1} |\answer| \meta{match 1}\\
%   \rule{2em}{0pt}|\item| \meta{item 2} |\answer| \meta{match 2}\\
%   \rule{2em}{0pt}\quad$\vdots$ \\
%   \rule{2em}{0pt}|\item| \meta{item $m$} |\answer| \meta{match $m$}\\
%   \rule{2em}{0pt}|\item| |\answer| \meta{no match $1$}\\
%   \rule{2em}{0pt}\quad$\vdots$ \\
%   \rule{2em}{0pt}|\item| |\answer| \meta{no match $n$}\\
%   |\end{matching}|
% \end{quote}
% \meta{match}es $1$ through $m$ are separated from their corresponding
% \meta{item}s by the command \cmd{\answer}\DescribeMacro[matching]{\answer}.
% Answers that match no item can be proposed at the end of the list,
% preceded by an empty item.
%
% After import, \Moodle\ will recognize matches that are \emph{exact}
% duplicates. If you intend multiple questions to have the same match,
% make sure that they are entered identically.
%
% \DescribeBoolean[matching]{shuffle}\DescribeDefault{true}
% When students take a matching question, \Moodle\ always displays
% the proposed matches in random order.
% The \env{matching} question accepts the option |shuffle| to also
% randomly permute the items.
%
% \DescribeBoolean[matching]{drag and drop}\DescribeDefault{false}
% The standard matching question offered by \Moodle\ corresponds to
% a dropdown box for choosing the match for each item.
% There also exists a \href{https://docs.moodle.org/en/Drag_and_drop_matching_question_type}
% {``drag and drop matching''} plugin for \Moodle\ that
% shows all items in a column (left), all proposed matches in a second column
% (right), and asks students to drag the correct
% match to each item with the mouse.
% In this package, to enable drag-and-drop matching, use the key
% \optn{drag and drop} or \optn{dd}\DescribeKey[matching]{dd}\ for short.
% Beware that, with the standard format (\optn{drag and drop=false}), due
% to the limitations of dropdown boxes, \Moodle\ will not render \LaTeX\
% or \HTML\ code passed in the answers.
%
% \subsubsection{\href{https://docs.moodle.org/en/Embedded_Answers_(Cloze)_question_type}
% {Cloze Questions and Subquestions}}
%
% \DescribeEnv[quiz]{cloze}\oarg{question options}\marg{question name}
% is an environment that defines a \emph{Cloze} question.
%
% A ``cloze question'' has one or more subquestions embedded within a passage of text.
% For example, you might ask students to fill in several missing words within
% a sentence, or calculate several coefficients of a polynomial.
% To encode cloze questions in \LaTeX\ using this package is easy:
% you simply nest one or more \env{multi}, \env{numerical}, or \env{shortanswer} environments
% within a \env{cloze} environment, as in the following example:
% \begin{VerbatimOut}[gobble=1]{minted.doc.out}
%   \begin{cloze}{my cloze question}
%     Thanks to calculus, invented by Isaac
%     \begin{shortanswer}[usecase]
%       \item Newton
%     \end{shortanswer},
%     we know that the derivative of $x^2$ is
%     \begin{multi}[horizontal]
%       \item $2x$
%       \item* $\frac{1}{3} x^3 + C$
%       \item $0$
%     \end{multi}
%     and that $\int_0^2 x^2\,dx$ equals
%     \begin{numerical}
%       \item[tolerance={4e-4}] 2.667
%     \end{numerical}.
%     Thanks, Isaac!
%   \end{cloze}
% \end{VerbatimOut}
% \inputminted[gobble=2,frame=lines]{latex}{minted.doc.out}
%
% Note that, when used as a subquestion within a \env{cloze} question,
% the question environments are \emph{not} followed by a question
% name in braces.
% \begin{description}
% \ItemDescribeEnv[cloze]{multi}\oarg{subquestion options} defines a
%             Multiple Choice question inside a Cloze question,
% \ItemDescribeEnv[cloze]{numerical}\oarg{subquestion options} defines a
%             Numerical question inside a Cloze question, and
% \ItemDescribeEnv[cloze]{shortanswer}\oarg{subquestion options} defines a
%             Shortanswer question inside a Cloze question,
% \end{description}
%
% \DescribeKey[cloze]{points}\DescribeDefault{1}
% \DescribeKey[cloze]{default grade}
% Inside \env{cloze} environments, the \optn{points} or \optn{default grade}
% keys can be used to weight the worth of each subquestion. A specific
% constraint applies: the values should be positive integers.
%
% \DescribeKey[cloze]{fraction}
% Inside \env{cloze} environments, the \optn{fraction} key can be used to give
% partial credit or sanction for certain answers. The values
% specified must be integers and are independent of the admissible
% values listed in Table~\vref{tab:fraction}.
%
% \DescribeBoolean[cloze]{single}\DescribeDefault{true}
% \DescribeKey[cloze]{multiple}
% Prior to \Moodle\ version 3.5, within a \env{cloze} question, a multiple choice question
% was necessarily of type \optn{single}, i.e. with a single good answer. If you intend
% to export your quiz to \Moodle\ 3.5+, the option \optn{multiple} can be used for
% questions where students must be able to select several answers.
% The modalities described in Section~\vref{subsec:multi} for setting the
% weight of answers apply. The only difference occurs in \emph{advanced mode}:
% the sum of positive fractions may not be 100\%. In this case, after
% importing the \XML\ file, \Moodle\ will automatically scale the positive
% fractions for a total of 100\% and leave intact the negative fractions.
% ^^A\footnote{\url{https://tracker.moodle.org/browse/MDL-3782?focusedCommentId=421564\#comment-421564}}.
%
% \DescribeKey[cloze]{vertical}\DescribeKey[cloze]{horizontal}\DescribeKey[cloze]{inline}
% \DescribeDefault{inline}
% Within a \env{cloze} question, by default, a multiple choice question is implemented
% as an \optn{inline} dropdown box. This is visually compact, but it also prevents
% the use of mathematical or \HTML\ formatting.
% Adding the option \optn{vertical} displays the subquestion as a vertical column
% of radio buttons instead; likewise the option \optn{horizontal} creates a horizontal
% row of radio buttons.
% The option \optn{inline} is incompatible\watchout\ with \optn{multiple} or \optn{single=false}
% (dropdown boxes don't let you pick up several answers!).
%
% \DescribeBoolean[cloze]{shuffle}
% Starting from \Moodle\ version 3.0, within a \env{cloze} question, the items of a
% multiple choice question can be shuffled. Setting \optn{shuffle=false} will
% guarantee that the answer appear in the order they were typed; the
% default is \optn{shuffle=true}.
%
% \DescribeBoolean[cloze]{case sensitive}\DescribeDefault{false}
% \DescribeBoolean[cloze]{usecase}
% Within a \env{cloze} question, the \env{shortanswer} question can be made case sensitive.
% This option, disabled by default, is selected with \optn{case sensitive} or \optn{usecase}.
%
%\subsubsection{\href{https://docs.moodle.org/en/Description_question_type}{Description}}
%
% \DescribeEnv[quiz]{description}\oarg{question options}\marg{question name}
% is an environment that defines a so-called \emph{Description} question.
%
% The \Moodle\ description type is not really a question. It is more like a label.
% One can set a \optn{feedback} that the student gets when reviewing the submission.
% Tags can be set as well.
%
% For descriptions, \pkg{moodle} redefines \LaTeX's \env{description} environment.
% The scope of this redefinition is limited to the \env{quiz} environment.
%
% The syntax for a Description question is as follows:
% \begin{quote}
%   |\begin{description}|\oarg{question options}\marg{question name} \\
%   \rule{2em}{0pt}\meta{question text} \\
%   |\end{description}|
% \end{quote}
%
% \subsubsection{A Word About \href{https://docs.moodle.org/en/Calculated_question_type}{Calculated Questions}}
% \Moodle's calculated questions are not supported by this package.
%
% However, as \href{https://github.com/avohns/python-latex-moodle-quiz}{demonstrated by A.
% Vohns}, an advanced scripting language may be used to generate a series of questions sharing
% the same prototype.
%
% We suggest to apply a specific tag to these questions. After import in \Moodle, when creating
% a quiz, this tag can be selected to narrow down a random selection of questions.
% This would mimic the behavior of calculated questions while bringing the flexibility of your
% favorite scripting language.
%
% Here are two examples inspired from the work of A. Vohns. The first one relies on the
% native \prog{Lua} capabilities of \LuaLaTeX.
% \begin{VerbatimOut}[gobble=1]{minted.doc.out}
%   \begin{quiz}[tags={calculated}]{Example Quiz}
%   \directlua{
% \end{VerbatimOut}
% \inputminted[gobble=2,frame=topline]{latex}{minted.doc.out}
% \vspace{-.7cm}
% \begin{VerbatimOut}[gobble=1]{minted.doc.out}
%   function clozenum_print(pair,op,result)
%     tex.print("\\begin{numerical}$"..pair[1].." "..op.." "..pair[2].."
%     =$".."\\item ",result,"\\end{numerical}")
%   end
%   function cloze_print(pair,points)
%     tex.print("\\begin{cloze}[points="..points.."]{Arithmetic Quiz
%     ("..pair[1]..", "..pair[2]..")}Solve the following tasks!\\\\")
%     clozenum_print(pair,"+",pair[1]+pair[2])
%     clozenum_print(pair,"-",pair[1]-pair[2])
%     clozenum_print(pair,"*",pair[1]*pair[2])
%     if pair[1]/pair[2]==math.floor(pair[1]/pair[2]) then
%       clozenum_print(pair,":",math.floor(pair[1]/pair[2]))
%     end
%     tex.print("\\end{cloze}")
%   end
%   for x = 2,4 do
%     for y = 2,4 do
%       if x>y then
%         if x/y==math.floor(x/y) then points=4 else points=3 end
%         cloze_print({x,y},points)
%       end
%     end
%   end
% \end{VerbatimOut}
% \inputminted[]{lua}{minted.doc.out}
% \vspace{-.7cm}
% \begin{VerbatimOut}[gobble=1]{minted.doc.out}
%   }
%   \end{quiz}
% \end{VerbatimOut}
% \inputminted[gobble=2,frame=bottomline]{latex}{minted.doc.out}
% The second example makes use of the \pkg{python} package (|\usepackage{python}|).
% \begin{VerbatimOut}[gobble=1]{minted.doc.out}
%   \begin{quiz}[tags={calculated}]{Example Quiz}
%   \begin{python}
% \end{VerbatimOut}
% \inputminted[gobble=2,frame=topline]{latex}{minted.doc.out}
% \vspace{-.7cm}
% \begin{VerbatimOut}[gobble=1]{minted.doc.out}
%   def clozenum_print(pair,op,result):
%     print(rf"""\begin{{numerical}}
%   ${pair[0]} {op} {pair[1]} =$\item {result}
%   \end{{numerical}}""")
%   def cloze_print(pair,points):
%     print(rf"""\begin{{cloze}}[points={points}]{{Arithmetic Quiz
%     {(pair[0],pair[1])}}}Solve the following tasks!\\""")
%     clozenum_print([x,y],"+",x+y)
%     clozenum_print([x,y],"-",x-y)
%     clozenum_print([x,y],"*",x*y)
%     if pair[0]/pair[1] == pair[0]//pair[1]:
%       clozenum_print([x,y],":",x//y)
%     print("\end{cloze}")
%   for x in range(2,5):
%     for y in range(2,5):
%       if x > y:
%         if x/y == x//y:
%           points=4
%         else:
%           points=3
%         cloze_print([x,y],points)
% \end{VerbatimOut}
% \inputminted[]{python}{minted.doc.out}
% \vspace{-.7cm}
% \begin{VerbatimOut}[gobble=1]{minted.doc.out}
%   \end{python}
%   \end{quiz}
% \end{VerbatimOut}
% \inputminted[gobble=2,frame=bottomline]{latex}{minted.doc.out}
% These two codes yield the same \XML\ content.
%
% \subsection{Summary of the Key Options}
%
% Table~\vref{tab:key-options}, summarizes
% the key options available at the question and answer levels depending on the
% question type. For the essay questions, please refer to Section~\vref{subsubsect:essay}.
%
% \begin{table}[tbp]
% \centering
% \caption{Options offered at the question and answer levels for each question type.}
% \label{tab:key-options}
% \small
% \begin{tabular}{*{15}{l}}
% \toprule
% & \multicolumn{11}{l}{Question} & \multicolumn{3}{l}{Answer}\\
% \cmidrule(lr){2-12}\cmidrule(lr){13-15}
% Question type & \rotatebox{90}{points} &
% \rotatebox{90}{penalty} & \rotatebox{90}{feedback} & \rotatebox{90}{tags} &
% \rotatebox{90}{shuffle} & \rotatebox{90}{numbering} & \rotatebox{90}{multiple} &
% \rotatebox{90}{allornothing} &\rotatebox{90}{usecase} & \rotatebox{90}{tolerance} &
% \rotatebox{90}{dd} & \rotatebox{90}{fraction} & \rotatebox{90}{feedback} &
% \rotatebox{90}{tolerance}\\\cmidrule(lr){1-1}\cmidrule(lr){2-2}\cmidrule(lr){3-3}
% \cmidrule(lr){4-4}\cmidrule(lr){5-5}\cmidrule(lr){6-6}\cmidrule(lr){7-7}
% \cmidrule(lr){8-8}\cmidrule(lr){9-9}\cmidrule(lr){10-10}\cmidrule(lr){11-11}
% \cmidrule(lr){12-12}\cmidrule(lr){13-13}\cmidrule(lr){14-14}\cmidrule(lr){15-15}
% \href{https://docs.moodle.org/35/en/Multiple_Choice_question_type}
% {Multichoice} & $\bullet$ & $\bullet$ & $\bullet$ & $\bullet$ &
% $\bullet$ & $\bullet$ & $\bullet$ & $\bullet$ & & & & $\bullet$ & $\bullet$ \\
% \href{https://docs.moodle.org/35/en/Numerical_question_type}{Numerical}
% & $\bullet$ & $\bullet$ & $\bullet$ & $\bullet$ & & & &
% & & $\bullet$ & & $\bullet$ & $\bullet$ & $\bullet$ \\
% \href{https://docs.moodle.org/35/en/Short-Answer_question_type}{Short
% Answer} & $\bullet$ & $\bullet$ & $\bullet$ & $\bullet$ & & & &
% & $\bullet$ & & & $\bullet$ & $\bullet$ \\
% \href{https://docs.moodle.org/35/en/Matching_question_type}{Matching}
% & $\bullet$ & $\bullet$ & $\bullet$ & $\bullet$ & $\bullet$ & & &
% & & & $\bullet$ & & \\
% \href{https://docs.moodle.org/35/en/True/False_question_type}
% {True/False} & $\bullet$ & & $\bullet$ & $\bullet$ & & & &
% & & & & & $\bullet$ \\
% \href{https://docs.moodle.org/35/en/Description_question_type}
% {Description} & & & $\bullet$ & $\bullet$ & & & &
% & & & & & \\
% ^^A\href{https://docs.moodle.org/35/en/Essay_question_type}{Essay} & \\\hline%
% \href{https://docs.moodle.org/35/en/Embedded_Answers_(Cloze)_question_type}{Cloze}
% & & $\bullet$ & $\bullet$ & $\bullet$ & & & &
% & & & & & \\\cmidrule(lr){1-1}
% \hspace{1em}Numerical & $\bullet$ & & & & & & &
% & &$\bullet$ & & $\bullet$ & $\bullet$ & $\bullet$ \\
% \hspace{1em}Short Answer & $\bullet$ & & & & & & &
% & $\bullet$ & & & $\bullet$ & $\bullet$ \\
% \hspace{1em}Multi (inline) & $\bullet$ & & & & $\bullet$ & & &
% & & & & $\bullet$ & $\bullet$ \\
% \hspace{1em}Multi (horiz.) & $\bullet$ & & & & $\bullet$ & & $\bullet$
% & & & & & $\bullet$ & $\bullet$ \\
% \hspace{1em}Multi (vert.) & $\bullet$ & & & & $\bullet$ & & $\bullet$
% & & & & & $\bullet$ & $\bullet$ \\
% \bottomrule
% \end{tabular}
% \end{table}
%
% \section{Conversion to HTML}
%
% \subsection{Level of Support}
%
% The package \pkg{moodle.sty} tries to automatically
% convert the \LaTeX\ code included in the questions
% into \HTML\ for web display.
%
% With this aim, a number of \TeX\ and \LaTeX\ macros,
% commands, and environments undergo a tailored treatment
% when \pkg{moodle} generates the \XML\ file. A few tables
% describe the current level of support:
% \begin{enumerate}
% \item text mode diacritic macros (e.g.~|\"u|) in Table~\vref{tab:diacritics},
% \item text mode macros for ligatures (e.g.~\cmd{\oe}) and other glyphs (e.g.~\cmd{\aa}) in
%       Table~\vref{tab:ligatures_and_glyphs},
% \item horizontal spacing (e.g.~\cmd{\quad}) and line breaking (e.g.~\cmd{\\}) macros in
%       Table~\vref{tab:spacing},
% \item text mode symbols (e.g.~\cmd{\$}) and punctuation (e.g.~\cmd{\textexclamdown}) macros in
%       Table~\vref{tab:symbols_and_punctuation}, and finally
% \item other \LaTeX\ commands (e.g.~\cmd{\emph}) and environments (e.g.~\env{center})
%       in Table~\vref{tab:commands_and_environments}.
% \end{enumerate}
%^^A \pagebreak
%
%^^A Inspired by https://tex.stackexchange.com/a/163717/228515
% \begin{table}[tbp]
% \centering
% \caption{Text mode diacritic macros undergoing a tailored conversion to \HTML.}
% \label{tab:diacritics}
% \begin{threeparttable}
% \begin{tabular}{clll}
% \toprule
% Definition & Letter List\tnote{1} & Description & Samples\\
% \cmidrule(lr){1-1}\cmidrule(lr){2-2}\cmidrule(lr){3-3}\cmidrule(lr){4-4}
% \cmd{\"}\marg{letter} & a, e, i, o, u, y &
% \href{https://en.wikipedia.org/wiki/Diaeresis_(diacritic)}{umlaut or diaeresis}
% & \"{a} \"{A} \"{e} \"{E} \"{i} \"{I} \"{o} \"{O} \dots\ \"{Y}\\
% \cmd{\'}\marg{letter} & a, e, i, o, u &
% \href{https://en.wikipedia.org/wiki/Acute\_accent}{acute accent}
% & \'{a} \'{A} \'{e} \'{E} \'{i} \'{I} \'{o} \'{O} \'{u} \'{U}\\
% \cmd{\.}\marg{letter} & c, e, g, i, z &
% \href{https://en.wikipedia.org/wiki/Dot_(diacritic)#Overdot}{overdot}
% & \.{c} \.{C} \.{e} \.{E} \.{g} \.{G} \.{i} \.{I} \.{z} \.{Z}\\
% \cmd{\=}\marg{letter} & a, e, g, i, o, u, y &
% \href{https://en.wikipedia.org/wiki/Macron_(diacritic)}{macron}
% & \={a} \={A} \={e} \={E} \={g} \={G} \={i} \={I} \dots\ \={Y}\\
% \cmd{\^}\marg{letter} & a, e, i, o, u &
% \href{https://en.wikipedia.org/wiki/Circumflex}{circumflex} &
% \^{a} \^{A} \^{e} \^{E} \^{i} \^{I} \^{o} \^{O} \^{u} \^{U}\\
% \cmd{\`}\marg{letter} & a, e, i, o, u &
% \href{https://en.wikipedia.org/wiki/Grave\_accent}{grave accent}
% & \`{a} \`{A} \`{e} \`{E} \`{i} \`{I} \`{o} \`{O} \`{u} \`{U}\\
% \cmd{\~}\marg{letter} & a, n, o &
% \href{https://en.wikipedia.org/wiki/Tilde}{tilde}
% & \~{a} \~{A} \~{n} \~{N} \~{o} \~{O}\\
% \cmd{\b}\marg{letter} & b, d, k, l, n, t, z &
% \href{https://en.wikipedia.org/wiki/Macron_below}{macron below}
% & \b{b} \b{B} \b{d} \b{D} \b{k} \b{K} \b{l} \dots\ \b{Z}\\
% \cmd{\c}\marg{letter} & c, s, t &
% \href{https://en.wikipedia.org/wiki/Cedilla}{cedilla}
% & \c{c} \c{C} \c{s} \c{S} \c{t} \c{T}\\
% \cmd{\d}\marg{letter} & a, b &
% \href{https://en.wikipedia.org/wiki/Dot_(diacritic)#Underdot}{underdot}
% & \d{a} \d{A} \d{b} \d{B}\\
% \cmd{\H}\marg{letter} & o, u &
% \href{https://en.wikipedia.org/wiki/Double\_acute\_accent}{double acute accent}
% & \H{o} \H{O} \H{u} \H{U}\\
% \cmd{\k}\marg{letter} & a, e, i, o, u &
% \href{https://en.wikipedia.org/wiki/Ogonek}{ogonek}
% & \k{a} \k{A} \k{e} \k{E} \k{i} \k{I} \k{o} \k{O} \k{u} \k{U}\\
% \cmd{\r}\marg{letter} & a, u &
% \href{https://en.wikipedia.org/wiki/Ring_(diacritic)#Overring}{overring}
% & \r{a} \r{A} \r{u} \r{U} \\
%^^A \cmd{\t}\marg{letter} & a, b, c & tie-after accent & \t{oo}\\
% \cmd{\u}\marg{letter} & a, e, g, i\tnote{2}, \i, o, u &
% \href{https://en.wikipedia.org/wiki/Breve}{breve}
% & \u{a} \u{A} \u{e} \u{E} \u{i} \u{\i} \u{I} \u{o} \u{O} \u{u} \u{U}\\
% \cmd{\v}\marg{letter} & c, d, e, l, n, r, s, t, z &
% \href{https://en.wikipedia.org/wiki/Caron}{caron or h\'a\v{c}ek}
% & \v{c} \v{C} \v{d} \v{D} \v{e} \v{E} \v{l} \dots\ \v{Z}\\
% \bottomrule
% \end{tabular}
% \begin{tablenotes}
%  \item[1]  The lowercase letters listed also stand for their uppercase
%  equivalent.
%  \item[2] pdf\TeX\ v3.14159265 typesets |\u{i}| with an objectionable tittle.
%  Use |\u{\i}|.
% \end{tablenotes}
% \end{threeparttable}
% \end{table}
%
% \begin{table}[tbp]
% \centering
% \caption{Text mode ligature and glyph macros undergoing a tailored conversion
% to \HTML.}
% \label{tab:ligatures_and_glyphs}
% \begin{threeparttable}
% \begin{tabular}{cccc}
% \toprule
% \multicolumn{2}{c}{Lowercase} & \multicolumn{2}{c}{Uppercase} \\
% \cmidrule(lr){1-2}\cmidrule(lr){3-4}
% Definition & Sample & Definition & Sample \\
% \cmidrule(lr){1-1}\cmidrule(lr){2-2}\cmidrule(lr){3-3}\cmidrule(lr){4-4}
% \cmd{\aa} & \aa & \cmd{\AA} & \AA \\
% \cmd{\ae} & \ae & \cmd{\AE} & \AE \\
% \cmd{\dh} & \dh & \cmd{\DH} & \DH \\
% \cmd{\dj} & \dj & \cmd{\DJ} & \DJ \\
% \cmd{\i} & \i &  &  \\
% \cmd{\ij} & \ij & \cmd{\IJ} & \IJ \\
% \cmd{\j} & \j &  &  \\
% \cmd{\l} & \l & \cmd{\L} & \L \\
% \cmd{\ng} & \ng & \cmd{\NG} & \NG \\
% \cmd{\o} & \o & \cmd{\O} & \O \\
% \cmd{\oe} & \oe & \cmd{\OE} & \OE \\
% \cmd{\ss} & \ss & \cmd{\SS}\tnote{1} & \SS\tnote{2} \\
% \cmd{\th} & \th & \cmd{\TH} & \TH \\
% \bottomrule
% \end{tabular}
% \begin{tablenotes}
%  \item[1] Contrarily to most fonts, Libertine, used in this documentation and
%  available for instance via the package \pkg{libertine}, defines the glyph
%  \SS.
%  \item[2] \LaTeX\ defines the \cmd{\SS} macro but pdf\TeX\ renders it as a
%  doubled capital S.
% \end{tablenotes}
% \end{threeparttable}
% \end{table}
%
% \begin{table}[tbp]
% \centering
% \caption{Text mode horizontal spacing and line breaking macros undergoing a
% tailored conversion to \HTML.}
% \label{tab:spacing}
%^^A Inspired by https://tex.stackexchange.com/a/74354/228515
% \begin{threeparttable}
% \begin{tabular}{ll}
% \toprule
% Definition & Description \\
% \cmidrule(lr){1-1}\cmidrule(lr){2-2}
% \cmd{\,} or \cmd{\thinspace} & narrow non-breaking space \\
% |~| or \texttt{\textbackslash\textvisiblespace} & non-breaking space \\
% \cmd{\>}\tnote{1},{ } \cmd{\:}{}\tnote{2}{ } or \cmd{\medspace}\tnote{2} & mid space \\
% \cmd{\;}\tnote{2}{ } or \cmd{\thickspace}\tnote{2} & thick space \\
% \cmd{\enspace} & nut (1en wide space)\\
% \cmd{\quad} & mutton (1em wide space)\\
% \cmd{\qquad} & doubled mutton (2em wide space)\\
% \cmd{\textvisiblespace} & sample: \textvisiblespace \\
% \cmd{\\} or \cmd{\newline} & start a new line \\
% |\par| or \meta{blank line} & start a new paragraph \\
% \bottomrule
% \end{tabular}
% \begin{tablenotes}
%  \item[1] \cmd{\>} is defined in math-mode only.
%  \item[2] \cmd{\:}, \cmd{\medspace}, \cmd{\;}, and \cmd{\thickspace} require the package
%  \pkg{amsmath}.
% \end{tablenotes}
% \end{threeparttable}
% \end{table}
%
% \begin{table}[tbp]
% \centering
% \caption{Text mode punctuation marks and symbol macros undergoing a tailored
% conversion to \HTML. A baseline is represented in the samples of quotations
% marks, in order to draw attention to their vertical positioning.}
% \label{tab:symbols_and_punctuation}
% \newsavebox\textbox
% \newcommand\showbaseline[1]{\leavevmode\sbox\textbox{#1}
% \rlap{\rule{\wd\textbox}{.1pt}}\usebox\textbox}
% \begin{threeparttable}
% \begin{tabular}{llc}
% \toprule
% Package & Definition & Sample \\
% \cmidrule(lr){1-1}\cmidrule(lr){2-2}\cmidrule(lr){3-3}
% \LaTeX\ base & \cmd{\%} & \% \\
% & \cmd{\#} & \# \\
% & \cmd{\_} & \_ \\
% & \cmd{\textbackslash} & \textbackslash \\
% & \cmd{\$} & \$ \\
% & \cmd{\&} & \& \\
% & \cmd{\S} & \S \\
% & |\{| & \{ \\
% & |\}| & \} \\
% & \cmd{\texteuro} & \texteuro \\
% & \cmd{\dots} or \cmd{\ldots} & \dots \\
% & \cmd{\textexclamdown} & \textexclamdown \\
% & \cmd{\textquestiondown} & \textquestiondown \\
% & |--| & -- \\
% & |=| & = \\
% & |`| and |'|\tnote{1} & \showbaseline{` '} \\
% & \cmd{\textquoteleft} and \cmd{\textquoteright}
% & \showbaseline{\textquoteleft\ \textquoteright} \\
% & |``| and |''|\tnote{1,2} & \showbaseline{`` ''} \\
% & \cmd{\textquotedblleft} and \cmd{\textquotedblright}
% & \showbaseline{\textquotedblleft\ \textquotedblright} \\
% & \cmd{\textquotesingle} & \showbaseline{\textquotesingle\ } \\
% & |"| & \showbaseline{" }\tnote{2} \\
% & \cmd{\textquotedbl} & \showbaseline{\textquotedbl\ }\\
% & \cmd{\guilsinglleft} and \cmd{\guilsinglright}
% & \showbaseline{\guilsinglleft\ \guilsinglright} \\
% & \cmd{\guillemotleft} and \cmd{\guillemotright}
% & \showbaseline{\guillemotleft\ \guillemotright}\\
% & \cmd{\quotesinglbase} & \showbaseline{\quotesinglbase\ }\\
% & \cmd{\quotedblbase} & \showbaseline{\quotedblbase\ } \\
% \cmidrule(lr){1-1}
% \pkg{eurosym} & \cmd{\euro} & \euro\\
% \cmidrule(lr){1-1}
% \pkg{babel} & \cmd{\flq} and \cmd{\frq} & \showbaseline{\flq\ \frq} \\
% & \cmd{\flqq} and \cmd{\frqq} & \showbaseline{\flqq\ \frqq} \\
% \cmidrule(lr){1-1}
% \pkg{babel}, opt. \optn{french} & \cmd{\og} and \cmd{\fg}\tnote{3}
% & \showbaseline{\flqq\ \frqq} \\
% \cmidrule(lr){1-1}
% \pkg{babel}, opt. \optn{german} & \cmd{\glq} and \cmd{\grq}
% & \showbaseline{\glq\ \grq} \\
% & \cmd{\dq} & \showbaseline{\dq\ }\tnote{2} \\
% & \cmd{\glqq} and \cmd{\grqq} & \showbaseline{\glqq\ \grqq} \\
% \bottomrule
% \end{tabular}
% \begin{tablenotes}
%  \item[1] When placed in math mode, the single straight quote (\texttt{'}) is
%           passed as-is to the XML, even when doubled (double prime symbol).
%  \item[2] In roman type, \LuaLaTeX\ and \XeLaTeX\ typeset the double straight
%  quote symbol (\texttt{"}) and the command \cmd{\dq} as a double right quote
%  (\textquotedblright). Instead,  \pkg{moodle} follows pdf\LaTeX: whatever
%  the \TeX\ engine used, the double straight quote (\texttt{"}) is passed to the \XML.
%   \item[3] The way \cmd{\og} and \cmd{\fg} are typeset in the \PDF\ depends on the
%   current babel language. Regardless, \pkg{moodle} passes the symbols
%   \texttt{\guillemotleft} and \texttt{\guillemotright} to the \XML.
% \end{tablenotes}
% \end{threeparttable}
% \end{table}
%
% \begin{table}[tbp]
% \centering
% \caption{\LaTeX\ commands and environments undergoing a tailored conversion
% to \HTML.}
% \label{tab:commands_and_environments}
% \begin{tabular}{lll}
% \toprule
% Package & Commands & Environments \\
% \cmidrule(lr){1-1}\cmidrule(lr){2-2}\cmidrule(lr){3-3}
% \LaTeX\ base & |{}|& |$|$\cdots$|$| (inline math) \\
% & \cmd{\relax} & |\(|$\cdots$|\)| (inline math) \\
% & \cmd{\LaTeX} &  |$$|$\cdots$|$$| (display math) \\
% & \cmd{\TeX} & \cmd{\[}$\cdots$\cmd{\]} (display math) \\
% & \cmd{\emph}\marg{\dots} & \env{center} \\
% & \cmd{\textbf}\marg{\dots} & \env{enumerate} \\
% & \cmd{\textit}\marg{\dots} & \env{itemize} \\
% & \cmd{\textsc}\marg{\dots} & \env{quote} \\
% & \cmd{\textsuperscript}\marg{\dots} & \env{quotation}\\
% & \cmd{\textsubscript}\marg{\dots} &\\
% & \cmd{\texttt}\marg{\dots} &\\
% & \cmd{\underline}\marg{\dots} &\\
% \cmidrule(lr){1-1}
% \pkg{hyperref} & \cmd{\href}\marg{\dots}\marg{\dots} &\\
% \pkg{url} or \pkg{hyperref} & \cmd{\url}\marg{\dots} &\\
% \cmidrule(lr){1-1}
% \pkg{babel}, opt. \optn{french} & \cmd{\fup}\marg{\dots} &\\
% & \cmd{\up}\marg{\dots} &\\
% \cmidrule(lr){1-1}
% \pkg{graphics} or \pkg{graphicx} &
% \cmd{\includegraphics}\oarg{\dots}\marg{\dots} \\
% \cmidrule(lr){1-1}
% \pkg{tikz} & \cmd{\tikz}\oarg{\dots}\marg{\dots} & \env{tikzpicture}\oarg{\dots} \\
% \cmidrule(lr){1-1}
% \pkg{verbatim} & \cmd{\verbatiminput}\marg{\dots} &\\
% \cmidrule(lr){1-1}
% \pkg{fancyverb} or
% \pkg{fvextra}& \cmd{\VerbatimInput}\oarg{\dots}\marg{\dots} &\\
% & \cmd{\LVerbatimInput}\oarg{\dots}\marg{\dots} &\\
% & \cmd{\BVerbatimInput}\oarg{\dots}\marg{\dots}& \\
% \cmidrule(lr){1-1}
% \pkg{minted} & \cmd{\inputminted}\oarg{\dots}\marg{\dots}\marg{\dots} &\\
% \bottomrule
% \end{tabular}
% \end{table}
% 
% In addition, |<| and |>| will be converted to their HTML equivalents |&lt;|
% and |&gt;| in the \XML\ file. This prevents  portions of the code to be
% interpreted by \Moodle\ as \HTML\ tags.
% 
% A doubled dash will be converted to en-dash |&ndash;| \emph{outside math mode}.
% Empty groups |{}| will be passed to the XML only \emph{in math mode}.
%
% Be aware that, apart what is described previously, \pkg{moodle} \emph{does
% not know how to convert any other \TeX\ or \LaTeX\ commands to \HTML.}
% If other sequences are used, they may be passed verbatim to the \XML\ file
% or may cause unpredicted results.
%
% If you think of another \LaTeX\ command that should be changed to an HTML
% equivalent, please have a look at Section~\vref{sec:dev}.
%
% \DescribeMacro[question,answer,feedback]{\htmlonly}\oarg{Content for traditional output}
% \marg{HTML content} is a command to be used inside question environments (text,
% answers, or feedback). It lets you pass directly code to the \XML\ file while being
% ignored for the traditional output (\PDF). The \HTML\ content passed as an argument
% is subject to no particular processing and users should not expect to be able to
% pass dangerous characters like |\|, |%|, or |#|. An optional argument allows to
% pass contents to be processed for the traditional output. This argument is
% ignored for the \XML\ output.
% For instance, one can write code like this in a question environment
% \begin{quote}
%   |\htmlonly[\fbox{PDF contents}]{|\\
%   |<div style="border: 1px solid; display: inline-block;">|\\
%   \rule{10em}{0pt}|HTML contents</div>|\\
%   |}|
% \end{quote}
%
% \DescribeMacro[moodle]{\htmlregister}\marg{command} is a command that lets you
% specify a macro that must be expanded in the \XML\ file. It works only when
% the macro is defined without optional argument.\watchout
%
% \DescribeMacro[moodle]{\moodleregisternewcommands}
% When the list of user-defined macros is long, it becomes cumbersome to
% record them individually for expansion. Calling \cmd{\moodleregisternewcommands}
% watches for subsequent calls to \cmd{\newcommand}, \cmd{\renewcommand},
% \cmd{\providecommand}, and their starred variants such that the corresponding
% commands are automatically expanded by \pkg{moodle}.
% Again, this works only if the macros are defined \emph{without} optional
% argument.\watchout
%
% \subsection{Graphics}
% The \pkg{moodle} package can handle two kinds of graphics seamlessly.
% External graphics files may be included with the \cmd{\includegraphics} command
% from the \pkg{graphicx} package, and graphics may be generated internally using \TikZ.
% In either case, the graphics will be embedded in base-64 encoding directly within
% the \Moodle\ \XML\ produced.  This prevents the hassle of managing separate
% graphics files on the \Moodle\ server, as \Moodle\ will store the picture
% within the question in the question bank.
%
% \subsubsection{Default \texttt{includegraphics}}
%
% \DescribeMacro[graphics,graphicx]{\includegraphics}\oarg{options}\marg{file} can be used to
% include graphic files in both \PDF\ and the \XML\ outputs.
% The only options currently supported are \optn{height}\DescribeKey[includegraphics]{height}\
% and \optn{width}\DescribeKey[includegraphics]{width}.  Attempts to use other
% \cmd{\includegraphics} options, such as \optn{scale} or \optn{angle}, will affect the \PDF\
% but not the \XML\ output.
% The dimensions set by \optn{height} and \optn{width} are \TeX\ dimensions such as
% \texttt{4\,in} or \texttt{2.3\,cm}.
% In order to prepare the image for web viewing, this package converts those
% dimensions to pixels using a default of
% 103 pixels per inch.\footnote{This
%   number was selected because an image with
%   |<IMG HEIGHT=103 WIDTH=103 SRC="...">| showed up as almost exactly 1 inch
%   tall and 1 inch wide on several of this author's devices and browsers
%   as of January 2016.}
% \DescribeKey[includegraphics]{ppi}
% That value may be changed by setting the \optn{ppi} key (e.g., \texttt{ppi=72});
% this is probably best done for the entire document with a \cmd{\moodleset} command,
% rather than image-by-image.
%
% \DescribeMacro[graphics,graphicx]{\graphicspath}|{{|\meta{path1}|},{|\meta{path2}|},...}| can be
% used to specify the locations of the pictures to be included.
%
% A special rule was added for the inclusion of \GIF\ pictures (\filenm{.gif} extension).
% These files are passed as-is to the \XML, preserving potential animations.
% However, as pdf\TeX\ engines do not support the \GIF\ format, the picture is
% passed to the \PDF\ output after a conversion to the \PNG\ format. When the \GIF\
% file is animated, only its first frame is passed to the \PDF.
%
%^^A \DescribeMacro[moodle]{\DeclareGraphicsAlien} Users may declare other graphic
%^^A formats with the command \cmd{\DeclareGraphicsAlien}\marg{alien extension}
%^^A \marg{native extension}\marg{command line for conversion}.
%
% \subsubsection{\TikZ\ Pictures}
% \TikZ\ is a user-friendly syntax layer for \acro{PGF}, the macro \LaTeX\ package for creating
% graphics. More information on \TikZ\ can be found at \url{https://ctan.org/pkg/pgf}.
%
% When \TikZ\ is loaded and used to define pictures, \pkg{moodle} invokes
% the \pkg{external} \TikZ\ library, so that each \env{tikzpicture} environment is compiled
% to a freestanding \PDF\ file.
%
% \subsubsection{Package Option \optn{tikz}}\label{subsec:tikz}
% \DescribeOption[moodle]{tikz}
% The \pkg{moodle} package admits a \optn{tikz} option which has the following effects:
% \begin{itemize}
% \item the package \pkg{tikz} is loaded.
% \item \DescribeMacro[tikz]{\includegraphics}\cmd{\includegraphics} embeds graphics in a \TikZ\ picture. Consequences are that
%   \begin{itemize}
%     \item the pictures encoded in the \XML\ file are resampled. This prevents encoding
%           images at a higher resolution than rendered by \Moodle.
%     \item the full set of \cmd{\includegraphics} options is accessible,
% e.g.~\optn{scale=.5}, \optn{angle=90}, or |width=.2\textwidth|.
%   \end{itemize}
% \item \DescribeMacro[tikz]{\embedaspict} a macro \cmd{\embedaspict}\marg{\LaTeX\ contents}
% is provided for the conversion of inline \LaTeX\ material as images. This can serve
% as a workaround to overcome limitations of this package---like the conversion of
% \emph{tabular}s to HTML--- or limitations of \Moodle\ itself.
% For the definition of this macro, the package \pkg{varwidth} is loaded.
% \item optimizations of the \TikZ-external library are disabled: compilation might get
% sensibly slower.
% \end{itemize}
%

% \subsubsection{External Tools}
% The mechanisms used for handling graphics are somewhat fragile and rely upon
% three free external programs.
% \begin{enumerate}
%   \item \prog{GhostScript} (\url{www.ghostscript.com}) is used to convert the \PDF\ output
%         from \TikZ\ into a \PNG\ raster graphics file.
%         The default command line is presumed to be \prog{gswin64c.exe}
%%%%%%%%%%%%%%%%%%%%%%%%%%%%%%%%%%%%%%%%%%%%%%%%%%%%%%%%%%
%         (if |\ifwindows| from the \pkg{ifplatform} package returns true)
%         or \prog{gs} (if |\ifwindows| returns false).
%%%%%%%%%%%%%%%%%%%%%%%%%%%%%%%%%%%%%%%%%%%%%%%%%%%%%%%%%
%         If your system requires a different command line to invoke \prog{Ghostscript},
%         \DescribeMacro[moodle]{\ghostscriptcommand}
%         you may change it by invoking:
%            \begin{quote}
%             |\ghostscriptcommand|\marg{executable filename}
%            \end{quote}
%   \item When external graphics files such as \PDF\ are included,
%         the \prog{ImageMagick} software (\url{www.imagemagick.org})
%         converts each file to \PNG\ format.
%         The command line for \prog{ImageMagick} is the nondescript word \progcode{convert},
%         \DescribeMacro[moodle]{\imagemagickcommand}
%         but may be changed by invoking \cmd{\imagemagickcommand}\marg{executable filename}.
%   \item \prog{OptiPNG} (\url{http://optipng.sourceforge.net/}) is used to optimize the \PNG\ images.
%         The command line is presumed to be \prog{optipng}, but can be changed with
%         \DescribeMacro[moodle]{\optipngcommand}\cmd{\optipngcommand}\marg{executable filename}.
% \end{enumerate}

%
% Please note the following vital points to make the graphics handling work:
% \begin{itemize}
%   \item As of now, graphics are only supported when compiling directly to a \PDF\
%         with (\prog{pdf}$\mid$\prog{xe}$\mid$\prog{lua})\prog{latex}. Including \acro{PS} graphics or using \TikZ\ with the \DVI$\to$\acro{PS} workflow is not
%         yet supported.
%^^A   \item Filenames should not contains spaces or, under windows, special characters like |_| or |\|.
%   \item You must have \prog{Ghostscript} and \prog{ImageMagick} installed on your system
%         to fully use the graphics-handling capabilities of \pkg{moodle}.
%   \item If \prog{OptiPNG} is not installed, the corresponding system calls will fail with otherwise no
%         impact on the compilation process: \PNG\ files are passed unoptimized to the \XML\ output.
%   \item \LaTeX\ must be able to call system commands; that is, \cmd{\write18} must be enabled.
%         For Mik\TeX, this means adding \progcode{--enable-write18} to the command line of
%         (\prog{pdf}$\mid$\prog{xe}$\mid$\prog{lua})\prog{latex}; for \TeX\ Live, this means adding \progcode{--shell-escape=true}.
%   \item Due to security issues with old versions of \prog{Ghostcript}, some systems default to a
%         policy that prevents the conversion of \PDF\ and \acro{PS} to \PNG. Assuming that, as a user of
%         \pkg{moodle} which requires shell escape capabilities, you either use a sandboxed
%         environment or trust the files handled at the system-level, you may want to disable this
%         over-zealous security policy. For example,
%         \href{https://stackoverflow.com/a/52661288/14608059}{see this}.
%   \item Users of the \pkg{circuitikz} package (\url{https://www.ctan.org/pkg/circuitikz}) must
%         enclose their circuits' \TikZ\ code in the |tikzpicture| environment instead of
%         the historical \env{circuitikz} environment. That is required, as of \TikZ\ 2.1,
%         by the \pkg{external} library.
% \end{itemize}
%
% \subsubsection{Package Option \optn{svg}}\label{subsec:svg}
%
% \paragraph{Important Notice}\watchout[experimental] \emph{The \optn{svg} option is
% an experimental feature introduced in \pkg{moodle} v0.8. It has been tested only
% under Linux and Windows, with TeX Live 2020, Inkscape v1.0.1 and Scour 0.38.2.}
%
% \DescribeOption[moodle]{svg} The \pkg{moodle} package admits an experimental
% \optn{svg} option which has the following effects:
% \begin{itemize}
% \item \DescribeMacro[svg]{\includegraphics}\cmd{\includegraphics} can be used to import
% \SVG\ graphic files directly (extension \filenm{.svg} or \filenm{.SVG}). In this case, the \SVG\ file
% is passed as-is to the \XML\ output and is converted using \href{https://inkscape.org/}
% {\prog{Inkscape}} (must be installed) for inclusion in the \PDF\ output.
% \item the graphic files in \PDF\ format are converted to the \SVG\ format using
% \href{https://inkscape.org/}{Inkscape} (must be installed), rather than being
% rasterized. Before inclusion to the \XML\ output, the \SVG\ file is optimized using
% the \href{https://github.com/scour-project/scour}{Scour} utility. This
% optimization step is optional in the sense that, if the Scour call fails,
% the un-optimized \SVG\ file will be passed to the \XML\ output.
% Two processes benefit from this \PDF$\to$\SVG\ conversion:
%   \begin{itemize}
%     \item inclusion of \PDF\ graphics with \cmd{\includegraphics}, and
%     \item \TikZ\ pictures that are externalized.
%   \end{itemize}
% \end{itemize}
%
%
% The call of external \SVG\ manipulation utilities can be modified using the macros:
% \begin{description}
%   \item \ItemDescribeMacro[svg]{\PDFtoSVGcommand}\ for conversion
%         from \PDF\ to \SVG,
%   \item \ItemDescribeMacro[svg]{\SVGtoPDFcommand}\ for conversion
%         from \SVG\ to \PDF, and
%   \item \ItemDescribeMacro[svg]{\optiSVGcommand}\ for the
%         optimization of \SVG\ files.
% \end{description}
%
% \subsection{Verbatim Code}
%
% Because, for \HTML\ translation, \pkg{moodle} parses the body of questions, the use of
% verbatim code results in compilation errors. This is why the use of |\verb|,
% \env{verbatim} and other standard utilities is not supported.
%
% However, using the following three utilities, verbatim code can be imported from an external file:
% \begin{enumerate}
%   \item \cmd{\verbatiminput}\marg{filename}\DescribeMacro[verbatim]{\verbatiminput}\
%         from the \pkg{verbatim} package inserts verbatim code in both the \PDF\ and the
%         \acro{XML} for \Moodle, without fancy additions.
%   \item The macro \cmd{\VerbatimInput}\marg{options}\marg{filename}
%         \DescribeMacro[fancyvrb,fvextra]{\VerbatimInput}\ from
%         \pkg{fancyvrb} or \pkg{fvextra} does more, with several options and settings
%         offered (see below).
%         The variants \DescribeMacro[fancyvrb,fvextra]{\BVerbatimInput}\cmd{\VBerbatimInput} and
%         \DescribeMacro[fancyvrb,fvextra]{\LVerbatimInput}\cmd{\LVerbatimInput} are also supported,
%         with no difference on the \XML\ output.
%         The variants with a star are unsupported and result in errors when used.
%   \item On top of that \cmd{\inputminted}\oarg{options}\marg{lang}\marg{filename}\
%         \DescribeMacro[minted]{\inputminted} from the \pkg{minted} package offers syntax
%         highlighting tailored to the specified language.
% \end{enumerate}
% The \pkg{moodle} package handles these three commands to pass the code in the output \XML.
% With \cmd{\inputminted}, an external Python tool, \prog{pygmentize}, performs syntax analysis and
% its \HTML\ formatter is used to populate the \XML. With the other commands, the contents of the
% file is passed almost as-is to the XML: in order to survive \Moodle\ import and \HTML\ rendering,
% characters |<|, |>|, |&|, |'|, and |"| are converted to \HTML\ equivalents.
%
% With \cmd{\VerbatimInput} and \cmd{\inputminted}, the options that are taken care of for XML
% generation are listed in Table~\vref{tab:verbatim-options}.
% Using \cmd{\fvset}\marg{key=value,\dots}, options can be set globally. Equivalently, with \pkg{minted},
% \cmd{\setminted}\oarg{lang}\marg{key=value,\dots} is available.
%
% \begin{table}[tbp]
% \centering
% \begin{threeparttable}[b]
% \caption{Options and corresponding values considered for \XML\ generation of verbatim material
% with \cmd{\VerbatimInput} and \cmd{\inputminted}.}
% \label{tab:verbatim-options}
% \begin{tabular}{ll}
% \toprule
% Option keys & Possible values\\\cmidrule(lr){1-1}\cmidrule(lr){2-2}
% ^^A\optn{commentchar} & \meta{character}\\
% \optn{gobble} & \meta{integer}\\
% \optn{autogobble}\tnote{1} & \texttt{true} or \texttt{false}\\
% \optn{tabsize} & \meta{integer}\\
% \optn{numbers} & \texttt{none}, \texttt{left}, \texttt{right}, or \texttt{both}\tnote{2}\\
% \optn{firstnumber} & \texttt{auto}, \texttt{last}, or \meta{integer}\\
% \optn{firstline} & \meta{integer}\\
% \optn{lastline} & \meta{integer}\\
% \optn{numberblanklines} & \texttt{true} or \texttt{false}\\
% \optn{highlightlines}\tnote{2} & \marg{coma-separated list of integers or ranges}\\
% \optn{style}\tnote{1} & \meta{string}\\
% \bottomrule
% \end{tabular}
%\begin{tablenotes}
%\item[1] \optn{autogobble}, \optn{numbers=both}, and \optn{style} are from \pkg{minted}.
%\item[2] line highlighting is offered only with \pkg{fvextra} or \pkg{minted} loaded.
%\end{tablenotes}
%\end{threeparttable}
%\end{table}
%
% In order to define the verbatim code from the \LaTeX\ document itself, it is still possible
% to use, outside the scope of the \pkg{moodle} questions, the environments \env{filecontents*}
% (from the \pkg{filecontents} package or \LaTeX\ kernel itself since 2019) or \env{VerbatimOut}
% (from the \pkg{fancyvrb} and \pkg{fvextra} packages).
%
% Here is an example:
% \begin{VerbatimOut}[gobble=2]{minted.doc.out}
%   \documentclass[12pt,a4paper]{article}
%   \usepackage[section]{moodle}
%   \usepackage{minted}
%   \begin{document}
%   \begin{quiz}[tags={minted}]{LaTeX Quiz}
%     \begin{filecontents*}{excerpt.tex}
%   % !TeX encoding = UTF-8
%   % !TeX spellcheck = en_US
%   % !TEX TS-program = lualatex
%   \documentclass{article}
%   \usepackage[nostamp]{moodle}
%   \ifPDFTeX % FOR LATEX and PDFLATEX
%     \usepackage[utf8]{inputenc} % necessary
%     \usepackage[T1]{fontenc} % necessary
%   \else % assuming XELATEX or LUALATEX
%     \usepackage{fontspec}
%   \fi
%     \end{filecontents*}
%     \begin{numerical}[tolerance=0]{Loading a Class}
%       Consider the following \LaTeX\ code excerpt.\\
%       \inputminted[numbers=left]{latex}{excerpt.tex}
%       On which line is the class loaded?
%       \item[feedback={
%            yes! \inputminted[highlightlines={4}]{latex}{excerpt.tex} }] 4
%       \item[feedback={Line 3 is just a comment.},fraction=0] 3
%       \item[feedback={Line 5 loads the package \texttt{moodle}},fraction=0] 5
%     \end{numerical}
%     \begin{multi}[single]{IDE}
%       Consider the following \LaTeX\ code excerpt.\\
%       \inputminted[numbers=left]{latex}{excerpt.tex}
%       Which \TeX\ engine will be used by the IDE for compilation.
%       \item[feedback={Have a closer look at line 3}] \texttt{tex}
%       \item[feedback={Have a closer look at line 3}] \texttt{latex}
%       \item[feedback={Have a closer look at line 3}] \texttt{pdflatex}
%       \item[feedback={Have a closer look at line 3}] \texttt{xelatex}
%       \item* \texttt{lualatex}
%     \end{multi}
%   \end{quiz}
%   \end{document}
% \end{VerbatimOut}
% \inputminted[gobble=2,frame=lines]{latex}{minted.doc.out}
%
% When code decorated with left-side line numbers is placed in question items, the output \PDF\ could
% show a collision between numbers of the item and the first line. To avoid this, \cmd{\LVerbatimInput} or
% \cmd{\BVerbatimInput} can be used. Instead, when \pkg{minted} is used, the ``left-right'' mode can be
% enforced with the \LaTeX\ command:
% \begin{quote}
%   |\RecustomVerbatimEnvironment{Verbatim}{LVerbatim}{}|
%  \end{quote}
%
% When using utilities from \pkg{fancyvrb}, \pkg{fvextra}, or \pkg{minted}, \pkg{moodle}
% sets framing options for the display of code in the output \PDF:
% \begin{quote}
%   |\fvset{frame=lines,label={[Beginning of code]End of code},|\\
%   |       framesep=3mm,numbersep=9pt}|
% \end{quote}
% These settings can be overridden using \cmd{\fvset} after the preamble.
%
% \section{Internationalization}
%
% This section is intended for authors of \Moodle\ quizzes writing in one or several
% languages other than English. It seems reasonable to assume that those users have heard
% of the packages \pkg{babel} or \pkg{polyglossia} (\XeTeX\ or \LuaTeX\ only), that
% both aim at enforcing language-related rules to \LaTeX\ documents.
%
% The contents of the \XML\ file that is generated by \pkg{moodle} depends entirely
% on the user input and \Moodle's \XML\ syntax: there is little room for \pkg{moodle} to
% internationalize something here. Instead, the focus of this section  is the \PDF\
% typesetting that is determined by \pkg{moodle}, where some internalization
% efforts can be of help.
%
% Both \pkg{babel} and \pkg{polyglossia} provide ways of specifiying a language for
% the document or a part of it. If one of these packages is loaded in the preamble,
% \pkg{moodle} will automatically load the package \pkg{translator} and rely on it
% to provide translations of its keys (see Table~\vref{tab:language_keys}) in
% different languages, with some knowledge on aliases. The package
% \pkg{translations} does a very similar job and can be loaded in the preamble
% to serve, if desired, as a replacement for \pkg{translator}.
%
% Currently, full-support is provided for English, French, German, Italian, and
% Spanish. Very limited support is provided for Catalan, Croatian, Czech, Danish, Dutch,
% Estonian, Finnish, Hungarian, Icelandic, Lithuanian, Norsk, Polish, Portuguese,
% Romanian, Swedish, and Turkish.
%
% Users may define their own translations in the preamble. For instance
% \begin{quote}
%   |\AfterEndPreamble{|\\
%     |  \DeclareTranslation|\marg{babel language name}|{True}{Foo}%|\\
%     |  \DeclareTranslation|\marg{babel language name}|{False}{Bar}%|\\
%   |}|
% \end{quote}
% Note that, when the package \pkg{translations} is loaded,
% \cmd{\AfterEndPreamble} must not be used.
%
% Contributions to improve or broaden the linguistic support are very welcome.
%
% \begin{center}
% \begin{longtable}{ l p{.55\textwidth} }
% \caption{\pkg{moodle}'s language keys for internationalization of the \PDF\ typesetting.}
% \label{tab:language_keys}\\
% \toprule
% Key & Context\\
% \cmidrule(lr){1-1}\cmidrule(lr){2-2}
% \endfirsthead
% \multicolumn{2}{c}%
% {\tablename\ \thetable\ -- \textit{Continued from previous page}} \\
% \hline
% Key & Context\\
% \cmidrule(lr){1-1}\cmidrule(lr){2-2}
% \endhead
% \hline\multicolumn{2}{r}{\textit{Continued on next page}} \\
% \endfoot
% \bottomrule
% \endlastfoot
% \optn{True} & indicates option ``True'' in a True/False question.\\
% \optn{False} & indicates option ``False'' in a True/False question.\\
% \optn{cloze} & tag indicating a ``cloze'' question.\\
% \optn{description} & tag indicating a ``description'' question.\\
% \optn{essay} & tag indicating an ``essay'' question.\\
% \optn{matching} & tag indicating a ``matching'' question.\\
% \optn{multi} & tag indicating a ``multichoice'' question.\\
% \optn{numerical} & tag indicating a ``numerical'' question.\\
% \optn{shortanswer} & tag indicating a ``shortanswer'' question.\\
% \optn{truefalse} & tag indicating a ``true/false'' question.\\
% \optn{Shuffle} & tag indicating that options offered will be shuffled in a
% ``multichoice'' or ``matching'' question.\\
% \optn{Single} & tag indicating that only one option can be selected in a
% ``multichoice'' question.\\
% \optn{Multiple} & tag indicating that several options can be selected in a
% ``multichoice'' question.\\
% \optn{marked out of} & tag indicating the weight of the question (maximum
% number of points), followed by a number.\\
% \optn{penalty} & tag indicating the penalty factor applied for each wrong
%  attempt in adaptive mode.\\
% \optn{tags} & indicates the beginning of a tag list characterizing the
% question.\\
% \optn{All-or-nothing} & tag indicating that all correct options must be
% selected to get credited a good answer.\\
% \optn{Case-Sensitive} & tag indicating that the case of characters matters
% for a ``Shortanswer'' question.\\
% \optn{Case-Insensitive} & tag indicating that the case of characters
% does not matter for a ``Shortanswer'' question.\\
% \optn{Drag and drop} & tag indicating that a ``matching'' question relies
% on the Drag-and-drop plugin.\\
% \optn{Information for graders} & indicates the beginning of a paragraph
% where instructions for the graders of an ``essay'' question are given.\\
% \optn{Response template} & indicates the beginning of a paragraph where
% the answer template of an ``essay'' question is represented.\\
% \optn{editor} & tag indicating that, for answering an essay question, an
% editor with \HTML\ support will be proposed.\\
% \optn{editorfilepicker} & tag indicating that, for answering an essay question,
% an editor with \HTML\ support and a file picker will be proposed.\\
% \optn{plain} & tag indicating that, for answering an essay question, an
% editor with no markup support will be proposed.\\
% \optn{monospaced} & tag indicating that, for answering an essay question,
% an editor with fixed-width font and no markup support will be proposed.\\
% \optn{noinline} & tag indicating that, for answering an essay question, a file
% picker will be proposed.\\
% \optn{Total of marks} & at the end of the quiz, indicates the sum of the
% weights of all questions, followed by a colon and a number. \\
% \end{longtable}
% \end{center}
%
% \section{Package Development}\label{sec:dev}
%
% \subsection{Feature Requests, Bug Reports, and Contributions}
%
% This package is developed as a collaborative project, currently hosted
% on the Gitlab server instance \url{https://framagit.org/mattgk/moodle}.
% The project's activity can be monitored there: reported issues, last
% modifications, \dots
%
% Contributions, either bug reports or fixes, are welcome. Users willing to
% help can either sign-in with an existing \brand{GitHub}, \brand{Gitlab.com}, or
% \brand{Bitbucket} account or register a new account.
%
% Of course, getting in touch with the package maintainer by
% \href{mailto:guerquin-kernATcrans.org}{email} works as well.
%
% The authors have used this package together with a limited number of colleagues
% for a few semesters of teaching. If other users adopt this package, we fully expect them
% to find bugs.
%
% When experiencing a problem and before reporting it, please check whether
% or not something similar has already been filed as an issue
% \href{https://framagit.org/mattgk/moodle/-/issues?scope=all&state=all}{here}.
% If the problem appears to be new, please report it by following these steps:
% \begin{enumerate}
% \item Prepare a \emph{minimal} working example, i.e. a \filenm{.tex} file shrunk down
% to the strict minimum (loaded packages, code, \dots) while still showing the
% faulty behavior upon compilation.
% \item Gather and send the \filenm{*.tex}, \filenm{*.log} and \filenm{*-moodle.xml} files together
% with an explanation about
%   \begin{itemize}
%     \item your local working environment (\TeX\ engine and distribution, platform,
%           external tools used, \dots),
%     \item the version of the \Moodle\ instance you are using
%           (for instance see the file |https://|\meta{server-domain}|/lib/upgrade.txt| )
%     \item the faulty behavior, and
%     \item what you expected instead.
%   \end{itemize}
% \end{enumerate}
%
% \subsection{Known Limitations}
%
% When using with pdf\TeX\ for compilation, \pkg{moodle} supports
% question contents with \acro{ASCII} characters only. The use of
% non-\acro{ASCII} characters may work in some cases but will most
% probably yield compilation errors or undesired \XML\ contents.
%
% Instead, \pkg{moodle} will work flawlessly when \XeTeX\ and
% \LuaTeX\ are used to compile \UTF-8 encoded documents.
%
% Table~\vref{tab:support-limitations} lists
% some features supported, limitations, and bugs.
%\begin{table}[tbp]
%\centering
%\begin{threeparttable}[b]
%\caption{Content enrichment (pictures, equations) support after \XML\ import in \Moodle\ v3.5.7,
% depending on the question type.}
%\label{tab:support-limitations}
%\begin{tabular}{lccc}
% \toprule
% & \multicolumn{3}{l}{XML rendering in\dots}\\\cmidrule(lr){2-4}
%Question type & Question & Answer & Feedback\\\cmidrule(lr){1-1}\cmidrule(lr){2-2}
%\cmidrule(lr){3-3}\cmidrule(lr){4-4}
% \href{https://docs.moodle.org/en/Multiple_Choice_question_type}{Multichoice}
%& yes & yes & yes \\
% \href{https://docs.moodle.org/en/Numerical_question_type}{Numerical}
%& yes & no\tnote{1} & yes \\
% \href{https://docs.moodle.org/en/Short-Answer_question_type}{Short Answer}
% & yes & no\tnote{1} & yes \\
% Matching (\href{https://docs.moodle.org/en/Matching_question_type}{std})
%& yes & no\tnote{2} & no\tnote{3} \\
% Matching (\href{https://docs.moodle.org/en/Drag_and_drop_matching_question_type}{dd})
% & yes & yes\tnote{4} & no\tnote{3} \\
% \href{https://docs.moodle.org/en/Essay_question_type}{Essay}
%& yes & yes\tnote{5,6} & yes\tnote{5} \\
% \href{https://docs.moodle.org/en/True/False_question_type}{True/False}
%& yes & no & yes \\
% \href{https://docs.moodle.org/en/Description_question_type}{Description}
%& yes & $\varnothing$ & yes \\
%\href{https://docs.moodle.org/en/Embedded_Answers_(Cloze)_question_type}{Cloze} &
%yes & $\varnothing$& $\varnothing$ \\\cmidrule(lr){1-1}
%\hspace{1em}Numerical & yes & no\tnote{1} & yes \\
%\hspace{1em}Short Answer & yes & no\tnote{1} & yes\tnote{7} \\
%\hspace{1em}Multi (inline) & yes & no\tnote{2} & yes\tnote{7} \\
%\hspace{1em}Multi (horizontal) & yes & yes & yes \\
%\hspace{1em}Multi (vertical) & yes & yes & yes \\\bottomrule
%\end{tabular}
%\begin{tablenotes}
%\item[1] \Moodle\ prompts the student for an answer and then compares it to the
%solutions provided. This is text-only.
%\item[2] \Moodle\ uses a dropdown list to let one choose among the possible
%answers. This forbids either picture inclusion and \LaTeX\ rendering.
%\item[3] Not supported by \Moodle\ (in this context, answer-specific feedback
%represents lots of possible combinations).
%\item[4] The drag-and-drop-matching plugin seems broken before version 1.6
%20190409. \Moodle's \XML\ import fails with a \texttt{dmlwriteexception} when
%the field content exceeds few hundreds characters. This prevents the inclusion
%of most base64 images and maybe some complicated equations.
%\item[5] For this question type and in the context of \XML\ generation, the
%Answer column represents the ``template" while the Feedback column represents
%the ``notes for the grader". Obviously, the grading process is not automatic
%and there is no answer-specific feedback.
%\item[6] Picture and \LaTeX\ rendering could be done, but only after
%submission and only if the keyval ``response format" is set to ``html".
%\item[7] \Moodle\ only reveals the feedback when hovering the checkmark or X
%mark with the mouse.
%\end{tablenotes}
%\end{threeparttable}
%\end{table}
%
% Some features of \Moodle\ quizzes have not yet been implemented in \pkg{moodle}.
% Here is a non-exhaustive list.
% \begin{itemize}
%   \item \Moodle's feature of designating feedback for correct,
%         partially correct, and incorrect answers.
%^^A   \item Calculated questions; that is, automatically generated \env{numerical} questions
%^^A         using randomly chosen numbers.
%   \item Hints
%   \item Units handling in \env{numerical} questions
% \end{itemize}
%
% \subsection{Compatibility}
% This package has been originally written for and tested with the implementation of
% \Moodle\ 2.9 run by Moodlerooms for St.~Norbert College in January 2016.
% Since then, it has been successfully combined with \Moodle\ 3.5.
%
% The package option \optn{LMS} lets you specify the targeted \Moodle\ version
% and helps ensuring compatibility.
%
% As the ultimate purpose of this package is the generation of \XML\ files,
% future versions of \pkg{moodle} will attempt to maintain backwards
% compatibility with earlier versions regarding the \XML\ output, apart from
% bug fixes.
% Backwards compatibility of the \PDF\ output is not yet guaranteed, however,
% in case the author or users discover better ways for the \PDF\ to display
% the underlying \XML\ data to be proofread.
%
% In other words, compiling your current \filenm{.tex} file with a future version
% of \pkg{moodle} should produce the same \XML\ file it does now
% (apart from bug fixes),
% but it might produce a more informative, and hence different,
% \PDF\ output.
%
% \subsection{Unrelated Tip: Quality of \Moodle\ \TeX\ Images}
% This has nothing to do with \pkg{moodle}, but is a Frequently Asked Question
% in is own right.
% On some servers, at least, \Moodle's default ``\TeX\ Filter'' for
% displaying mathematical notation is of abysmally poor quality, rending mathematics
% as low-resolution \PNG's.  One solution that has worked for me is to go
% to ``Course Administration $\to$ Filters,'' turn ``\TeX\ Notation'' \emph{off},
% but turn ``MathJax'' \emph{on}.  This forces \TeX\ code to be rendered by MathJax
% instead of \Moodle, producing much higher-quality results.
%
% ^^A\clearpage
% \changes{v0.5}{2016/01/05}{Initial version}
% \changes{v0.6}{2019/02/18}{Bug-fixing release}
% \changes{v0.7}{2020/07/09}{Feature extensions}
% \changes{v0.8}{2021/01/04}{Bug fixes and feature extensions}
% \changes{v0.9}{2021/02/07}{Bug fixes and feature extensions}
% \begin{changelog}[title={Version History},sectioncmd=\section*,author={Matthieu Guerquin-Kern}]
%   \begin{version}[version=\fileversion,]^^Adate=]
%     \added
%       \item Support for the \env{quote} and \env{quotation} environments in the HTML  
%             conversion process (courtesy of Gerald Teschl).
%       \item Experimental package option \optn{pluginfile} mimics the way
%             \Moodle\ embeds pictures in the \XML.
%       \item Answer box for Essay questions with \optn{handout}.
%       \item Key-value package option \optn{LMS} to target a specific version
%             of \Moodle.
%       \item Support for more text mode diacritic and horizontal spacing
%             commands.
%       \item Macros \cmd{\moodleversion} and \cmd{\moodledate} are defined.
%       \item Experimental package option \optn{samepage} to avoid splitting
%             questions across multiple pages.
%       \item \optn{sanction} key to set default mark for incorrect answers.
%       \item Started the internationalization of the package based on
%             \pkg{translator} (optionally \pkg{translations}). English,
%             French, German, Italian, and Spanish
%             are fully supported (with help from J\"urgen Vollmer for German
%             and Romano Giannetti for Italian and Spanish). Contributions are welcome.
%       \item Warn user of the \pkg{babel} package set for Turkish
%             that using the shorthands will not play well with \pkg{moodle}.
%       \item Support for \cmd{\i} in conversion to \HTML.
%       \item \SVG\ option support for Windows (courtesy of Wolf M\"uller).
%     \changed
%       \item Single straight quotes (apostrophe) in math mode are passed as-is
%             to the \XML\ (with help from Keno Wehr).
%       \item Empty groups |{}| in math-mode are passed as-is to the \XML\
%             (with help from Keno Wehr).
%       \item Inside \env{cloze} subquestions, \optn{points} are forced to the
%             nearest positive integer. ^^AA warning is issued when the user setting
%             ^^Ais modified.
%       \item \XML\ stamp exposes compilation time.
%       \item Labels of proposed answers now following \Moodle's convention
%             (\emph{1., 2., \dots}) in the \PDF\ typesetting of \env{multi} questions.
%       \item Automatic sanction mechanism for incorrect choices in \env{multi}
%             questions with multiple answers allowed. Now applies also inside
%             \env{cloze} questions.
%       \item In \env{cloze} subquestions, non-integer fractions are rounded.
%       \item Rewording of some indications in the \PDF, related to internationalization.
%       \item The answers of \env{matching} questions are now converted to \HTML\
%              (accents and \cmd{\htmlonly} in mind).
%     \removed
%       \item Support for non-\acro{ASCII} characters abandoned when compiling
%             with \prog{pdflatex}.
%     \fixed
%       \item Symbols |<| and |>| are translated to \HTML\ equivalents, also in
%             text-mode.
%       \item Warning german-writing and (Xe$\mid$Lua)\TeX\ users about troubles
%             caused by text-mode umlauts.
%       \item In \env{cloze} subquestions items and outside math environments,
%             the equal symbol (|=|) no longer interferes with the Cloze syntax.
%       \item A pagebreak occuring inside cloze shortanswers would reset page numbers.
%       \item The redefinition of the \env{description} environment is limited to
%             the scope of the \env{quiz}.
%       \item PDF typesetting of matching answers (line breaks, repeated matches).
%       \item Symbols \cmd{\%} found inside math mode are escaped for \HTML.
%       \item The \optn{fraction} keys specified inside \env{cloze} questions are
%             forced to integer values as required by \Moodle's \XML\ format.
%       \item True/False no longer broken when \pkg{hyperref} is loaded.
%       \item Tolerances in \env{numerical} answers now correctly displayed in \PDF.
%       \item Warning for users running a too old version of \pkg{graphics}.
%       \item In the code included with \pkg{verbatim}, \pkg{fancyvrb}, or
%             \pkg{fvextra}, characters that matter for \HTML\ are escaped.
%   \end{version}
%   \begin{version}[version=0.9,date=2021-02-07]
%     \added
%       \item Support for \href{https://docs.moodle.org/en/All_or_nothing_multiple_choice_question_type}
%             {all-or-nothing multiple choice} questions.
%       \item Support for the ogonek diacritical mark via \cmd{\k}\marg{\dots}.
%       \item Warn user of the \pkg{babel} package set for a German-related language
%             that using the character |"| will not play well with \pkg{moodle}.
%       \item Support for \pkg{babel} commands related to German quotes.
%       \item Support for en-dash (--) outside of math mode.
%       \item Support for \cmd{\%} in conversion to \HTML.
%       \item Command |\htmlonly[]{}| to pass \HTML\ contents directly to the \XML\ file.
%     \changed
%       \item An error is issued when a graphics conversion step fails.
%       \item Irrelevant points are no longer written at the \env{cloze} question level in \PDF\ and \XML.
%       \item The total number of marks is shown in the \PDF\ at the end of each quiz.
%       \item The \optn{tags} key can now be used to specify a comma-separated list of tags.
%     \fixed
%       \item Answer text of \env{shortanswer} questions is converted to \HTML.
%       \item Paragraph breaks in \env{multi} and \env{essay} items no longer break compilation.
%       \item Question text in Essays was not shown in \PDF\ file.
%       \item Commands \cmd{\textsc}, \cmd{\underline}, \cmd{\url}, and \cmd{\href} yielded \HTML\ code with
%             inadequate double quotation marks.
%       \item Broken base64-encoding pipeline for images under Windows (thanks to Andreas Vohns).
%       \item Repeated single right quotation marks no longer merged in math mode
%             (thanks to Alberto Albano).
%   \end{version}
%   \begin{version}[version=0.8,date=2021-01-04]
%     \added
%       \item Support for inclusion of \GIF\ pictures.
%       \item Added package option \optn{svg} to avoid the rasterization of vector graphics.
%       \item New macro \cmd{\setsubcategory} to define subcategories, reflected in \PDF\ and \XML.
%       \item Package option \optn{handout} for sharing \PDF\ with students.
%       \item Extensions can be omitted when including pictures.
%       \item Description question type.
%       \item \LuaLaTeX\ is now supported (and recommended for \UTF-8 coded sources).
%       \item Examples of ways to reproduce the behavior of calculated questions.
%       \item Command to trigger the automatic recording of new commands.
%       \item Mechanism to match \optn{fraction} key to values accepted by \Moodle.
%       \item A \optn{fractiontol} key to control the tolerance in this mechanism.
%       \item Support for inverted punctuation marks |¿| and |¡|.
%       \item Support for \cmd{\_} and \cmd{\textbackslash}.
%       \item Support for the wildcard character as an answer in \env{numerical} questions.
%     \changed
%       \item Template of Essay questions is now shown in \PDF.
%       \item The macro \cmd{\setcategory} is reflected by a new section in \PDF.
%       \item In \env{matching} questions, warnings are raised if the number of items is insufficient.
%       \item Improved display of \env{matching} questions in \PDF.
%       \item The package \pkg{iftex} is now required.
%       \item An error is thrown when \optn{fraction} is set to an invalid value.
%       \item In \env{numerical} questions, the tolerance can be set in exponent form.
%       \item Nicer \PDF\ rendering of numbers in \env{numerical} questions if \pkg{siunitx} is loaded.
%       \item Included \PNG\ and \acro{JPEG} files are now directly converted to base64.
%     \fixed
%       \item TeX's inline math (\texttt{$...$}) can now contain escaped dollar signs (\cmd{\$}).
%       \item Closing braces escaped in \env{cloze} subquestions outside math environment.
%       \item The scope of the \optn{tolerance} key is now respected.
%   \end{version}
%   \begin{version}[version=0.7,date=2020-09-06]
%     \added
%       \item Support for inclusion of verbatim code.
%       \item Package option \optn{tikz}.
%       \item Support for \cmd{\"Y} and \cmd{\"y}.
%       \item New commands converted to \XML.
%       \item Adding a stamp comment in \XML, package option offered to disable
%             this behavior.
%       \item Support for the \cmd{\tikz} command.
%       \item A different directory can be specified for picture inclusion.
%       \item Warn user of the \pkg{babel} package set for French that autospacing
%             must be deactivated.
%       \item Square bracket math delimiters are recognized and converted properly.
%       \item Support of breve and caron diacritical marks.
%     \changed
%       \item In \env{multi} with multiple answers allowed, choosing all options no longer
%             results in a good grade. An automatic penalty mechanism is introduced.
%             Can be overridden by manually setting fractions.
%     \removed
%       \item Irrelevant \optn{penalty} tag in \env{cloze} subquestions.
%     \fixed
%       \item Non-integer fractions can now be specified in \env{cloze} subquestions.
%       \item Significantly squeeze \PNG\ images size by skipping ancillary data.
%       \item Enumerate or itemize environment can now be nested in question items.
%       \item Several pictures can be included in a question without being mixed
%             in the \XML\ file.
%       \item Management and rendering of fraction in questions.
%       \item Correctly handling a \LaTeX\ command starting the last item of a question.
%       \item Closing braces escaped in \env{cloze} subquestions. This allows \LaTeX\
%             equations or images to be included.
%       \item Image inclusion with macOS.
%   \end{version}
%   \begin{version}[version=0.6b,date=2019-11-27]
%     \added
%       \item New package options to set section or subsection at the quiz level.
%       \item True/False question type is now supported.
%       \item \Moodle\ tags can now be specified for questions (and rendered in \PDF\
%             as well).
%       \item In \env{cloze} questions, the multiresponse subquestion type is
%             now supported.
%     \removed
%       \item External dependency on \prog{OpenSSL}.
%       \item Irrelevant tags were written in \XML\ for \env{matching} questions.
%     \fixed
%       \item \TikZ\ externalization now works when using \XeLaTeX.
%       \item It is now possible to set points manually among several correct
%             answers in multichoice questions.
%       \item General feedback can now contain backslashes.
%       \item Several quizzes can now be defined in a single source file, each
%             specifying a category for \Moodle's question bank.
%       \item Correct encoding information is now written in \XML\ depending on
%             the \LaTeX\ compiler used.
%   \end{version}
%   \begin{version}[version=0.6a,date=2019-06-21]
%     \added
%       \item \XeLaTeX\ is now recommended when using \UTF-8 encoded sources (support
%             of accents).
%       \item Feedbacks are now displayed in the \PDF\ file produced.
%       \item The \prog{OptiPNG} utility is used to reduce the size
%             of images embedded in the \XML\ file.
%       \item Question options and settings are now displayed in the \PDF\ file
%       \item Supporting more \LaTeX\ macros for symbols and accents (mostly
%             diacritical marks and ligatures).
%       \item Introduce shuffle options in cloze-multi subquestions.
%       \item Package option \optn{final}.
%     \changed
%       \item In draft mode, \TikZ\ externalization is no longer triggered.
%     \fixed
%       \item In the different question types, the feedback fields are now converted
%             for \HTML\ allowing \LaTeX\ equations and images.
%       \item Documentation improvements (limitations and previously undocumented
%             features).
%   \end{version}
%   \shortversion{version=0.5,date=2016-01-05,simple,changes=Initial version,author={Anders O.F. Hendrickson}}
% \end{changelog}
% \addcontentsline{toc}{section}{Version History}%
%
% \StopEventually{%
%^^A \PrintChanges
%    \PrintIndex
%    \addcontentsline{toc}{section}{Index}%
% }
%
% \section{Implementation}
% \subsection{Packages, Options, and Utilities}
%    \begin{macrocode}
\newif\ifmoodle@draftmode
\newif\ifmoodle@handout
\newif\ifmoodle@stampmode
\newif\ifmoodle@tikz
\newif\ifmoodle@tikzloaded
\newif\ifmoodle@svg
\newif\ifmoodle@section
\newif\ifmoodle@subsection
\newif\ifmoodle@numbered
\newif\ifmoodle@international
\newif\ifmoodle@feedbackLeft% contribution of Juergen Vollmer, 2021-03-05
\newif\ifmoodle@samepage% contribution of Romano Giannetti, 2021-03-07
\newif\ifmoodle@pluginfile%

%% Moodle version
\RequirePackage{xkeyval}%For key-handling
\newcount\moodle@LMSmajor% major version number of the LMS targeted
\newcount\moodle@LMSminor% minor version number of the LMS targeted

%%DECLARATION OF OPTIONS
\DeclareOptionX{draft}{\moodle@draftmodetrue}%
\DeclareOptionX{handout}{\moodle@handouttrue}%
\DeclareOptionX{final}{\moodle@draftmodefalse}%
\DeclareOptionX{nostamp}{\moodle@stampmodefalse}%
\DeclareOptionX{tikz}{\moodle@tikztrue}%
\DeclareOptionX{svg}{\moodle@svgtrue}%
\DeclareOptionX{section}{\moodle@sectiontrue\moodle@numberedtrue}%
\DeclareOptionX{section*}{\moodle@sectiontrue\moodle@numberedfalse}%
\DeclareOptionX{subsection}{\moodle@sectionfalse\moodle@numberedtrue}%
\DeclareOptionX{subsection*}{\moodle@sectionfalse\moodle@numberedfalse}%
\DeclareOptionX{feedbackleft}{\moodle@feedbackLefttrue}% contribution of Juergen Vollmer, 2021-03-05
\DeclareOptionX{feedbackright}{\moodle@feedbackLeftfalse}% default
\DeclareOptionX{samepage}{\moodle@samepagetrue}% contribution of Romano Giannetti, 2021-03-07
\DeclareOptionX{LMS}[0.0]{%
  \filename@parse{#1}%
  \moodle@LMSmajor=\filename@base
  \moodle@LMSminor=\filename@ext
}%
\DeclareOptionX{pluginfile}{\moodle@pluginfiletrue}%

\moodle@draftmodefalse
\moodle@handoutfalse
\moodle@stampmodetrue
\moodle@tikzfalse
\moodle@tikzloadedfalse
\moodle@svgfalse
\moodle@subsectiontrue
\moodle@numberedfalse
\moodle@feedbackLeftfalse% contribution of Juergen Vollmer, 2021-03-05
\moodle@samepagefalse% contribution of Romano Giannetti, 2021-03-07
\moodle@pluginfilefalse

\ProcessOptionsX*% the star allows to inherit 'draft' and 'final' from the class

\ifmoodle@samepage
    \def\moodle@begin@samepage{\minipage[t]{\linewidth}}
    \def\moodle@end@samepage{\endminipage\vskip 0pt plus 50pt\relax}
\else
    % defining to relax is safer versus adding spurious spaces
    \def\moodle@begin@samepage{\relax}
    \def\moodle@end@samepage{\relax}
\fi

\RequirePackage{environ} %To be able to take environment body as a macro argument
\RequirePackage{amssymb} %For \checkmark symbol
%\RequirePackage{trimspaces} %To remove extra spaces from strings (loaded by environ)
\@ifpackageloaded{iftex}{}{\RequirePackage{iftex}}
% iftex already required by recent versions of ifplatform. Needed to know:
%     1) whether we can convert output from PDF to \PNG\ (ifpdf),
%     2) when output pdf is latin1-encoded (ifpdf)
%     3) when output xml is utf8-encoded (if?tex)
\@ifundefined{ifeTeX}{% version of 'iftex' prior to November 2019.
  \RequirePackage{etex}{}
}{
  \ifeTeX\else
    \RequirePackage{etex}{} % Recent (2015+) TeX engines should be e-TeX.
    %This is needed for expansion control, detokenization, etc.
  \fi
}
\RequirePackage{etoolbox}%List management
\RequirePackage{xpatch}  %To patch commands easily in \HTML\ mode
\RequirePackage{array}   %For formatting tables in the LaTeX mode of Clozes
\RequirePackage{ifplatform} % To choose Ghostscript commands
\@ifundefined{ifpdf}{\RequirePackage{ifpdf}}{}% old iftex would not define the conditional
\RequirePackage{shellesc} %Luatex-compatible way of getting system access
\RequirePackage{readprov} %To gather information on the package (version, date, ...)
\RequirePackage{fancybox} %For fancy LaTeX tags
\RequirePackage{getitems} %To gather the header and items
\ifmoodle@handout
  \RequirePackage[seed=42]{randomlist} %To randomize answers in matching questions
\fi
\ifmoodle@svg
  \RequirePackage{graphicx} %To include graphics
\fi

\GetFileInfo{moodle.sty}%
\edef\moodledate{\filedate}%
\edef\moodleversion{\fileversion}%

\let\xa=\expandafter
\def\@star{*}%
\def\@hundred{100}%
\def\@fifty{50}%
\def\@moodle@empty{}%
\def\@relax{\relax}%
\def\@moodle@par{\par}%

% Taken from https://tex.stackexchange.com/questions/47576/combining-ifxetex-and-ifluatex-with-the-logical-or-operation
\@ifundefined{ifTUTeX}{
  \newif\ifTUTeX % a new conditional starts as false
  \ifnum 0\ifXeTeX 1\fi\ifLuaTeX 1\fi>0\relax
    \TUTeXtrue
  \fi
}{}%
\newif\ifpdfoutput % a new conditional starts as false
\ifnum 0\ifTUTeX 1\fi\ifPDFTeX\ifpdf 1\fi\fi>0\relax
   \pdfoutputtrue
\fi
%    \end{macrocode}
% As the package involves a fair bit of file processing,
% we automate the naming of auxiliary files.
%    \begin{macrocode}
\def\jobnamewithsuffixtomacro#1#2{%
  \filenamewithsuffixtomacro{#1}{\jobname}{#2}%
}
\def\@jn@quote{"}%
\def\filenamewithsuffixtomacro#1#2#3{%
  % #1 = macro to create
  % #2 = filename to add suffix to
  % #3 = suffix to add
  \edef\jn@suffix{#3}%
  \def\jn@macro{#1}%
  \xa\testforquote#2\@jn@rdelim
}
\def\testforquote#1#2\@jn@rdelim{%
  \def\jn@test@i{#1}%
  \ifx\jn@test@i\@jn@quote
    % Involves quotes
    \edef\jn@next{"\jn@stripquotes#1#2\jn@suffix"}%
  \else
    \edef\jn@next{#1#2\jn@suffix}%
  \fi
  \xa\xdef\jn@macro{\jn@next}%
}
\def\jn@stripquotes"#1"{#1}%


\jobnamewithsuffixtomacro{\outputfilename}{-moodle.xml}
%    \end{macrocode}
% Next, we create macros to open and close the Moodle \XML\ file
% we will be writing.
%    \begin{macrocode}
\newwrite\moodle@outfile
\def\openmoodleout{%
  \immediate\openout\moodle@outfile=\outputfilename\relax
  \ifPDFTeX % latin1-based engines (pdflatex or latex)
    \writetomoodle{<?xml version="1.0" encoding="iso-8859-1"?>}%
  \else
    \ifTUTeX % UTF8-based engines (XeTeX or LuaTeX)
      \writetomoodle{<?xml version="1.0" encoding="UTF-8"?>}%
    \else % what shall we do?
      \writetomoodle{<?xml version="1.0" encoding="UTF-8"?>}%
      %\stop
    \fi
  \fi
  \ifmoodle@stampmode
    \def\moodle@stamp{This is a question bank made for Moodle }
    \ifnum\the\moodle@LMSmajor\the\moodle@LMSminor=\z@\else
      \g@addto@macro{\moodle@stamp}{\the\moodle@LMSmajor.\the\moodle@LMSminor.\space}
    \fi
    \writetomoodle{<!-- \moodle@stamp -->}%
    \newcount\hour\hour=\time
    \divide\hour by 60\relax
    \newcount\minute\minute=\hour
    \multiply\minute by -60\relax
    \advance\minute by \time\relax
    \def\moodle@stamp{It was generated on \the\year-\two@digits\month-\two@digits\day\space \the\hour:\the\minute}
    \ifPDFTeX % pdflatex or latex
      \ifpdf % pdflatex
        \g@addto@macro{\moodle@stamp}{ by pdfLaTeX }%
      \else % latex
        \g@addto@macro{\moodle@stamp}{ by LaTeX }%
      \fi
    \else
      \ifXeTeX % xetex
        \g@addto@macro{\moodle@stamp}{ by XeLaTeX }%
      \else
        \ifLuaTeX % luatex
          \g@addto@macro{\moodle@stamp}{ by LuaLaTeX }%
        \else
          \g@addto@macro{\moodle@stamp}{ a TeX engine }%
        \fi
      \fi
    \fi
    \writetomoodle{<!-- \moodle@stamp running -->}%
    \def\moodle@stamp{on \platformname}%
    \g@addto@macro{\moodle@stamp}{ with the package moodle \moodleversion\space (\moodledate) }%
    \writetomoodle{<!-- \moodle@stamp -->}%
  \fi
  \immediate\write\moodle@outfile{}%
  \writetomoodle{<quiz>}%
}%
\def\closemoodleout{%
  \writetomoodle{ }%
  \writetomoodle{</quiz>}%
  \immediate\closeout\moodle@outfile
}%
%    \end{macrocode}
%
% To both make this \filenm{.sty} file and the \XML\ output more readable,
% we create a mechanism for writing to the output file with indents.
% The macro |\calculateindent|\marg{$n$} globally defines
% |\moodle@indent| to be a string of \meta{$n$} |\otherspace|'s.
%    \begin{macrocode}
\newcount\moodle@indentnum
\def\calculateindent#1{%
  \bgroup
    \moodle@indentnum=\number#1\relax
    \gdef\moodle@indent{}%
    \calculateindent@int
  \egroup
}%
\def\calculateindent@int{%
  \ifnum\moodle@indentnum>0\relax
    \g@addto@macro{\moodle@indent}{\otherspace}%
    \advance\moodle@indentnum by -1\relax
    \expandafter
    \calculateindent@int
  \fi
}%
%    \end{macrocode}
% Now the command |\writetomoodle|\oarg{n}\marg{stuff} adds the line
% ``\meta{stuff}'' to the \XML\ file
% preceded by an indent of \meta{n} spaces.
% Note that this command |\edef|'s its second argument.
%    \begin{macrocode}
\newcommand\writetomoodle[2][0]{%
  \edef\test@ii{#2}%
  \ifnum#1=0\relax
    \immediate\write\moodle@outfile{\test@ii}%
  \else
    \calculateindent{#1}%
    \immediate\write\moodle@outfile{\moodle@indent\trim@pre@space{\test@ii}}%
  \fi
}%
%    \end{macrocode}
% We now create the main |quiz| environment that will contain
% the questions we write.  It outputs to \XML\ as a |<question type="category">| tag.
%    \begin{macrocode}
{\catcode`\$=12\catcode`\ =12% in this context we cannot indent with spaces...
\gdef\moodle@write@category@xml#1{%
\@moodle@ifgeneratexml{%
\writetomoodle{ }%
\writetomoodle{<question type="category">}%
\writetomoodle{  <category>}%
\writetomoodle{    <text>$course$/top/#1</text>}%
\writetomoodle{  </category>}%
\writetomoodle{</question>}%
\writetomoodle{ }%
}{}%
}}%
\newcommand*\@enumeratename{enumerate}%
\newdimen\moodle@totalmarks
\gdef\moodle@currentcategory{}%
\newenvironment{quiz}[2][]{%
  \setkeys{moodle}{#1}%
  \gdef\setcategory##1{%
    % At first call (end of \begin{quiz}) enumerate is not started yet
    \ifx\@currenvir\@enumeratename
      % In case no question is defined between two calls of \setcategory
      \def\@noitemerr{}%\@latex@warning{Empty question list}
      \end{enumerate}%
    \fi
    \gdef\moodle@currentcategory{##1}%
    \moodle@write@category@xml{##1}%
    \ifmoodle@section
      \ifmoodle@numbered
        \section{##1}%
      \else
        \section*{##1}%
      \fi
    \else
      \ifmoodle@numbered
        \subsection{##1}%
      \else
        \subsection*{##1}%
      \fi
    \fi
    \begin{enumerate}\renewcommand\labelenumi{(\theenumi)}%
  }%
  \gdef\setsubcategory##1{%
    \def\@noitemerr{}%\@latex@warning{Empty question list}
    \end{enumerate}%
    \moodle@write@category@xml{\moodle@currentcategory/##1}%
    \ifmoodle@section
      \ifmoodle@numbered
        \subsection{##1}%
      \else
        \subsection*{##1}%
      \fi
    \else
      \ifmoodle@numbered
        \subsubsection{##1}%
      \else
        \subsubsection*{##1}%
      \fi
    \fi
    \begin{enumerate}\renewcommand\labelenumi{(\theenumi)}%
  }%
  \setcategory{#2}%
   \moodle@totalmarks=0pt%
}{%
  \end{enumerate}%
  \emph{\GetTranslation{Total of marks}: \strip@pt\moodle@totalmarks}%
  \let\setcategory\relax
  \let\setsubcategory\relax
}%

%    \end{macrocode}
% The next utility takes a single macro control sequence |#1|,
% and allows that macro's current value to persist
% after the next |\egroup|, |}|, or |\endgroup|.
%    \begin{macrocode}
\def\passvalueaftergroup#1{%
  \xa\xa\xa\gdef\xa\xa\csname moodle@remember@\string#1\endcsname\xa{\xa\def\xa#1\xa{#1}}%
  \xa\aftergroup\csname moodle@remember@\string#1\endcsname
}
%    \end{macrocode}
%
% \subsubsection{Main Switch: to create \XML\ or not}
%
%    \begin{macrocode}
\long\def\@moodle@ifgeneratexml#1#2{%
  % If we are generating \XML, do #1; otherwise do #2.
  \tikzifexternalizing{%
    % This run of LaTeX is currently ONLY generating a TikZ image
    % to be saved in an external file.  We do NOT want to waste time
    % generating \XML, and moreover trying to do so would cause errors
    % because of file dependencies.
    #2%
  }{%
    \ifmoodle@draftmode
      #2%
    \else
      #1%
    \fi
  }%
}
%    \end{macrocode}
% Now the macros |openmoodleout| and |closemoodleout| are triggered at Begin and End Document, respectively
%    \begin{macrocode}
\AfterEndPreamble{
  \@moodle@ifgeneratexml{%
    \openmoodleout%
  }{}%
}
\AtEndDocument{
  \@moodle@ifgeneratexml{%
    \closemoodleout%
  }{}%
}
%    \end{macrocode}
%
% \subsection{Key-Value Pairs for Quiz Questions}
%
% The various options are set using key-value syntax of |xkeyval|.
%    \begin{macrocode}
\def\moodleset#1{\setkeys{moodle}{#1}}%
%    \end{macrocode}
% We first define some macros that will help us write other macros.
% Calling |\generate@moodle@write@code|\marg{name}|<|\meta{HTML tag}|>|\marg{text to write}
% creates a macro |\moodle@write|\meta{name}, taking no parameters,
% which writes the code |<|\meta{HTML tag}|>...</|\meta{HTML tag}|>|
% to the output \XML\ file.
%
% The ordinary version |\generate@moodle@write@code| passes its output text |#3|
% through the HTMLizer, producing \HTML\ code, while the starred variant
% |\generate@moodle@write@code*| passes |#3| verbatim as text.
%
% For example,
% |\generate@moodle@write@code{excuse}<EXC>{\theexcuse}|
% would expand to
% \begin{Verbatim}[gobble=4,frame=single]
%   \gdef\moodle@writeexcuse{%
%     \xa\def\xa\test@iii\xa{\theexcuse}%
%     \ifx\test@iii\@moodle@empty
%       \writetomoodle[2]{  <EXC format="html"><text/></EXC>}%
%     \else
%       \xa\converttohtmlmacro\xa\moodle@htmltowrite\xa{\theexcuse}%
%       \writetomoodle[2]{  <EXC format="html">}%
%       \writetomoodle[4]{    <text><![CDATA[<p>\moodle@htmltowrite</p>]]></text>}%
%       \writetomoodle[2]{  </EXC>}%
%     \fi
%   }%
% \end{Verbatim}
% but
% |\generate@moodle@write@code*{excuse}<EXC>{\theexcuse}|
% would expand only to
% \begin{Verbatim}[gobble=4,frame=single]
%   \gdef\moodle@writeexcuse{%
%      \writetomoodle[2]{  <EXC>\theexcuse</EXC>}%
%   }
% \end{Verbatim}
%    \begin{macrocode}
\def\generate@moodle@write@code{%
  \@ifnextchar*\generate@moodle@write@data\generate@moodle@write@html
}%

\def\generate@moodle@write@html#1<#2>#3{%
  % #1 = NAME for \moodle@writeNAME
  % #2 = \HTML\ tag
  % #3 = what, exactly, to write
  \xa\gdef\csname moodle@write#1\endcsname{%
    \xa\def\xa\test@iii\xa{#3}%
    \ifx\test@iii\@moodle@empty
      \writetomoodle[2]{  <#2 format="html"><text/></#2>}%
    \else
      \xa\converttohtmlmacro\xa\moodle@htmltowrite\xa{#3}%
      \writetomoodle[2]{  <#2 format="html">}%
      \writetomoodle[4]{    <text><![CDATA[<p>\moodle@htmltowrite</p>]]></text>}%
      \ifmoodle@pluginfile
        \writetomoodle[0]{\htmlize@embeddedfiletags}%
      \fi
      \writetomoodle[2]{  </#2>}%
    \fi
  }%
}%

\def\generate@moodle@write@html@noptag#1<#2>#3{%
  % No <P>..</P> introduced
  % #1 = NAME for \moodle@writeNAME
  % #2 = \HTML\ tag
  % #3 = what, exactly, to write
  \xa\gdef\csname moodle@write#1\endcsname{%
    \xa\def\xa\test@iii\xa{#3}%
    \ifx\test@iii\@moodle@empty
      \writetomoodle[2]{  <#2 format="html"><text/></#2>}%
    \else
      \xa\converttohtmlmacro\xa\moodle@htmltowrite\xa{#3}%
      \writetomoodle[2]{  <#2 format="html">}%
      \writetomoodle[4]{    <text><![CDATA[\moodle@htmltowrite]]></text>}%
      \ifmoodle@pluginfile
        \writetomoodle[0]{\htmlize@embeddedfiletags}%
      \fi
      \writetomoodle[2]{  </#2>}%
    \fi
  }%
}%

\def\generate@moodle@write@data*#1<#2>#3{%
  % #1 = NAME for \moodle@writeNAME
  % #2 = \HTML\ tag
  % #3 = what, exactly, to write
  \xa\gdef\csname moodle@write#1\endcsname{%
    \writetomoodle[2]{  <#2>#3</#2>}%
  }%
}%

\def\moodle@writetags{%
  \xa\xdef\xa\test@iii\xa{\moodle@tags}%
  \ifx\test@iii\@moodle@empty\relax\else
    \writetomoodle[2]{  <tags>}%
    \renewcommand*{\do}[1]{%
      \def\moodle@tagtext{##1}%
      \xa\converttohtmlmacro\xa\moodle@htmltowrite\xa{\moodle@tagtext}%
      \writetomoodle[4]{    <tag><text><![CDATA[\moodle@htmltowrite]]></text></tag>}%
    }
    \xa\docsvlist\xa{\moodle@tags}%
    \writetomoodle[2]{  </tags>}%
  \fi
}%
\newif\ifmoodle@firsttag
\moodle@firsttagtrue
\def\moodle@latex@writetags{%
  \xa\xdef\xa\test@iii\xa{\moodle@tags}%
  \ifx\test@iii\@moodle@empty\relax\else
    \hfill \GetTranslation{tags}: %
    \renewcommand*{\do}[1]{\ifmoodle@firsttag\moodle@firsttagfalse\else, \fi\texttt{##1}}%
    \xa\docsvlist\xa{\test@iii}%
  \fi
}%
%    \end{macrocode}
% \subsubsection{Keys for all question types}
%    \begin{macrocode}
%% QUESTIONNAME
      \define@cmdkey{moodle}[moodle@]{questionname}{}%
%      \gdef\moodle@writequestionname{%
%        \writetomoodle[2]{<name>}%
%        \writetomoodle[4]{  <text>\moodle@questionname</text>}%
%        \writetomoodle[2]{</name>}%
%      }%
%\generate@moodle@write@code{questionname}<name>{\moodle@questionname}%
\generate@moodle@write@html@noptag{questionname}<name>{\moodle@questionname}%

%% QUESTIONTEXT
      %I tried to use questiontext as a key, but it doesn't seem to work.
      %The trouble is that xkeyval has trouble parsing a key with a \par token followed by a comma within brackets,
      %like \setkeys{moodle}{questiontext={ABC\par [D,E]}}
      %It's not worth trying to fix.

      \long\def\questiontext#1{%
        %\converttohtmlmacro\myoutput{#1}%
        %\let\moodle@questiontext=\myoutput%
        \def\moodle@questiontext{#1}%
      }%
      \generate@moodle@write@code{questiontext}<questiontext>{\moodle@questiontext}%{%

%% PENALTY FOR WRONG ATTEMPT
      \define@cmdkey{moodle}[moodle@]{penalty}[0.10]{}%
      \generate@moodle@write@code*{penalty}<penalty>{\moodle@penalty}%
      \moodleset{penalty=0.10}%

%% FEEDBACK
      % Moodle allows for feedback tailored to each question,
      % and feedback tailored to each right or wrong answer.
      % We shall use the key 'feedback' to record both kinds of feedback,
      % relying on TeX's grouping mechanism to keep them apart.
      % When it comes time to write them to \XML,
      % \moodle@writegeneralfeedback uses the \HTML\ tag <generalfeedback>
      % whereas \moodle@writefeedback uses the tag <feedback>.
      % Note that the general feedback is NOT inherited by each answer!
      \define@cmdkey{moodle}[moodle@]{feedback}[]{}%
      \generate@moodle@write@code{generalfeedback}<generalfeedback>{\moodle@feedback}%
      \generate@moodle@write@code{feedback}<feedback>{\moodle@feedback}%
      \moodleset{feedback={}}%

%% DEFAULT GRADE
      %The next line creates \moodle@defaultgrade,
      %which is how many points the quiz question is worth.
      %Key calls like [default grade=7] set \moodle@defaultgrade.
      \define@cmdkey{moodle}[moodle@]{default grade}[1.0]{}%
      %Next, makes 'points' a synonym for 'default grade'
      \define@key{moodle}{points}[1.0]{\xa\def\csname moodle@default grade\endcsname{#1}}%
      \generate@moodle@write@code*{defaultgrade}<defaultgrade>{\csname moodle@default grade\endcsname}%
      \moodleset{default grade=1.0} %This sets the default.

%% HIDDEN
      \define@boolkey{moodle}[moodle@]{hidden}[true]{}%
      \generate@moodle@write@code*{hidden}<hidden>{\ifmoodle@hidden 1\else 0\fi}%
      \moodleset{hidden=false}%

\def\moodle@writecommondata{%
  \moodle@writequestionname%
  \moodle@writequestiontext%
  \moodle@writedefaultgrade%
  \moodle@writegeneralfeedback%
  \moodle@writepenalty%
  \moodle@writehidden%
}%

%% TAGS
      %The next line creates \moodle@tags,
      %which defines a "tag" (i.e., keyword) for the question.
      %Key calls like [tags={random}] set \moodle@tags.
      \define@cmdkey{moodle}[moodle@]{tags}[]{}%
      \moodleset{tags={}}%

%    \end{macrocode}
% \subsubsection{Keys for all answers}
%    \begin{macrocode}
%% FRACTION -- how much this answer is worth out of 100 percent
      \define@cmdkey{moodle}[moodle@]{fraction}[100]{}%
      %We do not create \moodle@writefraction, because the fraction occurs in
      %the \XML\ within the answer tag, like <answer fraction="75">.
      \moodleset{fraction=100} %This sets the default.
%    \end{macrocode}
%    \begin{macrocode}
%% FRACTIONTOL -- the tolerance for fractions with respect to valid values
      \define@cmdkey{moodle}[moodle@]{fractiontol}[0.1]{}%
      \moodleset{fractiontol=0.1} %This sets the default.
%    \end{macrocode}
% \subsubsection{Keys for multiple choice questions}
%    \begin{macrocode}

%% SINGLE and MULTIPLE -- for multichoice, is there 1 right answer or more than 1?
      \define@boolkey{moodle}[moodle@]{single}[true]{}%
      \generate@moodle@write@code*{single}<single>{\ifmoodle@single true\else false\fi}%
      \moodleset{single=true}%
      %The key 'multiple' is an antonym to 'single'.
      \define@boolkey{moodle}[moodle@]{multiple}[true]{\ifmoodle@multiple\moodle@singlefalse\else\moodle@singletrue\fi}%

%% SHUFFLE ANSWERS
      \define@boolkey{moodle}[moodle@]{shuffle}[true]{}%
      \generate@moodle@write@code*{shuffle}<shuffleanswers>{\ifmoodle@shuffle 1\else 0\fi}%
      \moodleset{shuffle=true}%

%% ALLORNOTHING -- for multichoice with multiple answers where all the points are given
%                  if and only if all the correct answers are selected.
      \define@boolkey{moodle}[moodle@]{allornothing}[true]{}%
      \moodleset{allornothing=false}%

%% SANCTION -- how much shall incorrect choices by sanctioned in multichoice questions (single)
      \define@cmdkey{moodle}[moodle@]{sanction}[]{}%
      \moodleset{sanction=0} %This sets the default.

%% TODO: CORRECTFEEDBACK
%% TODO: PARTIALLYCORRECTFEEDBACK
%% TODO: INCORRECTFEEDBACK
%% TODO: NUMCORRECT key

%% NUMBERING -- for numbering of multichoice questions
      \define@choicekey{moodle}{numbering}%
                       {alpha,alph,Alpha,Alph,arabic,roman,Roman,%
                        abc,ABCD,123,iii,IIII,none}[abc]{%
                        \def\moodle@numbering{#1}%
                        \def\test@@i{#1}%
                        \ifx\test@@i\@moodle@alpha
                          \def\moodle@numbering{abc}\fi
                        \ifx\test@@i\@moodle@alph
                          \def\moodle@numbering{abc}\fi
                        \ifx\test@@i\@moodle@Alpha
                          \def\moodle@numbering{ABCD}\fi
                        \ifx\test@@i\@moodle@Alph
                          \def\moodle@numbering{ABCD}\fi
                        \ifx\test@@i\@moodle@arabic
                          \def\moodle@numbering{123}\fi
                        \ifx\test@@i\@moodle@roman
                          \def\moodle@numbering{iii}\fi
                        \ifx\test@@i\@moodle@Roman
                          \def\moodle@numbering{IIII}\fi
                        }%
      %'answer numbering' will be a synonym to 'numbering'
      \define@key{moodle}{answer numbering}[abc]{\setkeys{moodle}{numbering={#1}}}%
      \generate@moodle@write@code*{answernumbering}<answernumbering>{\moodle@numbering}%
      %N.B. if we did not set the default here, then \moodle@numbering would be undefined, causing problems.
      \moodleset{answer numbering=abc}%

      \def\@moodle@alpha{alpha}%
      \def\@moodle@Alpha{Alpha}%
      \def\@moodle@alph{alph}%
      \def\@moodle@Alph{Alph}%
      \def\@moodle@arabic{arabic}%
      \def\@moodle@roman{roman}%
      \def\@moodle@Roman{Roman}%
      \def\@moodle@abc{abc}%
      \def\@moodle@ABCD{ABCD}%
      \def\@moodle@arabicnumbers{123}%
      \def\@moodle@iii{iii}%
      \def\@moodle@IIII{IIII}%
      \def\@moodle@none{none}%
      \def\moodle@obeynumberingstyle{%
        \renewcommand\labelenumii{\theenumii.}% follow Moodle's labeling convention
        \ifx\moodle@numbering\@moodle@abc
          \renewcommand\theenumii{\alph{enumii}}%
        \fi
        \ifx\moodle@numbering\@moodle@ABCD
          \renewcommand\theenumii{\Alph{enumii}}%
        \fi
        \ifx\moodle@numbering\@moodle@arabicnumbers
          \renewcommand\theenumii{\arabic{enumii}}%
        \fi
        \ifx\moodle@numbering\@moodle@iii
          \renewcommand\theenumii{\roman{enumii}}%
        \fi
        \ifx\moodle@numbering\@moodle@IIII
          \renewcommand\theenumii{\Roman{enumii}}%
        \fi
        \ifx\moodle@numbering\@moodle@none
          \renewcommand\labelenumii{$\bullet$~}%
        \fi
      }
      %TODO: * In the PDF, how should 'none' in a multi look different from
      %         short answer or numerical options?
      %       * Instead of \theenumi and \labelenumi,
      %         use \@enumdepth to automatically set the correct depth.

%% DISPLAY MODE -- affects Cloze multiple choice questions only.
      % 0 = inline, 1 = vertical, 2 = horizontal
      \def\moodle@multi@mode{0}%
      \define@key{moodle}{inline}[]{\def\moodle@multi@mode{0}}%
      \define@key{moodle}{vertical}[]{\def\moodle@multi@mode{1}}%
      \define@key{moodle}{horizontal}[]{\def\moodle@multi@mode{2}}%
%    \end{macrocode}
% \subsubsection{Keys for numerical questions}
%    \begin{macrocode}
%% TOLERANCE
      \define@cmdkey{moodle}[moodle@]{tolerance}[0]{}%
      \moodleset{tolerance=0}%
      %There is no \moodle@writetolerance, because in the \XML\ the
      %tolerance is given within the answer tag,
      %like <answer fraction=100 tolerance=0.03>.

% TODO: implement unit-handling for numerical questions!
%    \end{macrocode}
% \subsubsection{Keys for short answer questions}
%    \begin{macrocode}
%% CASE SENSITIVE
      \define@boolkey{moodle}[moodle@]{usecase}[true]{}%
      \generate@moodle@write@code*{usecase}<usecase>{\ifmoodle@usecase 1\else 0\fi}%
      % We make 'case sensitive' a synonym for 'usecase'.
      \define@boolkey{moodle}[moodle@]{case sensitive}[true]{\ifcsname moodle@case sensitive\endcsname \moodle@usecasetrue\else\moodle@usecasefalse\fi}%
      \moodleset{usecase=false}%
%    \end{macrocode}
% \subsubsection{Keys for matching questions}
%    \begin{macrocode}
%% DRAG-AND-DROP FORMAT
      \define@boolkey{moodle}[moodle@]{draganddrop}[true]{}%
      % We make 'dd' and 'draganddrop' and 'drag and drop' synonyms for 'draganddrop'.
      \define@boolkey{moodle}[moodle@]{dd}[true]{\ifmoodle@dd\moodle@draganddroptrue\else\moodle@draganddropfalse\fi}%
      \define@boolkey{moodle}[moodle@]{drag and drop}[true]{\moodle@ddsynonym}%
      \def\moodle@ddsynonym{%
        \csname ifmoodle@drag and drop\endcsname
          \moodle@draganddroptrue
        \else
          \moodle@draganddropfalse
        \fi
      }
      \moodleset{draganddrop=false}%
%    \end{macrocode}
% \subsubsection{Keys for essay questions}
%    \begin{macrocode}
%% EDITOR
      \def\@moodle@html{html}%
      \def\@moodle@htmlfile{html+file}%
      \def\@moodle@text{text}%
      \def\@moodle@plain{plain}%
      \def\@moodle@monospaced{monospaced}%
      \def\@moodle@file{file}%
      \def\@moodle@noinline{noinline}%
      \define@choicekey{moodle}{response format}%
                       {html,html+file,text,monospaced,file}[html]%
                       {\def\test@i{#1}%
                        \ifx\test@i\@moodle@html
                          % \HTML\ Editor
                          \def\moodle@responseformat{editor}%
                        \fi
                        \ifx\test@i\@moodle@htmlfile
                          % \HTML\ Editor with File Picker
                          \def\moodle@responseformat{editorfilepicker}%
                        \fi
                        \ifx\test@i\@moodle@text
                          % Plain text
                          \def\moodle@responseformat{plain}%
                        \fi
                        \ifx\test@i\@moodle@plain
                          % Plain text
                          \def\moodle@responseformat{plain}%
                        \fi
                        \ifx\test@i\@moodle@monospaced
                          % Plain text, monospaced font
                          \def\moodle@responseformat{monospaced}%
                        \fi
                        \ifx\test@i\@moodle@file
                          % No inline text (i.e., attachments only)
                          \def\moodle@responseformat{noinline}%
                        \fi
                        \ifx\test@i\@moodle@noinline
                          % No inline text (i.e., attachments only)
                          \def\moodle@responseformat{noinline}%
                        \fi
                       }%
      \generate@moodle@write@code*{responseformat}<responseformat>{\moodle@responseformat}%
      \moodleset{response format=html}%
      %N.B. if we did not set a default, then \moodle@responseformat would be undefined, causing problems.

%% RESPONSE REQUIRED
      \define@boolkey{moodle}[moodle@]{response required}[true]{}%
      % TODO: Make synonym 'required'
      \generate@moodle@write@code*{responserequired}<responserequired>{\csname ifmoodle@response required\endcsname 1\else 0\fi}%
      \moodleset{response required=false}%

%% RESPONSEFIELDLINES
      \define@cmdkey{moodle}[moodle@]{response field lines}[15]{}%
      \generate@moodle@write@code*{responsefieldlines}<responsefieldlines>{\csname moodle@response field lines\endcsname}%
      %Make synonyms 'input box size' or 'height' or 'lines'?
      \moodleset{response field lines=15}% N.B. if we do not set a default, then \moodle@responseformat will be undefined, causing problems.

%% ATTACHMENTS ALLOWED
      \def\@moodle@unlimited{unlimited}%
      \define@choicekey{moodle}{attachments allowed}{0,1,2,3,unlimited}[1]{%
        \def\test@i{#1}%
        \ifx\test@i\@moodle@unlimited
          \def\moodle@attachmentsallowed{-1}%
        \else
          \def\moodle@attachmentsallowed{#1}%
        \fi
      }
      \generate@moodle@write@code*{attachmentsallowed}<attachments>{\moodle@attachmentsallowed}
      \moodleset{attachments allowed=0}%

%% ATTACHMENTS REQUIRED
      \define@choicekey{moodle}{attachments required}{0,1,2,3}[1]{\def\moodle@attachmentsrequired{#1}}%
      \generate@moodle@write@code*{attachmentsrequired}<attachmentsrequired>{\moodle@attachmentsrequired}
      \moodleset{attachments required=0}%

%% RESPONSE TEMPLATE
      \define@key{moodle}{template}{\long\def\moodle@responsetemplate{#1}}%
      \generate@moodle@write@html@noptag{responsetemplate}<responsetemplate>{\moodle@responsetemplate}
      \moodleset{template={}}%
%    \end{macrocode}
% \subsubsection{Hint tags}
% The following are not yet fully implemented.
%    \begin{macrocode}
%% SHOWNUMCORRECT
      \define@boolkey{moodle}[moodle@]{shownumcorrect}[true]{}%
      \gdef\moodle@writeshownumcorrect{%
        \if\moodle@shownumcorrect
          \writetomoodle[4]{    <shownumcorrect/>}%
        \fi
      }%
      \moodleset{shownumcorrect=false}%

%% CLEARWRONG
      \define@boolkey{moodle}[moodle@]{clearwrong}[true]{}%
      \gdef\moodle@writeclearwrong{%
        \if\moodle@clearwrong
          \writetomoodle[4]{    <clearwrong/>}%
        \fi
      }%
      \moodleset{clearwrong=false}%

% TODO: Implement hints
%    \end{macrocode}
%
% \subsection{Answer handling}
%
%    \begin{macrocode}
%The Answers \XML\ depends heavily on the question type.
%Each type of question defines how it obtains answers from the LaTeX input,
%how it typesets those in a \PDF\ or \DVI, and how it writes them as \XML\ code.
%It will write that \XML\ to the macro \moodle@answers@xml,
%which them gets written to the file when \moodle@writeanswers
%is invoked.

\def\moodle@answers@xml{}%
\gdef\moodle@writeanswers{%
  \writetomoodle{\moodle@answers@xml}%
}%

\newcommand\addto@xml[3][0]{%
  % #1 = spaces to indent (default=0)
  % #2 = macro containing \XML\ code (possibly empty)
  % #3 = \XML\ text to be appended to that macro (will be \edef'd)
  \calculateindent{#1}%
  \edef\xml@to@add{\moodle@indent\trim@pre@space{#3}}%
  \ifx#2\@moodle@empty
    \edef\newxml{\noexpand#2\xml@to@add}%
  \else
    \edef\newxml{\noexpand#2^^J\xml@to@add}%
  \fi
  \xa\xa\xa\def\xa\xa\xa#2\xa\xa\xa{\newxml}%
}%
%    \end{macrocode}
%
% \subsubsection{Not yet implemented}
%
%    \begin{macrocode}

%%%%%%%%%%%%%%%%%%%%%%%%%%%%%%%%%%%%%%%%
%% CALCULATED %%%%%%%%%%%%%%%%%%%%%%%%%%

% TODO: I don't think I really want to handle this.  Not now.

%    \end{macrocode}
%
% \subsection{Typesetting options}
% This section provides the tools to allow on to control how the quiz gets typeset
% in the resulting PDF file.
%    \begin{macrocode}
\ifmoodle@feedbackLeft% contribution of Juergen Vollmer, 2021-03-05
  \newcommand{\moodle@preFeedback}{\\}%
\else%
  \newcommand{\moodle@preFeedback}{\hfill}%
\fi
%    \end{macrocode}
%
% \subsection{Front Ends}
% This section creates the user interface for the various question types.
% First, we define a generic command to create
% a front-end environment for a Moodle question type.
% In order to function, the following macros must be hard-coded:
% \begin{itemize}
%   \item |\moodle@|\meta{type}|@latexprocessing|:
%     Loops through the saved |\item|'s to typeset them in LaTeX,
%     usually inside an itemize or enumerate environment.
%   \item |\save|\meta{type}|answer#1|:
%     Processes the text of a single |\item| to save the information to memory,
%     usually inside |\moodle@answers@xml|.
%   \item |\write|\meta{type}|question|:
%     Writes the information, hitherto saved only in macros,
%     into the \XML\ file.
% \end{itemize}
% For example, to create the `shortanswer' question type,
% we shall call
% \begin{Verbatim}[gobble=5,frame=single]
%    \moodle@makefrontend{shortanswer}
%    \def\moodle@shortanswer@latexprocessing{...}
%    \def\saveshortansweranswer#1{...}
%    \def\writeshortanswerquestion{...}
% \end{Verbatim}
%
%    \begin{macrocode}

\def\moodle@makelatextagbox#1{%
%  \ifmoodle@tikzloaded
%    \tikzset{external/export next=false}
%    \tikz[baseline]{\node[draw,minimum height=1.2em,rounded corners,fill=black!20] {\tiny #1};}
%  % Fancy but interferes with the tikzexternalize counter
%  \else
    \Ovalbox{\tiny #1}
    %\ovalbox{\tiny #1}
    %\shadowbox{\tiny #1}
%  \fi
}%

\def\moodle@makelatextag@qtype#1{%
  \doublebox{\tiny \textsc{\GetTranslation{#1}}}
}%

\def\moodle@makelatextag@value#1#2{%
  \moodle@makelatextagbox{\GetTranslation{#2}~\csname moodle@#1\endcsname}%
}%

\def\moodle@makelatextag@key#1{%
  \moodle@makelatextagbox{\GetTranslation{#1}}
}%

\def\moodle@makelatextag@other#1{%
  \moodle@makelatextagbox{\GetTranslation{#1}}
}%

\def\moodle@makefrontend#1#2{%
  \NewEnviron{#1}[2][]{%
    \bgroup
      \setkeys{moodle}{##1,questionname={##2}}%
      \global\advance\moodle@totalmarks by \csname moodle@default grade\endcsname pt
      \expandafter\gatheritems\xa{\BODY}%
      \let\moodle@questionheader=\gatheredheader
      %First, the LaTeX processing
      \item \moodle@begin@samepage \textbf{\moodle@questionname}
      \ifmoodle@handout
        \moodle@makelatextag@qtype{#1}
      \else
        \moodle@latex@writetags
        \par
        \noindent
        \moodle@makelatextag@qtype{#1}%
        \moodle@makelatextag@value{default grade}{marked out of}
        \moodle@makelatextag@value{penalty}{penalty}%
      \fi
      #2\par
      \noindent
      \moodle@questionheader
      \edef\moodle@generalfeedback{\expandonce\moodle@feedback}
      \csname moodle@#1@latexprocessing\endcsname
      \moodle@end@samepage
      %Now, writing information to XML
      \@moodle@ifgeneratexml{%
        \xa\questiontext\xa{\moodle@questionheader}% Save the question text.
        \csname write#1question\endcsname
        \bgroup
          \gdef\moodle@answers@xml{}%
          \setkeys{moodle}{feedback={}}%
          \xa\loopthroughitemswithcommand\xa{\csname save#1answer\endcsname}%
          \passvalueaftergroup{\moodle@answers@xml}%
        \egroup
        \moodle@writeanswers%
        \moodle@writetags%
        \writetomoodle{</question>}%
      }{}%
    \egroup
  }%
}
%    \end{macrocode}
%
% \subsubsection{Description Question Front-End}
% Description and essay questions are the only question types whose front end
% is not yet created by |\moodle@makefrontend|.
% This is because of what need to be done with their contents.
%
% Description blocks can be empty. In this case, nothing is done.
%
%    \begin{macrocode}
\AtBeginEnvironment{quiz}{% protect existing description outside of quiz
  \let\description\relax% remove the meaning of existing \description and \enddescription
  \let\enddescription\relax
  \NewEnviron{description}[2][]{%
    \bgroup
      \setkeys{moodle}{#1,questionname={#2}}%
      \let\moodle@questiontext=\BODY
      \trim@spaces@in\moodle@questiontext
      \ifx\moodle@questiontext\@empty\relax\else%
        %First, the LaTeX processing.
        \item \moodle@begin@samepage\textbf{\moodle@questionname}
        \ifmoodle@handout\else
          \moodle@latex@writetags
          \par
          \noindent
        \fi
        \moodle@makelatextag@qtype{description}\par
        \noindent
        \moodle@questiontext\par
        \ifmoodle@handout\else
          \ifx\moodle@feedback\@empty\relax\else
            \fbox{\parbox{.96\linewidth}{\emph{\moodle@feedback}}}%
          \fi
        \fi
        \moodle@end@samepage
        %Now, writing information to memory.
        \@moodle@ifgeneratexml{%
          \writetomoodle{<question type="description">}%
          \moodle@writecommondata
          \moodle@writetags%
          \writetomoodle{</question>}%
        }{}%
      \fi
    \egroup
  }%
}%
%    \end{macrocode}
%
% \subsubsection{Essay Question Front-End}
% The front end is not yet created by |\moodle@makefrontend| because of
% what must must be done with the |\item|'s.
%
%    \begin{macrocode}
\def\moodle@essay@latexprocessing{%
  % Moodle cannot automatically grade an essay,
  % but if the user puts \item's in, we can list them in an itemize as notes.
  \par\noindent \emph{\GetTranslation{Information for graders}:}
  \ifnum\c@numgathereditems>0\relax
    \begin{itemize} \setlength\itemsep{0pt}\setlength\parskip{0pt}%
      \loopthroughitemswithcommand{\moodle@print@essay@answer}%
    \end{itemize}%
  \fi
  \ifx\moodle@generalfeedback\@empty\relax\else%
    \fbox{\parbox{.96\linewidth}{\emph{\moodle@generalfeedback}}}%
  \fi
}

\NewEnviron{essay}[2][]{%
  \bgroup
    \setkeys{moodle}{#1,questionname={#2}}%
    \global\advance\moodle@totalmarks by \csname moodle@default grade\endcsname pt%
    \moodle@checkresponsefieldlines
    \expandafter\gatheritems\expandafter{\BODY}%
    \let\moodle@questionheader=\gatheredheader
    %First, the LaTeX processing.
      \item \moodle@begin@samepage\textbf{\moodle@questionname}
      \ifmoodle@handout
        \moodle@makelatextag@qtype{essay}
      \else
        \moodle@latex@writetags
        \par
        \noindent
        \moodle@makelatextag@qtype{essay}
        \moodle@makelatextag@value{default grade}{marked out of}
        \moodle@makelatextag@value{penalty}{penalty}
        \xa\moodle@makelatextag@key\xa{\moodle@responseformat}
      \fi
      \par
      \noindent
      \moodle@questionheader
      \long\def\@lempty{}%
      \ifx\moodle@responsetemplate\@lempty\else
        \par\noindent\emph{\GetTranslation{Response template}:}
        \par\noindent\fbox{\parbox{.96\linewidth}{\moodle@responsetemplate}}\par
      \fi
      \edef\moodle@generalfeedback{\expandonce\moodle@feedback}
      \ifmoodle@handout
        \par\noindent
        \fbox{\parbox[t][\csname moodle@response field lines\endcsname\baselineskip]{.96\linewidth}{\phantom{Moodle}}}%
      \else
        \csname moodle@essay@latexprocessing\endcsname
      \fi
      \moodle@end@samepage
    %Now, writing information to memory.
    \@moodle@ifgeneratexml{%
      \xa\questiontext\xa{\moodle@questionheader}% Save the question text.
      \writeessayquestion
      \bgroup
        \gdef\moodle@answers@xml{}%
        %
        \ifnum\c@numgathereditems=0\relax
          \addto@xml[2]\moodle@answers@xml{<graderinfo format="html"><text/></graderinfo>}%
        \else
          \addto@xml[2]\moodle@answers@xml{<graderinfo format="html"><text><![CDATA[}%
          \ifnum\c@numgathereditems>1\relax
            \addto@xml[4]\moodle@answers@xml{<ul>}%
          \fi
          \loopthroughitemswithcommand{\moodle@savegraderinfo}%
          \ifnum\c@numgathereditems>1\relax
            \addto@xml[4]\moodle@answers@xml{</ul>}%
          \fi
          \addto@xml[2]\moodle@answers@xml{]]></text></graderinfo>}%
        \fi
        %
        \passvalueaftergroup{\moodle@answers@xml}%
      \egroup
      \moodle@writeanswers% The 'answers' \XML\ really contains the grader info.
      \moodle@writeresponsetemplate%
      \moodle@writetags%
      \writetomoodle{</question>}%
    }{}%
  \egroup
}%

\def\moodle@checkresponsefieldlines{%
  \newcount\a\a=\number\csname moodle@response field lines\endcsname
  \newcount\b\b=5%
  \ifnum\the\a>40\relax% if the value was more than 40
    \a=40%
  \fi
  \ifnum\the\a<5\relax% if the value was less than 5
    \a=5%
  \fi
  \divide\a by\b% integer division by 5
  \multiply\a by\b% multiply by 5
  \ifnum\a=\csname moodle@response field lines\endcsname\relax% equality holds if we had 5, 10, 15, 20, 25, 30, or 40
  \else
    \ifnum\csname moodle@response field lines\endcsname>5\relax%
      \ifnum\csname moodle@response field lines\endcsname<40\relax%
        \advance\a by\b% approximate with the next multiple of 5
      \fi
    \fi
    \PackageWarning{moodle}{"response field lines" admits only multiples of 5 between 5 and 40
       (You tried to set \csname moodle@response field lines\endcsname). The value
       \the\a\space will be used.}%
    \setkeys{moodle}{response field lines=\the\a}%
  \fi
}%

%%%% TODO
%%%% To make essay work will be tough.
%%%% Every line from \ifnum\c@numgathereditems=0\relax through its \else and \fi,
%%%% with the exception of
%%%%          \xa\loopthroughitemswithcommand\xa{\csname save#1answer\endcsname}%
%%%% , does not exist in our current \moodle@makefrontend code.
%%%% How can we cope?
%%%%
%%%% Idea: change \moodle@makefrontend so that
%%%%       1. if \c@numgathereditems=0, we don't do anything.
%%%%       2. it calls a preamble and postamble around the \loopthroughitemswithcommand.
%%%%          Like this:
%%%%
%%%%      \@moodle@ifgeneratexml{%
%%%%        \xa\questiontext\xa{\moodle@questionheader}% Save the question text.
%%%%        \bgroup
%%%%          \gdef\moodle@answers@xml{}%
%%%%          \setkeys{moodle}{feedback={}}%
%%%%          \@ifundefined{moodle@#1@answers@preamble}{}{}%
%%%%          \csname moodle@#1@answers@preamble\endcsname
%%%%          \ifnum\c@numgathereditems=0\relax
%%%%            \relax
%%%%          \else
%%%%            \xa\loopthroughitemswithcommand\xa{\csname save#1answer\endcsname}%
%%%%          \fi
%%%%          \@ifundefined{moodle@#1@answers@postamble}{}{}%
%%%%          \csname moodle@#1@answers@postamble\endcsname
%%%%          \passvalueaftergroup{\moodle@answers@xml}%
%%%%        \egroup
%%%%        \csname write#1question\endcsname
%%%%      }{}%
%%%% The \@ifundefined lines should automatically define the
%%%% \...@preamble \...@postamble macros to be \relax if they don't exist already.

\gdef\writeessayquestion{%
  \writetomoodle{<question type="essay">}%
    \moodle@writecommondata%
    \moodle@writeresponserequired%
    \moodle@writeresponseformat%
    \moodle@writeresponsefieldlines%
    \moodle@writeattachmentsallowed%
    \moodle@writeattachmentsrequired%
}%

\long\def\moodle@print@essay@answer#1{%
    \item #1%
}%

\long\def\moodle@savegraderinfo#1{%
  \def\moodle@answertext{#1}
  \xa\converttohtmlmacro\xa\moodle@answertext@html\xa{\moodle@answertext}%
  %\trim@spaces@in\moodle@answertext
  \ifnum\c@numgathereditems>1\relax
    \addto@xml[6]{\moodle@answers@xml}{<li>\moodle@answertext@html</li>}%
  \else
    \addto@xml[4]{\moodle@answers@xml}{\moodle@answertext@html}%
  \fi
}%
%    \end{macrocode}
%
% \subsubsection{Short Answer Question Front-End}
%
%    \begin{macrocode}
%\NewEnviron{shortanswer}[2][]{%
%   \bgroup
%     \setkeys{moodle}{#1,questionname={#2}}%
%     \expandafter\gatheritems\xa{\BODY}%
%     \let\moodle@questionheader=\gatheredheader
%     %First, the LaTeX processing.
%       \item \textbf{\moodle@questionname}
%       \csname ifmoodle@case sensitive\endcsname
%         \framebox{\tiny Case-Sensitive}\relax
%       \fi
%       \framebox{\tiny\csname moodle@default grade\endcsname~points}
%       \framebox{\tiny\csname moodle@penalty\endcsname~penalty}\par
%       \noindent
%       \moodle@questionheader
%       \csname moodle@shortanswer@latexprocessing\endcsname
%     %Now, writing information to memory.
%     \@moodle@ifgeneratexml{%
%       \xa\questiontext\xa{\moodle@questionheader}% Save the question text.
%       \bgroup
%         \gdef\moodle@answers@xml{}%
%         \setkeys{moodle}{feedback={}}%
%         \xa\loopthroughitemswithcommand\xa{\csname
%         saveshortansweranswer\endcsname}%
%         \passvalueaftergroup{\moodle@answers@xml}%
%       \egroup
%       \csname writeshortanswerquestion\endcsname
%     }{}%
%   \egroup
% }%

\moodle@makefrontend{shortanswer}{\moodle@makelatextag@shortanswer}%

% LATEX PROCESSING

\def\moodle@makelatextag@shortanswer{%
  \ifmoodle@usecase
    \moodle@makelatextag@other{Case-Sensitive}\relax
  \else
    \moodle@makelatextag@other{Case-Insensitive}\relax
  \fi
}

\ifmoodle@handout
  \let\moodle@shortanswer@latexprocessing\relax
\else
  \def\moodle@shortanswer@latexprocessing{%
    \begin{itemize} \setlength\itemsep{0pt}\setlength\parskip{0pt}%
      \loopthroughitemswithcommand{\moodle@print@shortanswer@answer}%
    \end{itemize}%
    \ifx\moodle@generalfeedback\@empty\relax\else%
      \fbox{\parbox{.96\linewidth}{\emph{\moodle@generalfeedback}}}%
    \fi
  }
\fi

   \def\moodle@print@shortanswer@answer#1{%
       \let\moodle@feedback=\@empty
       \moodle@print@shortanswer@answer@int#1\@rdelim
   }%
   \newcommand\moodle@print@shortanswer@answer@int[1][]{%
     \setkeys{moodle}{#1}%
     \moodle@print@shortanswer@answer@int@int%
   }%
   \def\moodle@print@shortanswer@answer@int@int#1\@rdelim{%
     \ifx\moodle@fraction\@hundred
       \item #1$~\checkmark$%
     \else
       \moodle@checkfraction
       \item #1$~(\moodle@fraction\%)$%
     \fi
     \ifx\moodle@feedback\@empty\relax\else
       \moodle@preFeedback \emph{$\rightarrow$ \moodle@feedback}
     \fi
   }%

% SAVING ANSWERS TO MEMORY
\def\saveshortansweranswer#1{%
  \bgroup
    \saveshortansweranswer@int#1\moodle@answer@rdelim
    \passvalueaftergroup{\moodle@answers@xml}%
  \egroup
}%
   \newcommand\saveshortansweranswer@int[1][]{%
     \setkeys{moodle}{fraction=100,#1}%                  %%%%%% DEFAULT VALUE IS 100%
     \saveshortansweranswer@int@int%
   }%
   \def\saveshortansweranswer@int@int#1\moodle@answer@rdelim{%
     \def\moodle@answertext{#1}%
     \trim@spaces@in\moodle@answertext
     \moodle@checkfraction
     \addto@xml[2]{\moodle@answers@xml}{<answer fraction="\moodle@fraction" format="plain_text">}%
     \xa\converttohtmlmacro\xa\moodle@answertext@html\xa{\moodle@answertext}%
     \addto@xml[4]{\moodle@answers@xml}{  <text>\moodle@answertext@html</text>}%
     \ifmoodle@pluginfile
       \writetomoodle[0]{\htmlize@embeddedfiletags}%
     \fi
     \ifx\moodle@feedback\@empty\relax\else
       \trim@spaces@in\moodle@feedback
       \xa\converttohtmlmacro\xa\moodle@feedback@html\xa{\moodle@feedback}%
       \addto@xml[4]{\moodle@answers@xml}{  <feedback format="html"><text><![CDATA[<p>\moodle@feedback@html</p>]]></text>\ifmoodle@pluginfile\htmlize@embeddedfiletags\fi</feedback>}%
     \fi
     \addto@xml[2]{\moodle@answers@xml}{</answer>}%
   }%

% WRITING QUESTION TO \XML\ FILE
\gdef\writeshortanswerquestion{%
  \writetomoodle{<question type="shortanswer">}%
    \moodle@writecommondata%
    \moodle@writeusecase%
}%
%    \end{macrocode}
%
% \subsubsection{Numerical Question Front-End}
%
%    \begin{macrocode}
\moodle@makefrontend{numerical}{\moodle@makelatextag@numerical}%

% LATEX PROCESSING

\def\moodle@makelatextag@numerical{}

\AtEndPreamble{
  \@ifpackageloaded{siunitx}{\def\moodle@printnum{\num[omit-uncertainty,copy-decimal-marker]}}{\let\moodle@printnum\trim@spaces}%
}

\ifmoodle@handout
  \let\moodle@numerical@latexprocessing\relax
\else
  \def\moodle@numerical@latexprocessing{%
    \begin{itemize} \setlength\itemsep{0pt}\setlength\parskip{0pt}%
      \loopthroughitemswithcommand{\moodle@print@numerical@answer}%
    \end{itemize}%
    \ifx\moodle@generalfeedback\@empty\relax\else%
      \fbox{\parbox{.96\linewidth}{\emph{\moodle@generalfeedback}}}%
    \fi
  }
\fi

   \def\moodle@print@numerical@answer#1{%
       \let\moodle@feedback=\@empty
       \bgroup
         \moodle@print@numerical@answer@int#1\@rdelim
       \egroup
   }%
   \newcommand\moodle@print@numerical@answer@int[1][]{%
     \setkeys{moodle}{#1}%
     \moodle@print@numerical@answer@int@int%
   }%
   \def\moodle@print@numerical@answer@int@int#1\@rdelim{%
     \def\test@i{#1}%
     \trim@spaces@in\test@i
     \ifx\test@i\@star
       \item \test@i
     \else
       \item \moodle@printnum{#1}%
       \ifx\moodle@tolerance\moodle@zero\else
         $\,\pm\,$\moodle@printnum{\expandonce\moodle@tolerance}%
       \fi
     \fi
     \ifx\moodle@fraction\@hundred
       $~\checkmark$%
     \else
       \moodle@checkfraction
       $~(\moodle@fraction\%)$%
     \fi
     \ifx\moodle@feedback\@empty\relax\else
       \moodle@preFeedback \emph{$\rightarrow$ \moodle@feedback}%
     \fi
   }%

% SAVING ANSWERS TO MEMORY
\def\savenumericalanswer#1{%
  \bgroup
    \savenumericalanswer@int#1\moodle@answer@rdelim
    \passvalueaftergroup{\moodle@answers@xml}%
  \egroup
}%
   \newcommand\savenumericalanswer@int[1][]{%
     \setkeys{moodle}{fraction=100,#1}%                  %%%%%% DEFAULT VALUE IS 100%
     \savenumericalanswer@int@int%
   }%
   \def\savenumericalanswer@int@int#1\moodle@answer@rdelim{%
     \def\moodle@answertext{#1}%
     \trim@spaces@in\moodle@answertext
     \moodle@checkfraction
     \addto@xml[2]{\moodle@answers@xml}{<answer fraction="\moodle@fraction" format="plain_text">}%
     \addto@xml[4]{\moodle@answers@xml}{  <text>\moodle@answertext</text>}%

     \ifx\moodle@answertext\@star\else
       \addto@xml[4]{\moodle@answers@xml}{  <tolerance>\expandonce\moodle@tolerance</tolerance>}%
     \fi
     \ifmoodle@pluginfile
       \writetomoodle[0]{\htmlize@embeddedfiletags}%
     \fi
     \ifx\moodle@feedback\@empty\relax\else
       \trim@spaces@in\moodle@feedback
       \xa\converttohtmlmacro\xa\moodle@feedback@html\xa{\moodle@feedback}%
       \addto@xml[4]{\moodle@answers@xml}{  <feedback format="html"><text><![CDATA[<p>\moodle@feedback@html</p>]]></text>\ifmoodle@pluginfile\htmlize@embeddedfiletags\fi</feedback>}%
     \fi
     \addto@xml[2]{\moodle@answers@xml}{</answer>}%
   }%


% WRITING QUESTION TO \XML\ FILE
\gdef\writenumericalquestion{%
  \writetomoodle{<question type="numerical">}%
    \moodle@writecommondata%
}%
%    \end{macrocode}
%
% \subsubsection{Multiple Choice Question Front-End}
%
%    \begin{macrocode}
%Multiple choice has the structure
% \begin{multi}[options]{name}%
%   What is 5+7?
%   \item 13
%   \item* 12
%   \item 11
% \end{multi}%

\moodle@makefrontend{multi}{\moodle@makelatextag@multi}%

% LATEX PROCESSING

\def\moodle@makelatextag@multi{%
  \ifmoodle@allornothing
    \ifx\endmulti\endclozemulti
      \PackageError{moodle}{Unsupported option "allornothing" for a multichoice subquestion}
      {Please set "allornothing=false"}%
    \else
      \moodle@makelatextag@other{All-or-nothing}%
    \fi
  \else
    \ifmoodle@multiple
      \moodle@makelatextag@other{Multiple}%
    \else
      \moodle@makelatextag@other{Single}%
    \fi
  \fi
  \ifmoodle@handout\else
    \ifmoodle@shuffle
      \moodle@makelatextag@other{Shuffle}\relax%
    \fi
  \fi
}

\def\moodle@multi@latexprocessing{%
  \ifmoodle@allornothing\moodle@singletrue\fi
  \ifmoodle@multiple\moodle@InspectMultipleAnswers\fi
  \ifmoodle@handout\NewList{answerlist}\fi
  \begin{enumerate}\moodle@obeynumberingstyle%
    %\renewcommand{\theenumi}{\alph{enumi}}%
    \setlength\itemsep{0pt}\setlength\parskip{0pt}%
    \loopthroughitemswithcommand{\moodle@print@multichoice@answer}%
    \ifmoodle@handout
      \ifmoodle@shuffle
        \let\moodle@multi@loop=\ForEachRandomItem
      \else
        \let\moodle@multi@loop=\ForEachFirstItem
      \fi
      \moodle@multi@loop{answerlist}{Answer}{\Answer}%
    \fi
  \end{enumerate}%
  \ifmoodle@handout\else
    \ifx\moodle@generalfeedback\@empty\relax\else%
      \fbox{\parbox{.96\linewidth}{\emph{\moodle@generalfeedback}}}%
    \fi
  \fi
}
  \long\def\moodle@print@multichoice@answer#1{%
    \let\moodle@feedback=\@empty%
    \moodle@print@multichoice@answer@int#1 \@rdelim%
  }%
  \newcommand\moodle@print@multichoice@answer@int[1][]{%
    \let\moodle@fraction\@empty%
    \setkeys{moodle}{#1}%
    \moodle@print@multichoice@answer@int@int%
  }%
  \long\def\moodle@print@multichoice@answer@int@int#1#2\@rdelim{%
    \def\test@i{#1}%
    \def\test@ii{#2}%
    \def\moodle@answertext{\item }%
    \ifx\test@i\@star%
      \g@addto@macro\moodle@answertext{#2}%
      \ifmoodle@single%
        \setkeys{moodle}{fraction=100}%
      \else
        \setkeys{moodle}{fraction=\moodle@AutoScore}%
      \fi
    \else
      \g@addto@macro\moodle@answertext{#1#2}%
    \fi
    \trim@spaces@in\moodle@answertext%
    \trim@spaces@in\moodle@answertext%
    \ifmoodle@handout\else
      \ifmoodle@single%
        \ifx\moodle@fraction\@empty\relax%
          \ifdim0pt<\moodle@sanction pt\relax
            \setkeys{moodle}{fraction=-\moodle@sanction}%
          \else
            \setkeys{moodle}{fraction=0}%
          \fi
        \fi
        \moodle@checkfraction
        \ifx\moodle@fraction\@hundred%
          \trim@spaces@in\moodle@answertext%
          \g@addto@macro\moodle@answertext{$~\checkmark$}%
        \else
          \ifdim0pt=\moodle@fraction pt\relax\else%
            \g@addto@macro\moodle@answertext{$~(\moodle@fraction\%)$}%
          \fi
        \fi
      \else% multiple
        \ifx\moodle@fraction\@empty\relax%
          \ifmoodle@AdvancedScoreMode
            \setkeys{moodle}{fraction=0}%
          \else
            \setkeys{moodle}{fraction=-\moodle@AutoScore}%
          \fi
        \fi
        \moodle@checkfraction
        \g@addto@macro\moodle@answertext{$~(\moodle@fraction\%)$}%
      \fi
    \fi
    \ifmoodle@handout
      \def\temp{\InsertLastItem{answerlist}}%
      \xa\temp\xa{\moodle@answertext}%
    \else
      \ifx\moodle@feedback\@empty\relax\else%
        \g@addto@macro\moodle@answertext{\moodle@preFeedback \emph{$\rightarrow$ \moodle@feedback}}%
      \fi
      \moodle@answertext
    \fi
  }%

% COMMON UTILITY: Inspecting answers of multiple choice questions (for option 'multiple')
\newcounter{moodle@NumStarredAnswers}% count the stars
\newdimen\moodle@autoscore@tmp%
\newdimen\moodle@TotalPositiveFractions% Total of user-set positive fractions
\newdimen\moodle@PositiveScoreFactor% scaling factor to impose the fractions of
%correct answers to add up to 100%
\newif\ifmoodle@AdvancedScoreMode%

\def\moodle@InspectMultipleAnswers{%
  \setcounter{moodle@NumStarredAnswers}{0}%
  \moodle@autoscore@tmp=100pt\relax% temporary variable
  \moodle@TotalPositiveFractions=0pt\relax%
  \moodle@PositiveScoreFactor=1pt\relax
  \moodle@AdvancedScoreModefalse
  \loopthroughitemswithcommand{\moodle@InspectMultipleAnswers@a}%
  \ifmoodle@AdvancedScoreMode% advanced mode
    \ifx\endmulti\endclozemulti% inside cloze
      \divide\moodle@TotalPositiveFractions by 100\relax
      \advance\moodle@TotalPositiveFractions by \c@moodle@NumStarredAnswers pt\relax
      \moodle@PositiveScoreFactor=1 pt\relax
      \moodle@PositiveScoreFactor=\dimexpr 1 pt * \moodle@PositiveScoreFactor / \moodle@TotalPositiveFractions\relax%
      \global\def\moodle@OtherScore{% for multianswer questions in cloze, with advanced mode
        \ifdim0pt<\moodle@fraction pt\relax
          \moodle@fraction
        \else
          \moodle@autoscore@tmp=\moodle@PositiveScoreFactor\relax%
          \multiply\moodle@autoscore@tmp by \moodle@fraction\relax%
          \strip@pt\moodle@autoscore@tmp
        \fi
      }%
    \else% outside cloze
      \advance\moodle@autoscore@tmp by-\moodle@TotalPositiveFractions\relax%
      \def\ds{\strip@pt\moodle@TotalPositiveFractions}%
      \ifnum0=\c@moodle@NumStarredAnswers\relax%
      % autopoints will never be used but we check if the sum of positive fractions is 100%
        \ifdim\moodle@autoscore@tmp<-\moodle@fractiontol pt\relax%
          \PackageWarning{moodle}{Positive fractions add up to more than 100 (here: \ds)}%
          % Here we issue only a warning because Moodle accepts the \XML\ without error.
        \else
          \ifdim\moodle@autoscore@tmp>\moodle@fractiontol pt\relax%
            \PackageError{moodle}{Positive fractions add up to less than 100 (here: \ds)}%
          \fi
        \fi
      \else% there are starred items
        \ifdim0pt<\moodle@autoscore@tmp\relax\else%
      % we have starred items so the sum of user-set positive fractions must be less than 100%
      % otherwise, starred items would lead to penalties
          \PackageError{moodle}{Positive fractions add up to 100 or more (here: \ds):
                             there is no positive points left to be given to starred items.}%
        \fi
        \divide\moodle@autoscore@tmp by \c@moodle@NumStarredAnswers\relax%
      \fi
    \fi
  \else% automatic score (not in advanced score mode)
    \global\divide\moodle@autoscore@tmp by \c@moodle@NumStarredAnswers\relax%
  \fi
  \xdef\moodle@AutoScore{\strip@pt\moodle@autoscore@tmp}%
}
\long\def\moodle@InspectMultipleAnswers@a#1{%
  %The grouping is to keep key answer-specific key changes local.
  \bgroup
    \moodle@InspectMultipleAnswers@b#1\moodle@answer@rdelim
  \egroup
}%
\newcommand\moodle@InspectMultipleAnswers@b[1][]{%
  %\ifx&#1&%
    \let\moodle@fraction\@empty%
    \setkeys{moodle}{#1}%
    \moodle@InspectMultipleAnswers@c%
  %\fi
}%
\long\def\moodle@InspectMultipleAnswers@c#1#2\moodle@answer@rdelim{%
  \def\test@i{#1}%
  \ifx\test@i\@star
    \stepcounter{moodle@NumStarredAnswers}%
  \else
    \ifx\moodle@fraction\@empty\else
      \global\moodle@AdvancedScoreModetrue
      \ifdim0pt<\moodle@fraction pt\relax%
        \global\advance\moodle@TotalPositiveFractions by \moodle@fraction pt\relax%
      \fi
    \fi
  \fi
}%
\newdimen\test@fraction%
\newdimen\test@lower%
\newdimen\test@upper%
\def\moodle@fractionerror{%
  \def\ds{\moodle@fraction}%
  \PackageError{moodle}{the current fraction is not an admissible value (here: \ds)}%
}
{\catcode`|=3\relax
\gdef\moodle@validfractionlist{0|5|10|11.11111|12.5|14.28571|16.66667|20|25|30|33.33333|40|50|60|66.66667|70|75|80|83.33333|90|100}}%
\def\moodle@isfractionnear#1{%
  \test@lower=\dimexpr #1 pt - \moodle@fractiontol pt\relax%
  \test@upper=\dimexpr #1 pt + \moodle@fractiontol pt\relax%
  \ifdim\test@upper>\test@fraction\relax
    \ifdim\test@lower<\test@fraction\relax
      \gdef\test@fractionmatched{#1}%
    \fi
  \fi
}
\def\moodle@checkfraction{%
  \ifmoodle@allornothing
    \ifnum\moodle@fraction>0\relax
      \setkeys{moodle}{fraction=100}%
    \fi
    \ifnum\moodle@fraction<0\relax
      \setkeys{moodle}{fraction=0}%
    \fi
  \else
    %\def\test@i{#1}%
    \test@fraction=\moodle@fraction pt\relax%
    % take the absolute value
    \ifdim0pt>\test@fraction\relax%
      \test@fraction=-\moodle@fraction pt\relax%
    \fi
    % test if the fraction is an admissible value
    \let\test@fractionmatched\@empty
    \forlistloop{\moodle@isfractionnear}{\moodle@validfractionlist}%
    \ifx\test@fractionmatched\@empty\relax
      \moodle@fractionerror%
    \fi
    \ifdim\moodle@fraction pt<-\moodle@fractiontol pt\relax%
      \xdef\moodle@fraction{-\test@fractionmatched}%
    \else
      \xdef\moodle@fraction{\test@fractionmatched}%
    \fi
  \fi
}
% TODO: Put these macros in same order as other sections'.

% SAVING ANSWERS TO MEMORY
\long\def\savemultianswer#1{%
  \bgroup
    \savemultianswer@int#1 \moodle@answer@rdelim
    \passvalueaftergroup{\moodle@answers@xml}%
  \egroup
}%
  \newcommand\savemultianswer@int[1][]{%
    \let\moodle@fraction\@empty%
    \setkeys{moodle}{#1}%
    \ifmoodle@allornothing
      \moodle@singletrue
    \fi
    \savemultianswer@int@int%
  }%
  \long\def\savemultianswer@int@int#1#2\moodle@answer@rdelim{%
    \def\test@i{#1}%
    \ifx\test@i\@star
      \ifmoodle@single
        \setkeys{moodle}{fraction=100}%
      \else
        \setkeys{moodle}{fraction=\moodle@AutoScore}%
      \fi
      \def\moodle@answertext{#2}%
    \else
      \def\moodle@answertext{#1#2}%
    \fi
    \ifx\moodle@fraction\@empty\relax%
      \ifmoodle@single\relax
        \ifdim0pt<\moodle@sanction pt\relax
          \setkeys{moodle}{fraction=-\moodle@sanction}%
        \else
          \setkeys{moodle}{fraction=0}%
        \fi
      \else% multiple
        \ifmoodle@AdvancedScoreMode
          \setkeys{moodle}{fraction=0}%
        \else
          \setkeys{moodle}{fraction=-\moodle@AutoScore}%
        \fi
      \fi
    \fi
    \trim@spaces@in\moodle@answertext
    \trim@spaces@in\moodle@answertext
    \moodle@checkfraction
    \addto@xml[2]{\moodle@answers@xml}{<answer fraction="\moodle@fraction" format="html">}%
    \xa\converttohtmlmacro\xa\moodle@answertext@html\xa{\moodle@answertext}%
    \addto@xml[4]{\moodle@answers@xml}{  <text><![CDATA[<p>\moodle@answertext@html</p>]]></text>}%
    \ifmoodle@pluginfile
      \addto@xml[4]{\moodle@answers@xml}{\htmlize@embeddedfiletags}%
    \fi
    \ifx\moodle@feedback\@empty\relax\else
      \trim@spaces@in\moodle@feedback
      \xa\converttohtmlmacro\xa\moodle@feedback@html\xa{\moodle@feedback}%
      \addto@xml[4]{\moodle@answers@xml}{  <feedback format="html"><text><![CDATA[<p>\moodle@feedback@html</p>]]></text>\ifmoodle@pluginfile\htmlize@embeddedfiletags\fi</feedback>}%
    \fi
    \addto@xml[2]{\moodle@answers@xml}{</answer>}%
  }%

% WRITING QUESTION TO \XML\ FILE
\gdef\writemultiquestion{%
    \writetomoodle{<question type="multichoice\ifmoodle@allornothing set\fi">}%
    \moodle@writecommondata%
    \ifmoodle@allornothing\else
      \moodle@writesingle%
    \fi
    \moodle@writeshuffle%
    \moodle@writeanswernumbering%
}%
%    \end{macrocode}
%
% \subsubsection{True/False Question Front-End}
%
%    \begin{macrocode}
% True/False has structure
% \begin{truefalse}[options]{name}%
%   This is a matching question.
%   \item[feedback={feedback for student answering incorrectly "true"}] % first item is for true
%   \item* this is an other way of specifying answer-specific feedback
% \end{truefalse}%

%\moodle@makefrontend{truefalse}{}% We dont use the generic frontend because truefalse has no tunable penalty

\NewEnviron{truefalse}[2][]{%
    \bgroup
      \setkeys{moodle}{#1,questionname={#2}}%
      \global\advance\moodle@totalmarks by \csname moodle@default grade\endcsname pt
      \expandafter\gatheritems\xa{\BODY}%
      \let\moodle@questionheader=\gatheredheader
      %First, the LaTeX processing
      \item \moodle@begin@samepage\textbf{\moodle@questionname}
      \ifmoodle@handout
        \moodle@makelatextag@qtype{truefalse}
      \else
        \moodle@latex@writetags
        \par
        \noindent
        \moodle@makelatextag@qtype{truefalse}
        \moodle@makelatextag@value{default grade}{marked out of}
      \fi
      \par
      \noindent
      \moodle@questionheader
      \edef\moodle@generalfeedback{\expandonce\moodle@feedback}
      \moodle@truefalse@latexprocessing
      \moodle@end@samepage
      %Now, writing information to XML
      \@moodle@ifgeneratexml{%
        \setkeys{moodle}{penalty=1}%
        \xa\questiontext\xa{\moodle@questionheader}% Save the question text.
        \csname writetruefalsequestion\endcsname
        \bgroup
          \gdef\moodle@answers@xml{}%
          \setkeys{moodle}{feedback={}}%
          \xa\loopthroughitemswithcommand\xa{\xa\savetruefalseanswer}%
          \ifnum\c@numgathereditems=1\relax
            \setcounter{currentitemnumber}{2}%
            \savetruefalseanswer{}
          \fi
          \passvalueaftergroup{\moodle@answers@xml}%
        \egroup
        \moodle@writeanswers%
        \moodle@writetags%
        \writetomoodle{</question>}%
      }{}%
    \egroup
  }%

% LATEX PROCESSING

\def\moodle@truefalse@latexprocessing{%
%  \ifnum\c@numgathereditems>2\relax%
%    \PackageError{moodle}{Expecting at max two answers with truefalse type}
%  \fi
  \setcounter{moodle@NumStarredAnswers}{0}%
  \begin{itemize} \setlength\itemsep{0pt}\setlength\parskip{0pt}%
    \loopthroughitemswithcommand{\moodle@print@truefalse@answer}%
    \ifnum\c@currentitemnumber=2\relax
      \item \textbf{\GetTranslation{False}}%
    \fi
  \end{itemize}
  \ifmoodle@handout\else
    \ifx\moodle@generalfeedback\@empty\relax\else%
      \fbox{\parbox{.96\linewidth}{\emph{\moodle@generalfeedback}}}%
    \fi
  \fi
  \ifnum\c@moodle@NumStarredAnswers=0\relax
    \PackageError{moodle}{No answer is explicitly marked as correct (*). Be sure one answer leads to points.}%
  \fi
  \ifnum\c@moodle@NumStarredAnswers>1\relax
    \PackageError{moodle}{Two answers are explicitly marked as correct (*). Be sure only one answer leads to points.}%
  \fi
}

   \def\moodle@print@truefalse@answer#1{% here # is all what comes after "\item", that is "[options]* text"
       \let\moodle@feedback=\@empty
       \moodle@print@truefalse@answer@int#1\@rdelim % add an end delimiter:
   }%
   \newcommand\moodle@print@truefalse@answer@int[1][]{% with the optional argument, catch options and set them as keys
     \setkeys{moodle}{#1}%
     \moodle@print@truefalse@answer@int@int% applies to the rest: "* text\@rdelim"
   }%
   \def\moodle@print@truefalse@answer@int@int#1\@rdelim{% this is just to treat appart the case where nothing follows
     \def\test@i{#1}
     \trim@spaces@in\test@i
     \ifx\test@i\@empty\relax
       \moodle@print@truefalse@answer@int@int@empty
     \else
       \moodle@print@truefalse@answer@int@int@int#1\@rdelim
     \fi
   }%
   \def\moodle@print@truefalse@answer@int@int@empty{%
     \ifnum\c@currentitemnumber=1\relax
       \def\moodle@answertext{\GetTranslation{True}}%
     \fi
     \ifnum\c@currentitemnumber=2\relax
       \def\moodle@answertext{\GetTranslation{False}}%
     \fi
     \item \textbf{\moodle@answertext}%
     \ifmoodle@handout\else
       \ifx\moodle@feedback\@empty\relax\else
         ~\moodle@preFeedback \emph{$\rightarrow$ \moodle@feedback}%
       \fi
     \fi
   }%
   \def\moodle@print@truefalse@answer@int@int@int#1#2\@rdelim{%
     \ifnum\c@currentitemnumber=1\relax
       \def\moodle@answertext{\GetTranslation{True}}%
     \fi
     \ifnum\c@currentitemnumber=2\relax
       \def\moodle@answertext{\GetTranslation{False}}%
     \fi
     \item \textbf{\moodle@answertext}%
     \ifnum\c@currentitemnumber<3\relax
       \def\test@i{#1}%
       %\trim@spaces@in\test@i
       \ifx\test@i\@star
         \ifmoodle@handout\else
           ~$\checkmark$%
         \fi
         \stepcounter{moodle@NumStarredAnswers}%
       \else
         ~%
       \fi
       \ifmoodle@handout\else
         \ifx\moodle@feedback\@empty\relax
           \def\test@ii{#2}%
           \trim@spaces@in\test@ii
           \ifx\test@ii\@empty\relax\else
             \ifx\test@i\@star%
               \moodle@preFeedback \emph{$\rightarrow$ #2}%
             \else%
               \moodle@preFeedback \emph{$\rightarrow$ #1#2}%
             \fi
           \fi
         \else
           \moodle@preFeedback \emph{$\rightarrow$ \moodle@feedback}%
         \fi
       \fi
     \fi
   }%

% SAVING ANSWERS TO MEMORY
\def\savetruefalseanswer#1{%
  \bgroup
    \savetruefalseanswer@int#1\moodle@answer@rdelim
    \passvalueaftergroup{\moodle@answers@xml}%
  \egroup
}%
   \newcommand\savetruefalseanswer@int[1][]{%
     \setkeys{moodle}{#1}%
     \savetruefalseanswer@int@int%
   }%
   \def\savetruefalseanswer@int@int#1\moodle@answer@rdelim{%
     \def\test@i{#1}
     \trim@spaces@in\test@i
     \ifx\test@i\@empty\relax
       \savetruefalseanswer@int@int@empty
     \else
       \savetruefalseanswer@int@int@int#1\moodle@answer@rdelim
     \fi
   }%
   \def\savetruefalseanswer@int@int@empty{%
     \setkeys{moodle}{fraction=0}%
     \ifnum\c@currentitemnumber=1\relax
       \def\moodle@answertext{true}%
     \fi
     \ifnum\c@currentitemnumber=2\relax
       \def\moodle@answertext{false}%
     \fi
     \ifnum\c@currentitemnumber<3\relax
       \addto@xml[2]{\moodle@answers@xml}{<answer fraction="\moodle@fraction" format="plain_text">}%
       \addto@xml[4]{\moodle@answers@xml}{  <text>\moodle@answertext</text>}%
       \ifmoodle@pluginfile
         \addto@xml[4]{\moodle@answers@xml}{\htmlize@embeddedfiletags}%
       \fi
       \ifx\moodle@feedback\@empty\relax\else
         \trim@spaces@in\moodle@feedback
         \xa\converttohtmlmacro\xa\moodle@feedback@html\xa{\moodle@feedback}%
         \addto@xml[4]{\moodle@answers@xml}{  <feedback format="html"><text><![CDATA[<p>\moodle@feedback@html</p>]]></text>\ifmoodle@pluginfile\htmlize@embeddedfiletags\fi</feedback>}%
       \fi
       \addto@xml[2]{\moodle@answers@xml}{</answer>}%
     \fi
   }%
   \def\savetruefalseanswer@int@int@int#1#2\moodle@answer@rdelim{%
     \def\test@i{#1}%
     \ifx\test@i\@star
       \setkeys{moodle}{fraction=100}%
     \else
       \setkeys{moodle}{fraction=0}%
     \fi
     \ifnum\c@currentitemnumber=1\relax
       \def\moodle@answertext{true}%
     \fi
     \ifnum\c@currentitemnumber=2\relax
       \def\moodle@answertext{false}%
     \fi
     \ifnum\c@currentitemnumber<3\relax
       \addto@xml[2]{\moodle@answers@xml}{<answer fraction="\moodle@fraction" format="plain_text">}%
       \addto@xml[4]{\moodle@answers@xml}{  <text>\moodle@answertext</text>}%
       \ifmoodle@pluginfile
         \addto@xml[4]{\moodle@answers@xml}{\htmlize@embeddedfiletags}%
       \fi
       \ifx\moodle@feedback\@empty\relax
         \def\test@ii{#2}
         \ifx\test@ii\@empty\relax\else
           \ifx\test@i\@star
             \xa\converttohtmlmacro\xa\moodle@feedback@html\xa{#2}%
           \else%
             \xa\converttohtmlmacro\xa\moodle@feedback@html\xa{#1#2}%
           \fi%
           \addto@xml[4]{\moodle@answers@xml}{  <feedback format="html"><text><![CDATA[<p>\moodle@feedback@html</p>]]></text>\ifmoodle@pluginfile\htmlize@embeddedfiletags\fi</feedback>}%
         \fi
       \else
         \trim@spaces@in\moodle@feedback
         \xa\converttohtmlmacro\xa\moodle@feedback@html\xa{\moodle@feedback}%
         \addto@xml[4]{\moodle@answers@xml}{  <feedback format="html"><text><![CDATA[<p>\moodle@feedback@html</p>]]></text>\ifmoodle@pluginfile\htmlize@embeddedfiletags\fi</feedback>}%
       \fi
       \addto@xml[2]{\moodle@answers@xml}{</answer>}%
     \fi
   }%

% WRITING QUESTION TO \XML\ FILE
\gdef\writetruefalsequestion{%
  \writetomoodle{<question type="truefalse">}%
    \moodle@writecommondata%
}%
%    \end{macrocode}
%
% \subsubsection{Matching Question Front-End}
%
%    \begin{macrocode}
%\let\answer=\hfill

\moodle@makefrontend{matching}{\moodle@makelatextag@matching}%

% LATEX PROCESSING

\def\moodle@makelatextag@matching{%
  \ifmoodle@handout\else
    \ifmoodle@draganddrop
      \moodle@makelatextag@other{Drag and drop}\relax%
    \fi
    \ifmoodle@shuffle
      \moodle@makelatextag@other{Shuffle}\relax%
    \fi
  \fi
}
\long\def\moodle@matching@left#1{%
  \parbox[c]{.9\linewidth}{\raggedleft #1}\hfil\vrule width .5pt$\bullet$%
}%
\long\def\moodle@matching@right#1{%
  $\bullet$\vrule width .5pt\hfil\parbox[c]{.9\linewidth}{#1}%
}%
\def\moodle@matching@latexprocessing{%
  \bgroup
    %\let\answer=\hfill
    \par\noindent
    \ifmoodle@handout
      \NewList{questionlist}%
      \gdef\matcheslist{}%
      \NewList{answerlist}%
    \else
      \long\def\matching@table@text{}%
    \fi
    \setcounter{moodle@NumStarredAnswers}{0}% Here this counter is for "questions" (items on the left column)
    \loopthroughitemswithcommand{\moodle@print@matching@answer}%
    \ifnum\c@numgathereditems<3\relax
      \PackageWarning{moodle}{Moodle expects at least three proposed matches in matching questions}%
    \fi
    \ifnum\c@moodle@NumStarredAnswers<2\relax
      \PackageWarning{moodle}{Moodle expects at least two items in matching questions}%
    \fi
    \ifmoodle@handout
      \ifmoodle@shuffle
        \let\moodle@matching@loop=\ForEachRandomItem
      \else
        \let\moodle@matching@loop=\ForEachFirstItem
      \fi
      \begin{minipage}{.43\linewidth}%
      \moodle@matching@loop{questionlist}{Question}{%
        \moodle@matching@left{\Question}%
        \vskip 4pt\relax
      }%
      \end{minipage}%
      \hfill
      \begin{minipage}{.43\linewidth}%
      \ForEachFirstItem{answerlist}{Answer}{%
        \moodle@matching@right{\Answer}%
        \vskip 4pt\relax
      }%
      \end{minipage}%
    \else
      \begin{tabular}{@{}p{.43\linewidth}@{}p{.1\linewidth}@{}p{.43\linewidth}@{}}%
        \matching@table@text
      \end{tabular}%
      \ifx\moodle@generalfeedback\@empty\relax\else%
        \fbox{\parbox{.96\linewidth}{\emph{\moodle@generalfeedback}}}%
      \fi
    \fi
  \egroup
}

\long\def\moodle@print@matching@answer#1{%
  \moodle@print@matching@answer@int#1 \@rdelim
}%
\newcommand\moodle@print@matching@answer@int[1][]{%
  \moodle@print@matching@answer@int@int
}%
\long\def\moodle@print@matching@answer@int@int#1\answer#2\@rdelim{%
  %\typeout{\string#1 \answer \string#2}%
  \def\test@i{#1}%
  \trim@spaces@in\test@i
  \ifmoodle@handout
    \ifx\test@i\@empty\else
      \stepcounter{moodle@NumStarredAnswers}%
      \InsertLastItem{questionlist}{#1}%
    \fi
    % add proposed match only if not already in the list
    \xifinlist{\trim@spaces{#2}}{\matcheslist}{}{%
      \listeadd\matcheslist{\trim@spaces{#2}}%
      % matches are always proposed shuffled
      \InsertRandomItem{answerlist}{#2}%
    }%
  \else
    \ifx\test@i\@empty
      \xa\g@addto@macro\xa\matching@table@text\xa{%
        &&\moodle@matching@right{#2}\\\\\relax%
      }%
    \else
      \stepcounter{moodle@NumStarredAnswers}%
      \xa\g@addto@macro\xa\matching@table@text\xa{%
        \moodle@matching@left{#1}&\leavevmode\leaders\hb@xt@.44em{\hss$\cdot$\hss}\hfill\kern\z@%
        &\moodle@matching@right{#2}\\\\\relax
      }%
    \fi
  \fi
}%


% SAVING ANSWERS TO MEMORY
\long\def\savematchinganswer#1{%
  \bgroup
    \savematchinganswer@int#1 \moodle@answer@rdelim%
    \passvalueaftergroup{\moodle@answers@xml}%
  \egroup
}%
   \newcommand\savematchinganswer@int[1][]{%
     \setkeys{moodle}{#1}%
     \savematchinganswer@int@int%\space
   }%
   \long\def\savematchinganswer@int@int#1\answer#2\moodle@answer@rdelim{%
     %\typeout{\string#1 \answer \string#2}%
     % Note that #1 may simply be \relax.
     \def\moodle@subquestiontext{#1}%
     \def\moodle@subanswertext{#2}%
     \trim@spaces@in\moodle@subquestiontext
     \xa\converttohtmlmacro\xa\moodle@subquestiontext@htmlized\xa{\moodle@subquestiontext}%
     \addto@xml[2]{\moodle@answers@xml}{<subquestion format="html">}%
     \ifx\moodle@subquestiontext\@empty
       \addto@xml[4]{\moodle@answers@xml}{  <text></text>}%
     \else
       \addto@xml[4]{\moodle@answers@xml}{  <text><![CDATA[<p>\moodle@subquestiontext@htmlized</p>]]></text>\ifmoodle@pluginfile\htmlize@embeddedfiletags\fi}%
     \fi
     \trim@spaces@in\moodle@subanswertext
     %\ifmoodle@draganddrop
       \xa\converttohtmlmacro\xa\moodle@subanswertext@htmlized\xa{\moodle@subanswertext}%
     %\fi
     \ifmoodle@draganddrop
%       \show\moodle@subanswertext@htmlized
       \addto@xml[4]{\moodle@answers@xml}{  <answer format="html"><text><![CDATA[<p>\moodle@subanswertext@htmlized</p>]]></text>\ifmoodle@pluginfile\htmlize@embeddedfiletags\fi</answer>}%
     \else
%       \show\moodle@subanswertext@htmlized
       \addto@xml[4]{\moodle@answers@xml}{  <answer format="html"><text><![CDATA[\moodle@subanswertext@htmlized]]></text></answer>}%
     \fi
%     \ifx\moodle@feedback\@empty\relax\else
%       \trim@spaces@in\moodle@feedback
%       \xa\converttohtmlmacro\xa\moodle@feedback@html\xa{\moodle@feedback}%
%       \addto@xml[4]{\moodle@answers@xml}{  <feedback
%format="html"><text><![CDATA[<p>\moodle@feedback@html</p>]]></text></feedback>}%
%     \fi
     \addto@xml[2]{\moodle@answers@xml}{</subquestion>}%
   }%

% WRITING QUESTION TO \XML\ FILE
\gdef\writematchingquestion{%
  \ifmoodle@draganddrop
    \writetomoodle{<question type="ddmatch">}%
  \else
    \writetomoodle{<question type="matching">}%
  \fi
    \moodle@writecommondata%
%    \moodle@writesingle% %irrelevant for the matching type
    \moodle@writeshuffle%
%    \moodle@writeanswernumbering% %irrelevant for the matching type
}%
%    \end{macrocode}
%
% \subsection{Cloze Questions}
% Because cloze questions are so complicated, they get their own section of code.
% The cloze strategy is as follows.
%
% All subquestions show up as part of the question body text.
% For each type of subquestion, we have a cloze-version environment
% that actually has 2 versions, depending on whether we are doing LaTeX or \XML\ processing.
% So the main environment is quite typical:
% \begin{enumerate}
%   \item Process the body as LaTeX.
%         During this run, a |\begin{multi}| etc.~will be processed for display onscreen.
%   \item Then save the body as the questiontext for \XML.
%         During this run, a |\begin{multi}| etc.~will be parsed and turned into
%         cloze code as part of the \XML\ questiontext.
% \end{enumerate}
%    \begin{macrocode}
% LATEX PROCESSING
% SAVING ANSWERS TO MEMORY

\newif\ifmoodle@clozemode
\moodle@clozemodefalse
\NewEnviron{cloze}[2][]{%
  \bgroup
    \setkeys{moodle}{default grade=1}%
    \setkeys{moodle}{#1,questionname={#2}}%
    % A cloze question won't have any \item's in it, so we just use \BODY.
    \moodle@enableclozeenvironments
    %First, the LaTeX processing.
      \item \moodle@begin@samepage\textbf{\moodle@questionname}
      \ifmoodle@handout
        \moodle@makelatextag@qtype{cloze}
      \else
        \moodle@latex@writetags
        \par
        \noindent
        \moodle@makelatextag@qtype{cloze}
        \moodle@makelatextag@value{penalty}{penalty}
      \fi
      \par
      \noindent
      \BODY
      \edef\moodle@generalfeedback{\expandonce\moodle@feedback}
      %\csname moodle@cloze@latexprocessing\endcsname
      \ifmoodle@handout\else
        \ifx\moodle@generalfeedback\@empty\relax\else%
          \fbox{\parbox{.96\linewidth}{\emph{\moodle@generalfeedback}}}%
        \fi
      \fi
      \moodle@end@samepage
    %Now, writing information to memory.
    \@moodle@ifgeneratexml{%
      \xa\questiontext\xa{\BODY}% Save the question text as \HTML.
      \writeclozequestion
    }{}%
  \egroup%
}

\def\moodle@enableclozeenvironments{%
  \let\multi=\clozemulti
  \let\endmulti=\endclozemulti
  \let\numerical=\clozenumerical
  \let\endnumerical=\endclozenumerical
  \let\shortanswer=\clozeshortanswer
  \let\endshortanswer=\endclozeshortanswer
}

% WRITING QUESTION TO \XML\ FILE
\gdef\writeclozequestion{%
  \writetomoodle{<question type="cloze">}%
    \moodle@writequestionname%
    \moodle@writequestiontext%
    \moodle@writegeneralfeedback%
    \moodle@writepenalty%
    \moodle@writehidden%
    \moodle@writetags%
  \writetomoodle{</question>}%
}%
%    \end{macrocode}
%
% Utility to check the validity of fractions and points inside cloze subquestions.
%    \begin{macrocode}
\newdimen\moodle@tmpdim
\def\moodle@checkclozefraction{%
  \moodle@tmpdim=\moodle@fraction pt\relax
  % Rounding is performed in three steps:
  % 1) round(x)=floor(x+1/2)
  \advance\moodle@tmpdim by .5pt\relax
  % 2) floor(x)=trunc(x) if x>0, and floor(x)=trunc(x-1) if x<0
  \ifdim\moodle@tmpdim<0pt\relax\advance\moodle@tmpdim by -1pt\fi
  % 3) Truncation because the quantum of TeX dimensions is 1/65536-th of a point.
  \divide\moodle@tmpdim by 65536\relax\multiply\moodle@tmpdim by 65536\relax
  \def\moodle@tmpval{\strip@pt\moodle@tmpdim}%
  \ifdim\moodle@tmpdim=\moodle@fraction pt\else
    \PackageWarning{moodle}{in cloze questions, fractions must be integers. Rounding \moodle@fraction\space to \moodle@tmpval}%
  \fi
  \xdef\moodle@fraction{\moodle@tmpval}% in any case, take eventual decimals out (e.g. change 10.0 for 10)
}%

\def\moodle@checkclozegrade{%
  \moodle@tmpdim=\csname moodle@default grade\endcsname pt\relax
  \ifdim\moodle@tmpdim<1 pt\relax
    \PackageWarning{moodle}{for cloze questions, the default grade must be a positive integer. Changing the default grade from \csname moodle@default grade\endcsname\space to 1}%
    \csgdef{moodle@default grade}{1}%
  \else % Rounding is performed in three steps:
    % 1) round(x)=floor(x+1/2)
    \advance\moodle@tmpdim by .5pt\relax
    % 2) floor(x)=trunc(x) if x>0, and floor(x)=trunc(x-1) if x<0
    \ifdim\moodle@tmpdim<0pt\relax\advance\moodle@tmpdim by -1pt\fi
    % 3) Truncation because the quantum of TeX dimensions is 1/65536-th of a point.
    \divide\moodle@tmpdim by 65536\relax\multiply\moodle@tmpdim by 65536\relax
    \def\moodle@tmpval{\strip@pt\moodle@tmpdim}%
    \ifdim\moodle@tmpdim=\csname moodle@default grade\endcsname pt\else
      \PackageWarning{moodle}{for cloze questions, the default grade must be a positive integer. Rounding \csname moodle@default grade\endcsname\space to \moodle@tmpval}%
    \fi
    \csxdef{moodle@default grade}{\moodle@tmpval}% in any case, take eventual decimals out (e.g. change 10.0 for 10)
  \fi
}%
%    \end{macrocode}
%
% \subsubsection{Cloze Multiple Choice Questions}
%
%    \begin{macrocode}
\NewEnviron{clozemulti}[1][]{%
  \bgroup
    \setkeys{moodle}{default grade=1}%
    \setkeys{moodle}{#1}%
    \moodle@checkclozegrade
    \expandafter\gatheritems\xa{\BODY}%
    \let\moodle@questionheader=\gatheredheader
    \ifhtmlizer@active
      %HTML version
      \def\moodle@clozemulti@output{}%
      \xa\g@addto@macro\xa\moodle@clozemulti@output\xa{\moodle@questionheader}%
      \def\clozemulti@coding{}%
      \edef\clozemulti@coding{\csname moodle@default grade\endcsname:}%
      \ifmoodle@multiple
        \moodle@WarningOrError{3}{5}{Cloze Multiresponse}%
        \g@addto@macro{\clozemulti@coding}{MULTIRESPONSE}%
      \else
        \g@addto@macro{\clozemulti@coding}{MULTICHOICE}%
      \fi
      \ifcase\moodle@multi@mode\relax
         % Case 0: dropdown box style
         \ifmoodle@shuffle
           \g@addto@macro{\clozemulti@coding}{_}%
         \fi
      \or
        % Case 1: vertical style
        \ifmoodle@multiple
          \PackageError{moodle}{Vertical mode (dropdown box) incompatible with multiresponse.}
        \else
          \g@addto@macro{\clozemulti@coding}{_V}%
        \fi
      \else
        % Case 2: horizontal radio buttons
        \g@addto@macro{\clozemulti@coding}{_H}%
      \fi
      \ifmoodle@shuffle
        \moodle@WarningOrError{3}{0}{Cloze Multi Shuffling}
        \g@addto@macro{\clozemulti@coding}{S:}%
      \else
        \g@addto@macro{\clozemulti@coding}{:}%
      \fi
      \bgroup
        \setkeys{moodle}{feedback={}}%
        \loopthroughitemswithcommand{\saveclozemultichoiceanswer}%
      \egroup
      %\xa\g@addto@macro\xa\clozemulti@coding\xa{\clozerbrace}%
      \xa\g@addto@macro\xa\moodle@clozemulti@output\xa{\xa\clozelbrace\clozemulti@coding\clozerbrace}%
      %\show\moodle@clozemulti@output
      \xa\gdef\xa\htmlize@afteraction@hook\xa{\moodle@clozemulti@output}%
      \def\endclozemulti@code{\htmlize@patchendenvironment}%
    \else
      %LaTeX version
      \global\advance\moodle@totalmarks by \csname moodle@default grade\endcsname pt
      \moodle@questionheader% %Any introductory text just continues to be typeset.
      \par
      \noindent
      \moodle@makelatextag@qtype{multi}
      \ifmoodle@handout\else
        \moodle@makelatextag@value{default grade}{marked out of}
        \moodle@makelatextag@multi
      \fi
      \def\cloze@multichoice@table@text{}%
      \ifmoodle@multiple\moodle@InspectMultipleAnswers\fi
      \ifmoodle@handout\NewList{answerlist}\fi
      %\let\moodle@feedback=\@empty
      \loopthroughitemswithcommand{\moodle@print@clozemultichoice@answer}%
      \ifmoodle@handout
        \ifmoodle@shuffle
          \let\moodle@clozemult@loop=\ForEachRandomItem
        \else
          \let\moodle@clozemult@loop=\ForEachFirstItem
        \fi
        \moodle@clozemult@loop{answerlist}{Answer}{
          \xdef\cloze@multichoice@table@text{\expandonce\cloze@multichoice@table@text\expandonce\Answer}%
        }%
      \fi
      \ifcase\moodle@multi@mode\relax
        %Case 0: dropdown box style
        \par\noindent
        \ifmoodle@handout
          \begin{tabular}[t]{|p{.45\linewidth}|}
        \else
          \begin{tabular}[t]{|p{.45\linewidth}|p{.45\linewidth}|}
%          answer & feedback \\\hline\hline
        \fi
          \firsthline% (\firsthline is from the array package.)
          \cloze@multichoice@table@text%
        \end{tabular}%
        \par%
      \or
        %Case 1: vertical style
        \par\noindent
        \begin{itemize}\setlength\itemsep{0pt}\setlength\parskip{0pt}%
          \cloze@multichoice@table@text%
        \end{itemize}%
        \par%
      \else
        %Case 2: horizontal radio buttons
        \par{\cloze@multichoice@table@text}\par%
      \fi
      \def\endclozemulti@code{\relax}%
    \fi
    \passvalueaftergroup\endclozemulti@code%
    \passvalueaftergroup\htmlize@afteraction@hook%
  \egroup%
}[\endclozemulti@code]%

\long\def\moodle@print@clozemultichoice@answer#1{%
%\bgroup
  \let\moodle@feedback=\@empty
  \moodle@print@clozemultichoice@answer@int#1 \@rdelim%
%\egroup
}%
\newcommand\moodle@print@clozemultichoice@answer@int[1][]{%
  \let\moodle@fraction\@empty%
  \setkeys{moodle}{#1}%
  \moodle@print@clozemultichoice@answer@int@int%
}%
\long\def\moodle@print@clozemultichoice@answer@int@int#1#2\@rdelim{%
  \def\moodle@answertext{}%
  % Case 0: "(answer) \\ \hline"
  % Case 1: "\item (answer)"
  % Case 2: "$\bullet~$(answer)\hfill"
  \ifcase\moodle@multi@mode\relax
    \relax% Case 0
  \or
    \g@addto@macro\moodle@answertext{\item}% Case 1
  \else
    \g@addto@macro\moodle@answertext{$\bullet~$}% Case 2
  \fi
  \def\test@i{#1}%
  \ifx\test@i\@star
    \setkeys{moodle}{fraction=100}%
    \g@addto@macro\moodle@answertext{#2}%
  \else
    \g@addto@macro\moodle@answertext{#1#2}%
  \fi
  \trim@spaces@in\moodle@answertext
  \trim@spaces@in\moodle@answertext
  \ifmoodle@handout\else
    \ifmoodle@single%
      \ifx\moodle@fraction\@empty\relax%
        \ifdim0pt<\moodle@sanction pt\relax
          \setkeys{moodle}{fraction=-\moodle@sanction}%
        \else
          \setkeys{moodle}{fraction=0}%
        \fi
      \fi
      \moodle@checkclozefraction
      \ifx\moodle@fraction\@hundred%
        \trim@spaces@in\moodle@answertext%
        \g@addto@macro\moodle@answertext{$~\checkmark$}%
      \else
        \ifdim0pt=\moodle@fraction pt\relax\else%
          \xdef\moodle@answertext{\expandonce\moodle@answertext$~(\moodle@fraction\%)$}%
        \fi
      \fi
    \else% multiple
      \ifx\moodle@fraction\@empty\relax%
        \ifmoodle@AdvancedScoreMode
          \setkeys{moodle}{fraction=0}%
        \else
          \setkeys{moodle}{fraction=-\moodle@AutoScore}%
        \fi
      \else
        \ifmoodle@AdvancedScoreMode
          \moodle@checkclozefraction
          \ifdim0pt<\moodle@fraction pt\relax
            \moodle@autoscore@tmp=\moodle@PositiveScoreFactor\relax%
            \multiply\moodle@autoscore@tmp by \moodle@fraction\relax%
            \xdef\moodle@fraction{\strip@pt\moodle@autoscore@tmp}%
          \fi
        \else
          \setkeys{moodle}{fraction=\moodle@AutoScore}%
        \fi
      \fi
      \xdef\moodle@answertext{\expandonce\moodle@answertext$~(\moodle@fraction\%)$}%
    \fi
  \fi
  \ifcase\moodle@multi@mode\relax
  % Case 0
    \ifmoodle@handout\else
      \xdef\moodle@answertext{\expandonce\moodle@answertext &\expandonce\emph{\expandonce\moodle@feedback}}%
    \fi
    \g@addto@macro\moodle@answertext{\\\hline}
  \or % Case 1
    \ifmoodle@handout\else
      \ifx\moodle@feedback\@empty\relax\else
        \xdef\moodle@answertext{\expandonce\moodle@answertext \moodle@preFeedback \expandonce\emph{$\rightarrow$ \expandonce\moodle@feedback}}%
      \fi
    \fi
  \else % otherwise
    \ifmoodle@handout\else
      \ifx\moodle@feedback\@empty\relax\else
        \xdef\moodle@answertext{\expandonce\moodle@answertext\,\expandonce\emph{$\rightarrow$ \expandonce\moodle@feedback}}%
      \fi
    \fi
    \g@addto@macro\moodle@answertext{\hfill}% Case 2
  \fi
  \ifmoodle@handout
    \def\temp{\InsertLastItem{answerlist}}%
    \xa\temp\xa{\moodle@answertext}%
  \else
    \xdef\cloze@multichoice@table@text{\expandonce\cloze@multichoice@table@text\expandonce\moodle@answertext}%
  \fi
}%

\long\def\saveclozemultichoiceanswer#1{%
  \bgroup
    \saveclozemultichoiceanswer@int#1 \moodle@answer@rdelim
  \egroup
}%
\newcommand\saveclozemultichoiceanswer@int[1][]{%
  \setkeys{moodle}{fraction=0,#1}%
  \saveclozemultichoiceanswer@int@int%
}%
\long\def\saveclozemultichoiceanswer@int@int#1#2\moodle@answer@rdelim{%
  \def\test@i{#1}%
  \ifgatherbeginningofloop\else
    \xa\gdef\xa\clozemulti@coding\xa{\clozemulti@coding\clozetilde}% separator between answers
  \fi
  \ifx\test@i\@star
    \setkeys{moodle}{fraction=100}%
    \def\moodle@answertext{#2}%
  \else
    \def\moodle@answertext{#1#2}%
  \fi
  \trim@spaces@in\moodle@answertext
  \trim@spaces@in\moodle@answertext
  \ifx\moodle@fraction\@hundred
    \g@addto@macro\clozemulti@coding{\clozecorrect}%
  \else
    \moodle@checkclozefraction
    \ifdim0pt=\moodle@fraction pt\relax\else
      \xdef\clozemulti@coding{\expandonce\clozemulti@coding\otherpercent\moodle@fraction\otherpercent}%
    \fi
  \fi
  \xdef\clozemulti@coding{\expandonce\clozemulti@coding\expandonce\moodle@answertext}%
  \ifx\moodle@feedback\@empty\else
    \xdef\clozemulti@coding{\expandonce\clozemulti@coding\otherbackslash\otherhash\expandonce\moodle@feedback}%
  \fi
}%
%    \end{macrocode}
%
% \subsubsection{Cloze Numerical Questions}
%
%    \begin{macrocode}
\NewEnviron{clozenumerical}[1][]{%
  \bgroup
    \expandafter\gatheritems\expandafter{\BODY}%
    \let\moodle@questionheader=\gatheredheader
    \setkeys{moodle}{default grade=1}%
    \setkeys{moodle}{#1}%
    \moodle@checkclozegrade
    \ifhtmlizer@active
      %HTML version
      \def\moodle@clozenumerical@output{}%
      \xa\g@addto@macro\xa\moodle@clozenumerical@output\xa{\moodle@questionheader}%
      \def\clozenumerical@coding{}%
      \edef\clozenumerical@coding{\csname moodle@default grade\endcsname:NUMERICAL:}%
      \bgroup
        \setkeys{moodle}{feedback={}}%
        \loopthroughitemswithcommand{\saveclozenumericalanswer}%
      \egroup
      %\xa\g@addto@macro\xa\clozenumerical@coding\xa{\otherrbrace}%
      \xa\g@addto@macro\xa\moodle@clozenumerical@output\xa{\xa\clozelbrace\clozenumerical@coding\clozerbrace}%
      \xa\gdef\xa\htmlize@afteraction@hook\xa{\moodle@clozenumerical@output}%
      \def\endclozenumerical@code{\htmlize@patchendenvironment}%
    \else
      %LaTeX version
      \global\advance\moodle@totalmarks by \csname moodle@default grade\endcsname pt
      \moodle@questionheader% %Any introductory text just continues to be typeset.
      \par
      \noindent
      \moodle@makelatextag@qtype{numerical}
      \ifmoodle@handout\else
        \moodle@makelatextag@value{default grade}{marked out of}
        \moodle@makelatextag@numerical
        \par
        \noindent
        \def\cloze@numerical@table@text{}%
        \loopthroughitemswithcommand{\moodle@print@clozenumerical@answer}%
        \begin{tabular}[t]{|p{.45\linewidth}|p{.45\linewidth}|}
          \firsthline% (\firsthline is from the array package.)
%         answer & feedback \\\hline\hline
          \cloze@numerical@table@text%
        \end{tabular}%
        \par%
      \fi
      \def\endclozenumerical@code{\relax}%
    \fi
    \passvalueaftergroup\endclozenumerical@code%
    \passvalueaftergroup\htmlize@afteraction@hook%
  \egroup
}[\endclozenumerical@code]%

\def\moodle@print@clozenumerical@answer#1{%
  \let\moodle@feedback=\@empty
  \bgroup
    \moodle@print@clozenumerical@answer@int#1\@rdelim
  \egroup
}%
\newcommand\moodle@print@clozenumerical@answer@int[1][]{%
  \setkeys{moodle}{#1}%
  \moodle@print@clozenumerical@answer@int@int%
}%
\def\moodle@zero{0}%
\def\moodle@print@clozenumerical@answer@int@int#1\@rdelim{%
  \ifx\moodle@fraction\@hundred
    \def\moodle@clozenumericalprint@fraction{$~\checkmark$}%
  \else
    \moodle@checkclozefraction
    \edef\moodle@clozenumericalprint@fraction{$(~\moodle@fraction\%)$}%
  \fi
  \ifx\moodle@zero\moodle@tolerance%
    \def\moodle@clozenumericalprint@tolerance{}%
  \else
    \edef\moodle@clozenumericalprint@tolerance{$\,\pm\,$\moodle@printnum{\expandonce\moodle@tolerance}}%
  \fi
  \def\test@i{#1}%
  \trim@spaces@in\test@i
  \ifx\test@i\@star
    \xdef\moodle@clozenumericalprint@line{\expandonce\test@i~\moodle@clozenumericalprint@fraction & \expandonce\emph{\expandonce\moodle@feedback}}%
  \else
    \xdef\moodle@clozenumericalprint@line{\moodle@printnum{\expandonce\test@i}\expandonce\moodle@clozenumericalprint@tolerance~\moodle@clozenumericalprint@fraction & \expandonce\emph{\expandonce\moodle@feedback}}%
  \fi
  \xa\g@addto@macro\xa\cloze@numerical@table@text\xa{\moodle@clozenumericalprint@line \\\hline}%
}%

\def\saveclozenumericalanswer#1{%
  \bgroup
    \saveclozenumericalanswer@int#1\moodle@answer@rdelim
  \egroup
}%
\newcommand\saveclozenumericalanswer@int[1][]{%
  \setkeys{moodle}{fraction=100,#1}%                  %%%%%% DEFAULT VALUE IS 100%
  \saveclozenumericalanswer@int@int%
}%
\def\saveclozenumericalanswer@int@int#1\moodle@answer@rdelim{%
  \ifgatherbeginningofloop\else
    \xa\gdef\xa\clozenumerical@coding\xa{\clozenumerical@coding\clozetilde}% separator between answers
  \fi
  \def\moodle@answertext{#1}%
  \trim@spaces@in\moodle@answertext
  \ifx\moodle@fraction\@hundred
    \g@addto@macro\clozenumerical@coding{\clozecorrect}%
  \else
    \moodle@checkclozefraction
    \ifdim0pt=\moodle@fraction pt\relax\else
      \xdef\clozenumerical@coding{\expandonce\clozenumerical@coding\otherpercent\moodle@fraction\otherpercent}%
    \fi
  \fi
  \ifx\moodle@answertext\@star
    \xdef\clozenumerical@coding{\expandonce\clozenumerical@coding\moodle@answertext}%
  \else
    \xdef\clozenumerical@coding{\expandonce\clozenumerical@coding\moodle@answertext:\moodle@tolerance}%
  \fi
  \ifx\moodle@feedback\@empty\else
    %\trim@spaces@in\moodle@feedback
    \xdef\clozenumerical@coding{\expandonce\clozenumerical@coding\otherbackslash\otherhash\expandonce\moodle@feedback}%
  \fi
}%
%    \end{macrocode}
%
% \subsubsection{Cloze Short Answer Questions}
%
%    \begin{macrocode}
\NewEnviron{clozeshortanswer}[1][]{%
  \bgroup
    \expandafter\gatheritems\expandafter{\BODY}%
    \let\moodle@questionheader=\gatheredheader
    \setkeys{moodle}{default grade=1}%
    \setkeys{moodle}{#1}%
    \moodle@checkclozegrade
    \ifhtmlizer@active
      %HTML version
      \def\moodle@clozeshortanswer@output{}%
      \xa\g@addto@macro\xa\moodle@clozeshortanswer@output\xa{\moodle@questionheader}%
      \def\clozeshortanswer@coding{}%
      \ifmoodle@usecase
        \edef\clozeshortanswer@coding{\csname moodle@default grade\endcsname:SHORTANSWER_C:}%
      \else
        \edef\clozeshortanswer@coding{\csname moodle@default grade\endcsname:SHORTANSWER:}%
      \fi
      \bgroup
        \setkeys{moodle}{feedback={}}%
        \loopthroughitemswithcommand{\saveclozeshortansweranswer}%
      \egroup
      %\xa\g@addto@macro\xa\clozeshortanswer@coding\xa{\otherrbrace}%
      \xa\g@addto@macro\xa\moodle@clozeshortanswer@output\xa{\xa\clozelbrace\clozeshortanswer@coding\clozerbrace}%
      \xa\gdef\xa\htmlize@afteraction@hook\xa{\moodle@clozeshortanswer@output}%
      \def\endclozeshortanswer@code{\htmlize@patchendenvironment}%
    \else
      %LaTeX version
      \global\advance\moodle@totalmarks by \csname moodle@default grade\endcsname pt
      \moodle@questionheader% %Any introductory text just continues to be typeset.
      \par
      \noindent
      \moodle@makelatextag@qtype{shortanswer}
      \ifmoodle@handout\else
        \moodle@makelatextag@value{default grade}{marked out of}
        \moodle@makelatextag@shortanswer
        \par
        \noindent
        \def\cloze@shortanswer@table@text{}%
        \loopthroughitemswithcommand{\moodle@print@clozeshortanswer@answer}%
        \begin{tabular}[t]{|p{.45\linewidth}|p{.45\linewidth}|}
          \firsthline% (\firsthline is from the array package.)
%         answer & feedback \\\hline\hline
          \cloze@shortanswer@table@text%
        \end{tabular}%
        \par%
      \fi
      \def\endclozeshortanswer@code{\relax}%
    \fi
    \passvalueaftergroup\endclozeshortanswer@code%
    \passvalueaftergroup\htmlize@afteraction@hook%
  \egroup
}[\endclozeshortanswer@code]%

\def\moodle@print@clozeshortanswer@answer#1{%
  \let\moodle@feedback=\@empty
  \bgroup
    \moodle@print@clozeshortanswer@answer@int#1\@rdelim
  \egroup
}%
\newcommand\moodle@print@clozeshortanswer@answer@int[1][]{%
  \setkeys{moodle}{#1}%
  \moodle@print@clozeshortanswer@answer@int@int%
}%
\def\moodle@print@clozeshortanswer@answer@int@int#1\@rdelim{%
  \ifx\moodle@fraction\@hundred
    \def\moodle@clozeshortanswerprint@fraction{$~\checkmark$}%
  \else
    \moodle@checkclozefraction
    \edef\moodle@clozeshortanswerprint@fraction{$~(\moodle@fraction\%)$}%
  \fi
  \xdef\moodle@clozeshortanswerprint@line{#1~\moodle@clozeshortanswerprint@fraction & \expandonce\emph{\expandonce\moodle@feedback}}%
  \xa\g@addto@macro\xa\cloze@shortanswer@table@text\xa{\moodle@clozeshortanswerprint@line \\\hline}%
}%

\def\saveclozeshortansweranswer#1{%
  \bgroup
    \saveclozeshortansweranswer@int#1\moodle@answer@rdelim
  \egroup
}%
\newcommand\saveclozeshortansweranswer@int[1][]{%
  \setkeys{moodle}{fraction=100,#1}%                  %%%%%% DEFAULT VALUE IS 100%
  \saveclozeshortansweranswer@int@int%
}%
\def\saveclozeshortansweranswer@int@int#1\moodle@answer@rdelim{%
  \ifgatherbeginningofloop\else
    \xa\gdef\xa\clozeshortanswer@coding\xa{\clozeshortanswer@coding\clozetilde}% separator between answers
  \fi
  \def\moodle@answertext{#1}%
  \trim@spaces@in\moodle@answertext
  \ifx\moodle@fraction\@hundred
    \g@addto@macro\clozeshortanswer@coding{\clozecorrect}%
  \else
    \moodle@checkclozefraction
    \ifdim0pt=\moodle@fraction pt\relax\else
      \xdef\clozeshortanswer@coding{\expandonce\clozeshortanswer@coding\otherpercent\moodle@fraction\otherpercent}%
    \fi
  \fi
  \xdef\clozeshortanswer@coding{\expandonce\clozeshortanswer@coding\moodle@answertext}%
  \ifx\moodle@feedback\@empty\else
    \xdef\clozeshortanswer@coding{\expandonce\clozeshortanswer@coding\otherbackslash\otherhash\expandonce\moodle@feedback}%
  \fi
}%
%    \end{macrocode}
%
% \section{Converting \LaTeX\ to HTML}
% A lot of work must now be done to convert the \LaTeX\ code
% of a question or answer into \HTML\ code with embedded \TeX\ for math.
%
% \subsection{Catcode Setup}
% First, we create versions of the special characters with catcode 12, ``other.''
%    \begin{macrocode}
{\catcode`\#=12\gdef\otherhash{#}%
 \catcode`\~=12\gdef\othertilde{~}%
 \catcode`\&=12\gdef\otherampersand{&}%
 \catcode`\^=12\gdef\othercaret{^}%
 \catcode`\$=12\gdef\otherdollar{$}%
 \catcode`\%=12\gdef\otherpercent{%}
 \catcode`\[=12\gdef\otherlbracket{[}
 \catcode`\]=12\gdef\otherrbracket{]}}%
{\catcode`\=\string=12\gdef\otherequal{=}}%
{\catcode`\ =12\gdef\otherspace{ }}%
{\ttfamily\catcode`\|=0\catcode`\\=12\relax|gdef|otherbackslash{\}}%
{\catcode`\[=1\catcode`\]=2\catcode`\{=12\catcode`\}=12%
 \gdef\otherlbrace[{]\gdef\otherrbrace[}]\gdef\clozelbrace[{]\gdef\clozerbrace[}]]%

\edef\@otherlbrace{\otherlbrace}%
\edef\@otherrbrace{\otherrbrace}%
\edef\@otherlbracket{\otherlbracket}%
\edef\@otherrbracket{\otherrbracket}%
\edef\@clozelbrace{\clozelbrace}%
\edef\@clozerbrace{\clozerbrace}%
\edef\@otherdollar{\otherdollar}%
\edef\@otherbackslash{\otherbackslash}%
\edef\@othertilde{\othertilde}%
\edef\@otherequal{\otherequal}%
%    \end{macrocode}
%
% Next, we define commands to change catcodes to a suitable verbatim mode
% for transcription.
%
%    \begin{macrocode}
{ \catcode`\[=1\relax
  \catcode`\]=2\relax
  \catcode`\|=0\relax
  |gdef|verbcatcodesweirdest[
    |catcode`\{=12|relax
    |catcode`\}=12|relax
    |catcode`\\=12|relax
  ]%
}%
\def\verbcatcodes{%
  \catcode`\$=12\relax
  \catcode`\&=12\relax
  \catcode`\#=12\relax
  \catcode`\^=12\relax
  \catcode`\_=12\relax
  \catcode`\~=12\relax
  \makeatletter
  \catcode`\%=12\relax
  \catcode`\ =12\relax\catcode\newlinechar=12\verbcatcodesweirdest}%

\def\normalcatcodes{%
  \catcode`\\=0\relax
  \catcode`\{=1\relax
  \catcode`\}=2\relax
  \catcode`\$=3\relax
  \catcode`\&=4\relax
  \catcode\endlinechar=5\relax
  \catcode`\#=6\relax
  \catcode`\^=7\relax
  \catcode`\_=8\relax
  \catcode`\ =10\relax
  \makeatletter% We will be detokenizing and retokenizing internal control sequences, so we need this.
  \catcode`\~=13\relax
  \catcode`\%=14\relax}%

\def\retokenizingcatcodes{%
  %For rescanning previously scanned text, all true comments will already be gone,
  %but % signs may have been inserted by Cloze questions, so we want to treat them as 'other.'
  %
  % TODO: #'s are more worrisome.
  \normalcatcodes
  \catcode`\%=12\relax
}
%    \end{macrocode}
%
% \subsection{Detokenization and Retokenization}
%
% We will be processing a \TeX\ token list.
% Based on its content, sometimes we will want it to be detokenized to
% individual characters, but other times we want it retokenized so that
% \TeX's own parsing mechanism can gather up the parameters of macros.
% We use the e\TeX\ primitive command |\scantokens| to do this.
%
% The following code (catcodes, groupings and all) defines a |\scantokens@to@macro| macro.
% That will assemble and disassemble strings of tokens
% using any changing schemes of catcodes we desire.
%
% We define |^^A| to be |\gdef\stm@saved|, while |^^B| and |^^C| are substitutes for |{| and |}|, respectively.
% This permits us to define |\scantokens@to@macro| in a peculiar catcode regime.
%    \begin{macrocode}
\begingroup
  \catcode`\^^A=13\gdef^^A{\gdef\stm@saved}%
  \catcode`\^^B=1\catcode`\^^C=2\relax
  \long\gdef\scantokens@to@macro#1#2#3{%
    % #1 = control sequence to be defined
    % #2 = command to change catcodes, e.g. \verbcatcodes,
    %      and define any command sequences
    % #3 = text to be retokenized and saved into #1.
    \bgroup
      \def\texttorescan{#3}%
      \catcode`\^^A=13\catcode`\^^B=1\catcode`\^^C=2\relax
      \xa\def\xa\arg\xa{\xa^^A\xa^^B\texttorescan^^C}%
      #2%
      \catcode\endlinechar=9\relax%
      %\scantokens always sees an end-of-line character at its end and converts it to a space.
      %The catcode change sets \scantokens to ignore end-of-line chars.
      %In practice, we're always calling \scantokens on previously scanned text anyway,
      %so we won't miss any real end-of-line chars, since they were already converted to spaces.
      \xa\scantokens\xa{\arg}%
    \egroup
    \xa\def\xa#1\xa{\stm@saved}%
  }%
\endgroup%

\long\def\ultradetokenize@to@macro#1#2{%
  \scantokens@to@macro#1{\verbcatcodes}{#2}%
}%
\def\retokenizenormal@to@macro#1#2{%
  \scantokens@to@macro#1{\retokenizingcatcodes}{#2}%
}%
%    \end{macrocode}
%
% \subsection{Level-Tracking}
%
% \subsubsection{TeX groups}
% While parsing, we'll need to keep track of how deeply nested in \TeX\ groups we are.
%    \begin{macrocode}
\newcount\grouplevel
%    \end{macrocode}
%
% \subsubsection{Math mode}
% While parsing, we'll need to keep track of whether
% we are in math mode (and how many levels deep the math mode might be nested).
%    \begin{macrocode}
\newcount\moodle@mathmodedepth
\moodle@mathmodedepth=0\relax
\def\moodle@ifmathmode#1#2{%
  \ifnum\moodle@mathmodedepth>0\relax
    #1%
  \else
    #2%
  \fi
}%
%    \end{macrocode}
%
% \subsubsection{Nested Lists}
% While parsing, we'll need to keep track levels of nested list.
%    \begin{macrocode}
\newcount\moodle@listdepth
\moodle@listdepth=0\relax

%    \end{macrocode}
%
% \subsection{Separation}
%
% This code separates a string of tokens into two parts.
% Its parameters, |#1#2|, consist of tokenized text,
% plus one terminal |\@htmlize@stop|.
% We are trying to break up the text into its first group and the remainder.
% This |\@htmlize@stop| is needed in case |#2| has the form ``|{...}|'',
% since we don't want \TeX\ to strip the braces off.
% Thus |\@htmlize@remainder| will definitely end in ``|\@htmlize@stop|''.
%    \begin{macrocode}
\long\def\htmlize@grabblock#1#2\htmlize@rdelim@ii{%
  \long\def\htmlize@blockinbraces{#1}%
  \long\def\htmlize@remainder{#2}%
}%
%    \end{macrocode}
%
% The next line defines the macro |\@htmlize@stop@detokenized| to contain
% the string of tokens |\@htmlize@stop|, all of category code 12 (other) or 10 (letter).
% We'll need this for comparison purposes later.
%    \begin{macrocode}
\ultradetokenize@to@macro\@htmlize@stop@detokenized{\@htmlize@stop}%
%    \end{macrocode}
%
% The next line creates the macro |\htmlize@remove@stopcode|,
% which removes the characters ``|\@htmlize@stop |'' from the end of a
% detokenized sequence.
% Its syntax when called is simply |\htmlize@remove@stopcode |\meta{material},
% with no delimiters, since the ``|\@htmlize@stop |'' is itself the delimiter.
%    \begin{macrocode}
\xa\def\xa\htmlize@remove@stopcode\xa#\xa1\@htmlize@stop@detokenized{#1}%
%    \end{macrocode}
%
% \subsection{Main Code: the HTMLizer}
%
%    \begin{macrocode}
\newif\ifhtmlizer@active
\htmlizer@activefalse
\newif\ifhtmlize@actioncs
\newif\ifhtmlize@expandcs
\newif\ifhtmlize@passcs

\long\def\@@begin@cs{\begin}%
\def\@@htmlize@stop{\@htmlize@stop}%

\long\def\converttohtmlmacro#1#2{%
  \grouplevel=0\relax
  \def\htmlize@output{}%
  \ifmoodle@pluginfile
    \gdef\htmlize@embeddedfiletags{}%
  \fi
  \htmlizer@activetrue%
  \converttohtml@int{#2}%
  \htmlizer@activefalse%
  \let#1=\htmlize@output\relax
}

\long\def\converttohtml@int#1{%
  \advance\grouplevel by 1\relax
  \bgroup
    \ultradetokenize@to@macro\htmlize@texttoscan{#1}%
    \xa\htmlize@recursive@i\htmlize@texttoscan\@htmlize@stop\@htmlize@stop\@htmlize@stop\htmlize@rdelim@i%
  \egroup
  \advance\grouplevel by -1\relax
}%

\def\@lt{<}%
\def\@gt{>}%
\def\@dash{-}%
\def\@dq{"}%

\long\def\htmlize@recursive@i#1#2#3\htmlize@rdelim@i{%
  % #1#2#3 is a sequence of tokens.  All should be categories 11 (letter) or 12 (other).
  % It terminates with the control sequences \@htmlize@stop\@htmlize@stop\@htmlize@stop.
  %\long\def\ds{(#1|#2|#3)}\show\ds
  \def\test@i{#1}%
  \def\test@ii{#2}%
  \ifx\test@i\@@htmlize@stop
    \let\htmlize@next@i=\relax
  \else
    \ifx\test@i\@otherlbrace%
      \ifx\test@ii\@otherrbrace%
        \moodle@ifmathmode{\g@addto@macro\htmlize@output{\otherlbrace\otherrbrace}}{}%
        \def\htmlize@next@i{\htmlize@recursive@i#3\htmlize@rdelim@i}%
      \else
        \xa\g@addto@macro\xa\htmlize@output\xa{\otherlbrace}%
        \bgroup
          \normalcatcodes
        %We need to rescan the input as TeX code,
        % so TeX can automatically pull off the first group in braces.
        % First, let's get rid of the terminal \@htmlize@stop codes.
          {\def\@htmlize@stop{}\xdef\htmlize@scrap{#1#2#3}}%
          \let\htmlize@text@to@rescan=\htmlize@scrap%
        % Next, we retokenize the code.
          \xa\retokenizenormal@to@macro\xa\htmlize@rescanned\xa{\htmlize@text@to@rescan}%
        % Now break it up into two pieces.
          \xa\htmlize@grabblock\htmlize@rescanned\@htmlize@stop\htmlize@rdelim@ii%
        % The first piece, \htmlize@blockinbraces, will be passed as a unit to \converttohtml@int.
        % The second part, \htmlize@remainder, will continue at this depth of grouping.
        % Therefore we'll detokenize \htmlize@remainder here.
          \xa\ultradetokenize@to@macro\xa\htmlize@remainder@detokenized\xa{\htmlize@remainder}%
          \edef\htmlize@remainder@detokenized{\xa\htmlize@remove@stopcode\htmlize@remainder@detokenized}%
        %
        % Now build \htmlize@next@i.
        % When done, should look like
        %   \converttohtml@int{\htmlize@blockinbraces}%
        %   \g@addto@macro\htmlize@output{\otherrbrace}%
        %   \htmlize@recursive@i\htmlize@remainder@detokenized\@htmlize@stop\@htmlize@stop\@htmlize@stop\htmlize@rdelim@i%
        % but with all three arguments expanded.
        % Note that we are running
          \gdef\htmlize@scrap{\converttohtml@int}%
          \xa\g@addto@macro\xa\htmlize@scrap\xa{\xa{\htmlize@blockinbraces}}%
          \g@addto@macro\htmlize@scrap{\g@addto@macro\htmlize@output}%
          \ifmoodle@clozemode
            \xa\g@addto@macro\xa\htmlize@scrap\xa{\xa{\otherbackslash\otherrbrace}}%
%          \moodle@ifmathmode{\xa\g@addto@macro\xa\htmlize@scrap\xa{\xa{\otherbackslash}}}%
%                            {}%
          \else
            \xa\g@addto@macro\xa\htmlize@scrap\xa{\xa{\otherrbrace}}%
          \fi
          \g@addto@macro\htmlize@scrap{\htmlize@recursive@i}%
          \xa\g@addto@macro\xa\htmlize@scrap\xa{\htmlize@remainder@detokenized\@htmlize@stop\@htmlize@stop\@htmlize@stop\htmlize@rdelim@i}%
        % Okay, that's done.  It's stored in a global macro.
        % Now we get it out of this group.
        \egroup
        \let\htmlize@next@i=\htmlize@scrap
      \fi
    \else
      \ifx\test@i\@otherdollar%
        % Math shift character.
        \ifx\test@ii\@otherdollar
          % Double dollar sign, so we're entering display math mode.
          % We grab everything between $$...$$, sanitize it, and add it verbatim to
          % our output.
          \htmlize@displaymathshift@replace#1#2#3\htmlize@rdelim@iii%
        \else
          % Single dollar sign, so we're entering inline math mode.
          % We grab everything between $...$, sanitize it, and add it verbatim to
          % our output.
          \htmlize@inlinemathshift@replace#1#2#3\htmlize@rdelim@iii%
        \fi% \ifx\test@ii\@otherdollar
        % Now we resume work.
        % The \htmlize@xxxxxxmathshift@replace macro stored the remaining text in \htmlize@remaining@text.
        % Note that since we never detokenized and retokenized #1#2#3,
        % \htmlize@remaining@text still includes the terminating \@htmlize@stop\@htmlize@stop\@htmlize@stop.
        \def\htmlize@next@i{\xa\htmlize@recursive@i\htmlize@remaining@text\htmlize@rdelim@i}%
      \else
        \ifx\test@i\@otherbackslash%
          % Control sequence.  Oh boy.
          % There are three possible things to do:
          % 1. Retokenize everything, so we get a token list.
          %    Expand this control sequence, the first one in the list,
          %    to obtain a new token list.  Then resume processing that list.
          %    Examples: \def\emph#1{<EM>#1</EM>}, \def\rec#1{\frac{1}{#1}}, \def\inv{^{-1}}
          %              \& --> &amp; \# --> #; etc.
          %    Environments: \begin{center}...\end{center} --> <CENTER>...</CENTER>
          % 2. Retokenize everything, so we get a token list.
          %    Let this first command (with its parameters) ACT.
          %    This may involve work in TeX's stomach (e.g., with counters)
          %    or with external files (e.g., image processing).
          %    The command may directly add material to \htmlize@output,
          %    but it should not typeset anything and should vanish from the
          %    input stream when it is done.
          %       When it's done, we somehow need to detokenize and resume
          %    processing the remainder of the input stream.
          %       Only commands explicitly crafted (or modified) to work
          %    with moodle.sty can possibly do all this!
          %    Examples: (modified) \includegraphics
          %    Environments: \begin{clozemulti}, \begin{enumerate}
          % 3. Ignore that it's a command.  Pass it right on as a character
          %    sequence to \htmlize@output.
          %    Examples: \alpha, \frac
          %    Environments: \begin{array}
          %
          % #2 is only for items on a specific list.
          % #1 is anything that runs in TeX's mouth.
          %    We could keep a list and give users a way to add to it.
          %    I could also try expanding macros, using \ifcsmacro from etoolbox.sty
          %
          % The first step is to figure out what control sequence we're dealing with.
          % First, let's get rid of the terminal \@htmlize@stop codes.
          {\def\@htmlize@stop{}\xdef\htmlize@scrap{#1#2#3}}%
          \let\htmlize@text@to@rescan=\htmlize@scrap%
          % Next, we retokenize the code.
          \xa\retokenizenormal@to@macro\xa\htmlize@rescanned\xa{\htmlize@text@to@rescan}%
          % Now break it up into two pieces.
          \xa\htmlize@grabblock\htmlize@rescanned\@htmlize@stop\htmlize@rdelim@ii%
          \let\@htmlize@cs\htmlize@blockinbraces%
          \edef\htmlize@cs@string{\xa\string\@htmlize@cs}%
          % The first piece, \htmlize@blockinbraces, will contain the single token in \@htmlize@cs.
          % We'll need to keep the second part, \htmlize@remainder, since it probably
          % contains arguments to the cs in \@htmlize@cs.
          %\edef\ds{Encountered '\xa\string\@htmlize@cs'}\show\ds
          %
          % N.B. that \@htmlize@cs is a macro *containing* a single control sequence.
          % This is good for testing with \ifx.
          % \htmlize@cs@string contains the cs as a string, e.g., the characters "\emph".
          %
          \ifx\@htmlize@cs\@@begin@cs
            %This is a \begin.  Begin environment-handling routine.
            %
            % Grab the first {...} from \htmlize@remainder, which is the argument
            % to \begin.
            \xa\htmlize@grabblock\htmlize@remainder\@htmlize@stop\htmlize@rdelim@ii%
            \let\htmlize@envname=\htmlize@blockinbraces%
            %We do not need the rest, so we won't pay any attention to the new
            %content of \htmlize@remainder.
            %
            %Now environments are non-expandable,
            %so there are only two possibilities: action or pass.
            \xa\ifinlist\xa{\htmlize@envname}{\htmlize@env@actionlist}%
              {% Action environment!
                %\bgroup
                  %\def\ds{Encountered active environment \string\begin\{{\htmlize@envname}\}}\show\ds
                  \def\htmlize@next@i{\xa\htmlize@do@actionenv\htmlize@rescanned\@htmlize@stop\htmlize@actionsequence@rdelim}%
                %The \bgroup is to active the environments.
                %The matching \egroup is found in \htmlize@do@actionenv.
              }{%An environment to pass to the HTML
                %We just pass the backslash from "\begin" and move on.
                \g@addto@macro\htmlize@output{#1}%
                \def\htmlize@next@i{\htmlize@recursive@i#2#3\htmlize@rdelim@i}%
              }%
          \else%
            %This is not an environment.  Begin macro-handling routine.
            \htmlize@actioncsfalse
            \htmlize@expandcsfalse
            \htmlize@passcsfalse
            \xa\ifinlist\xa{\htmlize@cs@string}{\htmlize@cs@actionlist}%
              {%Action sequence!
               \htmlize@actioncstrue}%
              {% Not action sequence!
               \xa\ifinlist\xa{\htmlize@cs@string}{\htmlize@cs@expandlist}%
                 {%CS to be expanded!
                  \htmlize@expandcstrue%
                 }%
                 {%CS to be transcribed to XML
                  \htmlize@passcstrue%
                 }%
              }%
            %Now exactly one of \ifhtmlize@actioncs, \ifhtmlize@expandcs, and \ifhtmlize@passcs is true.
            \ifhtmlize@actioncs
              % It's an action-sequence.
              %\edef\ds{Must let \xa\string\@htmlize@cs\ act!}\show\ds
              %\show\htmlize@rescanned
              \def\htmlize@next@i{\xa\htmlize@do@actioncs\htmlize@rescanned\@htmlize@stop\htmlize@actionsequence@rdelim}%
              %\show\htmlize@rescanned
              % Note that \htmlize@do@actioncs should patch the command to have it
              % restart the scanning in time.
            \else
              \ifhtmlize@expandcs
                % control sequence to be expanded
                %\edef\ds{Must expand \xa\string\@htmlize@cs}\show\ds
                \bgroup
                  \htmlize@redefine@expansionmacros
                  %The \expandafters first expand \htmlize@rescanned,
                  %and then expand its first token just once.
                  \xa\xa\xa\gdef\xa\xa\xa\htmlize@scrap\xa\xa\xa{\htmlize@rescanned}%
                \egroup
                \xa\ultradetokenize@to@macro\xa\htmlize@remaining@text\xa{\htmlize@scrap}%
                \def\htmlize@next@i{\xa\htmlize@recursive@i\htmlize@remaining@text\@htmlize@stop\@htmlize@stop\@htmlize@stop\htmlize@rdelim@i}%
              \else
                % control sequence to be transcribed to \XML.
                %\edef\ds{Must pass on \xa\string\@htmlize@cs}\show\ds
                \g@addto@macro\htmlize@output{#1}%
                \def\htmlize@next@i{\htmlize@recursive@i#2#3\htmlize@rdelim@i}%
              \fi% \ifhtmlize@expandcs
            \fi% \ifhtmlize@actioncs
          \fi% \ifx\@htmlize@cs\@@begin@cs
        \else%
          \ifx\test@i\@othertilde%
            % The ~ becomes non-breaking space &nbsp; outside of math mode
            \moodle@ifmathmode{\g@addto@macro\htmlize@output{\@othertilde}}%
                              {\g@addto@macro\htmlize@output{\otherampersand nbsp;}}%
            \def\htmlize@next@i{\htmlize@recursive@i#2#3\htmlize@rdelim@i}%
          \else
            \ifx\test@i\@lsinglequote%
              \ifx\test@ii\@lsinglequote%
                % Double left quote
                \g@addto@macro\htmlize@output{\otherampersand ldquo;}%
                \def\htmlize@next@i{\htmlize@recursive@i#3\htmlize@rdelim@i}%
              \else
                \g@addto@macro\htmlize@output{\otherampersand lsquo;}%
                \def\htmlize@next@i{\htmlize@recursive@i#2#3\htmlize@rdelim@i}%
              \fi% \ifx\test@ii\@lsinglequote%
            \else
              \ifx\test@i\@rsinglequote%
                \ifx\test@ii\@rsinglequote% Double right quote
                  %AAedit 2021.01.06: in math mode, you should write 2 single right quotes, for second derivatives
                  \moodle@ifmathmode{\g@addto@macro\htmlize@output{''}}%
                                    {\g@addto@macro\htmlize@output{\otherampersand rdquo;}}%
                  \def\htmlize@next@i{\htmlize@recursive@i#3\htmlize@rdelim@i}%
                \else% Single right quote
                  \moodle@ifmathmode{\g@addto@macro\htmlize@output{'}}%
                                    {\g@addto@macro\htmlize@output{\otherampersand rsquo;}}%
                  \def\htmlize@next@i{\htmlize@recursive@i#2#3\htmlize@rdelim@i}%
                \fi% \ifx\test@ii\@rsinglequote%
              \else
                \ifx\test@i\@doublequote
                  \g@addto@macro\htmlize@output{\otherampersand quot;}%
                  \def\htmlize@next@i{\htmlize@recursive@i#2#3\htmlize@rdelim@i}%
                \else
                  \ifx\test@i\@lt
                    \g@addto@macro\htmlize@output{\otherampersand lt;}
                    \def\htmlize@next@i{\htmlize@recursive@i#2#3\htmlize@rdelim@i}%
                  \else
                    \ifx\test@i\@gt
                      \g@addto@macro\htmlize@output{\otherampersand gt;}%
                      \def\htmlize@next@i{\htmlize@recursive@i#2#3\htmlize@rdelim@i}%
                    \else
                      \ifx\test@i\@dash
                        \ifx\test@ii\@dash% en-dash (--)
                          \moodle@ifmathmode{\g@addto@macro\htmlize@output{--}}%
                                    {\g@addto@macro\htmlize@output{\otherampersand ndash;}}%
                          \def\htmlize@next@i{\htmlize@recursive@i#3\htmlize@rdelim@i}%
                        \else
                          \g@addto@macro\htmlize@output{-}%
                          \def\htmlize@next@i{\htmlize@recursive@i#2#3\htmlize@rdelim@i}%
                        \fi
                      \else
                        \ifx\test@i\@otherequal
                          \ifmoodle@clozemode
                            \moodle@ifmathmode{\g@addto@macro\htmlize@output{=}}%
                                      {\g@addto@macro\htmlize@output{\otherampersand\otherhash 61;}}%
                          \else
                            \g@addto@macro\htmlize@output{=}%
                          \fi
                          \def\htmlize@next@i{\htmlize@recursive@i#2#3\htmlize@rdelim@i}%
                        \else
                          % Default case: write first token to output, call self on remaining tokens.
                          \g@addto@macro\htmlize@output{#1}%
                          \def\htmlize@next@i{\htmlize@recursive@i#2#3\htmlize@rdelim@i}%
                        \fi% \ifx\test@i\@otherequal
                      \fi% \ifx\test@i\@dash
                    \fi% \ifx\test@i\@gt
                  \fi% \ifx\test@i\@lt
                \fi% \ifx\test@i\@doublequote
              \fi% \ifx\test@i\@rsinglequote%
            \fi% \ifx\test@i\@lsinglequote%
          \fi% \ifx\test@i\@othertilde%
        \fi% \ifx\test@i\@otherbackslash%
      \fi% \ifx\test@i\@otherdollar%
    \fi% \ifx\test@i\@otherlbrace%
  \fi% \ifx\test@i\@@htmlize@stop
  \htmlize@next@i
}%

\def\@lsinglequote{`}%
\def\@rsinglequote{'}%
\def\@doublequote{"}%
%    \end{macrocode}
%
% \subsection{Math Mode handling}
%
% In the following, note that the |\|\meta{*}|mathrightdelim|'s gobble an argument.
% This is so ``|$a$ is...|" can turn into
%       ``\ldots |a\|\meta{*}|mathrightdelim{} is...|"
% and preserve a trailing space.
%    \begin{macrocode}
\edef\inlinemathleftdelim{\otherbackslash(}%
\def\inlinemathrightdelim#1{\advancemathmodecounter{-1}%
                            \g@addto@macro\htmlize@output{\otherbackslash)}}%
\edef\displaymathleftdelim{<CENTER>\otherbackslash[}%
\def\displaymathrightdelim#1{\advancemathmodecounter{-1}%
                             \g@addto@macro\htmlize@output{\otherbackslash]</CENTER>}}%
\long\def\htmlize@inlinemath@recursive@i#1#2#3\htmlize@rdelim@i{%
  % inspired by \htmlize@recursive@i
  \def\test@i{#1}%
  \ifx\test@i\@@htmlize@stop
    \let\htmlize@next@i=\relax
  \else
    \ifx\test@i\@otherbackslash
        \g@addto@macro\mathtext{#1#2}%
        \def\htmlize@next@i{\xa\htmlize@inlinemath@recursive@i#3\htmlize@rdelim@i}%
    \else
      \ifx\test@i\@otherdollar
        \let\htmlize@next@i=\relax
        \g@addto@macro\aftertext{#2#3}%
      \else
        \g@addto@macro\mathtext{#1}%
        \def\htmlize@next@i{\xa\htmlize@inlinemath@recursive@i#2#3\htmlize@rdelim@i}%
      \fi
    \fi
  \fi
  \htmlize@next@i
}
{\catcode`\$=12\relax%
  \gdef\htmlize@inlinemathshift@replace#1#2\htmlize@rdelim@iii{%
    %\def\ds{inline math shift has '#1' and '#2'}\show\ds
    \xa\g@addto@macro\xa\htmlize@output\xa{\inlinemathleftdelim}%
    \advancemathmodecounter{1}%
    \gdef\mathtext{}%
    \gdef\aftertext{}%
    \htmlize@inlinemath@recursive@i#2\htmlize@rdelim@i%
    \xdef\htmlize@remaining@text{\expandonce\mathtext%
                                 \otherbackslash inlinemathrightdelim{}%
                                 \expandonce\aftertext}%
    %\show\htmlize@remaining@text
  }%
  \gdef\htmlize@displaymathshift@replace$$#1$$#2\htmlize@rdelim@iii{%
    \xa\g@addto@macro\xa\htmlize@output\xa{\displaymathleftdelim}%
    \advancemathmodecounter{1}%
    \def\mathtext{#1}%
    \def\aftertext{#2}%
    \xdef\htmlize@remaining@text{\expandonce\mathtext%
                                 \otherbackslash displaymathrightdelim{}%
                                 \expandonce\aftertext}%
  }%
}

%    \end{macrocode}
%
% \subsection{Engines for Control Sequences}
%
% There are three kinds of control sequences that need special handling:
% \begin{enumerate}
%   \item Action environments
%   \item Action command sequences
%   \item Expansion macros
% \end{enumerate}
%
% \subsubsection{Engine for running action environments}
%    \begin{macrocode}
\long\def\htmlize@do@actionenv#1#2\@htmlize@stop\htmlize@actionsequence@rdelim{%
  \bgroup %The corresponding \egroup is given in \htmlize@proceedwiththerest,
          %to localize the changes to the environment definitions.
    \htmlize@activate@environments
    \gdef\htmlize@afteraction@hook{}%
    #1#2\@htmlize@stop\htmlize@actionsequence@rdelim%
}

\def\htmlize@patchendenvironment{\swaptotrueendenvironment{\xa\htmlize@proceedwiththerest\htmlize@afteraction@hook}}%

\def\swaptotrueendenvironment#1#2\if@ignore\@ignorefalse\ignorespaces\fi{#2\if@ignore\@ignorefalse\ignorespaces\fi#1}%

\long\def\htmlize@record@environment#1{%
  \listadd{\htmlize@env@actionlist}{#1}%
}
\long\def\html@newenvironment#1#2{%
  \listadd{\htmlize@env@actionlist}{#1}%
  \g@addto@macro\htmlize@activate@environments{%
    \xa\let\csname #1\endcsname\relax%
    \xa\let\csname end#1\endcsname\relax%
    \NewEnviron{#1}{%
      #2%
    }[\htmlize@patchendenvironment]%
  }%
}

\def\htmlize@activate@environments{}%
%    \end{macrocode}
%
% \subsubsection{Engine for running action command sequences}
%
% The following automatically adds the ``engine'' to do the command
% and then resume processing the \LaTeX\ into \HTML.
% It uses the |xpatch| package, which says it works with anything
% defined using |\newcommand| etc. and |\newenvironment| etc.
%    \begin{macrocode}
\gdef\htmlize@afteraction@hook{}%

\long\def\htmlize@do@actioncs#1#2\htmlize@actionsequence@rdelim{%
  % #1#2 contains the current string to be rendered into HTML;
  %      N.B. it has been tokenized at this point,
  %      so TeX can process it directly.
  % #1 = the command sequence we need to execute
  % #2 = the rest of the string
  %
  % First, we patch the desired command so that, when it is over,
  % it calls \htmlize@proceedwiththerest.
  % We do this within the group, so as not to permanently change the command.
  \bgroup
    % The matching \egroup is issued in \htmlize@proceedwiththerest,
    % so that the changes made by \htmlize@activate@css are localized to just the command itself.
    \gdef\htmlize@afteraction@hook{}%
    \htmlize@activate@css%
    \def\test@i{#1}%
    \ifx\test@i\@relax
      \def#1{\xa\htmlize@proceedwiththerest\htmlize@afteraction@hook}%
    \else
      \xapptocmd#1{\xa\htmlize@proceedwiththerest\htmlize@afteraction@hook}{}{\PackageError{Could not patch the command \string#1!}}%
    \fi
    % Now we call that patched command.
    #1#2\htmlize@actionsequence@rdelim%
  %The matching \egroup now is built into the command #1.
}
\long\def\htmlize@proceedwiththerest#1\htmlize@actionsequence@rdelim{%
    % The action cs has done its work.
    % Now we gather up the remaining tokens, detokenize them,
    % remove the \@htmlize@stop, and get back to work transcribing it.
  \egroup %This \egroup matches the \bgroup that was issued either in \htmlize@do@actioncs or in \htmlize@do@actionenv
  \ultradetokenize@to@macro\htmlize@remainder@detokenized{#1}%
  %This will contain an extra \@htmlize@stop, so we remove it.
  \xa\xa\xa\def\xa\xa\xa\htmlize@remainder@detokenized\xa\xa\xa{\xa\htmlize@remove@stopcode\htmlize@remainder@detokenized}%
  %Now we get back to work transcribing the remainder.
  \xa\htmlize@recursive@i\htmlize@remainder@detokenized\@htmlize@stop\@htmlize@stop\@htmlize@stop\htmlize@rdelim@i%
}

\long\def\htmlize@record@action#1{%
  \xa\listadd\xa\htmlize@cs@actionlist\xa{\string#1}%
}

\def\htmlize@activate@css{}%
\long\def\html@action@def#1{%
  \htmlize@record@action{#1}%
  \xa\def\xa\htmlize@scrap\xa{\xa\let\xa#1\csname html@\string#1\endcsname}%
  \xa\g@addto@macro\xa\htmlize@activate@css\xa{\htmlize@scrap}%
  \xa\def\csname html@\string#1\endcsname% %And this \def\html@\oldcsname is follows by the remainder of the definition.
}
\def\html@action@newcommand#1[#2][#3]#4{%
  %\message{>>> Defining #1[#2][#3]{...} ^^J}
  \ifmoodle@draftmode
  \else
    \xa\html@action@def\csname #1\endcsname{\csname moodle@#1@int\endcsname}%
  \fi
  % Note that \htmlize@do@actioncs will 'patch' this by putting
  % '\xa\htmlize@proceedwiththerest\htmlize@afteraction@hook'
  % at the end.  We want those 3 tokens to occur instead after
  % the graphics filename.
  \xa\csdef{moodle@#1@int}##1##2##3{\csname moodle@#1@int@int\endcsname}%
  % This gobbles up those three spurious tokens,
  % which we will re-insert after our work is done.
  \xa\newcommand\csname moodle@#1@int@int\endcsname[#2][#3]{%
    #4%
    % Now we re-insert the code to get the HTMLizing going again.
    \xa\htmlize@proceedwiththerest\htmlize@afteraction@hook
  }%
}
%    \end{macrocode}
%
% \subsubsection{Engine for expansion control sequences}
%
% Calling |\htmlize@redefine@expansionmacros| will redefine
% the macros for us.  It starts out empty.
%    \begin{macrocode}
\long\def\htmlize@redefine@expansionmacros{}%
%    \end{macrocode}
% If |\mymacro| needs no changes to be suited for expansion,
% you can simply call |\htmlize@record@expand{\mymacro}|
% or |\htmlregister{\mymacro}|
% to record that it should be expanded on its way to the \HTML.
% Examples would be user-built macros such as |\inv|$\to$|^{-1}|
% or |\N|$\to$|\mathbb{N}|.
%    \begin{macrocode}
\long\def\htmlize@record@expand#1{%
  \xa\listadd\xa\htmlize@cs@expandlist\xa{\string#1}%
}
\let\htmlregister=\htmlize@record@expand
%    \end{macrocode}
% Often users define a list of macros at the end of the preamble.
% It can be cumbersome to record individually these macros for expansion.
% By calling |\moodleregisternewcommands| they trigger the automatic
% expansion of macros defined subsequently using |\newcommand|,
% |\renewcommand|, |\providecommand| or their starred variants.
%    \begin{macrocode}
\def\moodleregisternewcommands{%
  %% INSPIRED FROM
  %https://tex.stackexchange.com/questions/73271/how-to-redefine-or-patch-the-newcommand-command
  \newcommand*{\saved@ifdefinable}{}
  \let\saved@ifdefinable\@ifdefinable
  \renewcommand{\@ifdefinable}[2]{%
    \saved@ifdefinable{##1}{##2}%
    \htmlregister{##1}
  }%
  \let\@@ifdefinable\@ifdefinable
}%
%    \end{macrocode}
% On the other hand, if an alternate version of the macro is
% needed for \HTML\ purposes, you can define its \HTML\ version with
% |\html@def\mymacro...|
% Parameters are okay.
% An example would be
% |\html@def\emph#1{<EM>#1</EM>}|.
%    \begin{macrocode}
\long\def\html@def#1{%
  \htmlize@record@expand{#1}%
  \xa\def\xa\htmlize@scrap\xa{\xa\let\xa#1\csname html@\string#1\endcsname}%
  \xa\g@addto@macro\xa\htmlize@redefine@expansionmacros\xa{\htmlize@scrap}%
  \xa\def\csname html@\string#1\endcsname%
}
%    \end{macrocode}
% Note that when |\html@def| expands out, it ends with |\def\html@\oldcsname|
% which abuts directly on the remainder of the definition.
%
% \subsection{Specific Control Sequences for Action and Expansion}
%
% Now that we have that machinery in place,
% we define specific environments, action control sequences, and macros to
% expand to accomplish our purposes.
%
% \subsubsection{Action Environments}
%    \begin{macrocode}
\htmlize@record@environment{clozemulti}
\htmlize@record@environment{multi}
\htmlize@record@environment{clozenumerical}
\htmlize@record@environment{numerical}
\htmlize@record@environment{clozeshortanswer}
\htmlize@record@environment{shortanswer}

\html@newenvironment{center}{\xdef\htmlize@afteraction@hook{\noexpand\HTMLtag{CENTER}\expandonce\BODY\noexpand\HTMLtag{/CENTER}}}%

\html@newenvironment{quote}{\xdef\htmlize@afteraction@hook{\noexpand\HTMLtag{BLOCKQUOTE}\expandonce\BODY\noexpand\HTMLtag{/BLOCKQUOTE}}}%

\html@newenvironment{quotation}{\xdef\htmlize@afteraction@hook{\noexpand\HTMLtag{BLOCKQUOTE}\expandonce\BODY\noexpand\HTMLtag{/BLOCKQUOTE}}}%

\def\moodle@save@getitems@state{%
  \global\xa\xdef\csname moodle@currentitemnumber@level@\the\moodle@listdepth\xa\endcsname\xa{\thecurrentitemnumber}%
  \global\xa\xdef\csname moodle@numgathereditems@level@\the\moodle@listdepth\xa\endcsname\xa{\thenumgathereditems}%
  \moodle@saveitems{\thenumgathereditems}%
}%
\def\moodle@restore@getitems@state{%
  \setcounter{numgathereditems}{\csname moodle@numgathereditems@level@\the\moodle@listdepth\endcsname}%
  \setcounter{currentitemnumber}{\csname moodle@currentitemnumber@level@\the\moodle@listdepth\endcsname}%
  \moodle@restoreitems{\thenumgathereditems}%
}%
\def\moodle@saveitems#1{%
  \ifnum#1>0\relax
    \global\csletcs{moodle@level@\the\moodle@listdepth @item@#1}{getitems@item@#1}%
    \xa\moodle@saveitems\xa{\number\numexpr#1-1\expandafter}%
  \fi
}%
\def\moodle@restoreitems#1{%
  \ifnum#1>0\relax
    \global\csletcs{getitems@item@#1}{moodle@level@\the\moodle@listdepth @item@#1}%
    \global\xa\let\csname moodle@level@\the\moodle@listdepth @item@#1\endcsname=\@undefined
    \xa\moodle@restoreitems\xa{\number\numexpr#1-1\expandafter}%
  \fi
}%
\def\moodle@makelistenv#1#2{%
  \html@newenvironment{#1}{%
    \advance\moodle@listdepth by 1\relax
    \moodle@save@getitems@state%
      \xa\gatheritems\xa{\BODY}%
      \gdef\htmlize@afteraction@hook{\HTMLtag{#2}}%
      \loopthroughitemswithcommand{\moodle@itemtoLI}%
      \g@addto@macro\htmlize@afteraction@hook{\HTMLtag{/#2}}%
    \moodle@restore@getitems@state%
    \advance\moodle@listdepth by -1\relax
  }%
}%

\moodle@makelistenv{enumerate}{OL}%
\moodle@makelistenv{itemize}{UL}%

\def\moodle@itemtoLI#1{%
  \g@addto@macro\htmlize@afteraction@hook{\HTMLtag{LI}#1}%
  \trim@spaces@in\htmlize@afteraction@hook%
  \g@addto@macro\htmlize@afteraction@hook{\HTMLtag{/LI}}%
}%

%    \end{macrocode}
%
% \subsubsection{Action Control Sequences}
%
%    \begin{macrocode}
\def\advancemathmodecounter#1{%
  \global\advance\moodle@mathmodedepth by #1\relax
}
\def\openclozemode{%
  \global\moodle@clozemodetrue\relax
}
\def\endclozemode{%
  \global\moodle@clozemodefalse\relax
}
\htmlize@record@action{\advancemathmodecounter}%
\htmlize@record@action{\openclozemode}%
\htmlize@record@action{\endclozemode}%
\htmlize@record@action{\relax}%

\html@action@def\HTMLtag#1{%
    \xa\g@addto@macro\xa\htmlize@output\xa{<#1>}%
  }%
\html@action@def\%{%
    \moodle@ifmathmode{\g@addto@macro\htmlize@output{\otherbackslash\otherpercent}}%
                      {\g@addto@macro\htmlize@output{\otherpercent}}%
  }%
\html@action@def\#{\g@addto@macro\htmlize@output{\otherhash}}%
\html@action@def\&{\g@addto@macro\htmlize@output{\otherampersand}}%
\html@action@def\\{%
    \moodle@ifmathmode{\g@addto@macro\htmlize@output{\otherbackslash\otherbackslash}}%
                      {\g@addto@macro\htmlize@output{<BR/>}}
  }%
\html@action@def\{{%
    \moodle@ifmathmode{\g@addto@macro\htmlize@output{\otherbackslash\otherlbrace}}%
                      {\g@addto@macro\htmlize@output{\otherlbrace}}%
  }%
\html@action@def\}{%
    \moodle@ifmathmode{\g@addto@macro\htmlize@output{\otherbackslash\otherrbrace}}%
       {\ifmoodle@clozemode\g@addto@macro\htmlize@output{\otherbackslash\otherrbrace}%
                           \else\g@addto@macro\htmlize@output{\otherrbrace}\fi}%
  }%
\html@action@def\[{%
    \advancemathmodecounter{1}
    \g@addto@macro\htmlize@output{<CENTER>\otherbackslash\otherlbracket}%
  }%
\html@action@def\]{%
    \g@addto@macro\htmlize@output{\otherbackslash\otherrbracket</CENTER>}%
    \advancemathmodecounter{-1}
  }%
\html@action@def\({%
    \advancemathmodecounter{1}
    \g@addto@macro\htmlize@output{\otherbackslash(}%
  }%
\html@action@def\){%
    \g@addto@macro\htmlize@output{\otherbackslash)}%
    \advancemathmodecounter{-1}
  }%
\html@action@def\ldots{%
    \moodle@ifmathmode{\g@addto@macro\htmlize@output{\string\ldots}}%
                      {\g@addto@macro\htmlize@output{\otherampersand hellip\othersemicol}}%
  }%
\html@action@def\dots{%
    \moodle@ifmathmode{\g@addto@macro\htmlize@output{\string\dots}}%
                      {\g@addto@macro\htmlize@output{\otherampersand hellip\othersemicol}}%
  }%
\html@action@def\ {%
    \moodle@ifmathmode{\g@addto@macro\htmlize@output{\otherbackslash\otherspace}}%
                      {\g@addto@macro\htmlize@output{\otherspace}}%
  }%
\html@action@def\,{%
    \moodle@ifmathmode{\g@addto@macro\htmlize@output{\string\,}}%
                      {\g@addto@macro\htmlize@output{\otherampersand\otherhash 8239\othersemicol}}%
  }%
\html@action@def\thinspace{%
    \moodle@ifmathmode{\g@addto@macro\htmlize@output{\string\thinspace\otherspace}}%
                      {\g@addto@macro\htmlize@output{\otherampersand\otherhash 8239\othersemicol}}%
  }%
\html@action@def\>{%
    \moodle@ifmathmode{\g@addto@macro\htmlize@output{\otherbackslash\otherampersand gt;}}%
                      {\g@addto@macro\htmlize@output{\otherampersand emsp14\othersemicol}}%
  }%
\html@action@def\:{%
    \moodle@ifmathmode{\g@addto@macro\htmlize@output{\string\:}}%
                      {\g@addto@macro\htmlize@output{\otherampersand emsp14\othersemicol}}%
  }%
\html@action@def\medspace{%
    \moodle@ifmathmode{\g@addto@macro\htmlize@output{\string\:}}%
                      {\g@addto@macro\htmlize@output{\otherampersand emsp14\othersemicol}}%
  }%
\html@action@def\;{%
    \moodle@ifmathmode{\g@addto@macro\htmlize@output{\otherbackslash\othersemicol}}%
                      {\g@addto@macro\htmlize@output{\otherampersand emsp13\othersemicol}}%
  }%
\html@action@def\thickspace{%
    \moodle@ifmathmode{\g@addto@macro\htmlize@output{\otherbackslash\othersemicol}}%
                      {\g@addto@macro\htmlize@output{\otherampersand emsp13\othersemicol}}%
  }%
\html@action@def\enspace{%
    \moodle@ifmathmode{\g@addto@macro\htmlize@output{\string\enspace\otherspace}}%
                      {\g@addto@macro\htmlize@output{\otherampersand ensp\othersemicol}}%
  }%
\html@action@def\quad{%
    \moodle@ifmathmode{\g@addto@macro\htmlize@output{\string\quad\otherspace}}%
                      {\g@addto@macro\htmlize@output{\otherampersand emsp\othersemicol}}%
  }%
\html@action@def\qquad{%
    \moodle@ifmathmode{\g@addto@macro\htmlize@output{\string\qquad\otherspace}}%
                      {\g@addto@macro\htmlize@output{\otherampersand emsp\othersemicol\otherampersand emsp\othersemicol}}%
  }%
\html@action@def\${%
     \g@addto@macro\htmlize@output{\otherdollar}%
  }%
\html@action@def\textquotesingle{%
     \g@addto@macro\htmlize@output{'}%
  }%
\html@action@def\@dquote{%
     \g@addto@macro\htmlize@output{"}%
  }%
\html@action@def\textquotedbl{%
     \g@addto@macro\htmlize@output{"}%
  }%
\html@action@def\clozetilde{%
    \xa\g@addto@macro\xa\htmlize@output\xa{\othertilde}%
  }%
\html@action@def\clozecorrect{%
    \xa\g@addto@macro\xa\htmlize@output\xa{\otherequal}%
  }%
\html@action@def\clozelbrace{%
    \openclozemode
    \xa\g@addto@macro\xa\htmlize@output\xa{\otherlbrace}%
  }%
\html@action@def\clozerbrace{%
    \xa\g@addto@macro\xa\htmlize@output\xa{\otherrbrace}%
    \endclozemode
  }%
\html@action@def\TeX{%
    \g@addto@macro\htmlize@output{\otherbackslash(\otherbackslash rm\otherbackslash TeX\otherbackslash)}%
  }%
\html@action@def\LaTeX{%
    \g@addto@macro\htmlize@output{\otherbackslash(\otherbackslash rm\otherbackslash LaTeX\otherbackslash)}%
  }%

{\catcode`;=12\relax\gdef\othersemicol{;}}

%Diacritical marks over vowels
{\catcode`|=3\relax
 \gdef\htmlize@vowels{a|e|i|o|u|A|E|I|O|U|}}%
\def\htmlize@define@diacritic#1#2{%
  \htmlize@record@action{#1}%
  \g@addto@macro\htmlize@activate@css{%
    \def#1##1{%
      \ifinlist{##1}{\htmlize@vowels}%
        {\g@addto@macro\htmlize@output{\otherampersand##1#2\othersemicol}}%
        {\xa\g@addto@macro\htmlize@output\xa{\string#1##1}}%
    }%
  }%
}
\htmlize@define@diacritic{\^}{circ}%
\htmlize@define@diacritic{\'}{acute}%
\htmlize@define@diacritic{\`}{grave}%

%Diaeresis/Tréma/Umlaut
{\catcode`|=3\relax
 \gdef\htmlize@diaeresis{a|e|i|o|u|y|A|E|I|O|U|Y|}}%
\html@action@def\"#1{%
    \ifinlist{#1}{\htmlize@diaeresis}%
      {\g@addto@macro\htmlize@output{\otherampersand#1uml\othersemicol}}%
      {\xa\g@addto@macro\htmlize@output\xa{\string\"#1}}%
}%

%Hungarian long-umlaut diacritics
\def\@o{o}\def\@O{O}\def\@u{u}\def\@U{U}%
\html@action@def\H#1{%
  \bgroup
    \def\test@i{#1}%
    \ifx\test@i\@O
      \def\toadd{\otherampersand\otherhash336\othersemicol}%
    \else
      \ifx\test@i\@o
        \def\toadd{\otherampersand\otherhash337\othersemicol}%
      \else
        \ifx\test@i\@U
          \def\toadd{\otherampersand\otherhash368\othersemicol}%
        \else
          \ifx\test@i\@u
            \def\toadd{\otherampersand\otherhash369\othersemicol}%
          \else
            \def\toadd{\otherbackslash H\otherlbrace#1\otherrbrace}%
          \fi
        \fi
      \fi
    \fi
    \xa\g@addto@macro\xa\htmlize@output\xa{\toadd}%
  \egroup
}%

%Cedilla
\def\@c{c}\def\@C{C}\def\@s{s}\def\@S{S}\def\@t{t}\def\@T{T}%
\html@action@def\c#1{%
  \bgroup
    \def\test@i{#1}%
    \ifx\test@i\@c
      \def\toadd{\otherampersand ccedil\othersemicol}%
    \else
      \ifx\test@i\@C
        \def\toadd{\otherampersand Ccedil\othersemicol}%
      \else
        \ifx\test@i\@s
          \def\toadd{\otherampersand\otherhash351\othersemicol}%
        \else
          \ifx\test@i\@S
            \def\toadd{\otherampersand\otherhash350\othersemicol}%
          \else
            \ifx\test@i\@t
              \def\toadd{\otherampersand\otherhash355\othersemicol}%
            \else
              \ifx\test@i\@T
                \def\toadd{\otherampersand\otherhash354\othersemicol}%
              \else
                \def\toadd{\otherbackslash c\otherlbrace#1\otherrbrace}%
              \fi
            \fi
          \fi
        \fi
      \fi
    \fi
    \xa\g@addto@macro\xa\htmlize@output\xa{\toadd}%
  \egroup
}%

%Tilde over a, n, o
{\catcode`|=3\relax
 \gdef\htmlize@tilde{A|N|O|a|n|o|}}%
\html@action@def\~#1{%
    \ifinlist{#1}{\htmlize@tilde}%
      {\g@addto@macro\htmlize@output{\otherampersand#1tilde\othersemicol}}%
      {\xa\g@addto@macro\htmlize@output\xa{\string\~#1}}%
}%

% breve diacritics
{\catcode`|=3\relax
 \gdef\htmlize@breve{A|G|U|a|g|u|}}%
\def\@e{e}\def\@E{E}\def\@i{i}\def\@ii{\i}\def\@I{I}\def\@o{o}\def\@O{O}%
\html@action@def\u#1{%
  \ifinlist{#1}{\htmlize@breve}%
     {\g@addto@macro\htmlize@output{\otherampersand#1breve\othersemicol}}%
     {%
      \bgroup
        \def\test@i{#1}%
        \ifx\test@i\@E
          \def\toadd{\otherampersand\otherhash276\othersemicol}%
        \else
          \ifx\test@i\@e
            \def\toadd{\otherampersand\otherhash277\othersemicol}%
          \else
            \ifx\test@i\@I
              \def\toadd{\otherampersand\otherhash300\othersemicol}%
            \else
              \ifx\test@i\@i
                \def\toadd{\otherampersand\otherhash301\othersemicol}%
              \else
                \ifx\test@i\@ii
                  \def\toadd{\otherampersand\otherhash301\othersemicol}%
                \else
                  \ifx\test@i\@O
                    \def\toadd{\otherampersand\otherhash334\othersemicol}%
                  \else
                    \ifx\test@i\@o
                      \def\toadd{\otherampersand\otherhash335\othersemicol}%
                    \else
                      \def\toadd{\otherbackslash u\otherlbrace#1\otherrbrace}%
                    \fi
                  \fi
                \fi
              \fi
            \fi
          \fi
        \fi
        \xa\g@addto@macro\xa\htmlize@output\xa{\toadd}%
      \egroup
    }%
}%

% caron diacritics
{\catcode`|=3\relax
 \gdef\htmlize@caron{C|D|E|L|N|R|S|T|Z|c|d|e|l|n|r|s|t|z|}}%
\html@action@def\v#1{%
    \ifinlist{#1}{\htmlize@caron}%
      {\g@addto@macro\htmlize@output{\otherampersand#1caron\othersemicol}}%
      {\xa\g@addto@macro\htmlize@output\xa{\string\v#1}}%
}%

% Ogonek diacritics
{\catcode`|=3\relax
 \gdef\htmlize@ogonek{A|E|I|U|a|e|i|u|}}%
\html@action@def\k#1{%
  \ifinlist{#1}{\htmlize@ogonek}%
    {\g@addto@macro\htmlize@output{\otherampersand#1ogon\othersemicol}}%
    {%
      \bgroup
        \def\test@i{#1}%
        \ifx\test@i\@O
          \def\toadd{\otherampersand\otherhash490\othersemicol}%
        \else
          \ifx\test@i\@o
            \def\toadd{\otherampersand\otherhash491\othersemicol}%
          \else
            \def\toadd{\otherbackslash k\otherlbrace#1\otherrbrace}%
          \fi
        \fi
        \xa\g@addto@macro\xa\htmlize@output\xa{\toadd}%
      \egroup
    }%
}%

% macron diacritics
{\catcode`|=3\relax
 \gdef\htmlize@macron{A|E|I|O|U|a|e|i|o|u|}}%
\def\@g{g}\def\@G{G}\def\@y{y}\def\@Y{Y}%
\html@action@def\=#1{%
  \ifinlist{#1}{\htmlize@macron}%
    {\g@addto@macro\htmlize@output{\otherampersand#1macr\othersemicol}}%
    {%
      \bgroup
        \def\test@i{#1}%
        \ifx\test@i\@g
          \def\toadd{\otherampersand\otherhash7713\othersemicol}%
        \else
          \ifx\test@i\@G
            \def\toadd{\otherampersand\otherhash7712\othersemicol}%
          \else
            \ifx\test@i\@y
              \def\toadd{\otherampersand\otherhash563\othersemicol}%
            \else
              \ifx\test@i\@Y
                \def\toadd{\otherampersand\otherhash562\othersemicol}%
              \else
                \def\toadd{\otherbackslash =\otherlbrace#1\otherrbrace}%
              \fi
            \fi
          \fi
        \fi
        \xa\g@addto@macro\xa\htmlize@output\xa{\toadd}%
      \egroup
    }%
}%

% macron below diacritics
\def\@b{b}\def\@B{B}\def\@d{d}\def\@D{D}\def\@k{k}\def\@K{K}\def\@l{l}%
\def\@L{L}\def\@n{n}\def\@N{N}\def\@t{t}\def\@T{T}\def\@z{z}\def\@Z{Z}%
\html@action@def\b#1{%
  \bgroup
    \def\test@i{#1}%
    \ifx\test@i\@b
      \def\toadd{\otherampersand\otherhash7687\othersemicol}%
    \else
      \ifx\test@i\@B
        \def\toadd{\otherampersand\otherhash7686\othersemicol}%
      \else
        \ifx\test@i\@d
          \def\toadd{\otherampersand\otherhash7695\othersemicol}%
        \else
          \ifx\test@i\@D
            \def\toadd{\otherampersand\otherhash7694\othersemicol}%
          \else
            \ifx\test@i\@k
              \def\toadd{\otherampersand\otherhash7733\othersemicol}%
            \else
              \ifx\test@i\@K
                \def\toadd{\otherampersand\otherhash7732\othersemicol}%
              \else
                \ifx\test@i\@l
                  \def\toadd{\otherampersand\otherhash7739\othersemicol}%
                \else
                  \ifx\test@i\@L
                    \def\toadd{\otherampersand\otherhash7738\othersemicol}%
                  \else
                    \ifx\test@i\@n
                      \def\toadd{\otherampersand\otherhash7753\othersemicol}%
                    \else
                      \ifx\test@i\@N
                        \def\toadd{\otherampersand\otherhash7752\othersemicol}%
                      \else
                        \ifx\test@i\@r
                          \def\toadd{\otherampersand\otherhash7775\othersemicol}%
                        \else
                          \ifx\test@i\@R
                            \def\toadd{\otherampersand\otherhash7774\othersemicol}%
                          \else
                            \ifx\test@i\@t
                              \def\toadd{\otherampersand\otherhash7791\othersemicol}%
                            \else
                              \ifx\test@i\@T
                                \def\toadd{\otherampersand\otherhash7790\othersemicol}%
                              \else
                                \ifx\test@i\@z
                                  \def\toadd{\otherampersand\otherhash7829\othersemicol}%
                                \else
                                  \ifx\test@i\@Z
                                    \def\toadd{\otherampersand\otherhash7828\othersemicol}%
                                  \else
                                    \def\toadd{\otherbackslash b\otherlbrace#1\otherrbrace}%
                                  \fi
                                \fi
                              \fi
                            \fi
                          \fi
                        \fi
                      \fi
                    \fi
                  \fi
                \fi
              \fi
            \fi
          \fi
        \fi
      \fi
    \fi
    \xa\g@addto@macro\xa\htmlize@output\xa{\toadd}%
  \egroup
}%

% Overdot diacritics
{\catcode`|=3\relax
 \gdef\htmlize@dotabove{C|E|G|I|Z|c|e|g|z|}}%
\def\@i{i}%
\html@action@def\.#1{%
  \ifinlist{#1}{\htmlize@dotabove}%
    {\g@addto@macro\htmlize@output{\otherampersand#1dot\othersemicol}}%
    {
      \bgroup
        \def\test@i{#1}%
        \ifx\test@i\@i
          \def\toadd{i}%
        \else
          \def\toadd{\otherbackslash .\otherlbrace#1\otherrbrace}%
        \fi
        \xa\g@addto@macro\xa\htmlize@output\xa{\toadd}%
      \egroup
    }%
}%

% Underdot diacritics
\def\@A{A}\def\@a{a}\def\@B{B}\def\@b{b}%
\html@action@def\d#1{%
  \bgroup
    \def\test@i{#1}%
    \ifx\test@i\@a
      \def\toadd{\otherampersand\otherhash7841\othersemicol}%
    \else
      \ifx\test@i\@A
        \def\toadd{\otherampersand\otherhash7840\othersemicol}%
      \else
        \ifx\test@i\@b
          \def\toadd{\otherampersand\otherhash7685\othersemicol}%
        \else
          \ifx\test@i\@B
            \def\toadd{\otherampersand\otherhash7684\othersemicol}%
          \else
            \def\toadd{\otherbackslash d\otherlbrace#1\otherrbrace}%
          \fi
        \fi
      \fi
    \fi
    \xa\g@addto@macro\xa\htmlize@output\xa{\toadd}%
  \egroup
}%

% Overring
{\catcode`|=3\relax
 \gdef\htmlize@ring{A|U|a|u|}}%
\html@action@def\r#1{%
    \ifinlist{#1}{\htmlize@ring}%
      {\g@addto@macro\htmlize@output{\otherampersand#1ring\othersemicol}}%
      {\xa\g@addto@macro\htmlize@output\xa{\string\r#1}}%
}%
%    \end{macrocode}
%
% \subsubsection{Command sequences for Expansion}
%
% First, commands defined by this package.
%   \begin{macrocode}
\def\blank{\rule{1in}{0.5pt}}%
% TODO: Make an optional argument for width?  This wouldn't affect Moodle,
%        only the appearance in the PDF.  It doesn't seem worth it.
\html@def\blank{____________}%
\htmlize@record@action\inlinemathrightdelim
\htmlize@record@action\displaymathrightdelim
%    \end{macrocode}
% Second, native \LaTeX\ commands.
%    \begin{macrocode}
\html@def\emph#1{\HTMLtag{EM}#1\HTMLtag{/EM}}%
\html@def\textbf#1{\HTMLtag{B}#1\HTMLtag{/B}}%
\html@def\textit#1{\HTMLtag{I}#1\HTMLtag{/I}}%
\html@def\textsc#1{\HTMLtag{SPAN STYLE="font-variant: small-caps;"}#1\HTMLtag{/SPAN}}%
\html@def\textsuperscript#1{\HTMLtag{SUP}#1\HTMLtag{/SUP}}%
\html@def\textsubscript#1{\HTMLtag{SUB}#1\HTMLtag{/SUB}}%
\html@def\texttt#1{\HTMLtag{CODE}#1\HTMLtag{/CODE}}%
\html@def\underline#1{\HTMLtag{SPAN STYLE="text-decoration: underline;"}#1\HTMLtag{/SPAN}}%
%    \end{macrocode}
% Third, popular \LaTeX\ commands from packages.
%    \begin{macrocode}
\html@def\url#1{\HTMLtag{A href=\@dq #1\@dq}#1\HTMLtag{/A}}% url or hyperref
\html@def\href#1#2{\HTMLtag{A href=\@dq #1\@dq}#2\HTMLtag{/A}}% hyperref
\html@def\up#1{\HTMLtag{SUP}#1\HTMLtag{/SUP}}% \pkg{babel}, option \optn{french}
\html@def\fup#1{\HTMLtag{SUP}#1\HTMLtag{/SUP}}% \pkg{babel}, option \optn{french}
%    \end{macrocode}
% Fourth, \LaTeX\ commands for ligature and other glyphs.
%    \begin{macrocode}
\html@def\aa{\&aring;}%
\html@def\AA{\&Aring;}%
\html@def\ae{\&aelig;}%
\html@def\AE{\&AElig;}%
\html@def\dh{\&eth;}%
\html@def\DH{\&ETH;}%
\html@def\dj{\&dstrok;}%
\html@def\DJ{\&Dstrok;}%
\html@def\i{\&imath;}%
\html@def\ij{\&ijlig;}%
\html@def\IJ{\&IJlig;}%
\html@def\j{\&jmath;}%
\html@def\l{\&lstrok;}%
\html@def\L{\&Lstrok;}%
\html@def\ng{\&eng;}%
\html@def\NG{\&ENG;}%
\html@def\o{\&oslash;}%
\html@def\O{\&Oslash;}%
\html@def\oe{\&oelig;}%
\html@def\OE{\&OElig;}%
\html@def\ss{\&szlig;}%
\html@def\SS{\&\#7838;}%
\html@def\th{\&thorn;}%
\html@def\TH{\&THORN;}%
%    \end{macrocode}
% Fifth, \LaTeX\ commands for horizontal spacing and line breaks.
% In a number of situations, paragraph breaks translated into
% |</p><p>| would bring trouble in the \HTML\ code. Examples:
% \begin{itemize}
% \item fields of cloze subquestions
% \item inside environments like |center|, |itemize|, or |enumerate|.
% \end{itemize}
%    \begin{macrocode}
\html@def\space{ }%
\html@def\textvisiblespace{\&blank;}%
\html@def\newline{\HTMLtag{BR/}}%
\html@def\par{\HTMLtag{BR/}}%
%    \end{macrocode}
% Sixth, \LaTeX\ commands for various symbols.
%    \begin{macrocode}
\html@def\textbackslash{\&\#92;}%
\html@def\_{\&\#95;}%
\html@def\textquestiondown{\&iquest;}%
\html@def\textexclamdown{\&iexcl;}%
\html@def\euro{\&euro;}%
\html@def\texteuro{\&euro;}%
\html@def\S{\&sect;}%
%    \end{macrocode}
% Seventh, \LaTeX\ commands for various quotation marks.
%    \begin{macrocode}
\html@def\textquoteleft{\&lsquo;}%
\html@def\textquoteright{\&rsquo;}%
\html@def\textquotedblleft{\&ldquo;}%
\html@def\textquotedblright{\&rdquo;}%
\html@def\guilsinglleft{\&lsaquo;}%
\html@def\guilsinglright{\&rsaquo;}%
\html@def\guillemotleft{\&laquo;}%
\html@def\guillemotright{\&raquo;}%
\html@def\quotesinglbase{\&sbquo;}%
\html@def\quotedblbase{\&bdquo;}%
\html@def\flq{\&lsaquo;}%
\html@def\frq{\&rsaquo;}%
\html@def\flqq{\&laquo;}%
\html@def\frqq{\&raquo;}%
\html@def\og{\&laquo;\&\#8239;}%
\html@def\fg{\&\#8239;\&raquo;}%
\html@def\glq{\&sbquo;}%
\html@def\grq{\&lsquo;}%
\html@def\glqq{\&bdquo;}%
\html@def\grqq{\&ldquo;}%
\html@def\dq{\&quot;}%
%    \end{macrocode}
%
% \subsubsection{Passing Code to \XML\ Only}
% Users may want to include pieces of \HTML\ code to the \XML\ file only.
% When producing the traditional output, the optional argument, empty
% by default, is used.
%    \begin{macrocode}
\newcommand\htmlonly[2][]{#1}%
%    \end{macrocode}
% When producing the \XML\ file, the mandatory argument is passed as-is.
%    \begin{macrocode}
\html@action@newcommand{htmlonly}[2][]{%
  \g@addto@macro\htmlize@output{#2}%
}%
%    \end{macrocode}
%
% \subsection{Graphics via {\ttfamily\string\includegraphics}}
%
% \subsubsection{Finding Media Files}
% The following code is adapted from the command |\Ginclude@graphics| as
% found in \filenm{graphics.sty}.
% Calling |\moodle@media@find|\marg{filename} looks for the file the
% same way |\includegraphics| does, with or without extension provided,
% in the current folder and in folders specified with |\graphicspath{}|.
% The allowed extensions and their relative priorities can be set
% via |\DeclareGraphicsExtensions{}|.
% The outcome is that the macros |\moodle@media@base| and |\moodle@media@ext|
% are set with the basename, including path, and extension, respectively.
%
%    \begin{macrocode}
\AtEndPreamble{%
  \@ifpackageloaded{graphics}{%
    \@ifpackagelater{graphics}{2019/10/08 v1.3c}{}%
    {\PackageError{moodle}{`moodle' is made to interact with the `graphics'\MessageBreak
                           package not older than 2019/10/08.}\@eha\endinput}%

  }{}%
}%
\def\moodle@media@find#1{%
  \ifx\detokenize\@undefined\else
    \edef\Gin@extensions{\detokenize\expandafter{\Gin@extensions}}%
  \fi
  \begingroup
  \let\input@path\Ginput@path
  \set@curr@file{#1}%
  \expandafter\filename@parse\expandafter{\@curr@file}%
  \ifx\filename@ext\Gin@gzext
    \expandafter\filename@parse\expandafter{\filename@base}%
    \ifx\filename@ext\relax
      \let\filename@ext\Gin@gzext
    \else
      \edef\Gin@ext{\Gin@ext\Gin@sepdefault\Gin@gzext}%
    \fi
  \fi
  \ifx\filename@ext\relax
    \@for\Gin@temp:=\Gin@extensions\do{%
      \ifx\Gin@ext\relax
        \Gin@getbase\Gin@temp
      \fi}%
  \else
    \Gin@getbase{\Gin@sepdefault\filename@ext}%
    \ifnum0%
        \ifx\Gin@ext\relax 1%
        \else \@ifundefined{Gin@rule@\Gin@ext}{1}{0}%
        \fi >0
      \let\Gin@ext\relax
      \let\Gin@savedbase\filename@base
      \let\Gin@savedext\filename@ext
      \edef\filename@base{\filename@base\Gin@sepdefault\filename@ext}%
      \let\filename@ext\relax
      \@for\Gin@temp:=\Gin@extensions\do{%
          \ifx\Gin@ext\relax
            \Gin@getbase\Gin@temp
          \fi}%
      \ifx\Gin@ext\relax
        \let\filename@base\Gin@savedbase
        \let\filename@ext\Gin@savedext
      \fi
    \fi
    \ifx\Gin@ext\relax
       \@warning{File `#1' not found}%
       \def\Gin@base{\filename@area\filename@base}%
       \edef\Gin@ext{\Gin@sepdefault\filename@ext}%
    \fi
  \fi
    \ifx\Gin@ext\relax
         \@latex@error{File `#1' not found}%
         {I could not locate the file with any of these extensions:^^J%
          \Gin@extensions^^J\@ehc}%
    \else
      % begin modified part
       \xdef\moodle@media@base{\detokenize\xa{\Gin@base}}%
       \xdef\moodle@media@ext{\detokenize\xa{\Gin@ext}}%
      % end modfied part
    \fi
  \endgroup
}
%    \end{macrocode}
%    \begin{macrocode}
%See
% * https://developer.mozilla.org/en-US/docs/Web/Media/Formats/Image_types
% * https://developer.mozilla.org/en-US/docs/Web/Media/Formats/Containers
\def\moodle@media@formats{}
\newcommand{\DeclareMediaFormat}[3]{%
  \xdef\moodle@media@formats{\ifx\moodle@media@formats\empty\else\moodle@media@formats,\fi#1}%
  \csgdef{moodle@media@#1list}{#2}%
  \csgdef{moodle@media@#1mime}{#3}%
}%
% Image Formats
\DeclareMediaFormat{PNG}{.png,.PNG}{image/png}%
\DeclareMediaFormat{JPEG}{.jpg,.JPG,.jpeg,.JPEG}{image/jpeg}%
\DeclareMediaFormat{SVG}{.svg,.SVG}{image/svg+xml}%
\DeclareMediaFormat{GIF}{.gif}{image/gif}%
% Audio Formats
\DeclareMediaFormat{WAV}{.wav,.WAV}{audio/wave}%
\DeclareMediaFormat{MP3}{.mp3,.MP3}{audio/mpeg}%
\DeclareMediaFormat{OGG}{.ogg,.opus,.og&}{audio/ogg}%
%\DeclareMediaFormat{FLAC}{.flac,.FLAC}{audio/flac}%
% Video Formats
\DeclareMediaFormat{WEBM}{.webm,.webm}{video/webm}%
\DeclareMediaFormat{MP4}{.mp4,.MP4,.m4v,.M4V}{video/mp4}%
\DeclareMediaFormat{OGV}{.ogv,.OGV}{video/ogg}%
%    \end{macrocode}
%    \begin{macrocode}
\def\moodle@media@mime@identify#1{%
  \edef\test@i{\detokenize\xa{#1}}%
  \let\moodle@media@mime@current\relax
  \edef\moodle@media@formats{\detokenize\xa{\moodle@media@formats}}%
  \@for\@format:=\moodle@media@formats\do{%
    \edef\@templist{\csname moodle@media@\@format list\endcsname}%
    \@for\@ext:=\@templist\do{%
      \ifx\moodle@media@mime@current\relax
        \edef\@ext{\detokenize\xa{\@ext}}%
        \ifx\test@i\@ext
          \message{<<extension \@ext (\@format) identified>>^^J}%
          \xdef\moodle@media@mime@current{\csname moodle@media@\@format mime\endcsname}%
          \xdef\moodle@media@ext@current{#1}%
        \fi
      \fi
    }%
  }%
}
%    \end{macrocode}
%
% \subsubsection{External program command lines}
% We first set up commands for the external programs.
%    \begin{macrocode}
\def\htmlize@setexecutable#1{%
  % Defines macro #1 to be #2 in a verbatim mode suitable for filenames
  \def\htmlize@executable@macro{#1}%
  \bgroup\catcode`\|=0\catcode`\\=12\relax%
  \htmlize@setexecutable@int
}
\def\htmlize@setexecutable@int#1{%
  \egroup
  \expandafter\def\htmlize@executable@macro{#1}%
}

\def\ghostscriptcommand{\htmlize@setexecutable\gs}%
\def\baselxivcommand{\htmlize@setexecutable\baselxiv}%
\def\imagemagickcommand{\htmlize@setexecutable\htmlize@imagemagick@convert}%
\def\optipngcommand{\htmlize@setexecutable\optipng}%
\def\PDFtoSVGcommand{\htmlize@setexecutable\PDFtoSVG}%
\def\SVGtoPDFcommand{\htmlize@setexecutable\SVGtoPDF}%
\def\optiSVGcommand{\htmlize@setexecutable\optiSVG}%
\def\DeleteFilecommand{\htmlize@setexecutable\DeleteFiles}%
\def\MoveFilecommand{\htmlize@setexecutable\MoveFiles}%
\def\DevNullcommand{\htmlize@setexecutable\DevNull}%

\ifwindows%
  \ghostscriptcommand{gswin64c.exe -dBATCH -dNOPAUSE -sDEVICE=pngalpha}%
  \baselxivcommand{certutil}%
  % Uses scour from inkscape default installation
  % The \Inkscape\bin  directory has to be in the users path and provides both:
  % {inkscape,python}.exe the scour-Package which is found by python in Inkscape\lib\python3.x\site-packages\
  \optiSVGcommand{python -m scour.scour -q --enable-id-stripping --enable-comment-stripping
                --shorten-ids --indent=none --remove-descriptive-elements}%
  \DeleteFilecommand{del}%
  \MoveFilecommand{move}%
  \DevNullcommand{NUL}%
\else%
  \ghostscriptcommand{gs -dBATCH -dNOPAUSE -sDEVICE=pngalpha}%
  \baselxivcommand{base64}%
  \optiSVGcommand{scour -q --enable-id-stripping --enable-comment-stripping
                --shorten-ids --indent=none --remove-descriptive-elements}%
  \DeleteFilecommand{rm -f}%
  \MoveFilecommand{mv}%
  \DevNullcommand{/dev/null}%
\fi%

\imagemagickcommand{convert -colorspace RGB}%
\optipngcommand{optipng -clobber -strip all -quiet}%
\PDFtoSVGcommand{inkscape --export-type=svg --export-area-page --vacuum-defs}%--pdf-poppler
% remove the "--pdf-poppler" import option if you want to preserve text (avoid conversion to path)
\SVGtoPDFcommand{inkscape --export-type=pdf --export-area-page}%
%    \end{macrocode}
%
% \subsubsection{Conversion and inclusion of non-native formats}
%    \begin{macrocode}
\AtEndPreamble{%
  \@ifpackageloaded{graphics}{%
    \def\@firstofthree#1#2#3{#1}%
    \def\@secondofthree#1#2#3{#2}%
    \newcommand{\DeclareGraphicsAlien}[3]{%
      \edef\Gin@extensions{\Gin@extensions,#1}%
      \DeclareGraphicsRule{#1}{\@gobble#1}{#1}{}%
      \csdef{Gread@\@gobble#1}##1{%
        \edef\SourceFile{\Gin@base\Gin@ext}%
        \edef\Gin@base{\Gin@base-\@gobble#1-converted-to}%
        \edef\Gin@ext{#2}%
        \edef\OutputFile{\Gin@base\Gin@ext}%
        \edef\targetfmt{\expandafter\expandafter\expandafter
                        \@firstofthree\csname Gin@rule@\Gin@ext\endcsname\relax}%
        \edef\targetext{\expandafter\expandafter\expandafter
                        \@secondofthree\csname Gin@rule@\Gin@ext\endcsname\relax}%
        \IfFileExists{\OutputFile}{}{\ShellEscape{#3}}%
        \csletcs{Ginclude@\@gobble#1}{Ginclude@\targetfmt}%
        \csname Gread@\targetfmt\endcsname{\Gin@base\targetext}%
      }%
    }%
% Support for \GIF\ files: passed as-is in \XML\ but converted in \PNG\ for PDF output.
% In case the \GIF\ is animated, we pick up the first frame in this conversion.
% While the picture will be animated after Moodle import, it will not be in the PDF
% output, whatever the viewer is.
    \DeclareGraphicsAlien{.gif}{.png}{\htmlize@imagemagick@convert\otherspace '\SourceFile[0]' \OutputFile}%
  }{
    \newcommand\includegraphics[2][]{\PackageError{moodle}{"\string\includegraphics" is not defined}%
                                          {Add "\string\usepackage{graphicx}" to you preamble.}}{}%}%
  }%
}%
%    \end{macrocode}
% \subsubsection{Graphics key-handling}
% Next, we get ready to handle keys like |height=4cm| or |width=3cm| or |ppi=72|.
%    \begin{macrocode}
\define@cmdkeys{moodle@includegraphics}[moodle@graphics@]{ppi}
\define@cmdkey{moodle}[moodle@graphics@]{ppi}{}% This is so the ppi key can be set at the document, quiz, or question level.
\define@cmdkeys{Gin}{ppi}% This is so the original \includegraphics will not object to a key of ppi.
\setkeys{moodle@includegraphics}{ppi=103}

\newdimen\moodle@graphics@temp@dimen
\newcount\moodle@graphics@height@pixels
\newcount\moodle@graphics@width@pixels
\def\moodle@graphics@dimentopixels#1#2{%
  \moodle@graphics@temp@dimen=#2\relax
  \moodle@graphics@temp@dimen=0.013837\moodle@graphics@temp@dimen
  \xa\moodle@graphics@temp@dimen\xa=\moodle@graphics@ppi\moodle@graphics@temp@dimen
  #1=\moodle@graphics@temp@dimen
  \divide #1 by 65536\relax
}
\define@key{moodle@includegraphics}{height}[]{%
  \moodle@graphics@dimentopixels{\moodle@graphics@height@pixels}{#1}%
}
\define@key{moodle@includegraphics}{width}[]{%
  \moodle@graphics@dimentopixels{\moodle@graphics@width@pixels}{#1}%
}
\setkeys{moodle@includegraphics}{height=0pt,width=0pt}
%    \end{macrocode}
%
% \subsubsection{Graphics conversion to HTML}
% If the |tikz| option is loaded, we define the |embedaspict| command.
% Furthermore, |includegraphics| is packed into a TikZ node.
% This allows externalization with regular options for |includegraphics|.
% Otherwise, |includegraphics| is redefined with a limited set of options supported.
%
% Option SVG
%    \begin{macrocode}
\ifmoodle@svg
  \AtEndPreamble{%
% Declaring \SVG\ to PDF conversion rule for includegraphics
%      \edef\Gin@extensions{\Gin@extensions,.svg,.SVG}%
%      \DeclareGraphicsRule{.svg}{pdf}{.pdf}{%
%        `\SVGtoPDF\otherspace '#1' \noexpand\Gin@base-svg-converted-to.pdf}%
%      \DeclareGraphicsRule{.SVG}{pdf}{.pdf}{%
%        `\SVGtoPDF\otherspace '#1' \noexpand\Gin@base-SVG-converted-to.pdf}%
    \DeclareGraphicsAlien{.svg}{.pdf}{%
      \SVGtoPDF\otherspace '\SourceFile' -o '\OutputFile' 2>\DevNull}%
    \DeclareGraphicsAlien{.SVG}{.pdf}{%
      \SVGtoPDF\otherspace '\SourceFile' -o '\OutputFile' 2>\DevNull}%
  }%
\fi
\def\moodle@checkconversionsuccess#1#2{%
  \IfFileExists{#1}{}{%
    \PackageError{moodle}{#2 failed}%
    {If the XML file is not of importance to you: use package option "draft"}%
  }%
}%
\ifmoodle@tikz
  \AfterEndPreamble{%
    %\htmlize@record@expand{\embedaspict}%
    \let\oldincludegraphics=\includegraphics
    % patching includegraphics to trigger externalization
    \renewcommand{\includegraphics}[2][]{%
      %\message{moodle.sty: Processing \string\includegraphics[#1]{#2} for HTML^^J}%
      \tikz{\node[inner sep=0pt]{\oldincludegraphics[#1]{#2}};}%
    }%
    % externalized images must be included with the regular command
    \pgfkeys{/pgf/images/include external/.code={\oldincludegraphics{#1}}}%
    \html@action@newcommand{includegraphics}[2][]{%
      \message{moodle.sty: Processing \string\includegraphics[#1]{#2} ^^J}
      \global\advance\numpicturesread by 1\relax
      \edef\htmlize@imagetag{<IMG SRC="data:\TikzExportMIME;base64,\csname picbaselxiv@\the\numpicturesread\endcsname">}%
      \xa\g@addto@macro\xa\htmlize@output\xa{\htmlize@imagetag}%
    }%
  }%
\else
\html@action@newcommand{includegraphics}[2][]{%
  \bgroup% The grouping is to localize the changes caused by \setkeys.
    \message{moodle.sty: Processing \string\includegraphics[#1]{#2} for HTML...^^J}
    \setkeys*{moodle@includegraphics}{#1}%
    % Height or width should be given in TeX dimensions like cm or pt or in,
    % and are converted to pixels for web use using the ppi key.
    % TODO: Can we modify \includegraphics to accept height or width in
    %        pixels?
    % TODO: What about \includegraphics[scale=0.7] ?
    %        Other keys: keepaspectratio=true|false, angle (rotation), clip & trim
    %           -> the package option 'tikz' offers a workaround for this
    \ifnum\moodle@graphics@height@pixels=0\relax
      \ifnum\moodle@graphics@width@pixels=0\relax
        % No size specified.  Default to height of 200 pixels.
        \def\moodle@graphics@geometry{x200}%
        \def\moodle@graphics@htmlgeometry{}%
      \else
        % Width only specified.
        \edef\moodle@graphics@geometry{\number\moodle@graphics@width@pixels}%
        \edef\moodle@graphics@htmlgeometry{width=\number\moodle@graphics@width@pixels}%
      \fi
    \else
      \ifnum\moodle@graphics@width@pixels=0\relax
        % Height only specified.  The `x' is part of the syntax.
        \edef\moodle@graphics@geometry{x\number\moodle@graphics@height@pixels}%
        \edef\moodle@graphics@htmlgeometry{height=\number\moodle@graphics@height@pixels}%
      \else
        % Height and width both specified.  The `!' is part of the syntax.
        \edef\moodle@graphics@geometry{\number\moodle@graphics@width@pixels x\number\moodle@graphics@height@pixels!}%
        \edef\moodle@graphics@htmlgeometry{width=\number\moodle@graphics@width@pixels\otherspace height=\number\moodle@graphics@height@pixels}%
      \fi
    \fi
    %Look for the file, even if no extension is provided
    \moodle@media@find{#2}%
    %\message{<<\moodle@media@base>>^^J}
    %\message{<<\moodle@media@ext>>^^J}
    %Try to identify corresponding MIME-type
    \moodle@media@mime@identify{\moodle@media@ext}%
    \ifx\moodle@media@mime@current\relax
      % conversion needed
      \edef\moodle@media@pdf{\detokenize{.pdf}}%
      \edef\moodle@media@ext{\detokenize\xa{\moodle@media@ext}}%
      \ifnum0\ifx\moodle@media@ext\moodle@media@pdf1\fi\ifmoodle@svg1\fi=11\relax% PDF file and \SVG\ option active
        \def\moodle@media@ext@current{.svg}%
        \edef\moodle@media@mime@current{\moodle@media@SVGmime}%
        \edef\cmdline{\PDFtoSVG\otherspace "\moodle@media@base\moodle@media@ext" -o "\moodle@media@base\moodle@media@ext@current" 2>\DevNull}%
        \message{moodle.sty:   Converting '#2' to SVG...^^J}%
        \xa\ShellEscape\xa{\cmdline}%
        \moodle@checkconversionsuccess{\moodle@media@base\moodle@media@ext@current}{PDFtoSVG conversion}%
        %Next, optimize inline
        \ifwindows\else
          \edef\cmdline{\optiSVG < "\moodle@media@base\moodle@media@ext@current" >
           "\moodle@media@base.tmp.svg" && \MoveFiles\otherspace "\moodle@media@base.tmp.svg"
           "\moodle@media@base\moodle@media@ext@current"}%
          \message{moodle.sty:   Optimizing '\moodle@media@base\moodle@media@ext@current'...^^J}%
          \xa\ShellEscape\xa{\cmdline}%
          \moodle@checkconversionsuccess{\moodle@media@base\moodle@media@ext@current}{SVG optimization}%
        \fi
      \else% call ImageMagick
        \def\moodle@media@ext@current{.png}%
        \def\moodle@media@mime@current{image/png}%
        \edef\cmdline{\htmlize@imagemagick@convert\otherspace "\moodle@media@base\moodle@media@ext" -resize \moodle@graphics@geometry\otherspace "\moodle@media@base\moodle@media@ext@current"}%
        \message{moodle.sty:   Converting '#2' to PNG...^^J}%
        \xa\ShellEscape\xa{\cmdline}%
        \moodle@checkconversionsuccess{\moodle@media@base\moodle@media@ext@current}{ImageMagick conversion}%
        %Next, optimize inline
        \edef\cmdline{\optipng\otherspace "\moodle@media@base\moodle@media@ext@current"}%
        \message{moodle.sty:   Optimizing '\moodle@media@base\moodle@media@ext@current'...^^J}%
        \xa\ShellEscape\xa{\cmdline}%
        \moodle@checkconversionsuccess{\moodle@media@base\moodle@media@ext@current}{PNG optimization}%
      \fi
    \fi
    %Next, convert the file to base64 encoding
    \ConvertToBaseLXIV{\moodle@media@base}{\moodle@media@ext@current}%
    %Now, save that base64 encoding in a TeX macro
    \def\moodle@newpic@baselxiv{}%
    \message{moodle.sty:   Reading base64 file '\moodle@media@base.enc'...^^J}%
    \openin\baseLXIVdatafile="\moodle@media@base.enc"
      \savebaselxivdata@recursive
    \closein\baseLXIVdatafile
    %Clean up files
    \ifx\moodle@media@ext@current\moodle@media@ext
      \ShellEscape{\DeleteFiles\otherspace "\moodle@media@base.enc"}%
    \else
      \ShellEscape{\DeleteFiles\otherspace "\moodle@media@base.enc" "\moodle@media@base\moodle@media@ext@current"}%
    \fi
    \xa\global\xa\let\csname picbaselxiv@graphics@#2\endcsname=\moodle@newpic@baselxiv%
    \ifmoodle@pluginfile
      \edef\htmlize@imagetag{<IMG \moodle@graphics@htmlgeometry\otherspace SRC="@@PLUGINFILE@@/\moodle@media@base\moodle@media@ext@current">}%
      \xa\filename@parse\xa{\moodle@media@base}%
      \xdef\moodle@media@path{\filename@area}%
      \xdef\moodle@media@base{\filename@base}%
      \xa\g@addto@macro\xa\htmlize@embeddedfiletags\xa{<file name="\moodle@media@base}%
      \xa\g@addto@macro\xa\htmlize@embeddedfiletags\xa{\moodle@media@ext@current"}
      \xa\g@addto@macro\xa\htmlize@embeddedfiletags\xa{ path="/\moodle@media@path" encoding="base64">}%
      \xa\g@addto@macro\xa\htmlize@embeddedfiletags\xa{\csname picbaselxiv@graphics@#2\endcsname</file>}%
    \else
      \edef\htmlize@imagetag{<IMG \moodle@graphics@htmlgeometry\otherspace SRC="data:\moodle@media@mime@current;base64,\csname picbaselxiv@graphics@#2\endcsname">}%
    \fi
    \xa\g@addto@macro\xa\htmlize@output\xa{\htmlize@imagetag}%
    \message{moodle.sty:   <IMG> tag inserted.^^J}%
  \egroup
}%
\fi
%    \end{macrocode}
% This macro is in charge of throwing a system call to convert local files to base64.
%    \begin{macrocode}
\def\ConvertToBaseLXIV#1#2{%
  \message{moodle.sty:   Converting '#1#2' to base64...^^J}%
  \ConvertToBaseLXIV@int{#1}{#2}
  \xa\ShellEscape\xa{\cmdline}%
  \moodle@checkconversionsuccess{#1.enc}{Base64 conversion}%
}%
\ifwindows
  \def\ConvertToBaseLXIV@int#1#2{%
    \def\cmdline{\baselxiv\otherspace -encode "#1#2"\otherspace tmp.b64 && findstr /vbc:"---" tmp.b64 > "#1.enc" && del tmp.b64}%
  }% Starting from Windows 7, CertUtil is included by default. There should be no windows XP still running
\else
  \ifmacosx
    \def\ConvertToBaseLXIV@int#1#2{%
      \def\cmdline{\baselxiv\otherspace -b 64 -i "#1#2"\otherspace -o "#1.enc"}%
    }%
  \else % Linux, Cygwin
    \def\ConvertToBaseLXIV@int#1#2{%
      \def\cmdline{\baselxiv\otherspace "#1#2"\otherspace > "#1.enc"}%
    }% base64 is part of coreutils, add "-w 64" to get exactly the previous behavior  %
  \fi
\fi
%    \end{macrocode}
% The following code accomplishes the reading of an \filenm{.enc} file into memory.
% It is also used by the \TikZ\ code below.
%    \begin{macrocode}
\newread\baseLXIVdatafile
\def\savebaselxivdata@recursive{%
  \ifeof\baseLXIVdatafile
    \let\baselxiv@next=\relax
  \else
    \read\baseLXIVdatafile to \datalinein
    %\message{<<\datalinein>>^^J}
    \ifx\datalinein\@moodle@par
      \let\baselxiv@next=\relax
    \else
      %We add tokens manually, rather than with \g@addto@macro, to save time.
      \xa\xa\xa\gdef\xa\xa\xa\moodle@newpic@baselxiv\xa\xa\xa{\xa\moodle@newpic@baselxiv\datalinein^^J}%
      \let\baselxiv@next=\savebaselxivdata@recursive
    \fi
  \fi
  \baselxiv@next
}
%    \end{macrocode}
%
% \subsection{\TikZ\ Picture Handling}
% If the user is not using the \TikZ\ package, there is no need to waste time
% loading it.  Without \TikZ\ loaded, however, many of the following commands
% are undefined.
% Our solution is to wait until |\AtBeginDocument| and then test whether
% \TikZ\ is loaded.  If so, we make the appropriate definitions.
% \begin{macro}{TikZ}
% \changes{v0.7}{2020/07/14}{Support \emph{tikz}\ command}
%    \begin{macrocode}
\AtBeginDocument{
  \ifx\tikzpicture\@undefined
    \moodle@tikzloadedfalse
  \else
    \moodle@tikzloadedtrue
  \fi
  \ifmoodle@draftmode
    \long\def\tikzifexternalizing#1#2{#2}%
  \else
  \ifmoodle@tikzloaded
    \usetikzlibrary{external}%
    \tikzexternalize%
    \tikzset{external/force remake}%
    \def\moodle@basename{\tikzexternalrealjob-tikztemp-\the\numconvertedpictures}%
    \ifmoodle@svg
      \def\TikzExportExtension{.svg}%
      \def\TikzExportMIME{image/svg+xml}%
      \ifpdfoutput% tex engines defaulting to PDF output (pdflatex, xelatex, lualatex)
        \def\ExportTikz{ \message{moodle.sty: Converting picture '\moodle@basename.pdf' to SVG...^^J}%
                         \edef\cmdline{\PDFtoSVG\otherspace "\moodle@basename.pdf" -o "\moodle@basename\TikzExportExtension" 2>\DevNull}%
\message{\cmdline}
                         \xa\ShellEscape\xa{\cmdline}%
                         \moodle@checkconversionsuccess{\moodle@basename\TikzExportExtension}{PDFtoSVG conversion}%
                       }%
      \else % latex
        \def\ExportTikz{ \message{moodle.sty: Converting picture '\moodle@basename.ps' to SVG...^^J}%
                         \edef\cmdline{\PDFtoSVG\otherspace "\moodle@basename.ps" -o "\moodle@basename\TikzExportExtension" 2>\DevNull}%
\message{\cmdline}
                        \xa\ShellEscape\xa{\cmdline}%
                         \moodle@checkconversionsuccess{\moodle@basename\TikzExportExtension}{PStoSVG conversion}%
                       }%
      \fi
      \def\OptimizeExport{ \message{moodle.sty:   Optimizing '\moodle@basename.svg'...^^J}%
                           \edef\cmdline{\optiSVG\otherspace < "\moodle@basename\TikzExportExtension" >
                                         "\moodle@basename.tmp.svg" && \MoveFiles\otherspace "\moodle@basename.tmp.svg"
                                         "\moodle@basename\TikzExportExtension" }%
\message{\cmdline}
                           \xa\ShellEscape\xa{\cmdline}%
                           \moodle@checkconversionsuccess{\moodle@basename\TikzExportExtension}{SVG optimization}%
                         }%
    \else
      \def\TikzExportExtension{.png}%
      \def\TikzExportMIME{image/png}%
      \ifpdfoutput% tex engines defaulting to PDF output (pdflatex, xelatex, lualatex)
        \def\ExportTikz{ \message{moodle.sty: Converting picture '\moodle@basename.pdf' to PNG...^^J}%
                         \edef\gscmdline{\gs\otherspace -sOutputFile=\moodle@basename\TikzExportExtension\otherspace -r150 \moodle@basename.pdf}%
                         \xa\ShellEscape\xa{\gscmdline}%
                         \moodle@checkconversionsuccess{\moodle@basename\TikzExportExtension}{Ghostscript conversion}%
                       }%
      \else % latex
        \def\ExportTikz{ \message{moodle.sty: Converting picture '\moodle@basename.ps' to PNG...^^J}%
                         \edef\gscmdline{\gs\otherspace -sOutputFile=\moodle@basename\TikzExportExtension\otherspace -r150 \\moodle@basename.ps}%
                         \xa\ShellEscape\xa{\gscmdline}%
                         \moodle@checkconversionsuccess{\moodle@basename\TikzExportExtension}{Ghostscript conversion}%
                       }%
      \fi
      \def\OptimizeExport{ \message{moodle.sty:   Optimizing '\moodle@basename.png'...^^J}%
                           \edef\cmdline{\optipng\otherspace \moodle@basename\TikzExportExtension}%
                           \xa\ShellEscape\xa{\cmdline}%
                           \moodle@checkconversionsuccess{\moodle@basename\TikzExportExtension}{PNG optimization}%
                         }%
    \fi
    \let\moodle@oldtikzpicture=\tikzpicture
    %The following code lets us run things *before* the normal \begin{tikzpicture}.
    \renewenvironment{tikzpicture}{%
      \global\advance\numconvertedpictures by 1\relax
      %\jobnamewithsuffixtomacro{\htmlize@picbasename}{-tikztemp-\the\numconvertedpictures}%
      %\xa\tikzsetnextfilename\xa{\htmlize@picbasename}%
      \tikzsetnextfilename{\tikzexternalrealjob-tikztemp-\the\numconvertedpictures}%
      \moodle@oldtikzpicture%
    }{}%
    % However, the tikz externalize library does *not* run \end{tikzpicture}.
    % In order to run commands after the tikz picture is done compiling, we need to
    % use a hook into \tikzexternal@closeenvironments.
    \g@addto@macro{\tikzexternal@closeenvironments}{%
      \moodle@endtikzpicture@hook
    }
% The following could replace calls to \pdftopng, \pngoptim and \pngtobaselxiv
%    \tikzset{external/system call/.add={}{;
%         gs -dBATCH -dNOPAUSE -sDEVICE=pngalpha -sOutputFile="\image.png" -r150 "\image.pdf";
%         optipng -clobber -strip all -quiet "\image.png";
%         base64 "\image.png" > "\image.enc"
%      }
%    }
% With the following mechanism, we could trigger something when the externalized images are included back.
%    \pgfkeys{/pgf/images/include external/.code={\pgfimage{#1}\@moodle@ifgeneratexml{\savebaselxivdata}{}}}
%
    \def\moodle@endtikzpicture@hook{%
      \@moodle@ifgeneratexml{%
        \ExportTikz
        \OptimizeExport
        \ConvertToBaseLXIV{\moodle@basename}{\TikzExportExtension}%
        \IfFileExists{\moodle@basename.enc}{}{\PackageError{moodle}{Conversion failed}{Check your base64 conversion utiliy}}%
        \message{moodle.sty:   Reading base64 file '\tikzexternalrealjob-tikztemp-\the\numconvertedpictures.enc'...^^J}%
        \savebaselxivdata
        \message{moodle.sty:   base64 data saved.^^J}%
      }{}%
    }
    \ifmoodle@tikz
      \tikzset{external/optimize=false}% due to redefinition, includegraphics must not be optimized away
    \else
      \tikzset{external/optimize=true}%
      \tikzset{external/optimize command away={\VerbatimInput}{1}}%
    \fi
    %
    % The HTMLizer version of the tikzpicture environment,
    % which writes an <IMG> tag to the \XML\ file.
    \htmlize@record@environment{tikzpicture}%
    \g@addto@macro\htmlize@activate@environments{%
      \let\tikzpicture\relax\let\endtikzpicture\relax
      \NewEnviron{tikzpicture}[1][]{%
        \global\advance\numpicturesread by 1\relax
        \ifmoodle@pluginfile
          \edef\htmlize@imagetag{<IMG SRC="@@PLUGINFILE@@/tikz/\tikzexternalrealjob-tikztemp-\the\numpicturesread\TikzExportExtension">}%
          \xa\g@addto@macro\xa\htmlize@embeddedfiletags\xa{<file name="\tikzexternalrealjob-tikztemp-}%
          \xa\g@addto@macro\xa\htmlize@embeddedfiletags\xa{\the\numpicturesread}%
          \xa\g@addto@macro\xa\htmlize@embeddedfiletags\xa{\TikzExportExtension"}
          \xa\g@addto@macro\xa\htmlize@embeddedfiletags\xa{ path="/tikz/" encoding="base64">}%
          \xa\g@addto@macro\xa\htmlize@embeddedfiletags\xa{\csname picbaselxiv@\the\numpicturesread\endcsname</file>}%
        \else
          \edef\htmlize@imagetag{<IMG SRC="data:\TikzExportMIME;base64,\csname picbaselxiv@\the\numpicturesread\endcsname">}%
        \fi
        \xa\g@addto@macro\xa\htmlize@output\xa{\htmlize@imagetag}%
      }[\htmlize@patchendenvironment]%
    }%
    \html@action@newcommand{tikz}[2][]{%
%      \message{>>> Processing \string\tikz[#1]{...} ^^J}
      \global\advance\numpicturesread by 1\relax
      \ifmoodle@pluginfile
        \edef\htmlize@imagetag{<IMG SRC="@@PLUGINFILE@@/tikz/\tikzexternalrealjob-tikztemp-\the\numpicturesread\TikzExportExtension">}%
        \xa\g@addto@macro\xa\htmlize@embeddedfiletags\xa{<file name="\tikzexternalrealjob-tikztemp-}%
        \xa\g@addto@macro\xa\htmlize@embeddedfiletags\xa{\the\numpicturesread}%
        \xa\g@addto@macro\xa\htmlize@embeddedfiletags\xa{\TikzExportExtension"}
        \xa\g@addto@macro\xa\htmlize@embeddedfiletags\xa{ path="/tikz/" encoding="base64">}%
        \xa\g@addto@macro\xa\htmlize@embeddedfiletags\xa{\csname picbaselxiv@\the\numpicturesread\endcsname</file>}%
      \else
        \edef\htmlize@imagetag{<IMG SRC="data:\TikzExportMIME;base64,\csname picbaselxiv@\the\numpicturesread\endcsname">}%
      \fi
      \xa\g@addto@macro\xa\htmlize@output\xa{\htmlize@imagetag}%
    }%
  \else
    %TikZ not loaded.  Provide dummy definitions for commands.
    \long\def\tikzifexternalizing#1#2{#2}%
  \fi
  \fi
  \ifmoodle@tikz
    \tikzstyle{moodlepict}=[minimum height=1em,inner sep=0pt,execute at begin node={\begin{varwidth}{\linewidth}},execute at end node={\end{varwidth}}]
    \newcommand\embedaspict[1]{\tikz[baseline=-\the\dimexpr\fontdimen22\textfont2\relax]{\node[moodlepict]{\mbox{#1}};}}
    \htmlize@record@expand{\embedaspict}
  \fi
}

\newcount\numconvertedpictures
\numconvertedpictures=0\relax
\newcount\numpicturesread
\numpicturesread=0\relax

\def\savebaselxivdata{%
  \def\moodle@newpic@baselxiv{}%
  \openin\baseLXIVdatafile=\tikzexternalrealjob-tikztemp-\the\numconvertedpictures.enc\relax
    \savebaselxivdata@recursive
  \closein\baseLXIVdatafile
  \xa\global\xa\let\csname picbaselxiv@\the\numconvertedpictures\endcsname=\moodle@newpic@baselxiv%
}

\ifmoodle@tikz
  \ifmoodle@tikzloaded
    \PackageWarning{moodle}{With package option 'tikz', you should not load TikZ manually.}%
  \fi
  \RequirePackage{tikz}%
  \RequirePackage{varwidth}% for the command |embedaspict|
\fi
%    \end{macrocode}
% Finally, we clean up our mess by deleting the temporary PDF, \PNG, and ENC
% files we created.
% With XeTeX, we cannot clean pictures pdf's because they are actually included in the final pdf afterwards, by xdvipdfmx.
% With LuaTeX, it looks like |\AtEndDocument| is too early to for this cleaning step.
%    \begin{macrocode}
\AfterEndDocument{%
  \ifmoodle@tikzloaded
    \@moodle@ifgeneratexml{%
      \ifXeTeX
        % we must keep picture pdf's for subsequent linking (xdvipdfmx)
        \ifwindows
          \ShellEscape{powershell.exe -noexit "del * -include \tikzexternalrealjob-tikztemp-*.* -exclude *.pdf}%
        \else
          \ShellEscape{find . -type f -name "\tikzexternalrealjob-tikztemp-*.*" -not -name "*.pdf" -delete}%
        \fi
      \else
        \ShellEscape{\DeleteFiles\otherspace \tikzexternalrealjob-tikztemp-*.*}%
      \fi
    }{}%
  \fi
}
%    \end{macrocode}
% TODO:
% * sizing options for TikZ pictures?
%\end{macro}
%
% \subsection{Other Media}
% Support for other media comes through the |\url|\marg{link} and |\href|\marg{link}\marg{text}
% commands.
\AtEndPreamble{%
  \@ifpackageloaded{hyperref}{%
    \let\oldhref\href
    \let\oldurl\url
    \def\moodle@hyper@readexternallink#1#2#3#4:#5:#6\\#7{%
% 1) The link type (the string "link" in the cases I observed)
% 2) The URL fragment (i.e. what comes after # in the end),
% 3) the text replacement,
% 4) the URL scheme (http, https, mailto, file, run etc),
% 5) the URL details (in general, that is whole authority + path + query).
%    If the URL contains a column (specification of a password in userinfo or a
%    port in authority), we get here whatever comes before the first column (:)
% 6) empty is URL contains no column. Otherwise, filled with whatever follows
%    a first column (:) after the scheme,
% 7) the whole URL again (fragment removed).
      \def\filename@ext{}%
      \def\filename@area{}%
      \def\filename@base{}%
      \ifx\\#6\\% if ##6 is empty (the URL contains no column -> no scheme -> local)
        %\@hyper@linkfile file:#7\\{#3}{#2}{#7}% local file
        %Local file (##1|##2|##3|##4|##5|##6): ##7\par
        \filename@parse{#4}%
        \ConvertToBaseLXIV{\filename@area\filename@base}{.\filename@ext}%
      \else
        \ifx\\#4\\%if ##4 is empty (no scheme was specified -> local file)
%          %\@hyper@linkfile file:#7\\{#3}{#2}{#7}% Mac filename ?
%          Local file (Mac? ##1|##2|##3|##4|##5|##6): ##7\par
          \filename@parse{#4}%
          \ConvertToBaseLXIV{\filename@area\filename@base}{.\filename@ext}%
        \else
          \def\@pdftempa{#4}%
          \ifx\@pdftempa\@pdftempwordfile% scheme is "file:"
%            %\@hyper@linkfile#7\\{#3}{#2}{#7}% file
%            Local file (no column ##1|##2|##3|##4|##5|##6): ##7\par
            \filename@parse{#5}%
            \ConvertToBaseLXIV{\filename@area\filename@base}{.\filename@ext}%
          \else
            \ifx\@pdftempa\@pdftempwordrun% scheme is "run:"
              \ifHy@pdfa
                \Hy@Error{%
                  PDF/A: Launch action is prohibited%
                }\@ehc
                \begingroup
                  \leavevmode
%                  ##2%
                \endgroup
              \else% not in PDF/A mode -> run is allowed
                %\@hyper@launch#7\\{#3}{#2}% run local file
%                Run local file (##1|##2|##3|##4|##5|##6): ##7\par
                \filename@parse{#5}%
                \ConvertToBaseLXIV{\filename@area\filename@base}{.\filename@ext}%
              \fi
            \else% scheme is neither "file" nor "run", assuming it is a web protocol
              %\hyper@linkurl{#3}{#7\ifx\\#2\\\else\hyper@hash#2\fi}% URL
%              URL (##1|##2|##3|##4|##5|##6): ##7\par
              \filename@parse{#7}%
            \fi
          \fi
        \fi
      \fi
      \xdef\moodle@media@ext{.\filename@ext}%
      \xdef\moodle@media@base{\filename@area\filename@base}%
    }%
%    \html@action@newcommand{href}[3][]{%
%      \bgroup% The grouping is to localize the changes caused by \setkeys.
%        \message{moodle.sty: Processing \string\href[#1]{#2}{#3} for HTML...^^J}%
% the following macro is a modified version of hyperref's |\hyper@readexternallink|
%        \let\@hyper@readexternallink\moodle@hyper@readexternallink
%        \oldhref[#1]{#2}{#3}%
        %\message{<<\moodle@media@base>>^^J}
        %\message{<<\moodle@media@ext>>^^J}
        %Try to identify corresponding MIME-type
%        \moodle@media@mime@identify{\moodle@media@ext}%
        %\edef\moodle@media@mime@current{\detokenize\xa{\moodle@media@mime@current}}%
%        \xa\message\xa{moodle.sty: \moodle@media@mime@current^^J}%
%        \ifx\moodle@media@mime@current\relax
%          \xa\g@addto@macro\xa\htmlize@output\xa{<A href=\otherampersand\otherhash34;\moodle@media@base\moodle@media@ext\otherampersand\otherhash34;>#3</A>}%
%          \message{moodle.sty:   <A> tag inserted.^^J}%
%        \else
%          \filename@parse{\moodle@media@mime@current}%
%          \def\@tmp{audio/}
%          \ifx\filename@area\@tmp
%            \IfFileExists{\moodle@media@base.enc}{%
%              \def\moodle@newpic@baselxiv{}%
%              \openin\baseLXIVdatafile=\moodle@media@base.enc\relax
%                \savebaselxivdata@recursive
%              \closein\baseLXIVdatafile
%              \xa\g@addto@macro\xa\htmlize@output\xa{<audio controls src="data:\moodle@media@mime@current;base64,\moodle@newpic@baselxiv">#3</audio>}%
%            }{%
%              \xa\g@addto@macro\xa\htmlize@output\xa{<audio controls src="\moodle@media@base\moodle@media@ext">#3</audio>}%
%            }%
%            \message{moodle.sty:   <audio> tag inserted.^^J}%
%          \else
%            \def\@tmp{video/}
%            \ifx\filename@area\@tmp
%              \IfFileExists{\moodle@media@base.enc}{%
%                \def\moodle@newpic@baselxiv{}%
%                \openin\baseLXIVdatafile=\moodle@media@base.enc\relax
%                  \savebaselxivdata@recursive
%                \closein\baseLXIVdatafile
%                \xa\g@addto@macro\xa\htmlize@output\xa{<video controls src="data:\moodle@media@mime@current;base64,\moodle@newpic@baselxiv">#3</video>}%
%              }{%
%                \xa\g@addto@macro\xa\htmlize@output\xa{<video controls src="\moodle@media@base\moodle@media@ext">#3</video>}%
%              }%
%              \message{moodle.sty:   <video> tag inserted.^^J}%
%            \else
%              \PackageWarning{moodle}{cannot recognize MIME type of #2. Ignoring it.}%
%            \fi
%          \fi
%        \fi
%      \egroup
%    }%
%    \html@action@def\url#1{%
%      \bgroup% The grouping is to localize the changes caused by \setkeys.
%        \message{moodle.sty: Processing \string\url{#1} for HTML...^^J}%
% the following macro is a modified version of hyperref's |\hyper@readexternallink|
%        \let\@hyper@readexternallink\moodle@hyper@readexternallink
%        \oldhref{#1}{#1}%
%        \xa\g@addto@macro\xa\htmlize@output\xa{<A href=\otherampersand\otherhash34;\moodle@media@base\moodle@media@ext\otherampersand\otherhash34;>#1</A>}%
%        \message{moodle.sty:   <A> tag inserted.^^J}%
%      \egroup
%    }%
  }{}%
}%
%
% \subsection{Verbatim Code}
%
% We start by defining a macro to parameter a style for code box display in Moodle
%    \begin{macrocode}
\def\xmlDisplayVerbatimBox{border-top: thin solid; border-bottom: thin solid}%
%    \end{macrocode}
% Then we set a macro to escape some characters that would not play well with \HTML.
%    \begin{macrocode}
\begingroup
\catcode`\<=\active\relax
\catcode`\>=\active\relax
\catcode`\"=\active\relax
\catcode`\'=\active\relax
\catcode`\&=\active\relax
\gdef\moodle@HackTML{%
  \catcode`\<=\active\relax
  \catcode`\>=\active\relax
  \catcode`\"=\active\relax
  \catcode`\'=\active\relax
  \catcode`\&=\active\relax
  \gdef<{\otherampersand lt;}%
  \gdef>{\otherampersand gt;}%
  \gdef"{\otherampersand quot;}%
  \gdef'{\otherampersand apos;}%
  \gdef&{\otherampersand amp;}%
}%
\endgroup
%    \end{macrocode}
% First, let us handle |\verbatiminput| from the `verbatim' package
%    \begin{macrocode}
\html@action@def\verbatiminput#1{%
  \message{moodle.sty: Processing \string\verbatiminput{#1} for HTML ^^J}%
  \g@addto@macro\htmlize@output{<PRE style="\xmlDisplayVerbatimBox"><CODE>}%
  %%%%%%%%%%%%%% from verbatim %%%%%%%%%%%%%%%%%
  \@bsphack
  \let\do\@makeother\dospecials
  \catcode`\^^M\active
  \moodle@HackTML
  \def\verbatim@processline{\xa\g@addto@macro\xa\htmlize@output\xa{\the\verbatim@line<BR/>}}
  \verbatim@readfile{#1}%
  \@esphack
   %%%%%%%%%%%%%%%%%%%%%%%%%%%%%%%%%%%%%%%%%%%%%%
  \g@addto@macro\htmlize@output{</CODE></PRE>}%
}%
%    \end{macrocode}
% Second, we deal with |\VerbatimInput| from `fancyvrb' or `fvextra'
%    \begin{macrocode}
\@ifpackageloaded{minted}{\PackageError{moodle}{'moodle' should be loaded before 'minted'.}%
{'moodle' loads 'fancybox' which, unfortunately, redefines verbatim commands.}}{\relax}%
\@ifpackageloaded{fvextra}{\PackageError{moodle}{'moodle' should be loaded before 'fvextra'.}%
{'moodle' loads 'fancybox' which, unfortunately, redefines verbatim commands.}}{\relax}%
\@ifpackageloaded{fancyvrb}{\PackageError{moodle}{'moodle' should be loaded before 'fancyvrb'.}%
{'moodle' loads 'fancybox' which, unfortunately, redefines verbatim commands.}}{\relax}%

\def\moodle@FV#1{%
  \html@action@newcommand{#1}[2][]{%
    \message{moodle.sty: Processing \@backslashchar#1[##1]{##2} for HTML ^^J}%
    \def\FV@KeyValues{##1}%
    \FV@UseKeyValues% import options defined in #1
    \def\FV@Input####1{
      \immediate\openin\FV@InFile ####1\relax
      \ifeof\FV@InFile
        \FV@Error{No verbatim file ####1}\FV@eha
        \immediate\closein\FV@InFile
      \else
        \FV@CatCodes
        \moodle@HackTML
        \expandafter\FV@@Input
      \fi}%
    \moodle@VerbatimInput{##2}%
  }%
}
\moodle@FV{VerbatimInput}%
\moodle@FV{LVerbatimInput}%
\moodle@FV{BVerbatimInput}%
\def\moodle@VerbatimInput#1{%
  \g@addto@macro\htmlize@output{<PRE style="\xmlDisplayVerbatimBox"><CODE>}%
  %%%%%%%% using material from fancyvrb and fvextra  %%%%%%%%
  %\begingroup
  \def\moodle@verbatim@addlinenumber##1{%
    \g@addto@macro\htmlize@output{<span style="font-size: 80\otherpercent;
         background-color: \otherhash f0f0f0; padding: 0 5px 0 5px; display:
         inline-block; width: 16pt; ##1">}%
    \if@FV@NumberBlankLines
      \xa\g@addto@macro\xa\htmlize@output\xa{\the\c@FancyVerbLine</span>}%
    \else
      \ifx\FV@Line\empty
        \xa\g@addto@macro\xa\htmlize@output\xa{\otherampersand nbsp;</span>}%
      \else
        \xa\g@addto@macro\xa\htmlize@output\xa{\the\c@FancyVerbLine</span>}%
      \fi
    \fi
  }
  % redefine the ProcessLine routine ('fancyvrb' and 'fvextra') for \XML\ output
  \def\FV@ProcessLine##1{%
    \ifcsname FV@HighlightLine:\number\c@FancyVerbLine\endcsname
      \xdef\moodle@FV@tagB{<mark>}% fvextra triggered highlighting for this line
      \xdef\moodle@FV@tagE{</mark>}%
    \else
      \xdef\moodle@FV@tagB{}% no highlighting
      \xdef\moodle@FV@tagE{}%
    \fi
    \catcode`\`=12%
    \def\FV@Line{##1}%
    \ifx\FV@LeftListNumber\relax

    \else% line numbers displayed on the left side
      \moodle@verbatim@addlinenumber{text-align: right}%
    \fi
    \xa\g@addto@macro\xa\htmlize@output\xa{\moodle@FV@tagB}%
    \xa\g@addto@macro\xa\htmlize@output\xa{\FV@Line}%
    \xa\g@addto@macro\xa\htmlize@output\xa{\moodle@FV@tagE}%
    \ifx\FV@RightListNumber\relax\else% line numbers on the right side
      \moodle@verbatim@addlinenumber{text-align: left; float: right}%
    \fi
    \g@addto@macro\htmlize@output{<BR/>}%linebreak
  }
  \global\FV@CodeLineNo\z@% reset codeline counter
  \frenchspacing% Cancels special punctuation spacing.
  \FV@DefineWhiteSpace
  \def\FV@Space{\space}
  \FV@DefineTabOut% replace tabs with a series a whitespaces
  \ifdefined\FV@HighlightLinesPrep
    \FV@HighlightLinesPrep% prepare highlighting if 'fvextra' is loaded
  \fi
  \FV@Input{#1}%
  %\endgroup
  %%%%%%%%%%%%%%%%%%%%%%%%%%%%%%%%%%%%%%%%%%%%%%%%%%%%%%%%
  \g@addto@macro\htmlize@output{</CODE></PRE>}%
}%
\AtEndPreamble{%
  \@ifpackageloaded{fancyvrb}{%
    % custom settings for display
    \fvset{frame=lines,label={[Beginning of code]End of code},framesep=3mm,numbersep=9pt}%
  }{\relax}%
}
%    \end{macrocode}
% Third, we patch `minted' so that it also calls pygmentize to generate \HTML\ code.
%    \begin{macrocode}
\AtEndPreamble{% this definition should prevail because `minted' gets loaded after `moodle'
\@ifpackageloaded{minted}{%
  \newcounter{moodle@pygmentizecounter}%
  \renewcommand{\minted@pygmentize}[2][\minted@outputdir\minted@jobname.pyg]{%
    \minted@checkstyle{\minted@get@opt{style}{default}}%
    \stepcounter{minted@pygmentizecounter}%
    \ifthenelse{\equal{\minted@get@opt{autogobble}{false}}{true}}%
      {\def\minted@codefile{\minted@outputdir\minted@jobname.pyg}}%
      {\def\minted@codefile{#1}}%
    \ifthenelse{\boolean{minted@isinline}}%
      {\def\minted@optlistcl@inlines{%
        \minted@optlistcl@g@i
        \csname minted@optlistcl@lang\minted@lang @i\endcsname}}%
      {\let\minted@optlistcl@inlines\@empty}%
    \def\minted@cmdtemplate##1##2{%
      \ifminted@kpsewhich
        \ifwindows
          \detokenize{for /f "usebackq tokens=*"}\space\@percentchar\detokenize{a
          in (`kpsewhich}\space\minted@codefile\detokenize{`) do}\space
        \fi
      \fi
      \MintedPygmentize\space -l #2 -f ##1 -F tokenmerge
      \minted@optlistcl@g \csname minted@optlistcl@lang\minted@lang\endcsname
      \minted@optlistcl@inlines
      \minted@optlistcl@cmd -o \minted@outputdir##2\space
      \ifminted@kpsewhich
        \ifwindows
          \@percentchar\detokenize{a}%
        \else
          \detokenize{`}kpsewhich \minted@codefile\space
            \detokenize{||} \minted@codefile\detokenize{`}%
        \fi
      \else
        \minted@codefile
      \fi}%
    \def\minted@cmd{\minted@cmdtemplate{latex -P commandprefix=PYG}{\minted@infile}}
    % For debugging, uncomment: %%%%
    \immediate\typeout{\minted@cmd}%
    % %%%%
    \def\minted@cmdHTML{\minted@cmdtemplate{html -P noclasses -P
      nowrap -P hl_lines="\FV@HighlightLinesList" -P
      style="\minted@get@opt{style}{default}"}{\csname minted@infileHTML\the\c@minted@pygmentizecounter\endcsname}}%
    \def\minted@cmdPNG{\minted@cmdtemplate{png -P
      line_numbers=True}{\minted@infilePNG}}%
    \def\minted@cmdSVG{\minted@cmdtemplate{svg -P
      linenos=True}{\minted@infileSVG}}%
    \ifthenelse{\boolean{minted@cache}}%
      {%
        \ifminted@frozencache
        \else
          \ifx\XeTeXinterchartoks\minted@undefined
            \ifthenelse{\equal{\minted@get@opt{autogobble}{false}}{true}}%
              {\edef\minted@hash{\pdf@filemdfivesum{#1}%
                \pdf@mdfivesum{\minted@cmd autogobble(\ifx\FancyVerbStartNum\z@
                0\else\FancyVerbStartNum\fi-\ifx\FancyVerbStopNum\z@
                0\else\FancyVerbStopNum\fi)}}}%
              {\edef\minted@hash{\pdf@filemdfivesum{#1}%
                \pdf@mdfivesum{\minted@cmd}}}%
          \else
            \ifx\mdfivesum\minted@undefined
              \immediate\openout\minted@code\minted@jobname.mintedcmd\relax
              \immediate\write\minted@code{\minted@cmd}%
              \ifthenelse{\equal{\minted@get@opt{autogobble}{false}}{true}}%
                {\immediate\write\minted@code{autogobble(\ifx\FancyVerbStartNum\z@
                 0\else\FancyVerbStartNum\fi-\ifx\FancyVerbStopNum\z@
                0\else\FancyVerbStopNum\fi)}}{}%
              \immediate\closeout\minted@code
              \edef\minted@argone@esc{#1}%
              \StrSubstitute{\minted@argone@esc}{\@backslashchar}{\@backslashchar\@backslashchar}[\minted@argone@esc]%
              \StrSubstitute{\minted@argone@esc}{"}{\@backslashchar"}[\minted@argone@esc]%
              \edef\minted@tmpfname@esc{\minted@outputdir\minted@jobname}%
              \StrSubstitute{\minted@tmpfname@esc}{\@backslashchar}{\@backslashchar\@backslashchar}[\minted@tmpfname@esc]%
              \StrSubstitute{\minted@tmpfname@esc}{"}{\@backslashchar"}[\minted@tmpfname@esc]%
              %Cheating a little here by using ASCII codes to write `{` and `}`
              %in the Python code
              \def\minted@hashcmd{%
                \detokenize{python -c "import hashlib; import os;
                  hasher = hashlib.sha1();
                  f =
                  open(os.path.expanduser(os.path.expandvars(\"}\minted@tmpfname@esc.mintedcmd\detokenize{\")),
                   \"rb\");
                  hasher.update(f.read());
                  f.close();
                  f =
                  open(os.path.expanduser(os.path.expandvars(\"}\minted@argone@esc\detokenize{\")),
                   \"rb\");
                  hasher.update(f.read());
                  f.close();
                  f =
                  open(os.path.expanduser(os.path.expandvars(\"}\minted@tmpfname@esc.mintedmd5\detokenize{\")),
                   \"w\");
                  macro = \"\\edef\\minted@hash\" + chr(123) + hasher.hexdigest()
                  + chr(125) + \"\";
                  f.write(\"\\makeatletter\" + macro +
                  \"\\makeatother\\endinput\n\");
                  f.close();"}}%
              \ShellEscape{\minted@hashcmd}%
              \minted@input{\minted@outputdir\minted@jobname.mintedmd5}%
            \else
              \ifthenelse{\equal{\minted@get@opt{autogobble}{false}}{true}}%
               {\edef\minted@hash{\mdfivesum file {#1}%
                  \mdfivesum{\minted@cmd autogobble(\ifx\FancyVerbStartNum\z@
                  0\else\FancyVerbStartNum\fi-\ifx\FancyVerbStopNum\z@
                  0\else\FancyVerbStopNum\fi)}}}%
               {\edef\minted@hash{\mdfivesum file {#1}%
                  \mdfivesum{\minted@cmd}}}%
            \fi
          \fi
          \edef\minted@infile{\minted@cachedir/\minted@hash.pygtex}%
          \edef\minted@temp@infileHTML{\minted@cachedir/\minted@hash.html}%
          \global\cslet{minted@infileHTML\the\c@minted@pygmentizecounter}{\minted@temp@infileHTML}%
          %\global\edef\minted@infilePNG{\minted@cachedir/\minted@hash.png}%
          %\global\edef\minted@infileSVG{\minted@cachedir/\minted@hash.svg}%
          \IfFileExists{\minted@infile}{}{%
            \ifthenelse{\equal{\minted@get@opt{autogobble}{false}}{true}}{%
              \minted@autogobble{#1}}{}%
            \ShellEscape{\minted@cmd}%
            \ShellEscape{\minted@cmdHTML}%
            %\ShellEscape{\minted@cmdPNG}%
            %\ShellEscape{\minted@cmdSVG}%
            }%
        \fi
        \ifthenelse{\boolean{minted@finalizecache}}%
         {%
            \edef\minted@cachefilename{listing\arabic{minted@pygmentizecounter}.pygtex}%
            \edef\minted@actualinfile{\minted@cachedir/\minted@cachefilename}%
            \ifwindows
              \StrSubstitute{\minted@infile}{/}{\@backslashchar}[\minted@infile@windows]
              \StrSubstitute{\minted@actualinfile}{/}{\@backslashchar}[\minted@actualinfile@windows]
              \ShellEscape{move /y
              \minted@outputdir\minted@infile@windows\space\minted@outputdir\minted@actualinfile@windows}%
            \else
              \ShellEscape{mv -f
              \minted@outputdir\minted@infile\space\minted@outputdir\minted@actualinfile}%
            \fi
            \let\minted@infile\minted@actualinfile
            \expandafter\minted@addcachefile\expandafter{\minted@cachefilename}%
         }%
         {\ifthenelse{\boolean{minted@frozencache}}%
           {%
              \edef\minted@cachefilename{listing\arabic{minted@pygmentizecounter}.pygtex}%
              \edef\minted@infile{\minted@cachedir/\minted@cachefilename}%
              \expandafter\minted@addcachefile\expandafter{\minted@cachefilename}}%
           {\expandafter\minted@addcachefile\expandafter{\minted@hash.pygtex}}%
         }%
        \minted@inputpyg}%
      {%
        \ifthenelse{\equal{\minted@get@opt{autogobble}{false}}{true}}{%
          \minted@autogobble{#1}}{}%
        \ShellEscape{\minted@cmd}%
        \ShellEscape{\minted@cmdHTML}%
        %\ShellEscape{\minted@cmdPNG}%
        %\ShellEscape{\minted@cmdSVG}%
        \minted@inputpyg}%
  }%
}{}%
}%
\html@action@newcommand{inputminted}[3][]{%
  \message{moodle.sty: Processing \string\inputminted[#1]{#2}{#3} for HTML ^^J}%
  % arguments #2 and #3 are thrown away: the job is done previously by minted when
  % calling pygmentize. The file |\minted@infileHTML| generated with our hack will be used.
  % Since minted is based upon `fvextra' the macro |\moodle@VerbatimInput| works here.
  \minted@configlang{#2}% grab options set for this specific language
  \setkeys{minted@opt@cmd}{#1}% grab options in #1
  \minted@fvset% import options
  \stepcounter{moodle@pygmentizecounter}
  \xa\moodle@VerbatimInput\xa{\csname minted@infileHTML\the\c@moodle@pygmentizecounter\endcsname}%
}%
%    \end{macrocode}
%
% \subsection{Internationalization}
% Here is an attempt to internationalize the PDF typesetting, relying on
% the package 'translations'.
%    \begin{macrocode}
\AtEndPreamble{%
  \@ifpackageloaded{translator}{\moodle@internationaltrue}{}%
  \@ifpackageloaded{translations}{\moodle@internationaltrue}{}%
  % polyglossia "fakes" babel
  \@ifpackageloaded{polyglossia}{%
    \moodle@internationaltrue
    % The following is commented because \xpg@bloaded is set very late and must be expanded
    %\PassOptionsToPackage{\xpg@bloaded}{translator}
  }{
    \@ifpackageloaded{babel}{%
      \moodle@internationaltrue
      % The following is commented because some languages of babel,
      % like lithuanian, are unknown to translator. Instead we
      % copied the aliases below.
      %\PassOptionsToPackage{\bbl@loaded}{translator}%
    }{}%
  }%
  \ifmoodle@international
    % By default, we load and rely on "translator".
    % The package is simple and has limited dependencies.
    % Since we borrow the syntax of "translations",
    % the switch is easy: just "\usepackage{translations}"
    % in the preamble.
    %\RequirePackage{translations}%
    \@ifpackageloaded{translations}{}{%
      \RequirePackage{translator}%
      % Borrow the syntax from 'translations'
      \newcommand\DeclareTranslation[3]{\deftranslation[to=#1]{#2}{#3}}%
      \newcommand\DeclareTranslationFallback[2]{\deftranslation[to=fallback]{#1}{#2}}%
      \let\GetTranslation=\translate
      % Set a fall-back if a translation is unknown (usually English, see below)
      \languagepath{\languagename,fallback}%
      % Set aliases (most of them taken directly from translator.sty)
      \languagealias{afrikaans} {Afrikaans,Dutch}%
      \languagealias{american}  {AmericanEnglish,English}%
      \languagealias{austrian}  {Austrian1997,Austrian,German1997,German}%
      \languagealias{brazil}    {Brazilian,Portuguese}%
      \languagealias{brazilian} {Brazilian,Portuguese}%
      \languagealias{british}   {BritishEnglish,English}%
      \languagealias{catalan}   {Catalan}% unknown to translator
      \languagealias{canadian}  {Canadian,English}%
      \languagealias{canadien}  {Canadien,French}%
      \languagealias{croatian}  {Croatian}%
      \languagealias{czech}     {Czech}%
      \languagealias{danish}    {Danish}%
      \languagealias{dutch}     {Dutch}%
      \languagealias{english}   {English}%
      \languagealias{estonian}  {Estonian}%
      \languagealias{finnish}   {Finnish}%
      \languagealias{french}    {French}%
      \languagealias{german}    {German1997,German}%
      \languagealias{hungarian} {Hungarian}%
      \languagealias{icelandic} {Icelandic}%
      \languagealias{italian}   {Italian}%
      \languagealias{lithuanian}{Lithuanian}% unknown to translator
      \languagealias{magyar}    {Magyar,Hungarian}% added Hungarian
      \languagealias{naustrian} {Austrian,German}%
      \languagealias{ngerman}   {German}%
      \languagealias{norsk}     {Norsk}%
      \languagealias{norwegian} {Norsk}% for polyglossia
      \languagealias{nynorsk}   {Nynorsk,Norsk}%
      \languagealias{polish}    {Polish}%
      \languagealias{portuges}  {Portuguese}%
      \languagealias{portuguese}{Portuguese}%
      \languagealias{romanian}  {Romanian}%
      \languagealias{slovak}    {Slovak,Czech}% added Czech
      \languagealias{spanish}   {Spanish}%
      \languagealias{swedish}   {Swedish}%
      \languagealias{turkish}   {Turkish}%
      \languagealias{UKenglish} {BritishEnglish,English}%
      \languagealias{USenglish} {AmericanEnglish,English}%
    }%
    \DeclareTranslation{Catalan}{True}{Vertader}%
    \DeclareTranslation{Catalan}{False}{Fals}%
    %\DeclareTranslation{Catalan}{cloze}{}%
    %\DeclareTranslation{Catalan}{description}{}%
    %\DeclareTranslation{Catalan}{essay}{}%
    %\DeclareTranslation{Catalan}{matching}{}%
    %\DeclareTranslation{Catalan}{multi}{}%
    %\DeclareTranslation{Catalan}{numerical}{}%
    %\DeclareTranslation{Catalan}{shortanswer}{}%
    %\DeclareTranslation{Catalan}{truefalse}{}%
    %\DeclareTranslation{Catalan}{Shuffle}{}%
    %\DeclareTranslation{Catalan}{Single}{}%
    %\DeclareTranslation{Catalan}{marked out of}{}%
    %\DeclareTranslation{Catalan}{penalty}{}%
    %\DeclareTranslation{Catalan}{tags}{}%
    %\DeclareTranslation{Catalan}{All-or-nothing}{}%
    %\DeclareTranslation{Catalan}{Case-Sensitive}{}%
    %\DeclareTranslation{Catalan}{Case-Insensitive}{}%
    %\DeclareTranslation{Catalan}{Drag and drop}{}%
    %\DeclareTranslation{Catalan}{Information for graders}{}%
    %\DeclareTranslation{Catalan}{Response template}{}%
    %\DeclareTranslation{Catalan}{editor}{}%
    %\DeclareTranslation{Catalan}{editorfilepicker}{}%
    %\DeclareTranslation{Catalan}{plain}{}%
    %\DeclareTranslation{Catalan}{monospaced}{}%
    %\DeclareTranslation{Catalan}{noinline}{}%
    %\DeclareTranslation{Catalan}{Total of marks}{}%
    \DeclareTranslation{Croatian}{True}{To\v{c}no}%
    \DeclareTranslation{Croatian}{False}{Neto\v{c}no}%
    %\DeclareTranslation{Croatian}{cloze}{}%
    %\DeclareTranslation{Croatian}{description}{}%
    %\DeclareTranslation{Croatian}{essay}{}%
    %\DeclareTranslation{Croatian}{matching}{}%
    %\DeclareTranslation{Croatian}{multi}{}%
    %\DeclareTranslation{Croatian}{numerical}{}%
    %\DeclareTranslation{Croatian}{shortanswer}{}%
    %\DeclareTranslation{Croatian}{truefalse}{}%
    %\DeclareTranslation{Croatian}{Shuffle}{}%
    %\DeclareTranslation{Croatian}{Single}{}%
    %\DeclareTranslation{Croatian}{Multiple}{}%
    %\DeclareTranslation{Croatian}{marked out of}{}%
    %\DeclareTranslation{Croatian}{penalty}{}%
    %\DeclareTranslation{Croatian}{tags}{}%
    %\DeclareTranslation{Croatian}{All-or-nothing}{}%
    %\DeclareTranslation{Croatian}{Case-Sensitive}{}%
    %\DeclareTranslation{Croatian}{Case-Insensitive}{}%
    %\DeclareTranslation{Croatian}{Drag and drop}{}%
    %\DeclareTranslation{Croatian}{Information for graders}{}%
    %\DeclareTranslation{Croatian}{Response template}{}%
    %\DeclareTranslation{Croatian}{editor}{}%
    %\DeclareTranslation{Croatian}{editorfilepicker}{}%
    %\DeclareTranslation{Croatian}{plain}{}%
    %\DeclareTranslation{Croatian}{monospaced}{}%
    %\DeclareTranslation{Croatian}{noinline}{}%
    %\DeclareTranslation{Croatian}{Total of marks}{}%
    \DeclareTranslation{Czech}{True}{Pravda}%
    \DeclareTranslation{Czech}{False}{Nepravda}%
    %\DeclareTranslation{Czech}{cloze}{}%
    %\DeclareTranslation{Czech}{description}{}%
    %\DeclareTranslation{Czech}{essay}{}%
    %\DeclareTranslation{Czech}{matching}{}%
    %\DeclareTranslation{Czech}{multi}{}%
    %\DeclareTranslation{Czech}{numerical}{}%
    %\DeclareTranslation{Czech}{shortanswer}{}%
    %\DeclareTranslation{Czech}{truefalse}{}%
    %\DeclareTranslation{Czech}{Shuffle}{}%
    %\DeclareTranslation{Czech}{Single}{}%
    %\DeclareTranslation{Czech}{Multiple}{}%
    %\DeclareTranslation{Czech}{marked out of}{}%
    %\DeclareTranslation{Czech}{penalty}{}%
    %\DeclareTranslation{Czech}{tags}{}%
    %\DeclareTranslation{Czech}{All-or-nothing}{}%
    %\DeclareTranslation{Czech}{Case-Sensitive}{}%
    %\DeclareTranslation{Czech}{Case-Insensitive}{}%
    %\DeclareTranslation{Czech}{Drag and drop}{}%
    %\DeclareTranslation{Czech}{Information for graders}{}%
    %\DeclareTranslation{Czech}{Response template}{}%
    %\DeclareTranslation{Czech}{editor}{}%
    %\DeclareTranslation{Czech}{editorfilepicker}{}%
    %\DeclareTranslation{Czech}{plain}{}%
    %\DeclareTranslation{Czech}{monospaced}{}%
    %\DeclareTranslation{Czech}{noinline}{}%
    %\DeclareTranslation{Czech}{Total of marks}{}%
    \DeclareTranslation{Danish}{True}{Sandt}%
    \DeclareTranslation{Danish}{False}{Falsk}%
    %\DeclareTranslation{Danish}{cloze}{}%
    %\DeclareTranslation{Danish}{description}{}%
    %\DeclareTranslation{Danish}{essay}{}%
    %\DeclareTranslation{Danish}{matching}{}%
    %\DeclareTranslation{Danish}{multi}{}%
    %\DeclareTranslation{Danish}{numerical}{}%
    %\DeclareTranslation{Danish}{shortanswer}{}%
    %\DeclareTranslation{Danish}{truefalse}{}%
    %\DeclareTranslation{Danish}{Shuffle}{}%
    %\DeclareTranslation{Danish}{Single}{}%
    %\DeclareTranslation{Danish}{Multiple}{}%
    %\DeclareTranslation{Danish}{marked out of}{}%
    %\DeclareTranslation{Danish}{penalty}{}%
    %\DeclareTranslation{Danish}{tags}{}%
    %\DeclareTranslation{Danish}{All-or-nothing}{}%
    %\DeclareTranslation{Danish}{Case-Sensitive}{}%
    %\DeclareTranslation{Danish}{Case-Insensitive}{}%
    %\DeclareTranslation{Danish}{Drag and drop}{}%
    %\DeclareTranslation{Danish}{Information for graders}{}%
    %\DeclareTranslation{Danish}{Response template}{}%
    %\DeclareTranslation{Danish}{editor}{}%
    %\DeclareTranslation{Danish}{editorfilepicker}{}%
    %\DeclareTranslation{Danish}{plain}{}%
    %\DeclareTranslation{Danish}{monospaced}{}%
    %\DeclareTranslation{Danish}{noinline}{}%
    %\DeclareTranslation{Danish}{Total of marks}{}%
    \DeclareTranslation{Dutch}{True}{Waar}%
    \DeclareTranslation{Dutch}{False}{Niet waar}%
    %\DeclareTranslation{Dutch}{cloze}{}%
    %\DeclareTranslation{Dutch}{description}{}%
    %\DeclareTranslation{Dutch}{essay}{}%
    %\DeclareTranslation{Dutch}{matching}{}%
    %\DeclareTranslation{Dutch}{multi}{}%
    %\DeclareTranslation{Dutch}{numerical}{}%
    %\DeclareTranslation{Dutch}{shortanswer}{}%
    %\DeclareTranslation{Dutch}{truefalse}{}%
    %\DeclareTranslation{Dutch}{Shuffle}{}%
    %\DeclareTranslation{Dutch}{Single}{}%
    %\DeclareTranslation{Dutch}{Multiple}{}%
    %\DeclareTranslation{Dutch}{marked out of}{}%
    %\DeclareTranslation{Dutch}{penalty}{}%
    %\DeclareTranslation{Dutch}{tags}{}%
    %\DeclareTranslation{Dutch}{All-or-nothing}{}%
    %\DeclareTranslation{Dutch}{Case-Sensitive}{}%
    %\DeclareTranslation{Dutch}{Case-Insensitive}{}%
    %\DeclareTranslation{Dutch}{Drag and drop}{}%
    %\DeclareTranslation{Dutch}{Information for graders}{}%
    %\DeclareTranslation{Dutch}{Response template}{}%
    %\DeclareTranslation{Dutch}{editor}{}%
    %\DeclareTranslation{Dutch}{editorfilepicker}{}%
    %\DeclareTranslation{Dutch}{plain}{}%
    %\DeclareTranslation{Dutch}{monospaced}{}%
    %\DeclareTranslation{Dutch}{noinline}{}%
    %\DeclareTranslation{Dutch}{Total of marks}{}%
    \DeclareTranslation{English}{True}{True}%
    \DeclareTranslation{English}{False}{False}%
    \DeclareTranslation{English}{cloze}{Embedded answers}%
    \DeclareTranslation{English}{description}{Description}%
    \DeclareTranslation{English}{essay}{Essay}%
    \DeclareTranslation{English}{matching}{Matching}%
    \DeclareTranslation{English}{multi}{Multiple choice}%
    \DeclareTranslation{English}{numerical}{Numerical}%
    \DeclareTranslation{English}{shortanswer}{Short answer}%
    \DeclareTranslation{English}{truefalse}{True/False}%
    \DeclareTranslation{English}{Shuffle}{Shuffle}%
    \DeclareTranslation{English}{Single}{One answer only}%
    \DeclareTranslation{English}{Multiple}{Multiple answers allowed}%
    \DeclareTranslation{English}{marked out of}{marked out of}%
    \DeclareTranslation{English}{penalty}{penalty}%
    \DeclareTranslation{English}{tags}{tags}%
    \DeclareTranslation{English}{All-or-nothing}{All-or-nothing}%
    \DeclareTranslation{English}{Case-Sensitive}{Case-Sensitive}%
    \DeclareTranslation{English}{Case-Insensitive}{Case-Insensitive}%
    \DeclareTranslation{English}{Drag and drop}{Drag and drop}%
    \DeclareTranslation{English}{Information for graders}{Information for graders}%
    \DeclareTranslation{English}{Response template}{Response template}%
    \DeclareTranslation{English}{editor}{HTML editor}%
    \DeclareTranslation{English}{editorfilepicker}{HTML editor + file picker}%
    \DeclareTranslation{English}{plain}{Plain text}%
    \DeclareTranslation{English}{monospaced}{Plain text, monospaced font}%
    \DeclareTranslation{English}{noinline}{File picker}%
    \DeclareTranslation{English}{Total of marks}{Total of marks}%
    \DeclareTranslation{Estonian}{True}{T\~oene}%
    \DeclareTranslation{Estonian}{False}{V\"a\"ar}%
    %\DeclareTranslation{Estonian}{cloze}{}%
    %\DeclareTranslation{Estonian}{description}{}%
    %\DeclareTranslation{Estonian}{essay}{}%
    %\DeclareTranslation{Estonian}{matching}{}%
    %\DeclareTranslation{Estonian}{multi}{}%
    %\DeclareTranslation{Estonian}{numerical}{}%
    %\DeclareTranslation{Estonian}{shortanswer}{}%
    %\DeclareTranslation{Estonian}{truefalse}{}%
    %\DeclareTranslation{Estonian}{Shuffle}{}%
    %\DeclareTranslation{Estonian}{Single}{}%
    %\DeclareTranslation{Estonian}{Multiple}{}%
    %\DeclareTranslation{Estonian}{marked out of}{}%
    %\DeclareTranslation{Estonian}{penalty}{}%
    %\DeclareTranslation{Estonian}{tags}{}%
    %\DeclareTranslation{Estonian}{All-or-nothing}{}%
    %\DeclareTranslation{Estonian}{Case-Sensitive}{}%
    %\DeclareTranslation{Estonian}{Case-Insensitive}{}%
    %\DeclareTranslation{Estonian}{Drag and drop}{}%
    %\DeclareTranslation{Estonian}{Information for graders}{}%
    %\DeclareTranslation{Estonian}{Response template}{}%
    %\DeclareTranslation{Estonian}{editor}{}%
    %\DeclareTranslation{Estonian}{editorfilepicker}{}%
    %\DeclareTranslation{Estonian}{plain}{}%
    %\DeclareTranslation{Estonian}{monospaced}{}%
    %\DeclareTranslation{Estonian}{noinline}{}%
    %\DeclareTranslation{Estonian}{Total of marks}{}%
    \DeclareTranslation{Finnish}{True}{Tosi}%
    \DeclareTranslation{Finnish}{False}{Ep\"atosi}%
    %\DeclareTranslation{Finnish}{cloze}{}%
    %\DeclareTranslation{Finnish}{description}{}%
    %\DeclareTranslation{Finnish}{essay}{}%
    %\DeclareTranslation{Finnish}{matching}{}%
    %\DeclareTranslation{Finnish}{multi}{}%
    %\DeclareTranslation{Finnish}{numerical}{}%
    %\DeclareTranslation{Finnish}{shortanswer}{}%
    %\DeclareTranslation{Finnish}{truefalse}{}%
    %\DeclareTranslation{Finnish}{Shuffle}{}%
    %\DeclareTranslation{Finnish}{Single}{}%
    %\DeclareTranslation{Finnish}{Multiple}{}%
    %\DeclareTranslation{Finnish}{marked out of}{}%
    %\DeclareTranslation{Finnish}{penalty}{}%
    %\DeclareTranslation{Finnish}{tags}{}%
    %\DeclareTranslation{Finnish}{All-or-nothing}{}%
    %\DeclareTranslation{Finnish}{Case-Sensitive}{}%
    %\DeclareTranslation{Finnish}{Case-Insensitive}{}%
    %\DeclareTranslation{Finnish}{Drag and drop}{}%
    %\DeclareTranslation{Finnish}{Information for graders}{}%
    %\DeclareTranslation{Finnish}{Response template}{}%
    %\DeclareTranslation{Finnish}{editor}{}%
    %\DeclareTranslation{Finnish}{editorfilepicker}{}%
    %\DeclareTranslation{Finnish}{plain}{}%
    %\DeclareTranslation{Finnish}{monospaced}{}%
    %\DeclareTranslation{Finnish}{noinline}{}%
    %\DeclareTranslation{Finnish}{Total of marks}{}%
    \DeclareTranslation{French}{True}{Vrai}%
    \DeclareTranslation{French}{False}{Faux}%
    \DeclareTranslation{French}{cloze}{R\'eponses int\'egr\'ees}%
    \DeclareTranslation{French}{description}{Description}%
    \DeclareTranslation{French}{essay}{Composition}%
    \DeclareTranslation{French}{matching}{Appariement}%
    \DeclareTranslation{French}{multi}{QCM}%
    \DeclareTranslation{French}{numerical}{Num\'erique}%
    \DeclareTranslation{French}{shortanswer}{R\'eponse courte}%
    \DeclareTranslation{French}{truefalse}{Vrai/Faux}%
    \DeclareTranslation{French}{Shuffle}{M\'elanger}%
    \DeclareTranslation{French}{Single}{R\'eponse unique}%
    \DeclareTranslation{French}{Multiple}{Plusieurs r\'eponses possibles}%
    \DeclareTranslation{French}{marked out of}{not\'e sur}%
    \DeclareTranslation{French}{penalty}{p\'enalit\'e}%
    \DeclareTranslation{French}{tags}{\'etiquettes}%
    \DeclareTranslation{French}{All-or-nothing}{Tout ou rien}%
    \DeclareTranslation{French}{Case-Sensitive}{Sensible \`a la casse}%
    \DeclareTranslation{French}{Case-Insensitive}{Insensible \`a la casse}%
    \DeclareTranslation{French}{Drag and drop}{Glisser-d\'eposer}%
    \DeclareTranslation{French}{Information for graders}{Information pour les évaluateurs}%
    \DeclareTranslation{French}{Response template}{Mod\`ele de r\'eponse}%
    \DeclareTranslation{French}{editor}{\'Editeur HTML}%
    \DeclareTranslation{French}{editorfilepicker}{\'Editeur \HTML\ + s\'electeur de fichier}%
    \DeclareTranslation{French}{plain}{\'Editeur texte}%
    \DeclareTranslation{French}{monospaced}{\'Editeur texte, police \`a chasse fixe}%
    \DeclareTranslation{French}{noinline}{S\'electeur de fichier}%
    \DeclareTranslation{French}{Total of marks}{Total des points}%
    \DeclareTranslation{German}{True}{Wahr}%
    \DeclareTranslation{German}{False}{Falsch}%
    \DeclareTranslation{German}{cloze}{L\"uckentext}%
    \DeclareTranslation{German}{description}{Beschreibung}%
    \DeclareTranslation{German}{essay}{Freitext}%
    \DeclareTranslation{German}{matching}{Zuordnung}%
    \DeclareTranslation{German}{multi}{Multiple-Choice}%
    \DeclareTranslation{German}{numerical}{Numerisch}%
    \DeclareTranslation{German}{shortanswer}{Kurzantwort}%
    \DeclareTranslation{German}{truefalse}{Wahr/Falsch}%
    \DeclareTranslation{German}{Shuffle}{Mischen}%
    \DeclareTranslation{German}{Single}{Nur eine Antwort erlaubt}%
    \DeclareTranslation{German}{Multiple}{Mehrere Antworten erlaubt}%
    \DeclareTranslation{German}{marked out of}{Punkte:}%
    \DeclareTranslation{German}{penalty}{Abzug}%
    \DeclareTranslation{German}{tags}{Tags}%
    \DeclareTranslation{German}{All-or-nothing}{Alles-oder-nichts}%
    \DeclareTranslation{German}{Case-Sensitive}{Gro\ss-/Kleinschreibung muss stimmen}%
    \DeclareTranslation{German}{Case-Insensitive}{Gro\ss-/Kleinschreibung unwichtig}%
    \DeclareTranslation{German}{Drag and drop}{Drag-and-Drop}%
    \DeclareTranslation{German}{Information for graders}{Information zur Bewertung}%
    \DeclareTranslation{German}{Response template}{Antwortvorlage}%
    \DeclareTranslation{German}{editor}{HTML Editor}%
    \DeclareTranslation{German}{editorfilepicker}{HTML Editor mit Dateiauswahl}%
    \DeclareTranslation{German}{plain}{Unformatierter Text}%
    \DeclareTranslation{German}{monospaced}{Unformatierter Text, Schriftart mit fester Laufweite}%
    \DeclareTranslation{German}{noinline}{Kein Textfeld}%
    \DeclareTranslation{German}{Total of marks}{Gesamtsumme der Punkte}%
    \DeclareTranslation{Hungarian}{True}{Igaz}%
    \DeclareTranslation{Hungarian}{False}{Hamis}%
    %\DeclareTranslation{Hungarian}{cloze}{}%
    %\DeclareTranslation{Hungarian}{description}{}%
    %\DeclareTranslation{Hungarian}{essay}{}%
    %\DeclareTranslation{Hungarian}{matching}{}%
    %\DeclareTranslation{Hungarian}{multi}{}%
    %\DeclareTranslation{Hungarian}{numerical}{}%
    %\DeclareTranslation{Hungarian}{shortanswer}{}%
    %\DeclareTranslation{Hungarian}{truefalse}{}%
    %\DeclareTranslation{Hungarian}{Shuffle}{}%
    %\DeclareTranslation{Hungarian}{Single}{}%
    %\DeclareTranslation{Hungarian}{Multiple}{}%
    %\DeclareTranslation{Hungarian}{marked out of}{}%
    %\DeclareTranslation{Hungarian}{penalty}{}%
    %\DeclareTranslation{Hungarian}{tags}{}%
    %\DeclareTranslation{Hungarian}{All-or-nothing}{}%
    %\DeclareTranslation{Hungarian}{Case-Sensitive}{}%
    %\DeclareTranslation{Hungarian}{Case-Insensitive}{}%
    %\DeclareTranslation{Hungarian}{Drag and drop}{}%
    %\DeclareTranslation{Hungarian}{Information for graders}{}%
    %\DeclareTranslation{Hungarian}{Response template}{}%
    %\DeclareTranslation{Hungarian}{editor}{}%
    %\DeclareTranslation{Hungarian}{editorfilepicker}{}%
    %\DeclareTranslation{Hungarian}{plain}{}%
    %\DeclareTranslation{Hungarian}{monospaced}{}%
    %\DeclareTranslation{Hungarian}{noinline}{}%
    %\DeclareTranslation{Hungarian}{Total of marks}{}%
    \DeclareTranslation{Icelandic}{True}{R\'ett}%
    \DeclareTranslation{Icelandic}{False}{Rangt}%
    %\DeclareTranslation{Icelandic}{cloze}{}%
    %\DeclareTranslation{Icelandic}{description}{}%
    %\DeclareTranslation{Icelandic}{essay}{}%
    %\DeclareTranslation{Icelandic}{matching}{}%
    %\DeclareTranslation{Icelandic}{multi}{}%
    %\DeclareTranslation{Icelandic}{numerical}{}%
    %\DeclareTranslation{Icelandic}{shortanswer}{}%
    %\DeclareTranslation{Icelandic}{truefalse}{}%
    %\DeclareTranslation{Icelandic}{Shuffle}{}%
    %\DeclareTranslation{Icelandic}{Single}{}%
    %\DeclareTranslation{Icelandic}{Multiple}{}%
    %\DeclareTranslation{Icelandic}{marked out of}{}%
    %\DeclareTranslation{Icelandic}{penalty}{}%
    %\DeclareTranslation{Icelandic}{tags}{}%
    %\DeclareTranslation{Icelandic}{All-or-nothing}{}%
    %\DeclareTranslation{Icelandic}{Case-Sensitive}{}%
    %\DeclareTranslation{Icelandic}{Case-Insensitive}{}%
    %\DeclareTranslation{Icelandic}{Drag and drop}{}%
    %\DeclareTranslation{Icelandic}{Information for graders}{}%
    %\DeclareTranslation{Icelandic}{Response template}{}%
    %\DeclareTranslation{Icelandic}{editor}{}%
    %\DeclareTranslation{Icelandic}{editorfilepicker}{}%
    %\DeclareTranslation{Icelandic}{plain}{}%
    %\DeclareTranslation{Icelandic}{monospaced}{}%
    %\DeclareTranslation{Icelandic}{noinline}{}%
    %\DeclareTranslation{Icelandic}{Total of marks}{}%
    \DeclareTranslation{Italian}{True}{Vero}%
    \DeclareTranslation{Italian}{False}{Falso}%
    \DeclareTranslation{Italian}{cloze}{Risposte inglobate}%
    \DeclareTranslation{Italian}{description}{Descrizione}%
    \DeclareTranslation{Italian}{essay}{Tema libero}%
    \DeclareTranslation{Italian}{matching}{Corrispondenza}%
    \DeclareTranslation{Italian}{multi}{Risposta multipla}%
    \DeclareTranslation{Italian}{numerical}{Numerica}%
    \DeclareTranslation{Italian}{shortanswer}{Risposta breve}%
    \DeclareTranslation{Italian}{truefalse}{Vero/Falso}%
    \DeclareTranslation{Italian}{Shuffle}{Ordine casuale}%
    \DeclareTranslation{Italian}{Single}{Una sola alternativa}%
    \DeclareTranslation{Italian}{Multiple}{Pi\`u alternative}%
    \DeclareTranslation{Italian}{marked out of}{punteggio max.}%
    \DeclareTranslation{Italian}{penalty}{penalit\`a}%
    \DeclareTranslation{Italian}{tags}{etichette}%
    \DeclareTranslation{Italian}{All-or-nothing}{Tutto o niente}%
    \DeclareTranslation{Italian}{Case-Sensitive}{Sensibile alle maiuscole}%
    \DeclareTranslation{Italian}{Case-Insensitive}{Maiuscolo/minuscolo indifferente}%
    \DeclareTranslation{Italian}{Drag and drop}{Drag and drop}%
    \DeclareTranslation{Italian}{Information for graders}{Informazioni per i valutatori}%
    \DeclareTranslation{Italian}{Response template}{Modello di risposta}%
    \DeclareTranslation{Italian}{editor}{Editor HTML}%
    \DeclareTranslation{Italian}{editorfilepicker}{Editor \HTML\ con file picker}%
    \DeclareTranslation{Italian}{plain}{Testo semplice}%
    \DeclareTranslation{Italian}{monospaced}{Testo semplice, caratteri a spaziatura fissa}%
    \DeclareTranslation{Italian}{noinline}{Solo file picker}%
    \DeclareTranslation{Italian}{Total of marks}{Punteggio complessivo}%
    \DeclareTranslation{Lithuanian}{True}{Tiesa}%
    \DeclareTranslation{Lithuanian}{False}{Netiesa}%
    %\DeclareTranslation{Lithuanian}{cloze}{}%
    %\DeclareTranslation{Lithuanian}{description}{}%
    %\DeclareTranslation{Lithuanian}{essay}{}%
    %\DeclareTranslation{Lithuanian}{matching}{}%
    %\DeclareTranslation{Lithuanian}{multi}{}%
    %\DeclareTranslation{Lithuanian}{numerical}{}%
    %\DeclareTranslation{Lithuanian}{shortanswer}{}%
    %\DeclareTranslation{Lithuanian}{truefalse}{}%
    %\DeclareTranslation{Lithuanian}{Shuffle}{}%
    %\DeclareTranslation{Lithuanian}{Single}{}%
    %\DeclareTranslation{Lithuanian}{Multiple}{}%
    %\DeclareTranslation{Lithuanian}{marked out of}{}%
    %\DeclareTranslation{Lithuanian}{penalty}{}%
    %\DeclareTranslation{Lithuanian}{tags}{}%
    %\DeclareTranslation{Lithuanian}{All-or-nothing}{}%
    %\DeclareTranslation{Lithuanian}{Case-Sensitive}{}%
    %\DeclareTranslation{Lithuanian}{Case-Insensitive}{}%
    %\DeclareTranslation{Lithuanian}{Drag and drop}{}%
    %\DeclareTranslation{Lithuanian}{Information for graders}{}%
    %\DeclareTranslation{Lithuanian}{Response template}{}%
    %\DeclareTranslation{Lithuanian}{editor}{}%
    %\DeclareTranslation{Lithuanian}{editorfilepicker}{}%
    %\DeclareTranslation{Lithuanian}{plain}{}%
    %\DeclareTranslation{Lithuanian}{monospaced}{}%
    %\DeclareTranslation{Lithuanian}{noinline}{}%
    %\DeclareTranslation{Lithuanian}{Total of marks}{}%
    \DeclareTranslation{Norsk}{True}{Riktig}%
    \DeclareTranslation{Norsk}{False}{Feil}%
    %\DeclareTranslation{Norsk}{cloze}{}%
    %\DeclareTranslation{Norsk}{description}{}%
    %\DeclareTranslation{Norsk}{essay}{}%
    %\DeclareTranslation{Norsk}{matching}{}%
    %\DeclareTranslation{Norsk}{multi}{}%
    %\DeclareTranslation{Norsk}{numerical}{}%
    %\DeclareTranslation{Norsk}{shortanswer}{}%
    %\DeclareTranslation{Norsk}{truefalse}{}%
    %\DeclareTranslation{Norsk}{Shuffle}{}%
    %\DeclareTranslation{Norsk}{Single}{}%
    %\DeclareTranslation{Norsk}{Multiple}{}%
    %\DeclareTranslation{Norsk}{marked out of}{}%
    %\DeclareTranslation{Norsk}{penalty}{}%
    %\DeclareTranslation{Norsk}{tags}{}%
    %\DeclareTranslation{Norsk}{All-or-nothing}{}%
    %\DeclareTranslation{Norsk}{Case-Sensitive}{}%
    %\DeclareTranslation{Norsk}{Case-Insensitive}{}%
    %\DeclareTranslation{Norsk}{Drag and drop}{}%
    %\DeclareTranslation{Norsk}{Information for graders}{}%
    %\DeclareTranslation{Norsk}{Response template}{}%
    %\DeclareTranslation{Norsk}{editor}{}%
    %\DeclareTranslation{Norsk}{editorfilepicker}{}%
    %\DeclareTranslation{Norsk}{plain}{}%
    %\DeclareTranslation{Norsk}{monospaced}{}%
    %\DeclareTranslation{Norsk}{noinline}{}%
    %\DeclareTranslation{Norsk}{Total of marks}{}%
    \DeclareTranslation{Polish}{True}{Prawda}%
    \DeclareTranslation{Polish}{False}{Fa\l sz}%
    %\DeclareTranslation{Polish}{cloze}{}%
    %\DeclareTranslation{Polish}{description}{}%
    %\DeclareTranslation{Polish}{essay}{}%
    %\DeclareTranslation{Polish}{matching}{}%
    %\DeclareTranslation{Polish}{multi}{}%
    %\DeclareTranslation{Polish}{numerical}{}%
    %\DeclareTranslation{Polish}{shortanswer}{}%
    %\DeclareTranslation{Polish}{truefalse}{}%
    %\DeclareTranslation{Polish}{Shuffle}{}%
    %\DeclareTranslation{Polish}{Single}{}%
    %\DeclareTranslation{Polish}{Multiple}{}%
    %\DeclareTranslation{Polish}{marked out of}{}%
    %\DeclareTranslation{Polish}{penalty}{}%
    %\DeclareTranslation{Polish}{tags}{}%
    %\DeclareTranslation{Polish}{All-or-nothing}{}%
    %\DeclareTranslation{Polish}{Case-Sensitive}{}%
    %\DeclareTranslation{Polish}{Case-Insensitive}{}%
    %\DeclareTranslation{Polish}{Drag and drop}{}%
    %\DeclareTranslation{Polish}{Information for graders}{}%
    %\DeclareTranslation{Polish}{Response template}{}%
    %\DeclareTranslation{Polish}{editor}{}%
    %\DeclareTranslation{Polish}{editorfilepicker}{}%
    %\DeclareTranslation{Polish}{plain}{}%
    %\DeclareTranslation{Polish}{monospaced}{}%
    %\DeclareTranslation{Polish}{noinline}{}%
    %\DeclareTranslation{Polish}{Total of marks}{}%
    \DeclareTranslation{Portuguese}{True}{Verdadeiro}%
    \DeclareTranslation{Portuguese}{False}{Falso}%
    %\DeclareTranslation{Portuguese}{cloze}{}%
    %\DeclareTranslation{Portuguese}{description}{}%
    %\DeclareTranslation{Portuguese}{essay}{}%
    %\DeclareTranslation{Portuguese}{matching}{}%
    %\DeclareTranslation{Portuguese}{multi}{}%
    %\DeclareTranslation{Portuguese}{numerical}{}%
    %\DeclareTranslation{Portuguese}{shortanswer}{}%
    %\DeclareTranslation{Portuguese}{truefalse}{}%
    %\DeclareTranslation{Portuguese}{Shuffle}{}%
    %\DeclareTranslation{Portuguese}{Single}{}%
    %\DeclareTranslation{Portuguese}{Multiple}{}%
    %\DeclareTranslation{Portuguese}{marked out of}{}%
    %\DeclareTranslation{Portuguese}{penalty}{}%
    %\DeclareTranslation{Portuguese}{tags}{}%
    %\DeclareTranslation{Portuguese}{All-or-nothing}{}%
    %\DeclareTranslation{Portuguese}{Case-Sensitive}{}%
    %\DeclareTranslation{Portuguese}{Case-Insensitive}{}%
    %\DeclareTranslation{Portuguese}{Drag and drop}{}%
    %\DeclareTranslation{Portuguese}{Information for graders}{}%
    %\DeclareTranslation{Portuguese}{Response template}{}%
    %\DeclareTranslation{Portuguese}{editor}{}%
    %\DeclareTranslation{Portuguese}{editorfilepicker}{}%
    %\DeclareTranslation{Portuguese}{plain}{}%
    %\DeclareTranslation{Portuguese}{monospaced}{}%
    %\DeclareTranslation{Portuguese}{noinline}{}%
    %\DeclareTranslation{Portuguese}{Total of marks}{}%
    \DeclareTranslation{Romanian}{True}{Adev\u{a}rat}%
    \DeclareTranslation{Romanian}{False}{Fals}%
    %\DeclareTranslation{Romanian}{cloze}{}%
    %\DeclareTranslation{Romanian}{description}{}%
    %\DeclareTranslation{Romanian}{essay}{}%
    %\DeclareTranslation{Romanian}{matching}{}%
    %\DeclareTranslation{Romanian}{multi}{}%
    %\DeclareTranslation{Romanian}{numerical}{}%
    %\DeclareTranslation{Romanian}{shortanswer}{}%
    %\DeclareTranslation{Romanian}{truefalse}{}%
    %\DeclareTranslation{Romanian}{Shuffle}{}%
    %\DeclareTranslation{Romanian}{Single}{}%
    %\DeclareTranslation{Romanian}{Multiple}{}%
    %\DeclareTranslation{Romanian}{marked out of}{}%
    %\DeclareTranslation{Romanian}{penalty}{}%
    %\DeclareTranslation{Romanian}{tags}{}%
    %\DeclareTranslation{Romanian}{All-or-nothing}{}%
    %\DeclareTranslation{Romanian}{Case-Sensitive}{}%
    %\DeclareTranslation{Romanian}{Case-Insensitive}{}%
    %\DeclareTranslation{Romanian}{Drag and drop}{}%
    %\DeclareTranslation{Romanian}{Information for graders}{}%
    %\DeclareTranslation{Romanian}{Response template}{}%
    %\DeclareTranslation{Romanian}{editor}{}%
    %\DeclareTranslation{Romanian}{editorfilepicker}{}%
    %\DeclareTranslation{Romanian}{plain}{}%
    %\DeclareTranslation{Romanian}{monospaced}{}%
    %\DeclareTranslation{Romanian}{noinline}{}%
    %\DeclareTranslation{Romanian}{Total of marks}{}%
    \DeclareTranslation{Spanish}{True}{Verdadero}%
    \DeclareTranslation{Spanish}{False}{Falso}%
    \DeclareTranslation{Spanish}{cloze}{Respuestas anidadas}%
    \DeclareTranslation{Spanish}{description}{Descripci\'on}%
    \DeclareTranslation{Spanish}{essay}{Ensayo}%
    \DeclareTranslation{Spanish}{matching}{Emparejamiento}%
    \DeclareTranslation{Spanish}{multi}{Opci\'on m\'ultiple}%
    \DeclareTranslation{Spanish}{numerical}{Num\'erica}%
    \DeclareTranslation{Spanish}{shortanswer}{Respuesta corta}%
    \DeclareTranslation{Spanish}{truefalse}{Verdadero/Falso}%
    \DeclareTranslation{Spanish}{Shuffle}{Barajar al azar}%
    \DeclareTranslation{Spanish}{Single}{S\'olo una respuesta}%
    \DeclareTranslation{Spanish}{Multiple}{Se permiten varias respuestas}%
    \DeclareTranslation{Spanish}{marked out of}{Punt\'ua como}%
    \DeclareTranslation{Spanish}{penalty}{penalizaci\'on}%
    \DeclareTranslation{Spanish}{tags}{marcas}%
    \DeclareTranslation{Spanish}{All-or-nothing}{Todo o nada}%
    \DeclareTranslation{Spanish}{Case-Sensitive}{May\'usculas y min\'usculas deben coincidir}%
    \DeclareTranslation{Spanish}{Case-Insensitive}{Igual may\'usculas que min\'usculas}%
    \DeclareTranslation{Spanish}{Drag and drop}{Arrastrar y soltar}%
    \DeclareTranslation{Spanish}{Information for graders}{Informaci\'on para evaluadores}%
    \DeclareTranslation{Spanish}{Response template}{Plantilla de respuesta}%
    \DeclareTranslation{Spanish}{editor}{Editor HTML}%
    \DeclareTranslation{Spanish}{editorfilepicker}{Editor \HTML\ con selector de archivos}%
    \DeclareTranslation{Spanish}{plain}{Texto sin formato}%
    \DeclareTranslation{Spanish}{monospaced}{Texto sin formato, tipografía monoespaciada}%
    \DeclareTranslation{Spanish}{noinline}{Sin texto}%
    \DeclareTranslation{Spanish}{Total of marks}{Total de puntos}%
    \DeclareTranslation{Swedish}{True}{Sant}%
    \DeclareTranslation{Swedish}{False}{Falskt}%
    %\DeclareTranslation{Swedish}{cloze}{}%
    %\DeclareTranslation{Swedish}{description}{}%
    %\DeclareTranslation{Swedish}{essay}{}%
    %\DeclareTranslation{Swedish}{matching}{}%
    %\DeclareTranslation{Swedish}{multi}{}%
    %\DeclareTranslation{Swedish}{numerical}{}%
    %\DeclareTranslation{Swedish}{shortanswer}{}%
    %\DeclareTranslation{Swedish}{truefalse}{}%
    %\DeclareTranslation{Swedish}{Shuffle}{}%
    %\DeclareTranslation{Swedish}{Single}{}%
    %\DeclareTranslation{Swedish}{Multiple}{}%
    %\DeclareTranslation{Swedish}{marked out of}{}%
    %\DeclareTranslation{Swedish}{penalty}{}%
    %\DeclareTranslation{Swedish}{tags}{}%
    %\DeclareTranslation{Swedish}{All-or-nothing}{}%
    %\DeclareTranslation{Swedish}{Case-Sensitive}{}%
    %\DeclareTranslation{Swedish}{Case-Insensitive}{}%
    %\DeclareTranslation{Swedish}{Drag and drop}{}%
    %\DeclareTranslation{Swedish}{Information for graders}{}%
    %\DeclareTranslation{Swedish}{Response template}{}%
    %\DeclareTranslation{Swedish}{editor}{}%
    %\DeclareTranslation{Swedish}{editorfilepicker}{}%
    %\DeclareTranslation{Swedish}{plain}{}%
    %\DeclareTranslation{Swedish}{monospaced}{}%
    %\DeclareTranslation{Swedish}{noinline}{}%
    %\DeclareTranslation{Swedish}{Total of marks}{}%
    \DeclareTranslation{Turkish}{True}{Do\u{g}ru}%
    \DeclareTranslation{Turkish}{False}{Yanl\i \c{s}}%
    %\DeclareTranslation{Turkish}{cloze}{}%
    %\DeclareTranslation{Turkish}{description}{}%
    %\DeclareTranslation{Turkish}{essay}{}%
    %\DeclareTranslation{Turkish}{matching}{}%
    %\DeclareTranslation{Turkish}{multi}{}%
    %\DeclareTranslation{Turkish}{numerical}{}%
    %\DeclareTranslation{Turkish}{shortanswer}{}%
    %\DeclareTranslation{Turkish}{truefalse}{}%
    %\DeclareTranslation{Turkish}{Shuffle}{}%
    %\DeclareTranslation{Turkish}{Single}{}%
    %\DeclareTranslation{Turkish}{Multiple}{}%
    %\DeclareTranslation{Turkish}{marked out of}{}%
    %\DeclareTranslation{Turkish}{penalty}{}%
    %\DeclareTranslation{Turkish}{tags}{}%
    %\DeclareTranslation{Turkish}{All-or-nothing}{}%
    %\DeclareTranslation{Turkish}{Case-Sensitive}{}%
    %\DeclareTranslation{Turkish}{Case-Insensitive}{}%
    %\DeclareTranslation{Turkish}{Drag and drop}{}%
    %\DeclareTranslation{Turkish}{Information for graders}{}%
    %\DeclareTranslation{Turkish}{Response template}{}%
    %\DeclareTranslation{Turkish}{editor}{}%
    %\DeclareTranslation{Turkish}{editorfilepicker}{}%
    %\DeclareTranslation{Turkish}{plain}{}%
    %\DeclareTranslation{Turkish}{monospaced}{}%
    %\DeclareTranslation{Turkish}{noinline}{}%
    %\DeclareTranslation{Turkish}{Total of marks}{}%
  \else% neither babel nor polyglossia loaded
    \newcommand\DeclareTranslationFallback[2]{\csdef{moodle@fallback@#1}{#2}}% Save fallback keys
    \def\GetTranslation#1{\csuse{moodle@fallback@#1}}% Use fallback keys
  \fi
  \DeclareTranslationFallback{True}{True}%
  \DeclareTranslationFallback{False}{False}%
  \DeclareTranslationFallback{cloze}{Embedded answers}%
  \DeclareTranslationFallback{description}{Description}%
  \DeclareTranslationFallback{essay}{Essay}%
  \DeclareTranslationFallback{matching}{Matching}%
  \DeclareTranslationFallback{multi}{Multiple choice}%
  \DeclareTranslationFallback{numerical}{Numerical}%
  \DeclareTranslationFallback{shortanswer}{Short answer}%
  \DeclareTranslationFallback{truefalse}{True/False}%
  \DeclareTranslationFallback{Shuffle}{Shuffle}%
  \DeclareTranslationFallback{Single}{One answer only}%
  \DeclareTranslationFallback{Multiple}{Multiple answers allowed}%
  \DeclareTranslationFallback{marked out of}{marked out of}%
  \DeclareTranslationFallback{penalty}{penalty}%
  \DeclareTranslationFallback{tags}{tags}%
  \DeclareTranslationFallback{All-or-nothing}{All-or-nothing}%
  \DeclareTranslationFallback{Case-Sensitive}{Case-Sensitive}%
  \DeclareTranslationFallback{Case-Insensitive}{Case-Insensitive}%
  \DeclareTranslationFallback{Drag and drop}{Drag and drop}%
  \DeclareTranslationFallback{Information for graders}{Information for graders}%
  \DeclareTranslationFallback{Response template}{Response template}%
  \DeclareTranslationFallback{editor}{HTML editor}%
  \DeclareTranslationFallback{editorfilepicker}{HTML editor + file picker}%
  \DeclareTranslationFallback{plain}{Plain text}%
  \DeclareTranslationFallback{monospaced}{Plain text, monospaced font}%
  \DeclareTranslationFallback{noinline}{File picker}%
  \DeclareTranslationFallback{Total of marks}{Total of marks}%
}%
%    \end{macrocode}
%
% Users of the |babel| package loaded with specific options may experience problems
% related to active characters and shorthands. We tell the problems and solutions we know.
%    \begin{macrocode}
\AtBeginDocument{%
  \@ifpackageloaded{babel}{%
    \@ifundefined{bbl@loaded}{}{% polyglossia fakes babel
      \forcsvlist{\listadd\moodle@babel@german@list}{german,austrian,swissgerman,swissgerman.toss,ngerman,naustrian,nswissgerman,nswissgerman.toss}%
      \forcsvlist{\listadd\moodle@babel@french@list}{acadian,french,frenchb,francais}%
      \ifPDFTeX
        \def\moodle@babel@french@warn{%
          \PackageWarning{moodle}{Be careful when using pdflatex, moodle, and the
            babel package loaded with a French-related option. Autospacing
            yields undesired symbols in the XML. Here are three workarounds:
            \MessageBreak 1) compile with xelatex/lualatex,\MessageBreak 2)
            add `\NoAutoSpacing' after `\begin{quiz}', or\MessageBreak 3)
            avoid the babel option.^^J}}%
        \def\moodle@babel@german@warn{%
          \PackageWarning{moodle}{Be careful when using pdflatex, moodle, and
            the babel package loaded with a German-related option. The symbol
            `"' is made active and your umlauts will confuse moodle. Here are
            three workarounds:\MessageBreak 1) compile with xelatex/lualatex,
            \MessageBreak 2) add `\string\shorthandoff{"}' after `\begin{quiz}',
            or\MessageBreak 3) avoid the babel option.^^J}}%
      \fi
      \def\moodle@babel@turkish@warn{%
        \PackageWarning{moodle}{Be careful when using moodle and the babel
          package loaded with option `turkish'. Shorthands are likely to
          break compilation. Here are three workarounds:\MessageBreak 1)
          rely on `polyglossia' (XeTeX or LuaTeX),\MessageBreak 2) add
          `\shorthandoff{=:!}' after `\begin{quiz}', or\MessageBreak
          3) avoid the babel option.^^J}}%
      \def\do#1{%
        \ifstrequal{#1}{turkish}{\moodle@babel@turkish@warn}{%
          \ifinlist{#1}{\moodle@babel@french@list}%
            {\ifPDFTeX\moodle@babel@french@warn\fi}{%
              \ifinlist{#1}{\moodle@babel@german@list}{%
                \ifPDFTeX\moodle@babel@german@warn\else
                  \gdef\moodle@babel@german@warn{%
                    \PackageWarning{moodle}{Be careful when using moodle and
            the babel package loaded with a German-related option that make the
            symbol `"' active. We recommend typing umlauts with UTF-8
            characters. By default, `\string\"a' is poorly translated in the XML
            file. This is fixed if you define `\string\shorthandoff{"}' after
            `\begin{quiz}'. But then `"a` will no longer work as expected.^^J}%
                  }%
                  \gdef\moodle@babel@german@warn@single{%
                    \moodle@babel@german@warn
                    \gdef\moodle@babel@german@warn@single{}%\relax
                  }%
                  \pretocmd{\"}%
                    {\moodle@babel@german@warn@single}%
                    {}%
                    { \PackageWarning{moodle}{Umlaut patch failed.}%
                      \moodle@babel@german@warn@single
                    }%
                \fi
              }{}%
            }

        }
      }
      \expandafter\docsvlist\expandafter{\bbl@loaded}%
    }%
  }{}%
}%
%    \end{macrocode}
%
% \subsection{Warning and Error Management}
%
%    \begin{macrocode}
\def\moodle@WarningOrError#1#2#3{%
  % #1 : major version
  % #2 : minor version
  % #3 : feature
  \ifnum\the\moodle@LMSmajor\the\moodle@LMSminor=\z@\relax
    \PackageWarning{moodle}{#3 is only supported by Moodle #1.#2+}%
  \else
    \ifnum\numexpr 1000*\the\moodle@LMSmajor+\the\moodle@LMSminor\relax<\numexpr 1000*#1+#2\relax
      \PackageError{moodle}{#3 is only supported by Moodle #1.#2+
       (you declared Moodle \the\moodle@LMSmajor.\the\moodle@LMSminor)}%
    \fi
  \fi
}%
%    \end{macrocode}
% \Finale
\endinput
